\PassOptionsToPackage{unicode=true}{hyperref} % options for packages loaded elsewhere
\PassOptionsToPackage{hyphens}{url}
%
\documentclass[]{article}
\usepackage{lmodern}
\usepackage{amssymb,amsmath}
\usepackage{ifxetex,ifluatex}
\usepackage{fixltx2e} % provides \textsubscript
\ifnum 0\ifxetex 1\fi\ifluatex 1\fi=0 % if pdftex
  \usepackage[T1]{fontenc}
  \usepackage[utf8]{inputenc}
  \usepackage{textcomp} % provides euro and other symbols
\else % if luatex or xelatex
  \usepackage{unicode-math}
  \defaultfontfeatures{Ligatures=TeX,Scale=MatchLowercase}
\fi
% use upquote if available, for straight quotes in verbatim environments
\IfFileExists{upquote.sty}{\usepackage{upquote}}{}
% use microtype if available
\IfFileExists{microtype.sty}{%
\usepackage[]{microtype}
\UseMicrotypeSet[protrusion]{basicmath} % disable protrusion for tt fonts
}{}
\IfFileExists{parskip.sty}{%
\usepackage{parskip}
}{% else
\setlength{\parindent}{0pt}
\setlength{\parskip}{6pt plus 2pt minus 1pt}
}
\usepackage{hyperref}
\hypersetup{
            pdftitle={Dplyr-like API for tech.ml.dataset},
            pdfauthor={GenerateMe},
            pdfborder={0 0 0},
            breaklinks=true}
\urlstyle{same}  % don't use monospace font for urls
\usepackage[margin=1in]{geometry}
\usepackage{color}
\usepackage{fancyvrb}
\newcommand{\VerbBar}{|}
\newcommand{\VERB}{\Verb[commandchars=\\\{\}]}
\DefineVerbatimEnvironment{Highlighting}{Verbatim}{commandchars=\\\{\}}
% Add ',fontsize=\small' for more characters per line
\usepackage{framed}
\definecolor{shadecolor}{RGB}{248,248,248}
\newenvironment{Shaded}{\begin{snugshade}}{\end{snugshade}}
\newcommand{\AlertTok}[1]{\textcolor[rgb]{0.94,0.16,0.16}{#1}}
\newcommand{\AnnotationTok}[1]{\textcolor[rgb]{0.56,0.35,0.01}{\textbf{\textit{#1}}}}
\newcommand{\AttributeTok}[1]{\textcolor[rgb]{0.77,0.63,0.00}{#1}}
\newcommand{\BaseNTok}[1]{\textcolor[rgb]{0.00,0.00,0.81}{#1}}
\newcommand{\BuiltInTok}[1]{#1}
\newcommand{\CharTok}[1]{\textcolor[rgb]{0.31,0.60,0.02}{#1}}
\newcommand{\CommentTok}[1]{\textcolor[rgb]{0.56,0.35,0.01}{\textit{#1}}}
\newcommand{\CommentVarTok}[1]{\textcolor[rgb]{0.56,0.35,0.01}{\textbf{\textit{#1}}}}
\newcommand{\ConstantTok}[1]{\textcolor[rgb]{0.00,0.00,0.00}{#1}}
\newcommand{\ControlFlowTok}[1]{\textcolor[rgb]{0.13,0.29,0.53}{\textbf{#1}}}
\newcommand{\DataTypeTok}[1]{\textcolor[rgb]{0.13,0.29,0.53}{#1}}
\newcommand{\DecValTok}[1]{\textcolor[rgb]{0.00,0.00,0.81}{#1}}
\newcommand{\DocumentationTok}[1]{\textcolor[rgb]{0.56,0.35,0.01}{\textbf{\textit{#1}}}}
\newcommand{\ErrorTok}[1]{\textcolor[rgb]{0.64,0.00,0.00}{\textbf{#1}}}
\newcommand{\ExtensionTok}[1]{#1}
\newcommand{\FloatTok}[1]{\textcolor[rgb]{0.00,0.00,0.81}{#1}}
\newcommand{\FunctionTok}[1]{\textcolor[rgb]{0.00,0.00,0.00}{#1}}
\newcommand{\ImportTok}[1]{#1}
\newcommand{\InformationTok}[1]{\textcolor[rgb]{0.56,0.35,0.01}{\textbf{\textit{#1}}}}
\newcommand{\KeywordTok}[1]{\textcolor[rgb]{0.13,0.29,0.53}{\textbf{#1}}}
\newcommand{\NormalTok}[1]{#1}
\newcommand{\OperatorTok}[1]{\textcolor[rgb]{0.81,0.36,0.00}{\textbf{#1}}}
\newcommand{\OtherTok}[1]{\textcolor[rgb]{0.56,0.35,0.01}{#1}}
\newcommand{\PreprocessorTok}[1]{\textcolor[rgb]{0.56,0.35,0.01}{\textit{#1}}}
\newcommand{\RegionMarkerTok}[1]{#1}
\newcommand{\SpecialCharTok}[1]{\textcolor[rgb]{0.00,0.00,0.00}{#1}}
\newcommand{\SpecialStringTok}[1]{\textcolor[rgb]{0.31,0.60,0.02}{#1}}
\newcommand{\StringTok}[1]{\textcolor[rgb]{0.31,0.60,0.02}{#1}}
\newcommand{\VariableTok}[1]{\textcolor[rgb]{0.00,0.00,0.00}{#1}}
\newcommand{\VerbatimStringTok}[1]{\textcolor[rgb]{0.31,0.60,0.02}{#1}}
\newcommand{\WarningTok}[1]{\textcolor[rgb]{0.56,0.35,0.01}{\textbf{\textit{#1}}}}
\usepackage{longtable,booktabs}
% Fix footnotes in tables (requires footnote package)
\IfFileExists{footnote.sty}{\usepackage{footnote}\makesavenoteenv{longtable}}{}
\usepackage{graphicx,grffile}
\makeatletter
\def\maxwidth{\ifdim\Gin@nat@width>\linewidth\linewidth\else\Gin@nat@width\fi}
\def\maxheight{\ifdim\Gin@nat@height>\textheight\textheight\else\Gin@nat@height\fi}
\makeatother
% Scale images if necessary, so that they will not overflow the page
% margins by default, and it is still possible to overwrite the defaults
% using explicit options in \includegraphics[width, height, ...]{}
\setkeys{Gin}{width=\maxwidth,height=\maxheight,keepaspectratio}
\setlength{\emergencystretch}{3em}  % prevent overfull lines
\providecommand{\tightlist}{%
  \setlength{\itemsep}{0pt}\setlength{\parskip}{0pt}}
\setcounter{secnumdepth}{0}
% Redefines (sub)paragraphs to behave more like sections
\ifx\paragraph\undefined\else
\let\oldparagraph\paragraph
\renewcommand{\paragraph}[1]{\oldparagraph{#1}\mbox{}}
\fi
\ifx\subparagraph\undefined\else
\let\oldsubparagraph\subparagraph
\renewcommand{\subparagraph}[1]{\oldsubparagraph{#1}\mbox{}}
\fi

% set default figure placement to htbp
\makeatletter
\def\fps@figure{htbp}
\makeatother


\title{Dplyr-like API for tech.ml.dataset}
\author{GenerateMe}
\date{2020-09-14}

\begin{document}
\maketitle

\hypertarget{introduction}{%
\subsection{Introduction}\label{introduction}}

\href{https://github.com/techascent/tech.ml.dataset}{tech.ml.dataset} is
a great and fast library which brings columnar dataset to the Clojure.
Chris Nuernberger has been working on this library for last year as a
part of bigger \texttt{tech.ml} stack.

I've started to test the library and help to fix uncovered bugs. My main
goal was to compare functionalities with the other standards from other
platforms. I focused on R solutions:
\href{https://dplyr.tidyverse.org/}{dplyr},
\href{https://tidyr.tidyverse.org/}{tidyr} and
\href{https://rdatatable.gitlab.io/data.table/}{data.table}.

During conversions of the examples I've come up how to reorganized
existing \texttt{tech.ml.dataset} functions into simple to use API. The
main goals were:

\begin{itemize}
\tightlist
\item
  Focus on dataset manipulation functionality, leaving other parts of
  \texttt{tech.ml} like pipelines, datatypes, readers, ML, etc.
\item
  Single entry point for common operations - one function dispatching on
  given arguments.
\item
  \texttt{group-by} results with special kind of dataset - a dataset
  containing subsets created after grouping as a column.
\item
  Most operations recognize regular dataset and grouped dataset and
  process data accordingly.
\item
  One function form to enable thread-first on dataset.
\end{itemize}

If you want to know more about \texttt{tech.ml.dataset} and
\texttt{tech.ml.datatype} please refer their documentation:

\begin{itemize}
\tightlist
\item
  \href{https://github.com/techascent/tech.datatype/blob/master/docs/cheatsheet.md}{Datatype}
\item
  \href{https://github.com/techascent/tech.datatype/blob/master/docs/datetime.md}{Date/time}
\item
  \href{https://github.com/techascent/tech.ml.dataset/blob/master/docs/walkthrough.md}{Dataset}
\end{itemize}

\href{https://github.com/scicloj/tablecloth}{SOURCE CODE}

Join the discussion on
\href{https://clojurians.zulipchat.com/\#narrow/stream/236259-tech.2Eml.2Edataset.2Edev/topic/api}{Zulip}

Let's require main namespace and define dataset used in most examples:

\begin{Shaded}
\begin{Highlighting}[]
\NormalTok{(}\KeywordTok{require}\NormalTok{ '[tablecloth.api }\AttributeTok{:as}\NormalTok{ api])}
\NormalTok{(}\BuiltInTok{def}\FunctionTok{ DS }\NormalTok{(api/dataset \{}\AttributeTok{:V1}\NormalTok{ (}\KeywordTok{take} \DecValTok{9}\NormalTok{ (}\KeywordTok{cycle}\NormalTok{ [}\DecValTok{1} \DecValTok{2}\NormalTok{]))}
                      \AttributeTok{:V2}\NormalTok{ (}\KeywordTok{range} \DecValTok{1} \DecValTok{10}\NormalTok{)}
                      \AttributeTok{:V3}\NormalTok{ (}\KeywordTok{take} \DecValTok{9}\NormalTok{ (}\KeywordTok{cycle}\NormalTok{ [}\FloatTok{0.5} \FloatTok{1.0} \FloatTok{1.5}\NormalTok{]))}
                      \AttributeTok{:V4}\NormalTok{ (}\KeywordTok{take} \DecValTok{9}\NormalTok{ (}\KeywordTok{cycle}\NormalTok{ [}\StringTok{"A"} \StringTok{"B"} \StringTok{"C"}\NormalTok{]))\}))}
\end{Highlighting}
\end{Shaded}

\begin{Shaded}
\begin{Highlighting}[]
\NormalTok{DS}
\end{Highlighting}
\end{Shaded}

\_unnamed {[}9 4{]}:

\begin{longtable}[]{@{}llll@{}}
\toprule
:V1 & :V2 & :V3 & :V4\tabularnewline
\midrule
\endhead
1 & 1 & 0.5 & A\tabularnewline
2 & 2 & 1.0 & B\tabularnewline
1 & 3 & 1.5 & C\tabularnewline
2 & 4 & 0.5 & A\tabularnewline
1 & 5 & 1.0 & B\tabularnewline
2 & 6 & 1.5 & C\tabularnewline
1 & 7 & 0.5 & A\tabularnewline
2 & 8 & 1.0 & B\tabularnewline
1 & 9 & 1.5 & C\tabularnewline
\bottomrule
\end{longtable}

\hypertarget{functionality}{%
\subsection{Functionality}\label{functionality}}

\hypertarget{dataset}{%
\subsubsection{Dataset}\label{dataset}}

Dataset is a special type which can be considered as a map of columns
implemented around \texttt{tech.ml.datatype} library. Each column can be
considered as named sequence of typed data. Supported types include
integers, floats, string, boolean, date/time, objects etc.

\hypertarget{dataset-creation}{%
\paragraph{Dataset creation}\label{dataset-creation}}

Dataset can be created from various of types of Clojure structures and
files:

\begin{itemize}
\tightlist
\item
  single values
\item
  sequence of maps
\item
  map of sequences or values
\item
  sequence of columns (taken from other dataset or created manually)
\item
  sequence of pairs
\item
  file types: raw/gzipped csv/tsv, json, xls(x) taken from local file
  system or URL
\item
  input stream
\end{itemize}

\texttt{api/dataset} accepts:

\begin{itemize}
\tightlist
\item
  data
\item
  options (see documentation of
  \texttt{tech.ml.dataset/-\textgreater{}dataset} function for full
  list):

  \begin{itemize}
  \tightlist
  \item
    \texttt{:dataset-name} - name of the dataset
  \item
    \texttt{:num-rows} - number of rows to read from file
  \item
    \texttt{:header-row?} - indication if first row in file is a header
  \item
    \texttt{:key-fn} - function applied to column names (eg.
    \texttt{keyword}, to convert column names to keywords)
  \item
    \texttt{:separator} - column separator
  \item
    \texttt{:single-value-column-name} - name of the column when single
    value is provided
  \end{itemize}
\end{itemize}

\begin{center}\rule{0.5\linewidth}{0.5pt}\end{center}

Empty dataset.

\begin{Shaded}
\begin{Highlighting}[]
\NormalTok{(api/dataset)}
\end{Highlighting}
\end{Shaded}

\begin{verbatim}
_unnamed [0 0]
\end{verbatim}

\begin{center}\rule{0.5\linewidth}{0.5pt}\end{center}

Dataset from single value.

\begin{Shaded}
\begin{Highlighting}[]
\NormalTok{(api/dataset }\DecValTok{999}\NormalTok{)}
\end{Highlighting}
\end{Shaded}

\_unnamed {[}1 1{]}:

\begin{longtable}[]{@{}l@{}}
\toprule
:\$value\tabularnewline
\midrule
\endhead
999\tabularnewline
\bottomrule
\end{longtable}

\begin{center}\rule{0.5\linewidth}{0.5pt}\end{center}

Set column name for single value. Also set the dataset name.

\begin{Shaded}
\begin{Highlighting}[]
\NormalTok{(api/dataset }\DecValTok{999}\NormalTok{ \{}\AttributeTok{:single-value-column-name} \StringTok{"my-single-value"}\NormalTok{\})}
\NormalTok{(api/dataset }\DecValTok{999}\NormalTok{ \{}\AttributeTok{:single-value-column-name} \StringTok{""}
                  \AttributeTok{:dataset-name} \StringTok{"Single value"}\NormalTok{\})}
\end{Highlighting}
\end{Shaded}

\_unnamed {[}1 1{]}:

\begin{longtable}[]{@{}l@{}}
\toprule
my-single-value\tabularnewline
\midrule
\endhead
999\tabularnewline
\bottomrule
\end{longtable}

Single value {[}1 1{]}:

\begin{longtable}[]{@{}l@{}}
\toprule
0\tabularnewline
\midrule
\endhead
999\tabularnewline
\bottomrule
\end{longtable}

\begin{center}\rule{0.5\linewidth}{0.5pt}\end{center}

Sequence of pairs (first = column name, second = value(s)).

\begin{Shaded}
\begin{Highlighting}[]
\NormalTok{(api/dataset [[}\AttributeTok{:A} \DecValTok{33}\NormalTok{] [}\AttributeTok{:B} \DecValTok{5}\NormalTok{] [}\AttributeTok{:C} \AttributeTok{:a}\NormalTok{]])}
\end{Highlighting}
\end{Shaded}

\_unnamed {[}1 3{]}:

\begin{longtable}[]{@{}lll@{}}
\toprule
:A & :B & :C\tabularnewline
\midrule
\endhead
33 & 5 & :a\tabularnewline
\bottomrule
\end{longtable}

\begin{center}\rule{0.5\linewidth}{0.5pt}\end{center}

Not sequential values are repeated row-count number of times.

\begin{Shaded}
\begin{Highlighting}[]
\NormalTok{(api/dataset [[}\AttributeTok{:A}\NormalTok{ [}\DecValTok{1} \DecValTok{2} \DecValTok{3} \DecValTok{4} \DecValTok{5} \DecValTok{6}\NormalTok{]] [}\AttributeTok{:B} \StringTok{"X"}\NormalTok{] [}\AttributeTok{:C} \AttributeTok{:a}\NormalTok{]])}
\end{Highlighting}
\end{Shaded}

\_unnamed {[}6 3{]}:

\begin{longtable}[]{@{}lll@{}}
\toprule
:A & :B & :C\tabularnewline
\midrule
\endhead
1 & X & :a\tabularnewline
2 & X & :a\tabularnewline
3 & X & :a\tabularnewline
4 & X & :a\tabularnewline
5 & X & :a\tabularnewline
6 & X & :a\tabularnewline
\bottomrule
\end{longtable}

\begin{center}\rule{0.5\linewidth}{0.5pt}\end{center}

Dataset created from map (keys = column names, vals = value(s)). Works
the same as sequence of pairs.

\begin{Shaded}
\begin{Highlighting}[]
\NormalTok{(api/dataset \{}\AttributeTok{:A} \DecValTok{33}\NormalTok{\})}
\NormalTok{(api/dataset \{}\AttributeTok{:A}\NormalTok{ [}\DecValTok{1} \DecValTok{2} \DecValTok{3}\NormalTok{]\})}
\NormalTok{(api/dataset \{}\AttributeTok{:A}\NormalTok{ [}\DecValTok{3} \DecValTok{4} \DecValTok{5}\NormalTok{] }\AttributeTok{:B} \StringTok{"X"}\NormalTok{\})}
\end{Highlighting}
\end{Shaded}

\_unnamed {[}1 1{]}:

\begin{longtable}[]{@{}l@{}}
\toprule
:A\tabularnewline
\midrule
\endhead
33\tabularnewline
\bottomrule
\end{longtable}

\_unnamed {[}3 1{]}:

\begin{longtable}[]{@{}l@{}}
\toprule
:A\tabularnewline
\midrule
\endhead
1\tabularnewline
2\tabularnewline
3\tabularnewline
\bottomrule
\end{longtable}

\_unnamed {[}3 2{]}:

\begin{longtable}[]{@{}ll@{}}
\toprule
:A & :B\tabularnewline
\midrule
\endhead
3 & X\tabularnewline
4 & X\tabularnewline
5 & X\tabularnewline
\bottomrule
\end{longtable}

\begin{center}\rule{0.5\linewidth}{0.5pt}\end{center}

You can put any value inside a column

\begin{Shaded}
\begin{Highlighting}[]
\NormalTok{(api/dataset \{}\AttributeTok{:A}\NormalTok{ [[}\DecValTok{3} \DecValTok{4} \DecValTok{5}\NormalTok{] [}\AttributeTok{:a} \AttributeTok{:b}\NormalTok{]] }\AttributeTok{:B} \StringTok{"X"}\NormalTok{\})}
\end{Highlighting}
\end{Shaded}

\_unnamed {[}2 2{]}:

\begin{longtable}[]{@{}ll@{}}
\toprule
:A & :B\tabularnewline
\midrule
\endhead
{[}3 4 5{]} & X\tabularnewline
{[}:a :b{]} & X\tabularnewline
\bottomrule
\end{longtable}

\begin{center}\rule{0.5\linewidth}{0.5pt}\end{center}

Sequence of maps

\begin{Shaded}
\begin{Highlighting}[]
\NormalTok{(api/dataset [\{}\AttributeTok{:a} \DecValTok{1} \AttributeTok{:b} \DecValTok{3}\NormalTok{\} \{}\AttributeTok{:b} \DecValTok{2} \AttributeTok{:a} \DecValTok{99}\NormalTok{\}])}
\NormalTok{(api/dataset [\{}\AttributeTok{:a} \DecValTok{1} \AttributeTok{:b}\NormalTok{ [}\DecValTok{1} \DecValTok{2} \DecValTok{3}\NormalTok{]\} \{}\AttributeTok{:a} \DecValTok{2} \AttributeTok{:b}\NormalTok{ [}\DecValTok{3} \DecValTok{4}\NormalTok{]\}])}
\end{Highlighting}
\end{Shaded}

\_unnamed {[}2 2{]}:

\begin{longtable}[]{@{}ll@{}}
\toprule
:a & :b\tabularnewline
\midrule
\endhead
1 & 3\tabularnewline
99 & 2\tabularnewline
\bottomrule
\end{longtable}

\_unnamed {[}2 2{]}:

\begin{longtable}[]{@{}ll@{}}
\toprule
:a & :b\tabularnewline
\midrule
\endhead
1 & {[}1 2 3{]}\tabularnewline
2 & {[}3 4{]}\tabularnewline
\bottomrule
\end{longtable}

\begin{center}\rule{0.5\linewidth}{0.5pt}\end{center}

Missing values are marked by \texttt{nil}

\begin{Shaded}
\begin{Highlighting}[]
\NormalTok{(api/dataset [\{}\AttributeTok{:a} \VariableTok{nil} \AttributeTok{:b} \DecValTok{1}\NormalTok{\} \{}\AttributeTok{:a} \DecValTok{3} \AttributeTok{:b} \DecValTok{4}\NormalTok{\} \{}\AttributeTok{:a} \DecValTok{11}\NormalTok{\}])}
\end{Highlighting}
\end{Shaded}

\_unnamed {[}3 2{]}:

\begin{longtable}[]{@{}ll@{}}
\toprule
:a & :b\tabularnewline
\midrule
\endhead
& 1\tabularnewline
3 & 4\tabularnewline
11 &\tabularnewline
\bottomrule
\end{longtable}

\begin{center}\rule{0.5\linewidth}{0.5pt}\end{center}

Import CSV file

\begin{Shaded}
\begin{Highlighting}[]
\NormalTok{(api/dataset }\StringTok{"data/family.csv"}\NormalTok{)}
\end{Highlighting}
\end{Shaded}

data/family.csv {[}5 5{]}:

\begin{longtable}[]{@{}lllll@{}}
\toprule
family & dob\_child1 & dob\_child2 & gender\_child1 &
gender\_child2\tabularnewline
\midrule
\endhead
1 & 1998-11-26 & 2000-01-29 & 1 & 2\tabularnewline
2 & 1996-06-22 & & 2 &\tabularnewline
3 & 2002-07-11 & 2004-04-05 & 2 & 2\tabularnewline
4 & 2004-10-10 & 2009-08-27 & 1 & 1\tabularnewline
5 & 2000-12-05 & 2005-02-28 & 2 & 1\tabularnewline
\bottomrule
\end{longtable}

\begin{center}\rule{0.5\linewidth}{0.5pt}\end{center}

Import from URL

\begin{Shaded}
\begin{Highlighting}[]
\NormalTok{(}\BuiltInTok{defonce}\FunctionTok{ ds }\NormalTok{(api/dataset }\StringTok{"https://vega.github.io/vega-lite/examples/data/seattle-weather.csv"}\NormalTok{))}
\end{Highlighting}
\end{Shaded}

\begin{Shaded}
\begin{Highlighting}[]
\NormalTok{ds}
\end{Highlighting}
\end{Shaded}

\url{https://vega.github.io/vega-lite/examples/data/seattle-weather.csv}
{[}1461 6{]}:

\begin{longtable}[]{@{}llllll@{}}
\toprule
date & precipitation & temp\_max & temp\_min & wind &
weather\tabularnewline
\midrule
\endhead
2012-01-01 & 0.0 & 12.8 & 5.0 & 4.7 & drizzle\tabularnewline
2012-01-02 & 10.9 & 10.6 & 2.8 & 4.5 & rain\tabularnewline
2012-01-03 & 0.8 & 11.7 & 7.2 & 2.3 & rain\tabularnewline
2012-01-04 & 20.3 & 12.2 & 5.6 & 4.7 & rain\tabularnewline
2012-01-05 & 1.3 & 8.9 & 2.8 & 6.1 & rain\tabularnewline
2012-01-06 & 2.5 & 4.4 & 2.2 & 2.2 & rain\tabularnewline
2012-01-07 & 0.0 & 7.2 & 2.8 & 2.3 & rain\tabularnewline
2012-01-08 & 0.0 & 10.0 & 2.8 & 2.0 & sun\tabularnewline
2012-01-09 & 4.3 & 9.4 & 5.0 & 3.4 & rain\tabularnewline
2012-01-10 & 1.0 & 6.1 & 0.6 & 3.4 & rain\tabularnewline
2012-01-11 & 0.0 & 6.1 & -1.1 & 5.1 & sun\tabularnewline
2012-01-12 & 0.0 & 6.1 & -1.7 & 1.9 & sun\tabularnewline
2012-01-13 & 0.0 & 5.0 & -2.8 & 1.3 & sun\tabularnewline
2012-01-14 & 4.1 & 4.4 & 0.6 & 5.3 & snow\tabularnewline
2012-01-15 & 5.3 & 1.1 & -3.3 & 3.2 & snow\tabularnewline
2012-01-16 & 2.5 & 1.7 & -2.8 & 5.0 & snow\tabularnewline
2012-01-17 & 8.1 & 3.3 & 0.0 & 5.6 & snow\tabularnewline
2012-01-18 & 19.8 & 0.0 & -2.8 & 5.0 & snow\tabularnewline
2012-01-19 & 15.2 & -1.1 & -2.8 & 1.6 & snow\tabularnewline
2012-01-20 & 13.5 & 7.2 & -1.1 & 2.3 & snow\tabularnewline
2012-01-21 & 3.0 & 8.3 & 3.3 & 8.2 & rain\tabularnewline
2012-01-22 & 6.1 & 6.7 & 2.2 & 4.8 & rain\tabularnewline
2012-01-23 & 0.0 & 8.3 & 1.1 & 3.6 & rain\tabularnewline
2012-01-24 & 8.6 & 10.0 & 2.2 & 5.1 & rain\tabularnewline
2012-01-25 & 8.1 & 8.9 & 4.4 & 5.4 & rain\tabularnewline
\bottomrule
\end{longtable}

\hypertarget{saving}{%
\paragraph{Saving}\label{saving}}

Export dataset to a file or output stream can be done by calling
\texttt{api/write-csv!}. Function accepts:

\begin{itemize}
\tightlist
\item
  dataset
\item
  file name with one of the extensions: \texttt{.csv}, \texttt{.tsv},
  \texttt{.csv.gz} and \texttt{.tsv.gz} or output stream
\item
  options:

  \begin{itemize}
  \tightlist
  \item
    \texttt{:separator} - string or separator char.
  \end{itemize}
\end{itemize}

\begin{Shaded}
\begin{Highlighting}[]
\NormalTok{(api/write-csv! ds }\StringTok{"output.tsv.gz"}\NormalTok{)}
\NormalTok{(.exists (clojure.java.io/file }\StringTok{"output.tsv.gz"}\NormalTok{))}
\end{Highlighting}
\end{Shaded}

\begin{verbatim}
nil
true
\end{verbatim}

\hypertarget{nippy}{%
\subparagraph{Nippy}\label{nippy}}

\begin{Shaded}
\begin{Highlighting}[]
\NormalTok{(api/write-nippy! DS }\StringTok{"output.nippy.gz"}\NormalTok{)}
\end{Highlighting}
\end{Shaded}

\begin{verbatim}
nil
\end{verbatim}

\begin{Shaded}
\begin{Highlighting}[]
\NormalTok{(api/read-nippy }\StringTok{"output.nippy.gz"}\NormalTok{)}
\end{Highlighting}
\end{Shaded}

\_unnamed {[}9 4{]}:

\begin{longtable}[]{@{}llll@{}}
\toprule
:V1 & :V2 & :V3 & :V4\tabularnewline
\midrule
\endhead
1 & 1 & 0.5 & A\tabularnewline
2 & 2 & 1.0 & B\tabularnewline
1 & 3 & 1.5 & C\tabularnewline
2 & 4 & 0.5 & A\tabularnewline
1 & 5 & 1.0 & B\tabularnewline
2 & 6 & 1.5 & C\tabularnewline
1 & 7 & 0.5 & A\tabularnewline
2 & 8 & 1.0 & B\tabularnewline
1 & 9 & 1.5 & C\tabularnewline
\bottomrule
\end{longtable}

\hypertarget{dataset-related-functions}{%
\paragraph{Dataset related functions}\label{dataset-related-functions}}

Summary functions about the dataset like number of rows, columns and
basic stats.

\begin{center}\rule{0.5\linewidth}{0.5pt}\end{center}

Number of rows

\begin{Shaded}
\begin{Highlighting}[]
\NormalTok{(api/row-count ds)}
\end{Highlighting}
\end{Shaded}

\begin{verbatim}
1461
\end{verbatim}

\begin{center}\rule{0.5\linewidth}{0.5pt}\end{center}

Number of columns

\begin{Shaded}
\begin{Highlighting}[]
\NormalTok{(api/column-count ds)}
\end{Highlighting}
\end{Shaded}

\begin{verbatim}
6
\end{verbatim}

\begin{center}\rule{0.5\linewidth}{0.5pt}\end{center}

Shape of the dataset, {[}row count, column count{]}

\begin{Shaded}
\begin{Highlighting}[]
\NormalTok{(api/shape ds)}
\end{Highlighting}
\end{Shaded}

\begin{verbatim}
[1461 6]
\end{verbatim}

\begin{center}\rule{0.5\linewidth}{0.5pt}\end{center}

General info about dataset. There are three variants:

\begin{itemize}
\tightlist
\item
  default - containing information about columns with basic statistics

  \begin{itemize}
  \tightlist
  \item
    \texttt{:basic} - just name, row and column count and information if
    dataset is a result of \texttt{group-by} operation
  \item
    \texttt{:columns} - columns' metadata
  \end{itemize}
\end{itemize}

\begin{Shaded}
\begin{Highlighting}[]
\NormalTok{(api/info ds)}
\NormalTok{(api/info ds }\AttributeTok{:basic}\NormalTok{)}
\NormalTok{(api/info ds }\AttributeTok{:columns}\NormalTok{)}
\end{Highlighting}
\end{Shaded}

\url{https://vega.github.io/vega-lite/examples/data/seattle-weather.csv}:
descriptive-stats {[}6 10{]}:

\begin{longtable}[]{@{}llllllllll@{}}
\toprule
\begin{minipage}[b]{0.08\columnwidth}\raggedright
:col-name\strut
\end{minipage} & \begin{minipage}[b]{0.11\columnwidth}\raggedright
:datatype\strut
\end{minipage} & \begin{minipage}[b]{0.05\columnwidth}\raggedright
:n-valid\strut
\end{minipage} & \begin{minipage}[b]{0.06\columnwidth}\raggedright
:n-missing\strut
\end{minipage} & \begin{minipage}[b]{0.06\columnwidth}\raggedright
:min\strut
\end{minipage} & \begin{minipage}[b]{0.06\columnwidth}\raggedright
:mean\strut
\end{minipage} & \begin{minipage}[b]{0.04\columnwidth}\raggedright
:mode\strut
\end{minipage} & \begin{minipage}[b]{0.06\columnwidth}\raggedright
:max\strut
\end{minipage} & \begin{minipage}[b]{0.11\columnwidth}\raggedright
:standard-deviation\strut
\end{minipage} & \begin{minipage}[b]{0.09\columnwidth}\raggedright
:skew\strut
\end{minipage}\tabularnewline
\midrule
\endhead
\begin{minipage}[t]{0.08\columnwidth}\raggedright
date\strut
\end{minipage} & \begin{minipage}[t]{0.11\columnwidth}\raggedright
:packed-local-date\strut
\end{minipage} & \begin{minipage}[t]{0.05\columnwidth}\raggedright
1461\strut
\end{minipage} & \begin{minipage}[t]{0.06\columnwidth}\raggedright
0\strut
\end{minipage} & \begin{minipage}[t]{0.06\columnwidth}\raggedright
2012-01-01\strut
\end{minipage} & \begin{minipage}[t]{0.06\columnwidth}\raggedright
2013-12-31\strut
\end{minipage} & \begin{minipage}[t]{0.04\columnwidth}\raggedright
\strut
\end{minipage} & \begin{minipage}[t]{0.06\columnwidth}\raggedright
2015-12-31\strut
\end{minipage} & \begin{minipage}[t]{0.11\columnwidth}\raggedright
3.64520463E+10\strut
\end{minipage} & \begin{minipage}[t]{0.09\columnwidth}\raggedright
1.98971418E-17\strut
\end{minipage}\tabularnewline
\begin{minipage}[t]{0.08\columnwidth}\raggedright
precipitation\strut
\end{minipage} & \begin{minipage}[t]{0.11\columnwidth}\raggedright
:float64\strut
\end{minipage} & \begin{minipage}[t]{0.05\columnwidth}\raggedright
1461\strut
\end{minipage} & \begin{minipage}[t]{0.06\columnwidth}\raggedright
0\strut
\end{minipage} & \begin{minipage}[t]{0.06\columnwidth}\raggedright
0.000\strut
\end{minipage} & \begin{minipage}[t]{0.06\columnwidth}\raggedright
3.029\strut
\end{minipage} & \begin{minipage}[t]{0.04\columnwidth}\raggedright
\strut
\end{minipage} & \begin{minipage}[t]{0.06\columnwidth}\raggedright
55.90\strut
\end{minipage} & \begin{minipage}[t]{0.11\columnwidth}\raggedright
6.68019432E+00\strut
\end{minipage} & \begin{minipage}[t]{0.09\columnwidth}\raggedright
3.50564372E+00\strut
\end{minipage}\tabularnewline
\begin{minipage}[t]{0.08\columnwidth}\raggedright
temp\_max\strut
\end{minipage} & \begin{minipage}[t]{0.11\columnwidth}\raggedright
:float64\strut
\end{minipage} & \begin{minipage}[t]{0.05\columnwidth}\raggedright
1461\strut
\end{minipage} & \begin{minipage}[t]{0.06\columnwidth}\raggedright
0\strut
\end{minipage} & \begin{minipage}[t]{0.06\columnwidth}\raggedright
-1.600\strut
\end{minipage} & \begin{minipage}[t]{0.06\columnwidth}\raggedright
16.44\strut
\end{minipage} & \begin{minipage}[t]{0.04\columnwidth}\raggedright
\strut
\end{minipage} & \begin{minipage}[t]{0.06\columnwidth}\raggedright
35.60\strut
\end{minipage} & \begin{minipage}[t]{0.11\columnwidth}\raggedright
7.34975810E+00\strut
\end{minipage} & \begin{minipage}[t]{0.09\columnwidth}\raggedright
2.80929992E-01\strut
\end{minipage}\tabularnewline
\begin{minipage}[t]{0.08\columnwidth}\raggedright
temp\_min\strut
\end{minipage} & \begin{minipage}[t]{0.11\columnwidth}\raggedright
:float64\strut
\end{minipage} & \begin{minipage}[t]{0.05\columnwidth}\raggedright
1461\strut
\end{minipage} & \begin{minipage}[t]{0.06\columnwidth}\raggedright
0\strut
\end{minipage} & \begin{minipage}[t]{0.06\columnwidth}\raggedright
-7.100\strut
\end{minipage} & \begin{minipage}[t]{0.06\columnwidth}\raggedright
8.235\strut
\end{minipage} & \begin{minipage}[t]{0.04\columnwidth}\raggedright
\strut
\end{minipage} & \begin{minipage}[t]{0.06\columnwidth}\raggedright
18.30\strut
\end{minipage} & \begin{minipage}[t]{0.11\columnwidth}\raggedright
5.02300418E+00\strut
\end{minipage} & \begin{minipage}[t]{0.09\columnwidth}\raggedright
-2.49458552E-01\strut
\end{minipage}\tabularnewline
\begin{minipage}[t]{0.08\columnwidth}\raggedright
weather\strut
\end{minipage} & \begin{minipage}[t]{0.11\columnwidth}\raggedright
:string\strut
\end{minipage} & \begin{minipage}[t]{0.05\columnwidth}\raggedright
1461\strut
\end{minipage} & \begin{minipage}[t]{0.06\columnwidth}\raggedright
0\strut
\end{minipage} & \begin{minipage}[t]{0.06\columnwidth}\raggedright
\strut
\end{minipage} & \begin{minipage}[t]{0.06\columnwidth}\raggedright
\strut
\end{minipage} & \begin{minipage}[t]{0.04\columnwidth}\raggedright
rain\strut
\end{minipage} & \begin{minipage}[t]{0.06\columnwidth}\raggedright
\strut
\end{minipage} & \begin{minipage}[t]{0.11\columnwidth}\raggedright
\strut
\end{minipage} & \begin{minipage}[t]{0.09\columnwidth}\raggedright
\strut
\end{minipage}\tabularnewline
\begin{minipage}[t]{0.08\columnwidth}\raggedright
wind\strut
\end{minipage} & \begin{minipage}[t]{0.11\columnwidth}\raggedright
:float64\strut
\end{minipage} & \begin{minipage}[t]{0.05\columnwidth}\raggedright
1461\strut
\end{minipage} & \begin{minipage}[t]{0.06\columnwidth}\raggedright
0\strut
\end{minipage} & \begin{minipage}[t]{0.06\columnwidth}\raggedright
0.4000\strut
\end{minipage} & \begin{minipage}[t]{0.06\columnwidth}\raggedright
3.241\strut
\end{minipage} & \begin{minipage}[t]{0.04\columnwidth}\raggedright
\strut
\end{minipage} & \begin{minipage}[t]{0.06\columnwidth}\raggedright
9.500\strut
\end{minipage} & \begin{minipage}[t]{0.11\columnwidth}\raggedright
1.43782506E+00\strut
\end{minipage} & \begin{minipage}[t]{0.09\columnwidth}\raggedright
8.91667519E-01\strut
\end{minipage}\tabularnewline
\bottomrule
\end{longtable}

\url{https://vega.github.io/vega-lite/examples/data/seattle-weather.csv}
:basic info {[}1 4{]}:

\begin{longtable}[]{@{}llll@{}}
\toprule
\begin{minipage}[b]{0.63\columnwidth}\raggedright
:name\strut
\end{minipage} & \begin{minipage}[b]{0.10\columnwidth}\raggedright
:grouped?\strut
\end{minipage} & \begin{minipage}[b]{0.06\columnwidth}\raggedright
:rows\strut
\end{minipage} & \begin{minipage}[b]{0.09\columnwidth}\raggedright
:columns\strut
\end{minipage}\tabularnewline
\midrule
\endhead
\begin{minipage}[t]{0.63\columnwidth}\raggedright
\url{https://vega.github.io/vega-lite/examples/data/seattle-weather.csv}\strut
\end{minipage} & \begin{minipage}[t]{0.10\columnwidth}\raggedright
false\strut
\end{minipage} & \begin{minipage}[t]{0.06\columnwidth}\raggedright
1461\strut
\end{minipage} & \begin{minipage}[t]{0.09\columnwidth}\raggedright
6\strut
\end{minipage}\tabularnewline
\bottomrule
\end{longtable}

\url{https://vega.github.io/vega-lite/examples/data/seattle-weather.csv}
:column info {[}6 4{]}:

\begin{longtable}[]{@{}llll@{}}
\toprule
:name & :size & :datatype & :categorical?\tabularnewline
\midrule
\endhead
date & 1461 & :packed-local-date &\tabularnewline
precipitation & 1461 & :float64 &\tabularnewline
temp\_max & 1461 & :float64 &\tabularnewline
temp\_min & 1461 & :float64 &\tabularnewline
wind & 1461 & :float64 &\tabularnewline
weather & 1461 & :string & true\tabularnewline
\bottomrule
\end{longtable}

\begin{center}\rule{0.5\linewidth}{0.5pt}\end{center}

Getting a dataset name

\begin{Shaded}
\begin{Highlighting}[]
\NormalTok{(api/dataset-name ds)}
\end{Highlighting}
\end{Shaded}

\begin{verbatim}
"https://vega.github.io/vega-lite/examples/data/seattle-weather.csv"
\end{verbatim}

\begin{center}\rule{0.5\linewidth}{0.5pt}\end{center}

Setting a dataset name (operation is immutable).

\begin{Shaded}
\begin{Highlighting}[]
\NormalTok{(}\KeywordTok{->>} \StringTok{"seattle-weather"}
\NormalTok{     (api/set-dataset-name ds)}
\NormalTok{     (api/dataset-name))}
\end{Highlighting}
\end{Shaded}

\begin{verbatim}
"seattle-weather"
\end{verbatim}

\hypertarget{columns-and-rows}{%
\paragraph{Columns and rows}\label{columns-and-rows}}

Get columns and rows as sequences. \texttt{column}, \texttt{columns} and
\texttt{rows} treat grouped dataset as regular one. See \texttt{Groups}
to read more about grouped datasets.

\begin{center}\rule{0.5\linewidth}{0.5pt}\end{center}

Select column.

\begin{Shaded}
\begin{Highlighting}[]
\NormalTok{(ds }\StringTok{"wind"}\NormalTok{)}
\NormalTok{(api/column ds }\StringTok{"date"}\NormalTok{)}
\end{Highlighting}
\end{Shaded}

\begin{verbatim}
#tech.ml.dataset.column<float64>[1461]
wind
[4.700, 4.500, 2.300, 4.700, 6.100, 2.200, 2.300, 2.000, 3.400, 3.400, 5.100, 1.900, 1.300, 5.300, 3.200, 5.000, 5.600, 5.000, 1.600, 2.300, ...]
#tech.ml.dataset.column<packed-local-date>[1461]
date
[2012-01-01, 2012-01-02, 2012-01-03, 2012-01-04, 2012-01-05, 2012-01-06, 2012-01-07, 2012-01-08, 2012-01-09, 2012-01-10, 2012-01-11, 2012-01-12, 2012-01-13, 2012-01-14, 2012-01-15, 2012-01-16, 2012-01-17, 2012-01-18, 2012-01-19, 2012-01-20, ...]
\end{verbatim}

\begin{center}\rule{0.5\linewidth}{0.5pt}\end{center}

Columns as sequence

\begin{Shaded}
\begin{Highlighting}[]
\NormalTok{(}\KeywordTok{take} \DecValTok{2}\NormalTok{ (api/columns ds))}
\end{Highlighting}
\end{Shaded}

\begin{verbatim}
(#tech.ml.dataset.column<packed-local-date>[1461]
date
[2012-01-01, 2012-01-02, 2012-01-03, 2012-01-04, 2012-01-05, 2012-01-06, 2012-01-07, 2012-01-08, 2012-01-09, 2012-01-10, 2012-01-11, 2012-01-12, 2012-01-13, 2012-01-14, 2012-01-15, 2012-01-16, 2012-01-17, 2012-01-18, 2012-01-19, 2012-01-20, ...] #tech.ml.dataset.column<float64>[1461]
precipitation
[0.000, 10.90, 0.8000, 20.30, 1.300, 2.500, 0.000, 0.000, 4.300, 1.000, 0.000, 0.000, 0.000, 4.100, 5.300, 2.500, 8.100, 19.80, 15.20, 13.50, ...])
\end{verbatim}

\begin{center}\rule{0.5\linewidth}{0.5pt}\end{center}

Columns as map

\begin{Shaded}
\begin{Highlighting}[]
\NormalTok{(}\KeywordTok{keys}\NormalTok{ (api/columns ds }\AttributeTok{:as-map}\NormalTok{))}
\end{Highlighting}
\end{Shaded}

\begin{verbatim}
("date" "precipitation" "temp_max" "temp_min" "wind" "weather")
\end{verbatim}

\begin{center}\rule{0.5\linewidth}{0.5pt}\end{center}

Rows as sequence of sequences

\begin{Shaded}
\begin{Highlighting}[]
\NormalTok{(}\KeywordTok{take} \DecValTok{2}\NormalTok{ (api/rows ds))}
\end{Highlighting}
\end{Shaded}

\begin{verbatim}
([#object[java.time.LocalDate 0x1f51cd74 "2012-01-01"] 0.0 12.8 5.0 4.7 "drizzle"] [#object[java.time.LocalDate 0x6fb00e8b "2012-01-02"] 10.9 10.6 2.8 4.5 "rain"])
\end{verbatim}

\begin{center}\rule{0.5\linewidth}{0.5pt}\end{center}

Rows as sequence of maps

\begin{Shaded}
\begin{Highlighting}[]
\NormalTok{(clojure.pprint/pprint (}\KeywordTok{take} \DecValTok{2}\NormalTok{ (api/rows ds }\AttributeTok{:as-maps}\NormalTok{)))}
\end{Highlighting}
\end{Shaded}

\begin{verbatim}
({"date" #object[java.time.LocalDate 0x30e61c13 "2012-01-01"],
  "precipitation" 0.0,
  "temp_min" 5.0,
  "weather" "drizzle",
  "temp_max" 12.8,
  "wind" 4.7}
 {"date" #object[java.time.LocalDate 0x689fe36d "2012-01-02"],
  "precipitation" 10.9,
  "temp_min" 2.8,
  "weather" "rain",
  "temp_max" 10.6,
  "wind" 4.5})
\end{verbatim}

\hypertarget{printing}{%
\paragraph{Printing}\label{printing}}

Dataset is printed using \texttt{dataset-\textgreater{}str} or
\texttt{print-dataset} functions. Options are the same as in
\texttt{tech.ml.dataset/dataset-data-\textgreater{}str}. Most important
is \texttt{:print-line-policy} which can be one of the:
\texttt{:single}, \texttt{:repl} or \texttt{:markdown}.

\begin{Shaded}
\begin{Highlighting}[]
\NormalTok{(api/print-dataset (api/group-by DS }\AttributeTok{:V1}\NormalTok{) \{}\AttributeTok{:print-line-policy} \AttributeTok{:markdown}\NormalTok{\})}
\end{Highlighting}
\end{Shaded}

\begin{verbatim}
_unnamed [2 3]:

| :name | :group-id |                                                                                                                                                                                                                                                             :data |
|-------|-----------|-------------------------------------------------------------------------------------------------------------------------------------------------------------------------------------------------------------------------------------------------------------------|
|     1 |         0 | Group: 1 [5 4]:<br><br>\| :V1 \| :V2 \| :V3 \| :V4 \|<br>\|-----\|-----\|-----\|-----\|<br>\|   1 \|   1 \| 0.5 \|   A \|<br>\|   1 \|   3 \| 1.5 \|   C \|<br>\|   1 \|   5 \| 1.0 \|   B \|<br>\|   1 \|   7 \| 0.5 \|   A \|<br>\|   1 \|   9 \| 1.5 \|   C \| |
|     2 |         1 |                                   Group: 2 [4 4]:<br><br>\| :V1 \| :V2 \| :V3 \| :V4 \|<br>\|-----\|-----\|-----\|-----\|<br>\|   2 \|   2 \| 1.0 \|   B \|<br>\|   2 \|   4 \| 0.5 \|   A \|<br>\|   2 \|   6 \| 1.5 \|   C \|<br>\|   2 \|   8 \| 1.0 \|   B \| |
\end{verbatim}

\begin{Shaded}
\begin{Highlighting}[]
\NormalTok{(api/print-dataset (api/group-by DS }\AttributeTok{:V1}\NormalTok{) \{}\AttributeTok{:print-line-policy} \AttributeTok{:repl}\NormalTok{\})}
\end{Highlighting}
\end{Shaded}

\begin{verbatim}
_unnamed [2 3]:

| :name | :group-id |                          :data |
|-------|-----------|--------------------------------|
|     1 |         0 | Group: 1 [5 4]:                |
|       |           |                                |
|       |           | \| :V1 \| :V2 \| :V3 \| :V4 \| |
|       |           | \|-----\|-----\|-----\|-----\| |
|       |           | \|   1 \|   1 \| 0.5 \|   A \| |
|       |           | \|   1 \|   3 \| 1.5 \|   C \| |
|       |           | \|   1 \|   5 \| 1.0 \|   B \| |
|       |           | \|   1 \|   7 \| 0.5 \|   A \| |
|       |           | \|   1 \|   9 \| 1.5 \|   C \| |
|     2 |         1 | Group: 2 [4 4]:                |
|       |           |                                |
|       |           | \| :V1 \| :V2 \| :V3 \| :V4 \| |
|       |           | \|-----\|-----\|-----\|-----\| |
|       |           | \|   2 \|   2 \| 1.0 \|   B \| |
|       |           | \|   2 \|   4 \| 0.5 \|   A \| |
|       |           | \|   2 \|   6 \| 1.5 \|   C \| |
|       |           | \|   2 \|   8 \| 1.0 \|   B \| |
\end{verbatim}

\begin{Shaded}
\begin{Highlighting}[]
\NormalTok{(api/print-dataset (api/group-by DS }\AttributeTok{:V1}\NormalTok{) \{}\AttributeTok{:print-line-policy} \AttributeTok{:single}\NormalTok{\})}
\end{Highlighting}
\end{Shaded}

\begin{verbatim}
_unnamed [2 3]:

| :name | :group-id |           :data |
|-------|-----------|-----------------|
|     1 |         0 | Group: 1 [5 4]: |
|     2 |         1 | Group: 2 [4 4]: |
\end{verbatim}

\hypertarget{group-by}{%
\subsubsection{Group-by}\label{group-by}}

Grouping by is an operation which splits dataset into subdatasets and
pack it into new special type of\ldots{} dataset. I distinguish two
types of dataset: regular dataset and grouped dataset. The latter is the
result of grouping.

Grouped dataset is annotated in by \texttt{:grouped?} meta tag and
consist following columns:

\begin{itemize}
\tightlist
\item
  \texttt{:name} - group name or structure
\item
  \texttt{:group-id} - integer assigned to the group
\item
  \texttt{:data} - groups as datasets
\end{itemize}

Almost all functions recognize type of the dataset (grouped or not) and
operate accordingly.

You can't apply reshaping or join/concat functions on grouped datasets.

\hypertarget{grouping}{%
\paragraph{Grouping}\label{grouping}}

Grouping is done by calling \texttt{group-by} function with arguments:

\begin{itemize}
\tightlist
\item
  \texttt{ds} - dataset
\item
  \texttt{grouping-selector} - what to use for grouping
\item
  options:

  \begin{itemize}
  \tightlist
  \item
    \texttt{:result-type} - what to return:

    \begin{itemize}
    \tightlist
    \item
      \texttt{:as-dataset} (default) - return grouped dataset
    \item
      \texttt{:as-indexes} - return rows ids (row number from original
      dataset)
    \item
      \texttt{:as-map} - return map with group names as keys and
      subdataset as values
    \item
      \texttt{:as-seq} - return sequens of subdatasets
    \end{itemize}
  \item
    \texttt{:select-keys} - list of the columns passed to a grouping
    selector function
  \end{itemize}
\end{itemize}

All subdatasets (groups) have set name as the group name, additionally
\texttt{group-id} is in meta.

Grouping can be done by:

\begin{itemize}
\tightlist
\item
  single column name
\item
  seq of column names
\item
  map of keys (group names) and row indexes
\item
  value returned by function taking row as map (limited to
  \texttt{:select-keys})
\end{itemize}

Note: currently dataset inside dataset is printed recursively so it
renders poorly from markdown. So I will use \texttt{:as-seq} result type
to show just group names and groups.

\begin{center}\rule{0.5\linewidth}{0.5pt}\end{center}

List of columns in grouped dataset

\begin{Shaded}
\begin{Highlighting}[]
\NormalTok{(}\KeywordTok{->}\NormalTok{ DS}
\NormalTok{    (api/group-by }\AttributeTok{:V1}\NormalTok{)}
\NormalTok{    (api/column-names))}
\end{Highlighting}
\end{Shaded}

\begin{verbatim}
(:V1 :V2 :V3 :V4)
\end{verbatim}

\begin{center}\rule{0.5\linewidth}{0.5pt}\end{center}

List of columns in grouped dataset treated as regular dataset

\begin{Shaded}
\begin{Highlighting}[]
\NormalTok{(}\KeywordTok{->}\NormalTok{ DS}
\NormalTok{    (api/group-by }\AttributeTok{:V1}\NormalTok{)}
\NormalTok{    (api/as-regular-dataset)}
\NormalTok{    (api/column-names))}
\end{Highlighting}
\end{Shaded}

\begin{verbatim}
(:name :group-id :data)
\end{verbatim}

\begin{center}\rule{0.5\linewidth}{0.5pt}\end{center}

Content of the grouped dataset

\begin{Shaded}
\begin{Highlighting}[]
\NormalTok{(api/columns (api/group-by DS }\AttributeTok{:V1}\NormalTok{) }\AttributeTok{:as-map}\NormalTok{)}
\end{Highlighting}
\end{Shaded}

\begin{verbatim}
{:name #tech.ml.dataset.column<int64>[2]
:name
[1, 2, ], :group-id #tech.ml.dataset.column<int64>[2]
:group-id
[0, 1, ], :data #tech.ml.dataset.column<dataset>[2]
:data
[Group: 1 [5 4]:

| :V1 | :V2 | :V3 | :V4 |
|-----|-----|-----|-----|
|   1 |   1 | 0.5 |   A |
|   1 |   3 | 1.5 |   C |
|   1 |   5 | 1.0 |   B |
|   1 |   7 | 0.5 |   A |
|   1 |   9 | 1.5 |   C |
, Group: 2 [4 4]:

| :V1 | :V2 | :V3 | :V4 |
|-----|-----|-----|-----|
|   2 |   2 | 1.0 |   B |
|   2 |   4 | 0.5 |   A |
|   2 |   6 | 1.5 |   C |
|   2 |   8 | 1.0 |   B |
, ]}
\end{verbatim}

\begin{center}\rule{0.5\linewidth}{0.5pt}\end{center}

Grouped dataset as map

\begin{Shaded}
\begin{Highlighting}[]
\NormalTok{(}\KeywordTok{keys}\NormalTok{ (api/group-by DS }\AttributeTok{:V1}\NormalTok{ \{}\AttributeTok{:result-type} \AttributeTok{:as-map}\NormalTok{\}))}
\end{Highlighting}
\end{Shaded}

\begin{verbatim}
(1 2)
\end{verbatim}

\begin{Shaded}
\begin{Highlighting}[]
\NormalTok{(}\KeywordTok{vals}\NormalTok{ (api/group-by DS }\AttributeTok{:V1}\NormalTok{ \{}\AttributeTok{:result-type} \AttributeTok{:as-map}\NormalTok{\}))}
\end{Highlighting}
\end{Shaded}

(Group: 1 {[}5 4{]}:

\begin{longtable}[]{@{}llll@{}}
\toprule
:V1 & :V2 & :V3 & :V4\tabularnewline
\midrule
\endhead
1 & 1 & 0.5 & A\tabularnewline
1 & 3 & 1.5 & C\tabularnewline
1 & 5 & 1.0 & B\tabularnewline
1 & 7 & 0.5 & A\tabularnewline
1 & 9 & 1.5 & C\tabularnewline
\bottomrule
\end{longtable}

Group: 2 {[}4 4{]}:

\begin{longtable}[]{@{}llll@{}}
\toprule
:V1 & :V2 & :V3 & :V4\tabularnewline
\midrule
\endhead
2 & 2 & 1.0 & B\tabularnewline
2 & 4 & 0.5 & A\tabularnewline
2 & 6 & 1.5 & C\tabularnewline
2 & 8 & 1.0 & B\tabularnewline
\bottomrule
\end{longtable}

)

\begin{center}\rule{0.5\linewidth}{0.5pt}\end{center}

Group dataset as map of indexes (row ids)

\begin{Shaded}
\begin{Highlighting}[]
\NormalTok{(api/group-by DS }\AttributeTok{:V1}\NormalTok{ \{}\AttributeTok{:result-type} \AttributeTok{:as-indexes}\NormalTok{\})}
\end{Highlighting}
\end{Shaded}

\begin{verbatim}
{1 [0 2 4 6 8], 2 [1 3 5 7]}
\end{verbatim}

\begin{center}\rule{0.5\linewidth}{0.5pt}\end{center}

Grouped datasets are printed as follows by default.

\begin{Shaded}
\begin{Highlighting}[]
\NormalTok{(api/group-by DS }\AttributeTok{:V1}\NormalTok{)}
\end{Highlighting}
\end{Shaded}

\_unnamed {[}2 3{]}:

\begin{longtable}[]{@{}lll@{}}
\toprule
:name & :group-id & :data\tabularnewline
\midrule
\endhead
1 & 0 & Group: 1 {[}5 4{]}:\tabularnewline
2 & 1 & Group: 2 {[}4 4{]}:\tabularnewline
\bottomrule
\end{longtable}

\begin{center}\rule{0.5\linewidth}{0.5pt}\end{center}

To get groups as sequence or a map can be done from grouped dataset
using \texttt{groups-\textgreater{}seq} and
\texttt{groups-\textgreater{}map} functions.

Groups as seq can be obtained by just accessing \texttt{:data} column.

I will use temporary dataset here.

\begin{Shaded}
\begin{Highlighting}[]
\NormalTok{(}\KeywordTok{let}\NormalTok{ [ds (}\KeywordTok{->}\NormalTok{ \{}\StringTok{"a"}\NormalTok{ [}\DecValTok{1} \DecValTok{1} \DecValTok{2} \DecValTok{2}\NormalTok{]}
              \StringTok{"b"}\NormalTok{ [}\StringTok{"a"} \StringTok{"b"} \StringTok{"c"} \StringTok{"d"}\NormalTok{]\}}
\NormalTok{             (api/dataset)}
\NormalTok{             (api/group-by }\StringTok{"a"}\NormalTok{))]}
\NormalTok{  (}\KeywordTok{seq}\NormalTok{ (ds }\AttributeTok{:data}\NormalTok{))) }\CommentTok{;; seq is not necessary but Markdown treats `:data` as command here}
\end{Highlighting}
\end{Shaded}

(Group: 1 {[}2 2{]}:

\begin{longtable}[]{@{}ll@{}}
\toprule
a & b\tabularnewline
\midrule
\endhead
1 & a\tabularnewline
1 & b\tabularnewline
\bottomrule
\end{longtable}

Group: 2 {[}2 2{]}:

\begin{longtable}[]{@{}ll@{}}
\toprule
a & b\tabularnewline
\midrule
\endhead
2 & c\tabularnewline
2 & d\tabularnewline
\bottomrule
\end{longtable}

)

\begin{Shaded}
\begin{Highlighting}[]
\NormalTok{(}\KeywordTok{->}\NormalTok{ \{}\StringTok{"a"}\NormalTok{ [}\DecValTok{1} \DecValTok{1} \DecValTok{2} \DecValTok{2}\NormalTok{]}
     \StringTok{"b"}\NormalTok{ [}\StringTok{"a"} \StringTok{"b"} \StringTok{"c"} \StringTok{"d"}\NormalTok{]\}}
\NormalTok{    (api/dataset)}
\NormalTok{    (api/group-by }\StringTok{"a"}\NormalTok{)}
\NormalTok{    (api/groups->seq))}
\end{Highlighting}
\end{Shaded}

(Group: 1 {[}2 2{]}:

\begin{longtable}[]{@{}ll@{}}
\toprule
a & b\tabularnewline
\midrule
\endhead
1 & a\tabularnewline
1 & b\tabularnewline
\bottomrule
\end{longtable}

Group: 2 {[}2 2{]}:

\begin{longtable}[]{@{}ll@{}}
\toprule
a & b\tabularnewline
\midrule
\endhead
2 & c\tabularnewline
2 & d\tabularnewline
\bottomrule
\end{longtable}

)

\begin{center}\rule{0.5\linewidth}{0.5pt}\end{center}

Groups as map

\begin{Shaded}
\begin{Highlighting}[]
\NormalTok{(}\KeywordTok{->}\NormalTok{ \{}\StringTok{"a"}\NormalTok{ [}\DecValTok{1} \DecValTok{1} \DecValTok{2} \DecValTok{2}\NormalTok{]}
     \StringTok{"b"}\NormalTok{ [}\StringTok{"a"} \StringTok{"b"} \StringTok{"c"} \StringTok{"d"}\NormalTok{]\}}
\NormalTok{    (api/dataset)}
\NormalTok{    (api/group-by }\StringTok{"a"}\NormalTok{)}
\NormalTok{    (api/groups->map))}
\end{Highlighting}
\end{Shaded}

\{1 Group: 1 {[}2 2{]}:

\begin{longtable}[]{@{}ll@{}}
\toprule
a & b\tabularnewline
\midrule
\endhead
1 & a\tabularnewline
1 & b\tabularnewline
\bottomrule
\end{longtable}

, 2 Group: 2 {[}2 2{]}:

\begin{longtable}[]{@{}ll@{}}
\toprule
a & b\tabularnewline
\midrule
\endhead
2 & c\tabularnewline
2 & d\tabularnewline
\bottomrule
\end{longtable}

\}

\begin{center}\rule{0.5\linewidth}{0.5pt}\end{center}

Grouping by more than one column. You can see that group names are maps.
When ungrouping is done these maps are used to restore column names.

\begin{Shaded}
\begin{Highlighting}[]
\NormalTok{(api/group-by DS [}\AttributeTok{:V1} \AttributeTok{:V3}\NormalTok{] \{}\AttributeTok{:result-type} \AttributeTok{:as-seq}\NormalTok{\})}
\end{Highlighting}
\end{Shaded}

(Group: \{:V3 1.0, :V1 1\} {[}1 4{]}:

\begin{longtable}[]{@{}llll@{}}
\toprule
:V1 & :V2 & :V3 & :V4\tabularnewline
\midrule
\endhead
1 & 5 & 1.0 & B\tabularnewline
\bottomrule
\end{longtable}

Group: \{:V3 0.5, :V1 1\} {[}2 4{]}:

\begin{longtable}[]{@{}llll@{}}
\toprule
:V1 & :V2 & :V3 & :V4\tabularnewline
\midrule
\endhead
1 & 1 & 0.5 & A\tabularnewline
1 & 7 & 0.5 & A\tabularnewline
\bottomrule
\end{longtable}

Group: \{:V3 0.5, :V1 2\} {[}1 4{]}:

\begin{longtable}[]{@{}llll@{}}
\toprule
:V1 & :V2 & :V3 & :V4\tabularnewline
\midrule
\endhead
2 & 4 & 0.5 & A\tabularnewline
\bottomrule
\end{longtable}

Group: \{:V3 1.0, :V1 2\} {[}2 4{]}:

\begin{longtable}[]{@{}llll@{}}
\toprule
:V1 & :V2 & :V3 & :V4\tabularnewline
\midrule
\endhead
2 & 2 & 1.0 & B\tabularnewline
2 & 8 & 1.0 & B\tabularnewline
\bottomrule
\end{longtable}

Group: \{:V3 1.5, :V1 1\} {[}2 4{]}:

\begin{longtable}[]{@{}llll@{}}
\toprule
:V1 & :V2 & :V3 & :V4\tabularnewline
\midrule
\endhead
1 & 3 & 1.5 & C\tabularnewline
1 & 9 & 1.5 & C\tabularnewline
\bottomrule
\end{longtable}

Group: \{:V3 1.5, :V1 2\} {[}1 4{]}:

\begin{longtable}[]{@{}llll@{}}
\toprule
:V1 & :V2 & :V3 & :V4\tabularnewline
\midrule
\endhead
2 & 6 & 1.5 & C\tabularnewline
\bottomrule
\end{longtable}

)

\begin{center}\rule{0.5\linewidth}{0.5pt}\end{center}

Grouping can be done by providing just row indexes. This way you can
assign the same row to more than one group.

\begin{Shaded}
\begin{Highlighting}[]
\NormalTok{(api/group-by DS \{}\StringTok{"group-a"}\NormalTok{ [}\DecValTok{1} \DecValTok{2} \DecValTok{1} \DecValTok{2}\NormalTok{]}
                  \StringTok{"group-b"}\NormalTok{ [}\DecValTok{5} \DecValTok{5} \DecValTok{5} \DecValTok{1}\NormalTok{]\} \{}\AttributeTok{:result-type} \AttributeTok{:as-seq}\NormalTok{\})}
\end{Highlighting}
\end{Shaded}

(Group: group-a {[}4 4{]}:

\begin{longtable}[]{@{}llll@{}}
\toprule
:V1 & :V2 & :V3 & :V4\tabularnewline
\midrule
\endhead
2 & 2 & 1.0 & B\tabularnewline
1 & 3 & 1.5 & C\tabularnewline
2 & 2 & 1.0 & B\tabularnewline
1 & 3 & 1.5 & C\tabularnewline
\bottomrule
\end{longtable}

Group: group-b {[}4 4{]}:

\begin{longtable}[]{@{}llll@{}}
\toprule
:V1 & :V2 & :V3 & :V4\tabularnewline
\midrule
\endhead
2 & 6 & 1.5 & C\tabularnewline
2 & 6 & 1.5 & C\tabularnewline
2 & 6 & 1.5 & C\tabularnewline
2 & 2 & 1.0 & B\tabularnewline
\bottomrule
\end{longtable}

)

\begin{center}\rule{0.5\linewidth}{0.5pt}\end{center}

You can group by a result of grouping function which gets row as map and
should return group name. When map is used as a group name, ungrouping
restore original column names.

\begin{Shaded}
\begin{Highlighting}[]
\NormalTok{(api/group-by DS (}\KeywordTok{fn}\NormalTok{ [row] (}\KeywordTok{*}\NormalTok{ (}\AttributeTok{:V1}\NormalTok{ row)}
\NormalTok{                             (}\AttributeTok{:V3}\NormalTok{ row))) \{}\AttributeTok{:result-type} \AttributeTok{:as-seq}\NormalTok{\})}
\end{Highlighting}
\end{Shaded}

(Group: 1.0 {[}2 4{]}:

\begin{longtable}[]{@{}llll@{}}
\toprule
:V1 & :V2 & :V3 & :V4\tabularnewline
\midrule
\endhead
2 & 4 & 0.5 & A\tabularnewline
1 & 5 & 1.0 & B\tabularnewline
\bottomrule
\end{longtable}

Group: 2.0 {[}2 4{]}:

\begin{longtable}[]{@{}llll@{}}
\toprule
:V1 & :V2 & :V3 & :V4\tabularnewline
\midrule
\endhead
2 & 2 & 1.0 & B\tabularnewline
2 & 8 & 1.0 & B\tabularnewline
\bottomrule
\end{longtable}

Group: 0.5 {[}2 4{]}:

\begin{longtable}[]{@{}llll@{}}
\toprule
:V1 & :V2 & :V3 & :V4\tabularnewline
\midrule
\endhead
1 & 1 & 0.5 & A\tabularnewline
1 & 7 & 0.5 & A\tabularnewline
\bottomrule
\end{longtable}

Group: 3.0 {[}1 4{]}:

\begin{longtable}[]{@{}llll@{}}
\toprule
:V1 & :V2 & :V3 & :V4\tabularnewline
\midrule
\endhead
2 & 6 & 1.5 & C\tabularnewline
\bottomrule
\end{longtable}

Group: 1.5 {[}2 4{]}:

\begin{longtable}[]{@{}llll@{}}
\toprule
:V1 & :V2 & :V3 & :V4\tabularnewline
\midrule
\endhead
1 & 3 & 1.5 & C\tabularnewline
1 & 9 & 1.5 & C\tabularnewline
\bottomrule
\end{longtable}

)

\begin{center}\rule{0.5\linewidth}{0.5pt}\end{center}

You can use any predicate on column to split dataset into two groups.

\begin{Shaded}
\begin{Highlighting}[]
\NormalTok{(api/group-by DS (}\KeywordTok{comp}\NormalTok{ #(}\KeywordTok{<} \VariableTok{%} \FloatTok{1.0}\NormalTok{) }\AttributeTok{:V3}\NormalTok{) \{}\AttributeTok{:result-type} \AttributeTok{:as-seq}\NormalTok{\})}
\end{Highlighting}
\end{Shaded}

(Group: false {[}6 4{]}:

\begin{longtable}[]{@{}llll@{}}
\toprule
:V1 & :V2 & :V3 & :V4\tabularnewline
\midrule
\endhead
2 & 2 & 1.0 & B\tabularnewline
1 & 3 & 1.5 & C\tabularnewline
1 & 5 & 1.0 & B\tabularnewline
2 & 6 & 1.5 & C\tabularnewline
2 & 8 & 1.0 & B\tabularnewline
1 & 9 & 1.5 & C\tabularnewline
\bottomrule
\end{longtable}

Group: true {[}3 4{]}:

\begin{longtable}[]{@{}llll@{}}
\toprule
:V1 & :V2 & :V3 & :V4\tabularnewline
\midrule
\endhead
1 & 1 & 0.5 & A\tabularnewline
2 & 4 & 0.5 & A\tabularnewline
1 & 7 & 0.5 & A\tabularnewline
\bottomrule
\end{longtable}

)

\begin{center}\rule{0.5\linewidth}{0.5pt}\end{center}

\texttt{juxt} is also helpful

\begin{Shaded}
\begin{Highlighting}[]
\NormalTok{(api/group-by DS (}\KeywordTok{juxt} \AttributeTok{:V1} \AttributeTok{:V3}\NormalTok{) \{}\AttributeTok{:result-type} \AttributeTok{:as-seq}\NormalTok{\})}
\end{Highlighting}
\end{Shaded}

(Group: {[}1 1.0{]} {[}1 4{]}:

\begin{longtable}[]{@{}llll@{}}
\toprule
:V1 & :V2 & :V3 & :V4\tabularnewline
\midrule
\endhead
1 & 5 & 1.0 & B\tabularnewline
\bottomrule
\end{longtable}

Group: {[}1 0.5{]} {[}2 4{]}:

\begin{longtable}[]{@{}llll@{}}
\toprule
:V1 & :V2 & :V3 & :V4\tabularnewline
\midrule
\endhead
1 & 1 & 0.5 & A\tabularnewline
1 & 7 & 0.5 & A\tabularnewline
\bottomrule
\end{longtable}

Group: {[}2 1.5{]} {[}1 4{]}:

\begin{longtable}[]{@{}llll@{}}
\toprule
:V1 & :V2 & :V3 & :V4\tabularnewline
\midrule
\endhead
2 & 6 & 1.5 & C\tabularnewline
\bottomrule
\end{longtable}

Group: {[}1 1.5{]} {[}2 4{]}:

\begin{longtable}[]{@{}llll@{}}
\toprule
:V1 & :V2 & :V3 & :V4\tabularnewline
\midrule
\endhead
1 & 3 & 1.5 & C\tabularnewline
1 & 9 & 1.5 & C\tabularnewline
\bottomrule
\end{longtable}

Group: {[}2 0.5{]} {[}1 4{]}:

\begin{longtable}[]{@{}llll@{}}
\toprule
:V1 & :V2 & :V3 & :V4\tabularnewline
\midrule
\endhead
2 & 4 & 0.5 & A\tabularnewline
\bottomrule
\end{longtable}

Group: {[}2 1.0{]} {[}2 4{]}:

\begin{longtable}[]{@{}llll@{}}
\toprule
:V1 & :V2 & :V3 & :V4\tabularnewline
\midrule
\endhead
2 & 2 & 1.0 & B\tabularnewline
2 & 8 & 1.0 & B\tabularnewline
\bottomrule
\end{longtable}

)

\begin{center}\rule{0.5\linewidth}{0.5pt}\end{center}

\texttt{tech.ml.dataset} provides an option to limit columns which are
passed to grouping functions. It's done for performance purposes.

\begin{Shaded}
\begin{Highlighting}[]
\NormalTok{(api/group-by DS }\KeywordTok{identity}\NormalTok{ \{}\AttributeTok{:result-type} \AttributeTok{:as-seq}
                           \AttributeTok{:select-keys}\NormalTok{ [}\AttributeTok{:V1}\NormalTok{]\})}
\end{Highlighting}
\end{Shaded}

(Group: \{:V1 1\} {[}5 4{]}:

\begin{longtable}[]{@{}llll@{}}
\toprule
:V1 & :V2 & :V3 & :V4\tabularnewline
\midrule
\endhead
1 & 1 & 0.5 & A\tabularnewline
1 & 3 & 1.5 & C\tabularnewline
1 & 5 & 1.0 & B\tabularnewline
1 & 7 & 0.5 & A\tabularnewline
1 & 9 & 1.5 & C\tabularnewline
\bottomrule
\end{longtable}

Group: \{:V1 2\} {[}4 4{]}:

\begin{longtable}[]{@{}llll@{}}
\toprule
:V1 & :V2 & :V3 & :V4\tabularnewline
\midrule
\endhead
2 & 2 & 1.0 & B\tabularnewline
2 & 4 & 0.5 & A\tabularnewline
2 & 6 & 1.5 & C\tabularnewline
2 & 8 & 1.0 & B\tabularnewline
\bottomrule
\end{longtable}

)

\hypertarget{ungrouping}{%
\paragraph{Ungrouping}\label{ungrouping}}

Ungrouping simply concats all the groups into the dataset. Following
options are possible

\begin{itemize}
\tightlist
\item
  \texttt{:order?} - order groups according to the group name ascending
  order. Default: \texttt{false}
\item
  \texttt{:add-group-as-column} - should group name become a column? If
  yes column is created with provided name (or \texttt{:\$group-name} if
  argument is \texttt{true}). Default: \texttt{nil}.
\item
  \texttt{:add-group-id-as-column} - should group id become a column? If
  yes column is created with provided name (or \texttt{:\$group-id} if
  argument is \texttt{true}). Default: \texttt{nil}.
\item
  \texttt{:dataset-name} - to name resulting dataset. Default:
  \texttt{nil} (\_unnamed)
\end{itemize}

If group name is a map, it will be splitted into separate columns. Be
sure that groups (subdatasets) doesn't contain the same columns already.

If group name is a vector, it will be splitted into separate columns. If
you want to name them, set vector of target column names as
\texttt{:add-group-as-column} argument.

After ungrouping, order of the rows is kept within the groups but groups
are ordered according to the internal storage.

\begin{center}\rule{0.5\linewidth}{0.5pt}\end{center}

Grouping and ungrouping.

\begin{Shaded}
\begin{Highlighting}[]
\NormalTok{(}\KeywordTok{->}\NormalTok{ DS}
\NormalTok{    (api/group-by }\AttributeTok{:V3}\NormalTok{)}
\NormalTok{    (api/ungroup))}
\end{Highlighting}
\end{Shaded}

\_unnamed {[}9 4{]}:

\begin{longtable}[]{@{}llll@{}}
\toprule
:V1 & :V2 & :V3 & :V4\tabularnewline
\midrule
\endhead
2 & 2 & 1.0 & B\tabularnewline
1 & 5 & 1.0 & B\tabularnewline
2 & 8 & 1.0 & B\tabularnewline
1 & 1 & 0.5 & A\tabularnewline
2 & 4 & 0.5 & A\tabularnewline
1 & 7 & 0.5 & A\tabularnewline
1 & 3 & 1.5 & C\tabularnewline
2 & 6 & 1.5 & C\tabularnewline
1 & 9 & 1.5 & C\tabularnewline
\bottomrule
\end{longtable}

\begin{center}\rule{0.5\linewidth}{0.5pt}\end{center}

Groups sorted by group name and named.

\begin{Shaded}
\begin{Highlighting}[]
\NormalTok{(}\KeywordTok{->}\NormalTok{ DS}
\NormalTok{    (api/group-by }\AttributeTok{:V3}\NormalTok{)}
\NormalTok{    (api/ungroup \{}\AttributeTok{:order}\NormalTok{? }\VariableTok{true}
                  \AttributeTok{:dataset-name} \StringTok{"Ordered by V3"}\NormalTok{\}))}
\end{Highlighting}
\end{Shaded}

Ordered by V3 {[}9 4{]}:

\begin{longtable}[]{@{}llll@{}}
\toprule
:V1 & :V2 & :V3 & :V4\tabularnewline
\midrule
\endhead
1 & 1 & 0.5 & A\tabularnewline
2 & 4 & 0.5 & A\tabularnewline
1 & 7 & 0.5 & A\tabularnewline
2 & 2 & 1.0 & B\tabularnewline
1 & 5 & 1.0 & B\tabularnewline
2 & 8 & 1.0 & B\tabularnewline
1 & 3 & 1.5 & C\tabularnewline
2 & 6 & 1.5 & C\tabularnewline
1 & 9 & 1.5 & C\tabularnewline
\bottomrule
\end{longtable}

\begin{center}\rule{0.5\linewidth}{0.5pt}\end{center}

Groups sorted descending by group name and named.

\begin{Shaded}
\begin{Highlighting}[]
\NormalTok{(}\KeywordTok{->}\NormalTok{ DS}
\NormalTok{    (api/group-by }\AttributeTok{:V3}\NormalTok{)}
\NormalTok{    (api/ungroup \{}\AttributeTok{:order}\NormalTok{? }\AttributeTok{:desc}
                  \AttributeTok{:dataset-name} \StringTok{"Ordered by V3 descending"}\NormalTok{\}))}
\end{Highlighting}
\end{Shaded}

Ordered by V3 descending {[}9 4{]}:

\begin{longtable}[]{@{}llll@{}}
\toprule
:V1 & :V2 & :V3 & :V4\tabularnewline
\midrule
\endhead
1 & 3 & 1.5 & C\tabularnewline
2 & 6 & 1.5 & C\tabularnewline
1 & 9 & 1.5 & C\tabularnewline
2 & 2 & 1.0 & B\tabularnewline
1 & 5 & 1.0 & B\tabularnewline
2 & 8 & 1.0 & B\tabularnewline
1 & 1 & 0.5 & A\tabularnewline
2 & 4 & 0.5 & A\tabularnewline
1 & 7 & 0.5 & A\tabularnewline
\bottomrule
\end{longtable}

\begin{center}\rule{0.5\linewidth}{0.5pt}\end{center}

Let's add group name and id as additional columns

\begin{Shaded}
\begin{Highlighting}[]
\NormalTok{(}\KeywordTok{->}\NormalTok{ DS}
\NormalTok{    (api/group-by (}\KeywordTok{comp}\NormalTok{ #(}\KeywordTok{<} \VariableTok{%} \DecValTok{4}\NormalTok{) }\AttributeTok{:V2}\NormalTok{))}
\NormalTok{    (api/ungroup \{}\AttributeTok{:add-group-as-column} \VariableTok{true}
                  \AttributeTok{:add-group-id-as-column} \VariableTok{true}\NormalTok{\}))}
\end{Highlighting}
\end{Shaded}

\_unnamed {[}9 6{]}:

\begin{longtable}[]{@{}llllll@{}}
\toprule
:\(group-name | :\)group-id & :V1 & :V2 & :V3 & :V4 &\tabularnewline
\midrule
\endhead
false & 0 & 2 & 4 & 0.5 & A\tabularnewline
false & 0 & 1 & 5 & 1.0 & B\tabularnewline
false & 0 & 2 & 6 & 1.5 & C\tabularnewline
false & 0 & 1 & 7 & 0.5 & A\tabularnewline
false & 0 & 2 & 8 & 1.0 & B\tabularnewline
false & 0 & 1 & 9 & 1.5 & C\tabularnewline
true & 1 & 1 & 1 & 0.5 & A\tabularnewline
true & 1 & 2 & 2 & 1.0 & B\tabularnewline
true & 1 & 1 & 3 & 1.5 & C\tabularnewline
\bottomrule
\end{longtable}

\begin{center}\rule{0.5\linewidth}{0.5pt}\end{center}

Let's assign different column names

\begin{Shaded}
\begin{Highlighting}[]
\NormalTok{(}\KeywordTok{->}\NormalTok{ DS}
\NormalTok{    (api/group-by (}\KeywordTok{comp}\NormalTok{ #(}\KeywordTok{<} \VariableTok{%} \DecValTok{4}\NormalTok{) }\AttributeTok{:V2}\NormalTok{))}
\NormalTok{    (api/ungroup \{}\AttributeTok{:add-group-as-column} \StringTok{"Is V2 less than 4?"}
                  \AttributeTok{:add-group-id-as-column} \StringTok{"group id"}\NormalTok{\}))}
\end{Highlighting}
\end{Shaded}

\_unnamed {[}9 6{]}:

\begin{longtable}[]{@{}llllll@{}}
\toprule
Is V2 less than 4? & group id & :V1 & :V2 & :V3 & :V4\tabularnewline
\midrule
\endhead
false & 0 & 2 & 4 & 0.5 & A\tabularnewline
false & 0 & 1 & 5 & 1.0 & B\tabularnewline
false & 0 & 2 & 6 & 1.5 & C\tabularnewline
false & 0 & 1 & 7 & 0.5 & A\tabularnewline
false & 0 & 2 & 8 & 1.0 & B\tabularnewline
false & 0 & 1 & 9 & 1.5 & C\tabularnewline
true & 1 & 1 & 1 & 0.5 & A\tabularnewline
true & 1 & 2 & 2 & 1.0 & B\tabularnewline
true & 1 & 1 & 3 & 1.5 & C\tabularnewline
\bottomrule
\end{longtable}

\begin{center}\rule{0.5\linewidth}{0.5pt}\end{center}

If we group by map, we can automatically create new columns out of group
names.

\begin{Shaded}
\begin{Highlighting}[]
\NormalTok{(}\KeywordTok{->}\NormalTok{ DS}
\NormalTok{    (api/group-by (}\KeywordTok{fn}\NormalTok{ [row] \{}\StringTok{"V1 and V3 multiplied"}\NormalTok{ (}\KeywordTok{*}\NormalTok{ (}\AttributeTok{:V1}\NormalTok{ row)}
\NormalTok{                                                      (}\AttributeTok{:V3}\NormalTok{ row))}
                            \StringTok{"V4 as lowercase"}\NormalTok{ (clojure.string/lower-case (}\AttributeTok{:V4}\NormalTok{ row))\}))}
\NormalTok{    (api/ungroup \{}\AttributeTok{:add-group-as-column} \VariableTok{true}\NormalTok{\}))}
\end{Highlighting}
\end{Shaded}

\_unnamed {[}9 6{]}:

\begin{longtable}[]{@{}llllll@{}}
\toprule
V1 and V3 multiplied & V4 as lowercase & :V1 & :V2 & :V3 &
:V4\tabularnewline
\midrule
\endhead
1.0 & a & 2 & 4 & 0.5 & A\tabularnewline
0.5 & a & 1 & 1 & 0.5 & A\tabularnewline
0.5 & a & 1 & 7 & 0.5 & A\tabularnewline
1.0 & b & 1 & 5 & 1.0 & B\tabularnewline
2.0 & b & 2 & 2 & 1.0 & B\tabularnewline
2.0 & b & 2 & 8 & 1.0 & B\tabularnewline
3.0 & c & 2 & 6 & 1.5 & C\tabularnewline
1.5 & c & 1 & 3 & 1.5 & C\tabularnewline
1.5 & c & 1 & 9 & 1.5 & C\tabularnewline
\bottomrule
\end{longtable}

\begin{center}\rule{0.5\linewidth}{0.5pt}\end{center}

We can add group names without separation

\begin{Shaded}
\begin{Highlighting}[]
\NormalTok{(}\KeywordTok{->}\NormalTok{ DS}
\NormalTok{    (api/group-by (}\KeywordTok{fn}\NormalTok{ [row] \{}\StringTok{"V1 and V3 multiplied"}\NormalTok{ (}\KeywordTok{*}\NormalTok{ (}\AttributeTok{:V1}\NormalTok{ row)}
\NormalTok{                                                      (}\AttributeTok{:V3}\NormalTok{ row))}
                            \StringTok{"V4 as lowercase"}\NormalTok{ (clojure.string/lower-case (}\AttributeTok{:V4}\NormalTok{ row))\}))}
\NormalTok{    (api/ungroup \{}\AttributeTok{:add-group-as-column} \StringTok{"just map"}
                  \AttributeTok{:separate}\NormalTok{? }\VariableTok{false}\NormalTok{\}))}
\end{Highlighting}
\end{Shaded}

\_unnamed {[}9 5{]}:

\begin{longtable}[]{@{}lllll@{}}
\toprule
just map & :V1 & :V2 & :V3 & :V4\tabularnewline
\midrule
\endhead
\{``V1 and V3 multiplied'' 1.0, ``V4 as lowercase'' ``a''\} & 2 & 4 &
0.5 & A\tabularnewline
\{``V1 and V3 multiplied'' 0.5, ``V4 as lowercase'' ``a''\} & 1 & 1 &
0.5 & A\tabularnewline
\{``V1 and V3 multiplied'' 0.5, ``V4 as lowercase'' ``a''\} & 1 & 7 &
0.5 & A\tabularnewline
\{``V1 and V3 multiplied'' 1.0, ``V4 as lowercase'' ``b''\} & 1 & 5 &
1.0 & B\tabularnewline
\{``V1 and V3 multiplied'' 2.0, ``V4 as lowercase'' ``b''\} & 2 & 2 &
1.0 & B\tabularnewline
\{``V1 and V3 multiplied'' 2.0, ``V4 as lowercase'' ``b''\} & 2 & 8 &
1.0 & B\tabularnewline
\{``V1 and V3 multiplied'' 3.0, ``V4 as lowercase'' ``c''\} & 2 & 6 &
1.5 & C\tabularnewline
\{``V1 and V3 multiplied'' 1.5, ``V4 as lowercase'' ``c''\} & 1 & 3 &
1.5 & C\tabularnewline
\{``V1 and V3 multiplied'' 1.5, ``V4 as lowercase'' ``c''\} & 1 & 9 &
1.5 & C\tabularnewline
\bottomrule
\end{longtable}

\begin{center}\rule{0.5\linewidth}{0.5pt}\end{center}

The same applies to group names as sequences

\begin{Shaded}
\begin{Highlighting}[]
\NormalTok{(}\KeywordTok{->}\NormalTok{ DS}
\NormalTok{    (api/group-by (}\KeywordTok{juxt} \AttributeTok{:V1} \AttributeTok{:V3}\NormalTok{))}
\NormalTok{    (api/ungroup \{}\AttributeTok{:add-group-as-column} \StringTok{"abc"}\NormalTok{\}))}
\end{Highlighting}
\end{Shaded}

\_unnamed {[}9 6{]}:

\begin{longtable}[]{@{}llllll@{}}
\toprule
:abc-0 & :abc-1 & :V1 & :V2 & :V3 & :V4\tabularnewline
\midrule
\endhead
1 & 1.0 & 1 & 5 & 1.0 & B\tabularnewline
1 & 0.5 & 1 & 1 & 0.5 & A\tabularnewline
1 & 0.5 & 1 & 7 & 0.5 & A\tabularnewline
2 & 1.5 & 2 & 6 & 1.5 & C\tabularnewline
1 & 1.5 & 1 & 3 & 1.5 & C\tabularnewline
1 & 1.5 & 1 & 9 & 1.5 & C\tabularnewline
2 & 0.5 & 2 & 4 & 0.5 & A\tabularnewline
2 & 1.0 & 2 & 2 & 1.0 & B\tabularnewline
2 & 1.0 & 2 & 8 & 1.0 & B\tabularnewline
\bottomrule
\end{longtable}

\begin{center}\rule{0.5\linewidth}{0.5pt}\end{center}

Let's provide column names

\begin{Shaded}
\begin{Highlighting}[]
\NormalTok{(}\KeywordTok{->}\NormalTok{ DS}
\NormalTok{    (api/group-by (}\KeywordTok{juxt} \AttributeTok{:V1} \AttributeTok{:V3}\NormalTok{))}
\NormalTok{    (api/ungroup \{}\AttributeTok{:add-group-as-column}\NormalTok{ [}\StringTok{"v1"} \StringTok{"v3"}\NormalTok{]\}))}
\end{Highlighting}
\end{Shaded}

\_unnamed {[}9 6{]}:

\begin{longtable}[]{@{}llllll@{}}
\toprule
v1 & v3 & :V1 & :V2 & :V3 & :V4\tabularnewline
\midrule
\endhead
1 & 1.0 & 1 & 5 & 1.0 & B\tabularnewline
1 & 0.5 & 1 & 1 & 0.5 & A\tabularnewline
1 & 0.5 & 1 & 7 & 0.5 & A\tabularnewline
2 & 1.5 & 2 & 6 & 1.5 & C\tabularnewline
1 & 1.5 & 1 & 3 & 1.5 & C\tabularnewline
1 & 1.5 & 1 & 9 & 1.5 & C\tabularnewline
2 & 0.5 & 2 & 4 & 0.5 & A\tabularnewline
2 & 1.0 & 2 & 2 & 1.0 & B\tabularnewline
2 & 1.0 & 2 & 8 & 1.0 & B\tabularnewline
\bottomrule
\end{longtable}

\begin{center}\rule{0.5\linewidth}{0.5pt}\end{center}

Also we can supress separation

\begin{Shaded}
\begin{Highlighting}[]
\NormalTok{(}\KeywordTok{->}\NormalTok{ DS}
\NormalTok{    (api/group-by (}\KeywordTok{juxt} \AttributeTok{:V1} \AttributeTok{:V3}\NormalTok{))}
\NormalTok{    (api/ungroup \{}\AttributeTok{:separate}\NormalTok{? }\VariableTok{false}
                  \AttributeTok{:add-group-as-column} \VariableTok{true}\NormalTok{\}))}
\CommentTok{;; => _unnamed [9 5]:}
\end{Highlighting}
\end{Shaded}

\_unnamed {[}9 5{]}:

\begin{longtable}[]{@{}lllll@{}}
\toprule
:\$group-name & :V1 & :V2 & :V3 & :V4\tabularnewline
\midrule
\endhead
{[}1 1.0{]} & 1 & 5 & 1.0 & B\tabularnewline
{[}1 0.5{]} & 1 & 1 & 0.5 & A\tabularnewline
{[}1 0.5{]} & 1 & 7 & 0.5 & A\tabularnewline
{[}2 1.5{]} & 2 & 6 & 1.5 & C\tabularnewline
{[}1 1.5{]} & 1 & 3 & 1.5 & C\tabularnewline
{[}1 1.5{]} & 1 & 9 & 1.5 & C\tabularnewline
{[}2 0.5{]} & 2 & 4 & 0.5 & A\tabularnewline
{[}2 1.0{]} & 2 & 2 & 1.0 & B\tabularnewline
{[}2 1.0{]} & 2 & 8 & 1.0 & B\tabularnewline
\bottomrule
\end{longtable}

\hypertarget{other-functions}{%
\paragraph{Other functions}\label{other-functions}}

To check if dataset is grouped or not just use \texttt{grouped?}
function.

\begin{Shaded}
\begin{Highlighting}[]
\NormalTok{(api/grouped? DS)}
\end{Highlighting}
\end{Shaded}

\begin{verbatim}
nil
\end{verbatim}

\begin{Shaded}
\begin{Highlighting}[]
\NormalTok{(api/grouped? (api/group-by DS }\AttributeTok{:V1}\NormalTok{))}
\end{Highlighting}
\end{Shaded}

\begin{verbatim}
true
\end{verbatim}

\begin{center}\rule{0.5\linewidth}{0.5pt}\end{center}

If you want to remove grouping annotation (to make all the functions
work as with regular dataset) you can use \texttt{unmark-group} or
\texttt{as-regular-dataset} (alias) functions.

It can be important when you want to remove some groups (rows) from
grouped dataset using \texttt{drop-rows} or something like that.

\begin{Shaded}
\begin{Highlighting}[]
\NormalTok{(}\KeywordTok{->}\NormalTok{ DS}
\NormalTok{    (api/group-by }\AttributeTok{:V1}\NormalTok{)}
\NormalTok{    (api/as-regular-dataset)}
\NormalTok{    (api/grouped?))}
\end{Highlighting}
\end{Shaded}

\begin{verbatim}
nil
\end{verbatim}

\begin{center}\rule{0.5\linewidth}{0.5pt}\end{center}

This is considered internal.

If you want to implement your own mapping function on grouped dataset
you can call \texttt{process-group-data} and pass function operating on
datasets. Result should be a dataset to have ungrouping working.

\begin{Shaded}
\begin{Highlighting}[]
\NormalTok{(}\KeywordTok{->}\NormalTok{ DS}
\NormalTok{    (api/group-by }\AttributeTok{:V1}\NormalTok{)}
\NormalTok{    (api/process-group-data #(}\KeywordTok{str} \StringTok{"Shape: "}\NormalTok{ (}\KeywordTok{vector}\NormalTok{ (api/row-count }\VariableTok\NormalTok{))))}
\NormalTok{    (api/as-regular-dataset))}
\end{Highlighting}
\end{Shaded}

\_unnamed {[}2 3{]}:

\begin{longtable}[]{@{}lll@{}}
\toprule
:name & :group-id & :data\tabularnewline
\midrule
\endhead
1 & 0 & Shape: {[}5 4{]}\tabularnewline
2 & 1 & Shape: {[}4 4{]}\tabularnewline
\bottomrule
\end{longtable}

\hypertarget{columns}{%
\subsubsection{Columns}\label{columns}}

Column is a special \texttt{tech.ml.dataset} structure based on
\texttt{tech.ml.datatype} library. For our purposes we cat treat columns
as typed and named sequence bound to particular dataset.

Type of the data is inferred from a sequence during column creation.

\hypertarget{names}{%
\paragraph{Names}\label{names}}

To select dataset columns or column names \texttt{columns-selector} is
used. \texttt{columns-selector} can be one of the following:

\begin{itemize}
\tightlist
\item
  \texttt{:all} keyword - selects all columns
\item
  column name - for single column
\item
  sequence of column names - for collection of columns
\item
  regex - to apply pattern on column names or datatype
\item
  filter predicate - to filter column names or datatype
\item
  \texttt{type} namespaced keyword for specific datatype or group of
  datatypes
\end{itemize}

Column name can be anything.

\texttt{column-names} function returns names according to
\texttt{columns-selector} and optional \texttt{meta-field}.
\texttt{meta-field} is one of the following:

\begin{itemize}
\tightlist
\item
  \texttt{:name} (default) - to operate on column names
\item
  \texttt{:datatype} - to operated on column types
\item
  \texttt{:all} - if you want to process all metadata
\end{itemize}

Datatype groups are:

\begin{itemize}
\tightlist
\item
  \texttt{:type/numerical} - any numerical type
\item
  \texttt{:type/float} - floating point number (\texttt{:float32} and
  \texttt{:float64})
\item
  \texttt{:type/integer} - any integer
\item
  \texttt{:type/datetime} - any datetime type
\end{itemize}

If qualified keyword starts with \texttt{:!type}, complement set is
used.

\begin{center}\rule{0.5\linewidth}{0.5pt}\end{center}

To select all column names you can use \texttt{column-names} function.

\begin{Shaded}
\begin{Highlighting}[]
\NormalTok{(api/column-names DS)}
\end{Highlighting}
\end{Shaded}

\begin{verbatim}
(:V1 :V2 :V3 :V4)
\end{verbatim}

or

\begin{Shaded}
\begin{Highlighting}[]
\NormalTok{(api/column-names DS }\AttributeTok{:all}\NormalTok{)}
\end{Highlighting}
\end{Shaded}

\begin{verbatim}
(:V1 :V2 :V3 :V4)
\end{verbatim}

In case you want to select column which has name \texttt{:all} (or is
sequence or map), put it into a vector. Below code returns empty
sequence since there is no such column in the dataset.

\begin{Shaded}
\begin{Highlighting}[]
\NormalTok{(api/column-names DS [}\AttributeTok{:all}\NormalTok{])}
\end{Highlighting}
\end{Shaded}

\begin{verbatim}
()
\end{verbatim}

\begin{center}\rule{0.5\linewidth}{0.5pt}\end{center}

Obviously selecting single name returns it's name if available

\begin{Shaded}
\begin{Highlighting}[]
\NormalTok{(api/column-names DS }\AttributeTok{:V1}\NormalTok{)}
\NormalTok{(api/column-names DS }\StringTok{"no such column"}\NormalTok{)}
\end{Highlighting}
\end{Shaded}

\begin{verbatim}
(:V1)
()
\end{verbatim}

\begin{center}\rule{0.5\linewidth}{0.5pt}\end{center}

Select sequence of column names.

\begin{Shaded}
\begin{Highlighting}[]
\NormalTok{(api/column-names DS [}\AttributeTok{:V1} \StringTok{"V2"} \AttributeTok{:V3} \AttributeTok{:V4} \AttributeTok{:V5}\NormalTok{])}
\end{Highlighting}
\end{Shaded}

\begin{verbatim}
(:V1 :V3 :V4)
\end{verbatim}

\begin{center}\rule{0.5\linewidth}{0.5pt}\end{center}

Select names based on regex, columns ends with \texttt{1} or \texttt{4}

\begin{Shaded}
\begin{Highlighting}[]
\NormalTok{(api/column-names DS }\SpecialStringTok{#".*[14]"}\NormalTok{)}
\end{Highlighting}
\end{Shaded}

\begin{verbatim}
(:V1 :V4)
\end{verbatim}

\begin{center}\rule{0.5\linewidth}{0.5pt}\end{center}

Select names based on regex operating on type of the column (to check
what are the column types, call \texttt{(api/info\ DS\ :columns)}. Here
we want to get integer columns only.

\begin{Shaded}
\begin{Highlighting}[]
\NormalTok{(api/column-names DS }\SpecialStringTok{#"^:int.*"} \AttributeTok{:datatype}\NormalTok{)}
\end{Highlighting}
\end{Shaded}

\begin{verbatim}
(:V1 :V2)
\end{verbatim}

or

\begin{Shaded}
\begin{Highlighting}[]
\NormalTok{(api/column-names DS }\AttributeTok{:type/integer}\NormalTok{)}
\end{Highlighting}
\end{Shaded}

\begin{verbatim}
(:V1 :V2)
\end{verbatim}

\begin{center}\rule{0.5\linewidth}{0.5pt}\end{center}

And finally we can use predicate to select names. Let's select double
precision columns.

\begin{Shaded}
\begin{Highlighting}[]
\NormalTok{(api/column-names DS #\{}\AttributeTok{:float64}\NormalTok{\} }\AttributeTok{:datatype}\NormalTok{)}
\end{Highlighting}
\end{Shaded}

\begin{verbatim}
(:V3)
\end{verbatim}

or

\begin{Shaded}
\begin{Highlighting}[]
\NormalTok{(api/column-names DS }\AttributeTok{:type/float64} \AttributeTok{:datatype}\NormalTok{)}
\end{Highlighting}
\end{Shaded}

\begin{verbatim}
(:V3)
\end{verbatim}

\begin{center}\rule{0.5\linewidth}{0.5pt}\end{center}

If you want to select all columns but given, use \texttt{complement}
function. Works only on a predicate.

\begin{Shaded}
\begin{Highlighting}[]
\NormalTok{(api/column-names DS (}\KeywordTok{complement}\NormalTok{ #\{}\AttributeTok{:V1}\NormalTok{\}))}
\NormalTok{(api/column-names DS (}\KeywordTok{complement}\NormalTok{ #\{}\AttributeTok{:float64}\NormalTok{\}) }\AttributeTok{:datatype}\NormalTok{)}
\NormalTok{(api/column-names DS :!type/float64)}
\end{Highlighting}
\end{Shaded}

\begin{verbatim}
(:V2 :V3 :V4)
(:V1 :V2 :V4)
(:V1 :V2 :V4)
\end{verbatim}

\begin{center}\rule{0.5\linewidth}{0.5pt}\end{center}

You can select column names based on all column metadata at once by
using \texttt{:all} metadata selector. Below we want to select column
names ending with \texttt{1} which have \texttt{long} datatype.

\begin{Shaded}
\begin{Highlighting}[]
\NormalTok{(api/column-names DS (}\KeywordTok{fn}\NormalTok{ [}\KeywordTok{meta}\NormalTok{]}
\NormalTok{                       (}\KeywordTok{and}\NormalTok{ (}\KeywordTok{=} \AttributeTok{:int64}\NormalTok{ (}\AttributeTok{:datatype} \KeywordTok{meta}\NormalTok{))}
\NormalTok{                            (clojure.string/ends-with? (}\AttributeTok{:name} \KeywordTok{meta}\NormalTok{) }\StringTok{"1"}\NormalTok{))) }\AttributeTok{:all}\NormalTok{)}
\end{Highlighting}
\end{Shaded}

\begin{verbatim}
(:V1)
\end{verbatim}

\hypertarget{select}\NormalTok{) }\AttributeTok{:datatype}\NormalTok{)}
\end{Highlighting}
\end{Shaded}

\_unnamed {[}9 1{]}:

\begin{longtable}[]{@{}l@{}}
\toprule
:V3\tabularnewline
\midrule
\endhead
0.5\tabularnewline
1.0\tabularnewline
1.5\tabularnewline
0.5\tabularnewline
1.0\tabularnewline
1.5\tabularnewline
0.5\tabularnewline
1.0\tabularnewline
1.5\tabularnewline
\bottomrule
\end{longtable}

\begin{center}\rule{0.5\linewidth}{0.5pt}\end{center}

Select all but \texttt{:V1} columns

\begin{Shaded}
\begin{Highlighting}[]
\NormalTok{(api/select-columns DS (}\KeywordTok{complement}\NormalTok{ #\{}\AttributeTok{:V1}\NormalTok{\}))}
\end{Highlighting}
\end{Shaded}

\_unnamed {[}9 3{]}:

\begin{longtable}[]{@{}lll@{}}
\toprule
:V2 & :V3 & :V4\tabularnewline
\midrule
\endhead
1 & 0.5 & A\tabularnewline
2 & 1.0 & B\tabularnewline
3 & 1.5 & C\tabularnewline
4 & 0.5 & A\tabularnewline
5 & 1.0 & B\tabularnewline
6 & 1.5 & C\tabularnewline
7 & 0.5 & A\tabularnewline
8 & 1.0 & B\tabularnewline
9 & 1.5 & C\tabularnewline
\bottomrule
\end{longtable}

\begin{center}\rule{0.5\linewidth}{0.5pt}\end{center}

If we have grouped data set, column selection is applied to every group
separately.

\begin{Shaded}
\begin{Highlighting}[]
\NormalTok{(}\KeywordTok{->}\NormalTok{ DS}
\NormalTok{    (api/group-by }\AttributeTok{:V1}\NormalTok{)}
\NormalTok{    (api/select-columns [}\AttributeTok{:V2} \AttributeTok{:V3}\NormalTok{])}
\NormalTok{    (api/groups->map))}
\end{Highlighting}
\end{Shaded}

\{1 Group: 1 {[}5 2{]}:

\begin{longtable}[]{@{}ll@{}}
\toprule
:V2 & :V3\tabularnewline
\midrule
\endhead
1 & 0.5\tabularnewline
3 & 1.5\tabularnewline
5 & 1.0\tabularnewline
7 & 0.5\tabularnewline
9 & 1.5\tabularnewline
\bottomrule
\end{longtable}

, 2 Group: 2 {[}4 2{]}:

\begin{longtable}[]{@{}ll@{}}
\toprule
:V2 & :V3\tabularnewline
\midrule
\endhead
2 & 1.0\tabularnewline
4 & 0.5\tabularnewline
6 & 1.5\tabularnewline
8 & 1.0\tabularnewline
\bottomrule
\end{longtable}

\}

\hypertarget{drop}\NormalTok{) }\AttributeTok{:datatype}\NormalTok{)}
\end{Highlighting}
\end{Shaded}

\_unnamed {[}9 3{]}:

\begin{longtable}[]{@{}lll@{}}
\toprule
:V1 & :V2 & :V4\tabularnewline
\midrule
\endhead
1 & 1 & A\tabularnewline
2 & 2 & B\tabularnewline
1 & 3 & C\tabularnewline
2 & 4 & A\tabularnewline
1 & 5 & B\tabularnewline
2 & 6 & C\tabularnewline
1 & 7 & A\tabularnewline
2 & 8 & B\tabularnewline
1 & 9 & C\tabularnewline
\bottomrule
\end{longtable}

\begin{center}\rule{0.5\linewidth}{0.5pt}\end{center}

Drop all columns but \texttt{:V1} and \texttt{:V2}

\begin{Shaded}
\begin{Highlighting}[]
\NormalTok{(api/drop-columns DS (}\KeywordTok{complement}\NormalTok{ #\{}\AttributeTok{:V1} \AttributeTok{:V2}\NormalTok{\}))}
\end{Highlighting}
\end{Shaded}

\_unnamed {[}9 2{]}:

\begin{longtable}[]{@{}ll@{}}
\toprule
:V1 & :V2\tabularnewline
\midrule
\endhead
1 & 1\tabularnewline
2 & 2\tabularnewline
1 & 3\tabularnewline
2 & 4\tabularnewline
1 & 5\tabularnewline
2 & 6\tabularnewline
1 & 7\tabularnewline
2 & 8\tabularnewline
1 & 9\tabularnewline
\bottomrule
\end{longtable}

\begin{center}\rule{0.5\linewidth}{0.5pt}\end{center}

If we have grouped data set, column selection is applied to every group
separately. Selected columns are dropped.

\begin{Shaded}
\begin{Highlighting}[]
\NormalTok{(}\KeywordTok{->}\NormalTok{ DS}
\NormalTok{    (api/group-by }\AttributeTok{:V1}\NormalTok{)}
\NormalTok{    (api/drop-columns [}\AttributeTok{:V2} \AttributeTok{:V3}\NormalTok{])}
\NormalTok{    (api/groups->map))}
\end{Highlighting}
\end{Shaded}

\{1 Group: 1 {[}5 2{]}:

\begin{longtable}[]{@{}ll@{}}
\toprule
:V1 & :V4\tabularnewline
\midrule
\endhead
1 & A\tabularnewline
1 & C\tabularnewline
1 & B\tabularnewline
1 & A\tabularnewline
1 & C\tabularnewline
\bottomrule
\end{longtable}

, 2 Group: 2 {[}4 2{]}:

\begin{longtable}[]{@{}ll@{}}
\toprule
:V1 & :V4\tabularnewline
\midrule
\endhead
2 & B\tabularnewline
2 & A\tabularnewline
2 & C\tabularnewline
2 & B\tabularnewline
\bottomrule
\end{longtable}

\}

\hypertarget{rename}{%
\paragraph{Rename}\label{rename}}

If you want to rename colums use \texttt{rename-columns} and pass map
where keys are old names, values new ones.

You can also pass mapping function with optional columns-selector

\begin{Shaded}
\begin{Highlighting}[]
\NormalTok{(api/rename-columns DS \{}\AttributeTok{:V1} \StringTok{"v1"}
                        \AttributeTok{:V2} \StringTok{"v2"}
                        \AttributeTok{:V3}\NormalTok{ [}\DecValTok{1} \DecValTok{2} \DecValTok{3}\NormalTok{]}
                        \AttributeTok{:V4}\NormalTok{ (Object.)\})}
\end{Highlighting}
\end{Shaded}

\_unnamed {[}9 4{]}:

\begin{longtable}[]{@{}llll@{}}
\toprule
v1 & v2 & {[}1 2 3{]} &
\href{mailto:java.lang.Object@68901263}{\nolinkurl{java.lang.Object@68901263}}\tabularnewline
\midrule
\endhead
1 & 1 & 0.5 & A\tabularnewline
2 & 2 & 1.0 & B\tabularnewline
1 & 3 & 1.5 & C\tabularnewline
2 & 4 & 0.5 & A\tabularnewline
1 & 5 & 1.0 & B\tabularnewline
2 & 6 & 1.5 & C\tabularnewline
1 & 7 & 0.5 & A\tabularnewline
2 & 8 & 1.0 & B\tabularnewline
1 & 9 & 1.5 & C\tabularnewline
\bottomrule
\end{longtable}

\begin{center}\rule{0.5\linewidth}{0.5pt}\end{center}

Map all names with function

\begin{Shaded}
\begin{Highlighting}[]
\NormalTok{(api/rename-columns DS (}\KeywordTok{comp} \KeywordTok{str} \KeywordTok{second} \KeywordTok{name}\NormalTok{))}
\end{Highlighting}
\end{Shaded}

\_unnamed {[}9 4{]}:

\begin{longtable}[]{@{}llll@{}}
\toprule
1 & 2 & 3 & 4\tabularnewline
\midrule
\endhead
1 & 1 & 0.5 & A\tabularnewline
2 & 2 & 1.0 & B\tabularnewline
1 & 3 & 1.5 & C\tabularnewline
2 & 4 & 0.5 & A\tabularnewline
1 & 5 & 1.0 & B\tabularnewline
2 & 6 & 1.5 & C\tabularnewline
1 & 7 & 0.5 & A\tabularnewline
2 & 8 & 1.0 & B\tabularnewline
1 & 9 & 1.5 & C\tabularnewline
\bottomrule
\end{longtable}

\begin{center}\rule{0.5\linewidth}{0.5pt}\end{center}

Map selected names with function

\begin{Shaded}
\begin{Highlighting}[]
\NormalTok{(api/rename-columns DS [}\AttributeTok{:V1} \AttributeTok{:V3}\NormalTok{] (}\KeywordTok{comp} \KeywordTok{str} \KeywordTok{second} \KeywordTok{name}\NormalTok{))}
\end{Highlighting}
\end{Shaded}

\_unnamed {[}9 4{]}:

\begin{longtable}[]{@{}llll@{}}
\toprule
1 & :V2 & 3 & :V4\tabularnewline
\midrule
\endhead
1 & 1 & 0.5 & A\tabularnewline
2 & 2 & 1.0 & B\tabularnewline
1 & 3 & 1.5 & C\tabularnewline
2 & 4 & 0.5 & A\tabularnewline
1 & 5 & 1.0 & B\tabularnewline
2 & 6 & 1.5 & C\tabularnewline
1 & 7 & 0.5 & A\tabularnewline
2 & 8 & 1.0 & B\tabularnewline
1 & 9 & 1.5 & C\tabularnewline
\bottomrule
\end{longtable}

\begin{center}\rule{0.5\linewidth}{0.5pt}\end{center}

Function works on grouped dataset

\begin{Shaded}
\begin{Highlighting}[]
\NormalTok{(}\KeywordTok{->}\NormalTok{ DS}
\NormalTok{    (api/group-by }\AttributeTok{:V1}\NormalTok{)}
\NormalTok{    (api/rename-columns \{}\AttributeTok{:V1} \StringTok{"v1"}
                         \AttributeTok{:V2} \StringTok{"v2"}
                         \AttributeTok{:V3}\NormalTok{ [}\DecValTok{1} \DecValTok{2} \DecValTok{3}\NormalTok{]}
                         \AttributeTok{:V4}\NormalTok{ (Object.)\})}
\NormalTok{    (api/groups->map))}
\end{Highlighting}
\end{Shaded}

\{1 Group: 1 {[}5 4{]}:

\begin{longtable}[]{@{}llll@{}}
\toprule
v1 & v2 & {[}1 2 3{]} &
\href{mailto:java.lang.Object@6cffde1d}{\nolinkurl{java.lang.Object@6cffde1d}}\tabularnewline
\midrule
\endhead
1 & 1 & 0.5 & A\tabularnewline
1 & 3 & 1.5 & C\tabularnewline
1 & 5 & 1.0 & B\tabularnewline
1 & 7 & 0.5 & A\tabularnewline
1 & 9 & 1.5 & C\tabularnewline
\bottomrule
\end{longtable}

, 2 Group: 2 {[}4 4{]}:

\begin{longtable}[]{@{}llll@{}}
\toprule
v1 & v2 & {[}1 2 3{]} &
\href{mailto:java.lang.Object@6cffde1d}{\nolinkurl{java.lang.Object@6cffde1d}}\tabularnewline
\midrule
\endhead
2 & 2 & 1.0 & B\tabularnewline
2 & 4 & 0.5 & A\tabularnewline
2 & 6 & 1.5 & C\tabularnewline
2 & 8 & 1.0 & B\tabularnewline
\bottomrule
\end{longtable}

\}

\hypertarget{add-or-update}{%
\paragraph{Add or update}\label{add-or-update}}

To add (or replace existing) column call \texttt{add-or-replace-column}
function. Function accepts:

\begin{itemize}
\tightlist
\item
  \texttt{ds} - a dataset
\item
  \texttt{column-name} - if it's existing column name, column will be
  replaced
\item
  \texttt{column} - can be column (from other dataset), sequence, single
  value or function. Too big columns are always trimmed. Too small are
  cycled or extended with missing values (according to
  \texttt{size-strategy} argument)
\item
  \texttt{size-strategy} (optional) - when new column is shorter than
  dataset row count, following strategies are applied:
\item
  \texttt{:cycle} (default) - repeat data

  \begin{itemize}
  \tightlist
  \item
    \texttt{:na} - append missing values
  \item
    \texttt{:strict} - throws an exception when sizes mismatch
  \end{itemize}
\end{itemize}

Function works on grouped dataset.

\begin{center}\rule{0.5\linewidth}{0.5pt}\end{center}

Add single value as column

\begin{Shaded}
\begin{Highlighting}[]
\NormalTok{(api/add-or-replace-column DS }\AttributeTok{:V5} \StringTok{"X"}\NormalTok{)}
\end{Highlighting}
\end{Shaded}

\_unnamed {[}9 5{]}:

\begin{longtable}[]{@{}lllll@{}}
\toprule
:V1 & :V2 & :V3 & :V4 & :V5\tabularnewline
\midrule
\endhead
1 & 1 & 0.5 & A & X\tabularnewline
2 & 2 & 1.0 & B & X\tabularnewline
1 & 3 & 1.5 & C & X\tabularnewline
2 & 4 & 0.5 & A & X\tabularnewline
1 & 5 & 1.0 & B & X\tabularnewline
2 & 6 & 1.5 & C & X\tabularnewline
1 & 7 & 0.5 & A & X\tabularnewline
2 & 8 & 1.0 & B & X\tabularnewline
1 & 9 & 1.5 & C & X\tabularnewline
\bottomrule
\end{longtable}

\begin{center}\rule{0.5\linewidth}{0.5pt}\end{center}

Replace one column (column is trimmed)

\begin{Shaded}
\begin{Highlighting}[]
\NormalTok{(api/add-or-replace-column DS }\AttributeTok{:V1}\NormalTok{ (}\KeywordTok{repeatedly} \KeywordTok{rand}\NormalTok{))}
\end{Highlighting}
\end{Shaded}

\_unnamed {[}9 4{]}:

\begin{longtable}[]{@{}llll@{}}
\toprule
:V1 & :V2 & :V3 & :V4\tabularnewline
\midrule
\endhead
0.5413 & 1 & 0.5 & A\tabularnewline
0.4942 & 2 & 1.0 & B\tabularnewline
0.5967 & 3 & 1.5 & C\tabularnewline
0.5360 & 4 & 0.5 & A\tabularnewline
0.2676 & 5 & 1.0 & B\tabularnewline
0.2873 & 6 & 1.5 & C\tabularnewline
0.2456 & 7 & 0.5 & A\tabularnewline
0.09065 & 8 & 1.0 & B\tabularnewline
0.4498 & 9 & 1.5 & C\tabularnewline
\bottomrule
\end{longtable}

\begin{center}\rule{0.5\linewidth}{0.5pt}\end{center}

Copy column

\begin{Shaded}
\begin{Highlighting}[]
\NormalTok{(api/add-or-replace-column DS }\AttributeTok{:V5}\NormalTok{ (DS }\AttributeTok{:V1}\NormalTok{))}
\end{Highlighting}
\end{Shaded}

\_unnamed {[}9 5{]}:

\begin{longtable}[]{@{}lllll@{}}
\toprule
:V1 & :V2 & :V3 & :V4 & :V5\tabularnewline
\midrule
\endhead
1 & 1 & 0.5 & A & 1\tabularnewline
2 & 2 & 1.0 & B & 2\tabularnewline
1 & 3 & 1.5 & C & 1\tabularnewline
2 & 4 & 0.5 & A & 2\tabularnewline
1 & 5 & 1.0 & B & 1\tabularnewline
2 & 6 & 1.5 & C & 2\tabularnewline
1 & 7 & 0.5 & A & 1\tabularnewline
2 & 8 & 1.0 & B & 2\tabularnewline
1 & 9 & 1.5 & C & 1\tabularnewline
\bottomrule
\end{longtable}

\begin{center}\rule{0.5\linewidth}{0.5pt}\end{center}

When function is used, argument is whole dataset and the result should
be column, sequence or single value

\begin{Shaded}
\begin{Highlighting}[]
\NormalTok{(api/add-or-replace-column DS }\AttributeTok{:row-count}\NormalTok{ api/row-count) }
\end{Highlighting}
\end{Shaded}

\_unnamed {[}9 5{]}:

\begin{longtable}[]{@{}lllll@{}}
\toprule
:V1 & :V2 & :V3 & :V4 & :row-count\tabularnewline
\midrule
\endhead
1 & 1 & 0.5 & A & 9\tabularnewline
2 & 2 & 1.0 & B & 9\tabularnewline
1 & 3 & 1.5 & C & 9\tabularnewline
2 & 4 & 0.5 & A & 9\tabularnewline
1 & 5 & 1.0 & B & 9\tabularnewline
2 & 6 & 1.5 & C & 9\tabularnewline
1 & 7 & 0.5 & A & 9\tabularnewline
2 & 8 & 1.0 & B & 9\tabularnewline
1 & 9 & 1.5 & C & 9\tabularnewline
\bottomrule
\end{longtable}

\begin{center}\rule{0.5\linewidth}{0.5pt}\end{center}

Above example run on grouped dataset, applies function on each group
separately.

\begin{Shaded}
\begin{Highlighting}[]
\NormalTok{(}\KeywordTok{->}\NormalTok{ DS}
\NormalTok{    (api/group-by }\AttributeTok{:V1}\NormalTok{)}
\NormalTok{    (api/add-or-replace-column }\AttributeTok{:row-count}\NormalTok{ api/row-count)}
\NormalTok{    (api/ungroup))}
\end{Highlighting}
\end{Shaded}

\_unnamed {[}9 5{]}:

\begin{longtable}[]{@{}lllll@{}}
\toprule
:V1 & :V2 & :V3 & :V4 & :row-count\tabularnewline
\midrule
\endhead
1 & 1 & 0.5 & A & 5\tabularnewline
1 & 3 & 1.5 & C & 5\tabularnewline
1 & 5 & 1.0 & B & 5\tabularnewline
1 & 7 & 0.5 & A & 5\tabularnewline
1 & 9 & 1.5 & C & 5\tabularnewline
2 & 2 & 1.0 & B & 4\tabularnewline
2 & 4 & 0.5 & A & 4\tabularnewline
2 & 6 & 1.5 & C & 4\tabularnewline
2 & 8 & 1.0 & B & 4\tabularnewline
\bottomrule
\end{longtable}

\begin{center}\rule{0.5\linewidth}{0.5pt}\end{center}

When column which is added is longer than row count in dataset, column
is trimmed. When column is shorter, it's cycled or missing values are
appended.

\begin{Shaded}
\begin{Highlighting}[]
\NormalTok{(api/add-or-replace-column DS }\AttributeTok{:V5}\NormalTok{ [}\AttributeTok{:r} \AttributeTok{:b}\NormalTok{])}
\end{Highlighting}
\end{Shaded}

\_unnamed {[}9 5{]}:

\begin{longtable}[]{@{}lllll@{}}
\toprule
:V1 & :V2 & :V3 & :V4 & :V5\tabularnewline
\midrule
\endhead
1 & 1 & 0.5 & A & :r\tabularnewline
2 & 2 & 1.0 & B & :b\tabularnewline
1 & 3 & 1.5 & C & :r\tabularnewline
2 & 4 & 0.5 & A & :b\tabularnewline
1 & 5 & 1.0 & B & :r\tabularnewline
2 & 6 & 1.5 & C & :b\tabularnewline
1 & 7 & 0.5 & A & :r\tabularnewline
2 & 8 & 1.0 & B & :b\tabularnewline
1 & 9 & 1.5 & C & :r\tabularnewline
\bottomrule
\end{longtable}

\begin{Shaded}
\begin{Highlighting}[]
\NormalTok{(api/add-or-replace-column DS }\AttributeTok{:V5}\NormalTok{ [}\AttributeTok{:r} \AttributeTok{:b}\NormalTok{] }\AttributeTok{:na}\NormalTok{)}
\end{Highlighting}
\end{Shaded}

\_unnamed {[}9 5{]}:

\begin{longtable}[]{@{}lllll@{}}
\toprule
:V1 & :V2 & :V3 & :V4 & :V5\tabularnewline
\midrule
\endhead
1 & 1 & 0.5 & A & :r\tabularnewline
2 & 2 & 1.0 & B & :b\tabularnewline
1 & 3 & 1.5 & C &\tabularnewline
2 & 4 & 0.5 & A &\tabularnewline
1 & 5 & 1.0 & B &\tabularnewline
2 & 6 & 1.5 & C &\tabularnewline
1 & 7 & 0.5 & A &\tabularnewline
2 & 8 & 1.0 & B &\tabularnewline
1 & 9 & 1.5 & C &\tabularnewline
\bottomrule
\end{longtable}

Exception is thrown when \texttt{:strict} strategy is used and column
size is not equal row count

\begin{Shaded}
\begin{Highlighting}[]
\NormalTok{(}\KeywordTok{try}
\NormalTok{  (api/add-or-replace-column DS }\AttributeTok{:V5}\NormalTok{ [}\AttributeTok{:r} \AttributeTok{:b}\NormalTok{] }\AttributeTok{:strict}\NormalTok{)}
\NormalTok{  (}\KeywordTok{catch}\NormalTok{ Exception }\KeywordTok{e}\NormalTok{ (}\KeywordTok{str} \StringTok{"Exception caught: "}\NormalTok{(ex-message }\KeywordTok{e}\NormalTok{))))}
\end{Highlighting}
\end{Shaded}

\begin{verbatim}
"Exception caught: Column size (2) should be exactly the same as dataset row count (9)"
\end{verbatim}

\begin{center}\rule{0.5\linewidth}{0.5pt}\end{center}

Tha same applies for grouped dataset

\begin{Shaded}
\begin{Highlighting}[]
\NormalTok{(}\KeywordTok{->}\NormalTok{ DS}
\NormalTok{    (api/group-by }\AttributeTok{:V3}\NormalTok{)}
\NormalTok{    (api/add-or-replace-column }\AttributeTok{:V5}\NormalTok{ [}\AttributeTok{:r} \AttributeTok{:b}\NormalTok{] }\AttributeTok{:na}\NormalTok{)}
\NormalTok{    (api/ungroup))}
\end{Highlighting}
\end{Shaded}

\_unnamed {[}9 5{]}:

\begin{longtable}[]{@{}lllll@{}}
\toprule
:V1 & :V2 & :V3 & :V4 & :V5\tabularnewline
\midrule
\endhead
2 & 2 & 1.0 & B & :r\tabularnewline
1 & 5 & 1.0 & B & :b\tabularnewline
2 & 8 & 1.0 & B &\tabularnewline
1 & 1 & 0.5 & A & :r\tabularnewline
2 & 4 & 0.5 & A & :b\tabularnewline
1 & 7 & 0.5 & A &\tabularnewline
1 & 3 & 1.5 & C & :r\tabularnewline
2 & 6 & 1.5 & C & :b\tabularnewline
1 & 9 & 1.5 & C &\tabularnewline
\bottomrule
\end{longtable}

\begin{center}\rule{0.5\linewidth}{0.5pt}\end{center}

Let's use other column to fill groups

\begin{Shaded}
\begin{Highlighting}[]
\NormalTok{(}\KeywordTok{->}\NormalTok{ DS}
\NormalTok{    (api/group-by }\AttributeTok{:V3}\NormalTok{)}
\NormalTok{    (api/add-or-replace-column }\AttributeTok{:V5}\NormalTok{ (DS }\AttributeTok{:V2}\NormalTok{))}
\NormalTok{    (api/ungroup))}
\end{Highlighting}
\end{Shaded}

\_unnamed {[}9 5{]}:

\begin{longtable}[]{@{}lllll@{}}
\toprule
:V1 & :V2 & :V3 & :V4 & :V5\tabularnewline
\midrule
\endhead
2 & 2 & 1.0 & B & 1\tabularnewline
1 & 5 & 1.0 & B & 2\tabularnewline
2 & 8 & 1.0 & B & 3\tabularnewline
1 & 1 & 0.5 & A & 1\tabularnewline
2 & 4 & 0.5 & A & 2\tabularnewline
1 & 7 & 0.5 & A & 3\tabularnewline
1 & 3 & 1.5 & C & 1\tabularnewline
2 & 6 & 1.5 & C & 2\tabularnewline
1 & 9 & 1.5 & C & 3\tabularnewline
\bottomrule
\end{longtable}

\begin{center}\rule{0.5\linewidth}{0.5pt}\end{center}

In case you want to add or update several columns you can call
\texttt{add-or-replace-columns} and provide map where keys are column
names, vals are columns.

\begin{Shaded}
\begin{Highlighting}[]
\NormalTok{(api/add-or-replace-columns DS \{}\AttributeTok{:V1}\NormalTok{ #(}\KeywordTok{map} \KeywordTok{inc}\NormalTok{ (}\VariableTok \AttributeTok{:V4}\NormalTok{))}
                               \AttributeTok{:V6} \DecValTok{11}\NormalTok{\})}
\end{Highlighting}
\end{Shaded}

\_unnamed {[}9 6{]}:

\begin{longtable}[]{@{}llllll@{}}
\toprule
:V1 & :V2 & :V3 & :V4 & :V5 & :V6\tabularnewline
\midrule
\endhead
2 & 1 & 0.5 & A & :A & 11\tabularnewline
3 & 2 & 1.0 & B & :B & 11\tabularnewline
2 & 3 & 1.5 & C & :C & 11\tabularnewline
3 & 4 & 0.5 & A & :A & 11\tabularnewline
2 & 5 & 1.0 & B & :B & 11\tabularnewline
3 & 6 & 1.5 & C & :C & 11\tabularnewline
2 & 7 & 0.5 & A & :A & 11\tabularnewline
3 & 8 & 1.0 & B & :B & 11\tabularnewline
2 & 9 & 1.5 & C & :C & 11\tabularnewline
\bottomrule
\end{longtable}

\hypertarget{update}{%
\paragraph{Update}\label{update}}

If you want to modify specific column(s) you can call
\texttt{update-columns}. Arguments:

\begin{itemize}
\tightlist
\item
  dataset
\item
  one of:

  \begin{itemize}
  \tightlist
  \item
    \texttt{columns-selector} and function (or sequence of functions)
  \item
    map where keys are column names and vals are function
  \end{itemize}
\end{itemize}

Functions accept column and have to return column or sequence

\begin{center}\rule{0.5\linewidth}{0.5pt}\end{center}

Reverse of columns

\begin{Shaded}
\begin{Highlighting}[]
\NormalTok{(api/update-columns DS }\AttributeTok{:all} \KeywordTok{reverse}\NormalTok{) }
\end{Highlighting}
\end{Shaded}

\_unnamed {[}9 4{]}:

\begin{longtable}[]{@{}llll@{}}
\toprule
:V1 & :V2 & :V3 & :V4\tabularnewline
\midrule
\endhead
1 & 9 & 1.500 & C\tabularnewline
2 & 8 & 1.000 & B\tabularnewline
1 & 7 & 0.5000 & A\tabularnewline
2 & 6 & 1.500 & C\tabularnewline
1 & 5 & 1.000 & B\tabularnewline
2 & 4 & 0.5000 & A\tabularnewline
1 & 3 & 1.500 & C\tabularnewline
2 & 2 & 1.000 & B\tabularnewline
1 & 1 & 0.5000 & A\tabularnewline
\bottomrule
\end{longtable}

\begin{center}\rule{0.5\linewidth}{0.5pt}\end{center}

Apply dec/inc on numerical columns

\begin{Shaded}
\begin{Highlighting}[]
\NormalTok{(api/update-columns DS }\AttributeTok{:type/numerical}\NormalTok{ [(}\KeywordTok{partial} \KeywordTok{map} \KeywordTok{dec}\NormalTok{)}
\NormalTok{                                        (}\KeywordTok{partial} \KeywordTok{map} \KeywordTok{inc}\NormalTok{)])}
\end{Highlighting}
\end{Shaded}

\_unnamed {[}9 4{]}:

\begin{longtable}[]{@{}llll@{}}
\toprule
:V1 & :V2 & :V3 & :V4\tabularnewline
\midrule
\endhead
0 & 2 & -0.5000 & A\tabularnewline
1 & 3 & 0.000 & B\tabularnewline
0 & 4 & 0.5000 & C\tabularnewline
1 & 5 & -0.5000 & A\tabularnewline
0 & 6 & 0.000 & B\tabularnewline
1 & 7 & 0.5000 & C\tabularnewline
0 & 8 & -0.5000 & A\tabularnewline
1 & 9 & 0.000 & B\tabularnewline
0 & 10 & 0.5000 & C\tabularnewline
\bottomrule
\end{longtable}

\begin{center}\rule{0.5\linewidth}{0.5pt}\end{center}

You can also assing function to a column by packing operations into the
map.

\begin{Shaded}
\begin{Highlighting}[]
\NormalTok{(api/update-columns DS \{}\AttributeTok{:V1} \KeywordTok{reverse}
                        \AttributeTok{:V2}\NormalTok{ (}\KeywordTok{comp}\NormalTok{ shuffle }\KeywordTok{seq}\NormalTok{)\})}
\end{Highlighting}
\end{Shaded}

\_unnamed {[}9 4{]}:

\begin{longtable}[]{@{}llll@{}}
\toprule
:V1 & :V2 & :V3 & :V4\tabularnewline
\midrule
\endhead
1 & 5 & 0.5 & A\tabularnewline
2 & 2 & 1.0 & B\tabularnewline
1 & 9 & 1.5 & C\tabularnewline
2 & 7 & 0.5 & A\tabularnewline
1 & 1 & 1.0 & B\tabularnewline
2 & 4 & 1.5 & C\tabularnewline
1 & 8 & 0.5 & A\tabularnewline
2 & 3 & 1.0 & B\tabularnewline
1 & 6 & 1.5 & C\tabularnewline
\bottomrule
\end{longtable}

\hypertarget{map}{%
\paragraph{Map}\label{map}}

The other way of creating or updating column is to map rows as regular
\texttt{map} function. The arity of mapping function should be the same
as number of selected columns.

Arguments:

\begin{itemize}
\tightlist
\item
  \texttt{ds} - dataset
\item
  \texttt{column-name} - target column name
\item
  \texttt{columns-selector} - columns selected
\item
  \texttt{map-fn} - mapping function
\end{itemize}

\begin{center}\rule{0.5\linewidth}{0.5pt}\end{center}

Let's add numerical columns together

\begin{Shaded}
\begin{Highlighting}[]
\NormalTok{(api/map-columns DS}
                 \AttributeTok{:sum-of-numbers}
\NormalTok{                 (api/column-names DS  #\{}\AttributeTok{:int64} \AttributeTok{:float64}\NormalTok{\} }\AttributeTok{:datatype}\NormalTok{)}
\NormalTok{                 (}\KeywordTok{fn}\NormalTok{ [& rows]}
\NormalTok{                   (}\KeywordTok{reduce} \KeywordTok{+}\NormalTok{ rows)))}
\end{Highlighting}
\end{Shaded}

\_unnamed {[}9 5{]}:

\begin{longtable}[]{@{}lllll@{}}
\toprule
:V1 & :V2 & :V3 & :V4 & :sum-of-numbers\tabularnewline
\midrule
\endhead
1 & 1 & 0.5 & A & 2.5\tabularnewline
2 & 2 & 1.0 & B & 5.0\tabularnewline
1 & 3 & 1.5 & C & 5.5\tabularnewline
2 & 4 & 0.5 & A & 6.5\tabularnewline
1 & 5 & 1.0 & B & 7.0\tabularnewline
2 & 6 & 1.5 & C & 9.5\tabularnewline
1 & 7 & 0.5 & A & 8.5\tabularnewline
2 & 8 & 1.0 & B & 11.0\tabularnewline
1 & 9 & 1.5 & C & 11.5\tabularnewline
\bottomrule
\end{longtable}

The same works on grouped dataset

\begin{Shaded}
\begin{Highlighting}[]
\NormalTok{(}\KeywordTok{->}\NormalTok{ DS}
\NormalTok{    (api/group-by }\AttributeTok{:V4}\NormalTok{)}
\NormalTok{    (api/map-columns }\AttributeTok{:sum-of-numbers}
\NormalTok{                     (api/column-names DS  #\{}\AttributeTok{:int64} \AttributeTok{:float64}\NormalTok{\} }\AttributeTok{:datatype}\NormalTok{)}
\NormalTok{                     (}\KeywordTok{fn}\NormalTok{ [& rows]}
\NormalTok{                       (}\KeywordTok{reduce} \KeywordTok{+}\NormalTok{ rows)))}
\NormalTok{    (api/ungroup))}
\end{Highlighting}
\end{Shaded}

\_unnamed {[}9 5{]}:

\begin{longtable}[]{@{}lllll@{}}
\toprule
:V1 & :V2 & :V3 & :V4 & :sum-of-numbers\tabularnewline
\midrule
\endhead
1 & 1 & 0.5 & A & 2.5\tabularnewline
2 & 4 & 0.5 & A & 6.5\tabularnewline
1 & 7 & 0.5 & A & 8.5\tabularnewline
2 & 2 & 1.0 & B & 5.0\tabularnewline
1 & 5 & 1.0 & B & 7.0\tabularnewline
2 & 8 & 1.0 & B & 11.0\tabularnewline
1 & 3 & 1.5 & C & 5.5\tabularnewline
2 & 6 & 1.5 & C & 9.5\tabularnewline
1 & 9 & 1.5 & C & 11.5\tabularnewline
\bottomrule
\end{longtable}

\hypertarget{reorder}{%
\paragraph{Reorder}\label{reorder}}

To reorder columns use columns selectors to choose what columns go
first. The unseleted columns are appended to the end.

\begin{Shaded}
\begin{Highlighting}[]
\NormalTok{(api/reorder-columns DS }\AttributeTok{:V4}\NormalTok{ [}\AttributeTok{:V3} \AttributeTok{:V2}\NormalTok{] }\AttributeTok{:V1}\NormalTok{)}
\end{Highlighting}
\end{Shaded}

\_unnamed {[}9 4{]}:

\begin{longtable}[]{@{}llll@{}}
\toprule
:V4 & :V2 & :V3 & :V1\tabularnewline
\midrule
\endhead
A & 1 & 0.5 & 1\tabularnewline
B & 2 & 1.0 & 2\tabularnewline
C & 3 & 1.5 & 1\tabularnewline
A & 4 & 0.5 & 2\tabularnewline
B & 5 & 1.0 & 1\tabularnewline
C & 6 & 1.5 & 2\tabularnewline
A & 7 & 0.5 & 1\tabularnewline
B & 8 & 1.0 & 2\tabularnewline
C & 9 & 1.5 & 1\tabularnewline
\bottomrule
\end{longtable}

\begin{center}\rule{0.5\linewidth}{0.5pt}\end{center}

This function doesn't let you select meta field, so you have to call
\texttt{column-names} in such case. Below we want to add integer columns
at the end.

\begin{Shaded}
\begin{Highlighting}[]
\NormalTok{(api/reorder-columns DS (api/column-names DS (}\KeywordTok{complement}\NormalTok{ #\{}\AttributeTok{:int64}\NormalTok{\}) }\AttributeTok{:datatype}\NormalTok{))}
\end{Highlighting}
\end{Shaded}

\_unnamed {[}9 4{]}:

\begin{longtable}[]{@{}llll@{}}
\toprule
:V3 & :V4 & :V1 & :V2\tabularnewline
\midrule
\endhead
0.5 & A & 1 & 1\tabularnewline
1.0 & B & 2 & 2\tabularnewline
1.5 & C & 1 & 3\tabularnewline
0.5 & A & 2 & 4\tabularnewline
1.0 & B & 1 & 5\tabularnewline
1.5 & C & 2 & 6\tabularnewline
0.5 & A & 1 & 7\tabularnewline
1.0 & B & 2 & 8\tabularnewline
1.5 & C & 1 & 9\tabularnewline
\bottomrule
\end{longtable}

\hypertarget{type-conversion}{%
\paragraph{Type conversion}\label{type-conversion}}

To convert column into given datatype can be done using
\texttt{convert-types} function. Not all the types can be converted
automatically also some types require slow parsing (every conversion
from string). In case where conversion is not possible you can pass
conversion function.

Arguments:

\begin{itemize}
\tightlist
\item
  \texttt{ds} - dataset
\item
  Two options:

  \begin{itemize}
  \tightlist
  \item
    \texttt{coltype-map} in case when you want to convert several
    columns, keys are column names, vals are new types
  \item
    \texttt{column-selector} and \texttt{new-types} - column name and
    new datatype (or datatypes as sequence)
  \end{itemize}
\end{itemize}

\texttt{new-types} can be:

\begin{itemize}
\tightlist
\item
  a type like \texttt{:int64} or \texttt{:string} or sequence of types
\item
  or sequence of pair of datetype and conversion function
\end{itemize}

After conversion additional infomation is given on problematic values.

The other conversion is casting column into java array
(\texttt{-\textgreater{}array}) of the type column or provided as
argument. Grouped dataset returns sequence of arrays.

\begin{center}\rule{0.5\linewidth}{0.5pt}\end{center}

Basic conversion

\begin{Shaded}
\begin{Highlighting}[]
\NormalTok{(}\KeywordTok{->}\NormalTok{ DS}
\NormalTok{    (api/convert-types }\AttributeTok{:V1} \AttributeTok{:float64}\NormalTok{)}
\NormalTok{    (api/info }\AttributeTok{:columns}\NormalTok{))}
\end{Highlighting}
\end{Shaded}

\_unnamed :column info {[}4 6{]}:

\begin{longtable}[]{@{}llllll@{}}
\toprule
:name & :size & :datatype & :unparsed-indexes & :unparsed-data &
:categorical?\tabularnewline
\midrule
\endhead
:V1 & 9 & :float64 & \{\} & {[}{]} &\tabularnewline
:V2 & 9 & :int64 & & &\tabularnewline
:V3 & 9 & :float64 & & &\tabularnewline
:V4 & 9 & :string & & & true\tabularnewline
\bottomrule
\end{longtable}

\begin{center}\rule{0.5\linewidth}{0.5pt}\end{center}

Using custom converter. Let's treat \texttt{:V4} as haxadecimal values.
See that this way we can map column to any value.

\begin{Shaded}
\begin{Highlighting}[]
\NormalTok{(}\KeywordTok{->}\NormalTok{ DS}
\NormalTok{    (api/convert-types }\AttributeTok{:V4}\NormalTok{ [[}\AttributeTok{:int16}\NormalTok{ #(Integer/parseInt }\VariableTok{%} \DecValTok{16}\NormalTok{)]]))}
\end{Highlighting}
\end{Shaded}

\_unnamed {[}9 4{]}:

\begin{longtable}[]{@{}llll@{}}
\toprule
:V1 & :V2 & :V3 & :V4\tabularnewline
\midrule
\endhead
1 & 1 & 0.5 & 10\tabularnewline
2 & 2 & 1.0 & 11\tabularnewline
1 & 3 & 1.5 & 12\tabularnewline
2 & 4 & 0.5 & 10\tabularnewline
1 & 5 & 1.0 & 11\tabularnewline
2 & 6 & 1.5 & 12\tabularnewline
1 & 7 & 0.5 & 10\tabularnewline
2 & 8 & 1.0 & 11\tabularnewline
1 & 9 & 1.5 & 12\tabularnewline
\bottomrule
\end{longtable}

\begin{center}\rule{0.5\linewidth}{0.5pt}\end{center}

You can process several columns at once

\begin{Shaded}
\begin{Highlighting}[]
\NormalTok{(}\KeywordTok{->}\NormalTok{ DS}
\NormalTok{    (api/convert-types \{}\AttributeTok{:V1} \AttributeTok{:float64}
                        \AttributeTok{:V2} \AttributeTok{:object}
                        \AttributeTok{:V3}\NormalTok{ [}\AttributeTok{:boolean}\NormalTok{ #(}\KeywordTok{<} \VariableTok{%} \FloatTok{1.0}\NormalTok{)]}
                        \AttributeTok{:V4} \AttributeTok{:object}\NormalTok{\})}
\NormalTok{    (api/info }\AttributeTok{:columns}\NormalTok{))}
\end{Highlighting}
\end{Shaded}

\_unnamed :column info {[}4 5{]}:

\begin{longtable}[]{@{}lllll@{}}
\toprule
:name & :size & :datatype & :unparsed-indexes &
:unparsed-data\tabularnewline
\midrule
\endhead
:V1 & 9 & :float64 & \{\} & {[}{]}\tabularnewline
:V2 & 9 & :object & \{\} & {[}{]}\tabularnewline
:V3 & 9 & :boolean & \{\} & {[}{]}\tabularnewline
:V4 & 9 & :object & &\tabularnewline
\bottomrule
\end{longtable}

\begin{center}\rule{0.5\linewidth}{0.5pt}\end{center}

Convert one type into another

\begin{Shaded}
\begin{Highlighting}[]
\NormalTok{(}\KeywordTok{->}\NormalTok{ DS}
\NormalTok{    (api/convert-types }\AttributeTok{:type/numerical} \AttributeTok{:int16}\NormalTok{)}
\NormalTok{    (api/info }\AttributeTok{:columns}\NormalTok{))}
\end{Highlighting}
\end{Shaded}

\_unnamed :column info {[}4 6{]}:

\begin{longtable}[]{@{}llllll@{}}
\toprule
:name & :size & :datatype & :unparsed-indexes & :unparsed-data &
:categorical?\tabularnewline
\midrule
\endhead
:V1 & 9 & :int16 & \{\} & {[}{]} &\tabularnewline
:V2 & 9 & :int16 & \{\} & {[}{]} &\tabularnewline
:V3 & 9 & :int16 & \{\} & {[}{]} &\tabularnewline
:V4 & 9 & :string & & & true\tabularnewline
\bottomrule
\end{longtable}

\begin{center}\rule{0.5\linewidth}{0.5pt}\end{center}

Function works on the grouped dataset

\begin{Shaded}
\begin{Highlighting}[]
\NormalTok{(}\KeywordTok{->}\NormalTok{ DS}
\NormalTok{    (api/group-by }\AttributeTok{:V1}\NormalTok{)}
\NormalTok{    (api/convert-types }\AttributeTok{:V1} \AttributeTok{:float32}\NormalTok{)}
\NormalTok{    (api/ungroup)}
\NormalTok{    (api/info }\AttributeTok{:columns}\NormalTok{))}
\end{Highlighting}
\end{Shaded}

\_unnamed :column info {[}4 6{]}:

\begin{longtable}[]{@{}llllll@{}}
\toprule
:name & :size & :datatype & :unparsed-indexes & :unparsed-data &
:categorical?\tabularnewline
\midrule
\endhead
:V1 & 9 & :float32 & \{\} & {[}{]} &\tabularnewline
:V2 & 9 & :int64 & & &\tabularnewline
:V3 & 9 & :float64 & & &\tabularnewline
:V4 & 9 & :string & & & true\tabularnewline
\bottomrule
\end{longtable}

\begin{center}\rule{0.5\linewidth}{0.5pt}\end{center}

Double array conversion.

\begin{Shaded}
\begin{Highlighting}[]
\NormalTok{(api/->array DS }\AttributeTok{:V1}\NormalTok{)}
\end{Highlighting}
\end{Shaded}

\begin{verbatim}
#object["[J" 0x549f9bfb "[J@549f9bfb"]
\end{verbatim}

\begin{center}\rule{0.5\linewidth}{0.5pt}\end{center}

Function also works on grouped dataset

\begin{Shaded}
\begin{Highlighting}[]
\NormalTok{(}\KeywordTok{->}\NormalTok{ DS}
\NormalTok{    (api/group-by }\AttributeTok{:V3}\NormalTok{)}
\NormalTok{    (api/->array }\AttributeTok{:V2}\NormalTok{))}
\end{Highlighting}
\end{Shaded}

\begin{verbatim}
(#object["[J" 0x76f60962 "[J@76f60962"] #object["[J" 0x35ed3c09 "[J@35ed3c09"] #object["[J" 0x224720c3 "[J@224720c3"])
\end{verbatim}

\begin{center}\rule{0.5\linewidth}{0.5pt}\end{center}

You can also cast the type to the other one (if casting is possible):

\begin{Shaded}
\begin{Highlighting}[]
\NormalTok{(api/->array DS }\AttributeTok{:V4} \AttributeTok{:string}\NormalTok{)}
\NormalTok{(api/->array DS }\AttributeTok{:V1} \AttributeTok{:float32}\NormalTok{)}
\end{Highlighting}
\end{Shaded}

\begin{verbatim}
#object["[Ljava.lang.String;" 0x552f8268 "[Ljava.lang.String;@552f8268"]
#object["[F" 0x4e2b6b80 "[F@4e2b6b80"]
\end{verbatim}

\hypertarget{rows}{%
\subsubsection{Rows}\label{rows}}

Rows can be selected or dropped using various selectors:

\begin{itemize}
\tightlist
\item
  row id(s) - row index as number or seqence of numbers (first row has
  index \texttt{0}, second \texttt{1} and so on)
\item
  sequence of true/false values
\item
  filter by predicate (argument is row as a map)
\end{itemize}

When predicate is used you may want to limit columns passed to the
function (\texttt{select-keys} option).

Additionally you may want to precalculate some values which will be
visible for predicate as additional columns. It's done internally by
calling \texttt{add-or-replace-columns} on a dataset. \texttt{:pre} is
used as a column definitions.

\hypertarget{select-1}{%
\paragraph{Select}\label{select-1}}

Select fifth row

\begin{Shaded}
\begin{Highlighting}[]
\NormalTok{(api/select-rows DS }\DecValTok{4}\NormalTok{)}
\end{Highlighting}
\end{Shaded}

\_unnamed {[}1 4{]}:

\begin{longtable}[]{@{}llll@{}}
\toprule
:V1 & :V2 & :V3 & :V4\tabularnewline
\midrule
\endhead
1 & 5 & 1.0 & B\tabularnewline
\bottomrule
\end{longtable}

\begin{center}\rule{0.5\linewidth}{0.5pt}\end{center}

Select 3 rows

\begin{Shaded}
\begin{Highlighting}[]
\NormalTok{(api/select-rows DS [}\DecValTok{1} \DecValTok{4} \DecValTok{5}\NormalTok{])}
\end{Highlighting}
\end{Shaded}

\_unnamed {[}3 4{]}:

\begin{longtable}[]{@{}llll@{}}
\toprule
:V1 & :V2 & :V3 & :V4\tabularnewline
\midrule
\endhead
2 & 2 & 1.0 & B\tabularnewline
1 & 5 & 1.0 & B\tabularnewline
2 & 6 & 1.5 & C\tabularnewline
\bottomrule
\end{longtable}

\begin{center}\rule{0.5\linewidth}{0.5pt}\end{center}

Select rows using sequence of true/false values

\begin{Shaded}
\begin{Highlighting}[]
\NormalTok{(api/select-rows DS [}\VariableTok{true} \VariableTok{nil} \VariableTok{nil} \VariableTok{true}\NormalTok{])}
\end{Highlighting}
\end{Shaded}

\_unnamed {[}2 4{]}:

\begin{longtable}[]{@{}llll@{}}
\toprule
:V1 & :V2 & :V3 & :V4\tabularnewline
\midrule
\endhead
1 & 1 & 0.5 & A\tabularnewline
2 & 4 & 0.5 & A\tabularnewline
\bottomrule
\end{longtable}

\begin{center}\rule{0.5\linewidth}{0.5pt}\end{center}

Select rows using predicate

\begin{Shaded}
\begin{Highlighting}[]
\NormalTok{(api/select-rows DS (}\KeywordTok{comp}\NormalTok{ #(}\KeywordTok{<} \VariableTok{%} \DecValTok{1}\NormalTok{) }\AttributeTok{:V3}\NormalTok{))}
\end{Highlighting}
\end{Shaded}

\_unnamed {[}3 4{]}:

\begin{longtable}[]{@{}llll@{}}
\toprule
:V1 & :V2 & :V3 & :V4\tabularnewline
\midrule
\endhead
1 & 1 & 0.5 & A\tabularnewline
2 & 4 & 0.5 & A\tabularnewline
1 & 7 & 0.5 & A\tabularnewline
\bottomrule
\end{longtable}

\begin{center}\rule{0.5\linewidth}{0.5pt}\end{center}

The same works on grouped dataset, let's select first row from every
group.

\begin{Shaded}
\begin{Highlighting}[]
\NormalTok{(}\KeywordTok{->}\NormalTok{ DS}
\NormalTok{    (api/group-by }\AttributeTok{:V1}\NormalTok{)}
\NormalTok{    (api/select-rows }\DecValTok{0}\NormalTok{)}
\NormalTok{    (api/ungroup))}
\end{Highlighting}
\end{Shaded}

\_unnamed {[}2 4{]}:

\begin{longtable}[]{@{}llll@{}}
\toprule
:V1 & :V2 & :V3 & :V4\tabularnewline
\midrule
\endhead
1 & 1 & 0.5 & A\tabularnewline
2 & 2 & 1.0 & B\tabularnewline
\bottomrule
\end{longtable}

\begin{center}\rule{0.5\linewidth}{0.5pt}\end{center}

If you want to select \texttt{:V2} values which are lower than or equal
mean in grouped dataset you have to precalculate it using \texttt{:pre}.

\begin{Shaded}
\begin{Highlighting}[]
\NormalTok{(}\KeywordTok{->}\NormalTok{ DS}
\NormalTok{    (api/group-by }\AttributeTok{:V4}\NormalTok{)}
\NormalTok{    (api/select-rows (}\KeywordTok{fn}\NormalTok{ [row] (}\KeywordTok{<=}\NormalTok{ (}\AttributeTok{:V2}\NormalTok{ row) (}\AttributeTok{:mean}\NormalTok{ row)))}
\NormalTok{                     \{}\AttributeTok{:pre}\NormalTok{ \{}\AttributeTok{:mean}\NormalTok{ #(tech.v2.datatype.functional/mean (}\VariableTok{%} \AttributeTok{:V2}\NormalTok{))\}\})}
\NormalTok{    (api/ungroup))}
\end{Highlighting}
\end{Shaded}

\_unnamed {[}6 4{]}:

\begin{longtable}[]{@{}llll@{}}
\toprule
:V1 & :V2 & :V3 & :V4\tabularnewline
\midrule
\endhead
1 & 1 & 0.5 & A\tabularnewline
2 & 4 & 0.5 & A\tabularnewline
2 & 2 & 1.0 & B\tabularnewline
1 & 5 & 1.0 & B\tabularnewline
1 & 3 & 1.5 & C\tabularnewline
2 & 6 & 1.5 & C\tabularnewline
\bottomrule
\end{longtable}

\hypertarget{drop-1}{%
\paragraph{Drop}\label{drop-1}}

\texttt{drop-rows} removes rows, and accepts exactly the same parameters
as \texttt{select-rows}

\begin{center}\rule{0.5\linewidth}{0.5pt}\end{center}

Drop values lower than or equal \texttt{:V2} column mean in grouped
dataset.

\begin{Shaded}
\begin{Highlighting}[]
\NormalTok{(}\KeywordTok{->}\NormalTok{ DS}
\NormalTok{    (api/group-by }\AttributeTok{:V4}\NormalTok{)}
\NormalTok{    (api/drop-rows (}\KeywordTok{fn}\NormalTok{ [row] (}\KeywordTok{<=}\NormalTok{ (}\AttributeTok{:V2}\NormalTok{ row) (}\AttributeTok{:mean}\NormalTok{ row)))}
\NormalTok{                   \{}\AttributeTok{:pre}\NormalTok{ \{}\AttributeTok{:mean}\NormalTok{ #(tech.v2.datatype.functional/mean (}\VariableTok{%} \AttributeTok{:V2}\NormalTok{))\}\})}
\NormalTok{    (api/ungroup))}
\end{Highlighting}
\end{Shaded}

\_unnamed {[}3 4{]}:

\begin{longtable}[]{@{}llll@{}}
\toprule
:V1 & :V2 & :V3 & :V4\tabularnewline
\midrule
\endhead
1 & 7 & 0.5 & A\tabularnewline
2 & 8 & 1.0 & B\tabularnewline
1 & 9 & 1.5 & C\tabularnewline
\bottomrule
\end{longtable}

\hypertarget{other}{%
\paragraph{Other}\label{other}}

There are several function to select first, last, random rows, or
display head, tail of the dataset. All functions work on grouped
dataset.

All random functions accept \texttt{:seed} as an option if you want to
fix returned result.

\begin{center}\rule{0.5\linewidth}{0.5pt}\end{center}

First row

\begin{Shaded}
\begin{Highlighting}[]
\NormalTok{(api/first DS)}
\end{Highlighting}
\end{Shaded}

\_unnamed {[}1 4{]}:

\begin{longtable}[]{@{}llll@{}}
\toprule
:V1 & :V2 & :V3 & :V4\tabularnewline
\midrule
\endhead
1 & 1 & 0.5 & A\tabularnewline
\bottomrule
\end{longtable}

\begin{center}\rule{0.5\linewidth}{0.5pt}\end{center}

Last row

\begin{Shaded}
\begin{Highlighting}[]
\NormalTok{(api/last DS)}
\end{Highlighting}
\end{Shaded}

\_unnamed {[}1 4{]}:

\begin{longtable}[]{@{}llll@{}}
\toprule
:V1 & :V2 & :V3 & :V4\tabularnewline
\midrule
\endhead
1 & 9 & 1.5 & C\tabularnewline
\bottomrule
\end{longtable}

\begin{center}\rule{0.5\linewidth}{0.5pt}\end{center}

Random row (single)

\begin{Shaded}
\begin{Highlighting}[]
\NormalTok{(api/rand-nth DS)}
\end{Highlighting}
\end{Shaded}

\_unnamed {[}1 4{]}:

\begin{longtable}[]{@{}llll@{}}
\toprule
:V1 & :V2 & :V3 & :V4\tabularnewline
\midrule
\endhead
1 & 5 & 1.0 & B\tabularnewline
\bottomrule
\end{longtable}

\begin{center}\rule{0.5\linewidth}{0.5pt}\end{center}

Random row (single) with seed

\begin{Shaded}
\begin{Highlighting}[]
\NormalTok{(api/rand-nth DS \{}\AttributeTok{:seed} \DecValTok{42}\NormalTok{\})}
\end{Highlighting}
\end{Shaded}

\_unnamed {[}1 4{]}:

\begin{longtable}[]{@{}llll@{}}
\toprule
:V1 & :V2 & :V3 & :V4\tabularnewline
\midrule
\endhead
2 & 6 & 1.5 & C\tabularnewline
\bottomrule
\end{longtable}

\begin{center}\rule{0.5\linewidth}{0.5pt}\end{center}

Random \texttt{n} (default: row count) rows with repetition.

\begin{Shaded}
\begin{Highlighting}[]
\NormalTok{(api/random DS)}
\end{Highlighting}
\end{Shaded}

\_unnamed {[}9 4{]}:

\begin{longtable}[]{@{}llll@{}}
\toprule
:V1 & :V2 & :V3 & :V4\tabularnewline
\midrule
\endhead
1 & 7 & 0.5 & A\tabularnewline
1 & 7 & 0.5 & A\tabularnewline
2 & 2 & 1.0 & B\tabularnewline
2 & 6 & 1.5 & C\tabularnewline
1 & 9 & 1.5 & C\tabularnewline
2 & 6 & 1.5 & C\tabularnewline
1 & 5 & 1.0 & B\tabularnewline
1 & 3 & 1.5 & C\tabularnewline
2 & 6 & 1.5 & C\tabularnewline
\bottomrule
\end{longtable}

\begin{center}\rule{0.5\linewidth}{0.5pt}\end{center}

Five random rows with repetition

\begin{Shaded}
\begin{Highlighting}[]
\NormalTok{(api/random DS }\DecValTok{5}\NormalTok{)}
\end{Highlighting}
\end{Shaded}

\_unnamed {[}5 4{]}:

\begin{longtable}[]{@{}llll@{}}
\toprule
:V1 & :V2 & :V3 & :V4\tabularnewline
\midrule
\endhead
2 & 6 & 1.5 & C\tabularnewline
2 & 8 & 1.0 & B\tabularnewline
1 & 1 & 0.5 & A\tabularnewline
2 & 2 & 1.0 & B\tabularnewline
1 & 7 & 0.5 & A\tabularnewline
\bottomrule
\end{longtable}

\begin{center}\rule{0.5\linewidth}{0.5pt}\end{center}

Five random, non-repeating rows

\begin{Shaded}
\begin{Highlighting}[]
\NormalTok{(api/random DS }\DecValTok{5}\NormalTok{ \{}\AttributeTok{:repeat}\NormalTok{? }\VariableTok{false}\NormalTok{\})}
\end{Highlighting}
\end{Shaded}

\_unnamed {[}5 4{]}:

\begin{longtable}[]{@{}llll@{}}
\toprule
:V1 & :V2 & :V3 & :V4\tabularnewline
\midrule
\endhead
1 & 9 & 1.5 & C\tabularnewline
1 & 7 & 0.5 & A\tabularnewline
2 & 8 & 1.0 & B\tabularnewline
2 & 6 & 1.5 & C\tabularnewline
1 & 5 & 1.0 & B\tabularnewline
\bottomrule
\end{longtable}

\begin{center}\rule{0.5\linewidth}{0.5pt}\end{center}

Five random, with seed

\begin{Shaded}
\begin{Highlighting}[]
\NormalTok{(api/random DS }\DecValTok{5}\NormalTok{ \{}\AttributeTok{:seed} \DecValTok{42}\NormalTok{\})}
\end{Highlighting}
\end{Shaded}

\_unnamed {[}5 4{]}:

\begin{longtable}[]{@{}llll@{}}
\toprule
:V1 & :V2 & :V3 & :V4\tabularnewline
\midrule
\endhead
2 & 6 & 1.5 & C\tabularnewline
1 & 5 & 1.0 & B\tabularnewline
1 & 3 & 1.5 & C\tabularnewline
1 & 1 & 0.5 & A\tabularnewline
1 & 9 & 1.5 & C\tabularnewline
\bottomrule
\end{longtable}

\begin{center}\rule{0.5\linewidth}{0.5pt}\end{center}

Shuffle dataset

\begin{Shaded}
\begin{Highlighting}[]
\NormalTok{(api/shuffle DS)}
\end{Highlighting}
\end{Shaded}

\_unnamed {[}9 4{]}:

\begin{longtable}[]{@{}llll@{}}
\toprule
:V1 & :V2 & :V3 & :V4\tabularnewline
\midrule
\endhead
1 & 3 & 1.5 & C\tabularnewline
2 & 2 & 1.0 & B\tabularnewline
1 & 5 & 1.0 & B\tabularnewline
2 & 4 & 0.5 & A\tabularnewline
2 & 6 & 1.5 & C\tabularnewline
1 & 7 & 0.5 & A\tabularnewline
2 & 8 & 1.0 & B\tabularnewline
1 & 1 & 0.5 & A\tabularnewline
1 & 9 & 1.5 & C\tabularnewline
\bottomrule
\end{longtable}

\begin{center}\rule{0.5\linewidth}{0.5pt}\end{center}

Shuffle with seed

\begin{Shaded}
\begin{Highlighting}[]
\NormalTok{(api/shuffle DS \{}\AttributeTok{:seed} \DecValTok{42}\NormalTok{\})}
\end{Highlighting}
\end{Shaded}

\_unnamed {[}9 4{]}:

\begin{longtable}[]{@{}llll@{}}
\toprule
:V1 & :V2 & :V3 & :V4\tabularnewline
\midrule
\endhead
1 & 5 & 1.0 & B\tabularnewline
2 & 2 & 1.0 & B\tabularnewline
2 & 6 & 1.5 & C\tabularnewline
2 & 4 & 0.5 & A\tabularnewline
2 & 8 & 1.0 & B\tabularnewline
1 & 3 & 1.5 & C\tabularnewline
1 & 7 & 0.5 & A\tabularnewline
1 & 1 & 0.5 & A\tabularnewline
1 & 9 & 1.5 & C\tabularnewline
\bottomrule
\end{longtable}

\begin{center}\rule{0.5\linewidth}{0.5pt}\end{center}

First \texttt{n} rows (default 5)

\begin{Shaded}
\begin{Highlighting}[]
\NormalTok{(api/head DS)}
\end{Highlighting}
\end{Shaded}

\_unnamed {[}5 4{]}:

\begin{longtable}[]{@{}llll@{}}
\toprule
:V1 & :V2 & :V3 & :V4\tabularnewline
\midrule
\endhead
1 & 1 & 0.5 & A\tabularnewline
2 & 2 & 1.0 & B\tabularnewline
1 & 3 & 1.5 & C\tabularnewline
2 & 4 & 0.5 & A\tabularnewline
1 & 5 & 1.0 & B\tabularnewline
\bottomrule
\end{longtable}

\begin{center}\rule{0.5\linewidth}{0.5pt}\end{center}

Last \texttt{n} rows (default 5)

\begin{Shaded}
\begin{Highlighting}[]
\NormalTok{(api/tail DS)}
\end{Highlighting}
\end{Shaded}

\_unnamed {[}5 4{]}:

\begin{longtable}[]{@{}llll@{}}
\toprule
:V1 & :V2 & :V3 & :V4\tabularnewline
\midrule
\endhead
1 & 5 & 1.0 & B\tabularnewline
2 & 6 & 1.5 & C\tabularnewline
1 & 7 & 0.5 & A\tabularnewline
2 & 8 & 1.0 & B\tabularnewline
1 & 9 & 1.5 & C\tabularnewline
\bottomrule
\end{longtable}

\begin{center}\rule{0.5\linewidth}{0.5pt}\end{center}

\texttt{by-rank} calculates rank on column(s). It's base on
\href{https://www.rdocumentation.org/packages/base/versions/3.6.1/topics/rank}{R
rank()} with addition of \texttt{:dense} (default) tie strategy which
give consecutive rank numbering.

\texttt{:desc?} options (default: \texttt{true}) sorts input with
descending order, giving top values under \texttt{0} value.

\texttt{rank} is zero based and is defined at
\texttt{tablecloth.api.utils} namespace.

\begin{center}\rule{0.5\linewidth}{0.5pt}\end{center}

\begin{Shaded}
\begin{Highlighting}[]
\NormalTok{(api/by-rank DS }\AttributeTok{:V3} \KeywordTok{zero?}\NormalTok{) }\CommentTok{;; most V3 values}
\end{Highlighting}
\end{Shaded}

\_unnamed {[}3 4{]}:

\begin{longtable}[]{@{}llll@{}}
\toprule
:V1 & :V2 & :V3 & :V4\tabularnewline
\midrule
\endhead
1 & 3 & 1.5 & C\tabularnewline
2 & 6 & 1.5 & C\tabularnewline
1 & 9 & 1.5 & C\tabularnewline
\bottomrule
\end{longtable}

\begin{Shaded}
\begin{Highlighting}[]
\NormalTok{(api/by-rank DS }\AttributeTok{:V3} \KeywordTok{zero?}\NormalTok{ \{}\AttributeTok{:desc}\NormalTok{? }\VariableTok{false}\NormalTok{\}) }\CommentTok{;; least V3 values}
\end{Highlighting}
\end{Shaded}

\_unnamed {[}3 4{]}:

\begin{longtable}[]{@{}llll@{}}
\toprule
:V1 & :V2 & :V3 & :V4\tabularnewline
\midrule
\endhead
1 & 1 & 0.5 & A\tabularnewline
2 & 4 & 0.5 & A\tabularnewline
1 & 7 & 0.5 & A\tabularnewline
\bottomrule
\end{longtable}

\begin{center}\rule{0.5\linewidth}{0.5pt}\end{center}

Rank also works on multiple columns

\begin{Shaded}
\begin{Highlighting}[]
\NormalTok{(api/by-rank DS [}\AttributeTok{:V1} \AttributeTok{:V3}\NormalTok{] }\KeywordTok{zero?}\NormalTok{ \{}\AttributeTok{:desc}\NormalTok{? }\VariableTok{false}\NormalTok{\})}
\end{Highlighting}
\end{Shaded}

\_unnamed {[}2 4{]}:

\begin{longtable}[]{@{}llll@{}}
\toprule
:V1 & :V2 & :V3 & :V4\tabularnewline
\midrule
\endhead
1 & 1 & 0.5 & A\tabularnewline
1 & 7 & 0.5 & A\tabularnewline
\bottomrule
\end{longtable}

\begin{center}\rule{0.5\linewidth}{0.5pt}\end{center}

Select 5 random rows from each group

\begin{Shaded}
\begin{Highlighting}[]
\NormalTok{(}\KeywordTok{->}\NormalTok{ DS}
\NormalTok{    (api/group-by }\AttributeTok{:V4}\NormalTok{)}
\NormalTok{    (api/random }\DecValTok{5}\NormalTok{)}
\NormalTok{    (api/ungroup))}
\end{Highlighting}
\end{Shaded}

\_unnamed {[}15 4{]}:

\begin{longtable}[]{@{}llll@{}}
\toprule
:V1 & :V2 & :V3 & :V4\tabularnewline
\midrule
\endhead
2 & 4 & 0.5 & A\tabularnewline
2 & 4 & 0.5 & A\tabularnewline
1 & 1 & 0.5 & A\tabularnewline
2 & 4 & 0.5 & A\tabularnewline
1 & 7 & 0.5 & A\tabularnewline
2 & 8 & 1.0 & B\tabularnewline
2 & 8 & 1.0 & B\tabularnewline
1 & 5 & 1.0 & B\tabularnewline
2 & 8 & 1.0 & B\tabularnewline
2 & 8 & 1.0 & B\tabularnewline
1 & 9 & 1.5 & C\tabularnewline
1 & 9 & 1.5 & C\tabularnewline
2 & 6 & 1.5 & C\tabularnewline
1 & 9 & 1.5 & C\tabularnewline
1 & 9 & 1.5 & C\tabularnewline
\bottomrule
\end{longtable}

\hypertarget{aggregate}{%
\subsubsection{Aggregate}\label{aggregate}}

Aggregating is a function which produces single row out of dataset.

Aggregator is a function or sequence or map of functions which accept
dataset as an argument and result single value, sequence of values or
map.

Where map is given as an input or result, keys are treated as column
names.

Grouped dataset is ungrouped after aggreation. This can be turned off by
setting \texttt{:ungroup} to false. In case you want to pass additional
ungrouping parameters add them to the options.

By default resulting column names are prefixed with \texttt{summary}
prefix (set it with \texttt{:default-column-name-prefix} option).

\begin{center}\rule{0.5\linewidth}{0.5pt}\end{center}

Let's calculate mean of some columns

\begin{Shaded}
\begin{Highlighting}[]
\NormalTok{(api/aggregate DS #(}\KeywordTok{reduce} \KeywordTok{+}\NormalTok{ (}\VariableTok \AttributeTok{:V2}\NormalTok{))\})}
\end{Highlighting}
\end{Shaded}

\_unnamed {[}1 1{]}:

\begin{longtable}[]{@{}l@{}}
\toprule
:sum-of-V2\tabularnewline
\midrule
\endhead
45\tabularnewline
\bottomrule
\end{longtable}

\begin{center}\rule{0.5\linewidth}{0.5pt}\end{center}

Sequential result is spread into separate columns

\begin{Shaded}
\begin{Highlighting}[]
\NormalTok{(api/aggregate DS #(}\KeywordTok{take} \DecValTok{5}\NormalTok{(}\VariableTok{%} \AttributeTok{:V2}\NormalTok{)))}
\end{Highlighting}
\end{Shaded}

\_unnamed {[}1 5{]}:

\begin{longtable}[]{@{}lllll@{}}
\toprule
:summary-0 & :summary-1 & :summary-2 & :summary-3 &
:summary-4\tabularnewline
\midrule
\endhead
1 & 2 & 3 & 4 & 5\tabularnewline
\bottomrule
\end{longtable}

\begin{center}\rule{0.5\linewidth}{0.5pt}\end{center}

You can combine all variants and rename default prefix

\begin{Shaded}
\begin{Highlighting}[]
\NormalTok{(api/aggregate DS [#(}\KeywordTok{take} \DecValTok{3}\NormalTok{ (}\VariableTok{%} \AttributeTok{:V2}\NormalTok{))}
\NormalTok{                   (}\KeywordTok{fn}\NormalTok{ [ds] \{}\AttributeTok{:sum-v1}\NormalTok{ (}\KeywordTok{reduce} \KeywordTok{+}\NormalTok{ (ds }\AttributeTok{:V1}\NormalTok{))}
                            \AttributeTok{:prod-v3}\NormalTok{ (}\KeywordTok{reduce} \KeywordTok{*}\NormalTok{ (ds }\AttributeTok{:V3}\NormalTok{))\})] \{}\AttributeTok{:default-column-name-prefix} \StringTok{"V2-value"}\NormalTok{\})}
\end{Highlighting}
\end{Shaded}

\_unnamed {[}1 5{]}:

\begin{longtable}[]{@{}lllll@{}}
\toprule
\begin{minipage}[b]{0.15\columnwidth}\raggedright
:V2-value-0-0\strut
\end{minipage} & \begin{minipage}[b]{0.15\columnwidth}\raggedright
:V2-value-0-1\strut
\end{minipage} & \begin{minipage}[b]{0.15\columnwidth}\raggedright
:V2-value-0-2\strut
\end{minipage} & \begin{minipage}[b]{0.20\columnwidth}\raggedright
:V2-value-1-sum-v1\strut
\end{minipage} & \begin{minipage}[b]{0.21\columnwidth}\raggedright
:V2-value-1-prod-v3\strut
\end{minipage}\tabularnewline
\midrule
\endhead
\begin{minipage}[t]{0.15\columnwidth}\raggedright
1\strut
\end{minipage} & \begin{minipage}[t]{0.15\columnwidth}\raggedright
2\strut
\end{minipage} & \begin{minipage}[t]{0.15\columnwidth}\raggedright
3\strut
\end{minipage} & \begin{minipage}[t]{0.20\columnwidth}\raggedright
13\strut
\end{minipage} & \begin{minipage}[t]{0.21\columnwidth}\raggedright
0.421875\strut
\end{minipage}\tabularnewline
\bottomrule
\end{longtable}

\begin{center}\rule{0.5\linewidth}{0.5pt}\end{center}

Processing grouped dataset

\begin{Shaded}
\begin{Highlighting}[]
\NormalTok{(}\KeywordTok{->}\NormalTok{ DS}
\NormalTok{    (api/group-by [}\AttributeTok{:V4}\NormalTok{])}
\NormalTok{    (api/aggregate [#(}\KeywordTok{take} \DecValTok{3}\NormalTok{ (}\VariableTok{%} \AttributeTok{:V2}\NormalTok{))}
\NormalTok{                    (}\KeywordTok{fn}\NormalTok{ [ds] \{}\AttributeTok{:sum-v1}\NormalTok{ (}\KeywordTok{reduce} \KeywordTok{+}\NormalTok{ (ds }\AttributeTok{:V1}\NormalTok{))}
                             \AttributeTok{:prod-v3}\NormalTok{ (}\KeywordTok{reduce} \KeywordTok{*}\NormalTok{ (ds }\AttributeTok{:V3}\NormalTok{))\})] \{}\AttributeTok{:default-column-name-prefix} \StringTok{"V2-value"}\NormalTok{\}))}
\end{Highlighting}
\end{Shaded}

\_unnamed {[}3 6{]}:

\begin{longtable}[]{@{}llllll@{}}
\toprule
\begin{minipage}[b]{0.05\columnwidth}\raggedright
:V4\strut
\end{minipage} & \begin{minipage}[b]{0.14\columnwidth}\raggedright
:V2-value-0-0\strut
\end{minipage} & \begin{minipage}[b]{0.14\columnwidth}\raggedright
:V2-value-0-1\strut
\end{minipage} & \begin{minipage}[b]{0.14\columnwidth}\raggedright
:V2-value-0-2\strut
\end{minipage} & \begin{minipage}[b]{0.18\columnwidth}\raggedright
:V2-value-1-sum-v1\strut
\end{minipage} & \begin{minipage}[b]{0.19\columnwidth}\raggedright
:V2-value-1-prod-v3\strut
\end{minipage}\tabularnewline
\midrule
\endhead
\begin{minipage}[t]{0.05\columnwidth}\raggedright
B\strut
\end{minipage} & \begin{minipage}[t]{0.14\columnwidth}\raggedright
2\strut
\end{minipage} & \begin{minipage}[t]{0.14\columnwidth}\raggedright
5\strut
\end{minipage} & \begin{minipage}[t]{0.14\columnwidth}\raggedright
8\strut
\end{minipage} & \begin{minipage}[t]{0.18\columnwidth}\raggedright
5\strut
\end{minipage} & \begin{minipage}[t]{0.19\columnwidth}\raggedright
1.000\strut
\end{minipage}\tabularnewline
\begin{minipage}[t]{0.05\columnwidth}\raggedright
C\strut
\end{minipage} & \begin{minipage}[t]{0.14\columnwidth}\raggedright
3\strut
\end{minipage} & \begin{minipage}[t]{0.14\columnwidth}\raggedright
6\strut
\end{minipage} & \begin{minipage}[t]{0.14\columnwidth}\raggedright
9\strut
\end{minipage} & \begin{minipage}[t]{0.18\columnwidth}\raggedright
4\strut
\end{minipage} & \begin{minipage}[t]{0.19\columnwidth}\raggedright
3.375\strut
\end{minipage}\tabularnewline
\begin{minipage}[t]{0.05\columnwidth}\raggedright
A\strut
\end{minipage} & \begin{minipage}[t]{0.14\columnwidth}\raggedright
1\strut
\end{minipage} & \begin{minipage}[t]{0.14\columnwidth}\raggedright
4\strut
\end{minipage} & \begin{minipage}[t]{0.14\columnwidth}\raggedright
7\strut
\end{minipage} & \begin{minipage}[t]{0.18\columnwidth}\raggedright
4\strut
\end{minipage} & \begin{minipage}[t]{0.19\columnwidth}\raggedright
0.125\strut
\end{minipage}\tabularnewline
\bottomrule
\end{longtable}

Result of aggregating is automatically ungrouped, you can skip this step
by stetting \texttt{:ungroup} option to \texttt{false}.

\begin{Shaded}
\begin{Highlighting}[]
\NormalTok{(}\KeywordTok{->}\NormalTok{ DS}
\NormalTok{    (api/group-by [}\AttributeTok{:V3}\NormalTok{])}
\NormalTok{    (api/aggregate [#(}\KeywordTok{take} \DecValTok{3}\NormalTok{ (}\VariableTok{%} \AttributeTok{:V2}\NormalTok{))}
\NormalTok{                    (}\KeywordTok{fn}\NormalTok{ [ds] \{}\AttributeTok{:sum-v1}\NormalTok{ (}\KeywordTok{reduce} \KeywordTok{+}\NormalTok{ (ds }\AttributeTok{:V1}\NormalTok{))}
                             \AttributeTok{:prod-v3}\NormalTok{ (}\KeywordTok{reduce} \KeywordTok{*}\NormalTok{ (ds }\AttributeTok{:V3}\NormalTok{))\})] \{}\AttributeTok{:default-column-name-prefix} \StringTok{"V2-value"}
                                                              \AttributeTok{:ungroup}\NormalTok{? }\VariableTok{false}\NormalTok{\}))}
\end{Highlighting}
\end{Shaded}

\_unnamed {[}3 3{]}:

\begin{longtable}[]{@{}lll@{}}
\toprule
:name & :group-id & :data\tabularnewline
\midrule
\endhead
\{:V3 1.0\} & 0 & \_unnamed {[}1 5{]}:\tabularnewline
\{:V3 0.5\} & 1 & \_unnamed {[}1 5{]}:\tabularnewline
\{:V3 1.5\} & 2 & \_unnamed {[}1 5{]}:\tabularnewline
\bottomrule
\end{longtable}

\hypertarget{column}\NormalTok{))}
\end{Highlighting}
\end{Shaded}

\_unnamed {[}1 3{]}:

\begin{longtable}[]{@{}lll@{}}
\toprule
:V1 & :V2 & :V3\tabularnewline
\midrule
\endhead
13 & 45 & 9.0\tabularnewline
\bottomrule
\end{longtable}

\begin{center}\rule{0.5\linewidth}{0.5pt}\end{center}

\begin{Shaded}
\begin{Highlighting}[]
\NormalTok{(api/aggregate-columns DS [}\AttributeTok{:V1} \AttributeTok{:V2} \AttributeTok{:V3}\NormalTok{] [#(}\KeywordTok{reduce} \KeywordTok{+} \VariableTok\NormalTok{)}
\NormalTok{                                         #(}\KeywordTok{reduce} \KeywordTok{*} \VariableTok{%}\NormalTok{)])}
\end{Highlighting}
\end{Shaded}

\_unnamed {[}1 3{]}:

\begin{longtable}[]{@{}lll@{}}
\toprule
:V1 & :V2 & :V3\tabularnewline
\midrule
\endhead
13 & 9 & 0.421875\tabularnewline
\bottomrule
\end{longtable}

\begin{center}\rule{0.5\linewidth}{0.5pt}\end{center}

\begin{Shaded}
\begin{Highlighting}[]
\NormalTok{(}\KeywordTok{->}\NormalTok{ DS}
\NormalTok{    (api/group-by [}\AttributeTok{:V4}\NormalTok{])}
\NormalTok{    (api/aggregate-columns [}\AttributeTok{:V1} \AttributeTok{:V2} \AttributeTok{:V3}\NormalTok{] #(}\KeywordTok{reduce} \KeywordTok{+} \VariableTok{%}\NormalTok{)))}
\end{Highlighting}
\end{Shaded}

\_unnamed {[}3 4{]}:

\begin{longtable}[]{@{}llll@{}}
\toprule
:V4 & :V1 & :V2 & :V3\tabularnewline
\midrule
\endhead
B & 5 & 15 & 3.0\tabularnewline
C & 4 & 18 & 4.5\tabularnewline
A & 4 & 12 & 1.5\tabularnewline
\bottomrule
\end{longtable}

\hypertarget{order}{%
\subsubsection{Order}\label{order}}

Ordering can be done by column(s) or any function operating on row.
Possible order can be:

\begin{itemize}
\tightlist
\item
  \texttt{:asc} for ascending order (default)
\item
  \texttt{:desc} for descending order
\item
  custom comparator
\end{itemize}

\texttt{:select-keys} limits row map provided to ordering functions.

\begin{center}\rule{0.5\linewidth}{0.5pt}\end{center}

Order by single column, ascending

\begin{Shaded}
\begin{Highlighting}[]
\NormalTok{(api/order-by DS }\AttributeTok{:V1}\NormalTok{)}
\end{Highlighting}
\end{Shaded}

\_unnamed {[}9 4{]}:

\begin{longtable}[]{@{}llll@{}}
\toprule
:V1 & :V2 & :V3 & :V4\tabularnewline
\midrule
\endhead
1 & 1 & 0.5 & A\tabularnewline
1 & 3 & 1.5 & C\tabularnewline
1 & 5 & 1.0 & B\tabularnewline
1 & 7 & 0.5 & A\tabularnewline
1 & 9 & 1.5 & C\tabularnewline
2 & 6 & 1.5 & C\tabularnewline
2 & 4 & 0.5 & A\tabularnewline
2 & 8 & 1.0 & B\tabularnewline
2 & 2 & 1.0 & B\tabularnewline
\bottomrule
\end{longtable}

\begin{center}\rule{0.5\linewidth}{0.5pt}\end{center}

Descending order

\begin{Shaded}
\begin{Highlighting}[]
\NormalTok{(api/order-by DS }\AttributeTok{:V1} \AttributeTok{:desc}\NormalTok{)}
\end{Highlighting}
\end{Shaded}

\_unnamed {[}9 4{]}:

\begin{longtable}[]{@{}llll@{}}
\toprule
:V1 & :V2 & :V3 & :V4\tabularnewline
\midrule
\endhead
2 & 2 & 1.0 & B\tabularnewline
2 & 4 & 0.5 & A\tabularnewline
2 & 6 & 1.5 & C\tabularnewline
2 & 8 & 1.0 & B\tabularnewline
1 & 5 & 1.0 & B\tabularnewline
1 & 3 & 1.5 & C\tabularnewline
1 & 7 & 0.5 & A\tabularnewline
1 & 1 & 0.5 & A\tabularnewline
1 & 9 & 1.5 & C\tabularnewline
\bottomrule
\end{longtable}

\begin{center}\rule{0.5\linewidth}{0.5pt}\end{center}

Order by two columns

\begin{Shaded}
\begin{Highlighting}[]
\NormalTok{(api/order-by DS [}\AttributeTok{:V1} \AttributeTok{:V2}\NormalTok{])}
\end{Highlighting}
\end{Shaded}

\_unnamed {[}9 4{]}:

\begin{longtable}[]{@{}llll@{}}
\toprule
:V1 & :V2 & :V3 & :V4\tabularnewline
\midrule
\endhead
1 & 1 & 0.5 & A\tabularnewline
1 & 3 & 1.5 & C\tabularnewline
1 & 5 & 1.0 & B\tabularnewline
1 & 7 & 0.5 & A\tabularnewline
1 & 9 & 1.5 & C\tabularnewline
2 & 2 & 1.0 & B\tabularnewline
2 & 4 & 0.5 & A\tabularnewline
2 & 6 & 1.5 & C\tabularnewline
2 & 8 & 1.0 & B\tabularnewline
\bottomrule
\end{longtable}

\begin{center}\rule{0.5\linewidth}{0.5pt}\end{center}

Use different orders for columns

\begin{Shaded}
\begin{Highlighting}[]
\NormalTok{(api/order-by DS [}\AttributeTok{:V1} \AttributeTok{:V2}\NormalTok{] [}\AttributeTok{:asc} \AttributeTok{:desc}\NormalTok{])}
\end{Highlighting}
\end{Shaded}

\_unnamed {[}9 4{]}:

\begin{longtable}[]{@{}llll@{}}
\toprule
:V1 & :V2 & :V3 & :V4\tabularnewline
\midrule
\endhead
1 & 9 & 1.5 & C\tabularnewline
1 & 7 & 0.5 & A\tabularnewline
1 & 5 & 1.0 & B\tabularnewline
1 & 3 & 1.5 & C\tabularnewline
1 & 1 & 0.5 & A\tabularnewline
2 & 8 & 1.0 & B\tabularnewline
2 & 6 & 1.5 & C\tabularnewline
2 & 4 & 0.5 & A\tabularnewline
2 & 2 & 1.0 & B\tabularnewline
\bottomrule
\end{longtable}

\begin{Shaded}
\begin{Highlighting}[]
\NormalTok{(api/order-by DS [}\AttributeTok{:V1} \AttributeTok{:V2}\NormalTok{] [}\AttributeTok{:desc} \AttributeTok{:desc}\NormalTok{])}
\end{Highlighting}
\end{Shaded}

\_unnamed {[}9 4{]}:

\begin{longtable}[]{@{}llll@{}}
\toprule
:V1 & :V2 & :V3 & :V4\tabularnewline
\midrule
\endhead
2 & 8 & 1.0 & B\tabularnewline
2 & 6 & 1.5 & C\tabularnewline
2 & 4 & 0.5 & A\tabularnewline
2 & 2 & 1.0 & B\tabularnewline
1 & 9 & 1.5 & C\tabularnewline
1 & 7 & 0.5 & A\tabularnewline
1 & 5 & 1.0 & B\tabularnewline
1 & 3 & 1.5 & C\tabularnewline
1 & 1 & 0.5 & A\tabularnewline
\bottomrule
\end{longtable}

\begin{Shaded}
\begin{Highlighting}[]
\NormalTok{(api/order-by DS [}\AttributeTok{:V1} \AttributeTok{:V3}\NormalTok{] [}\AttributeTok{:desc} \AttributeTok{:asc}\NormalTok{])}
\end{Highlighting}
\end{Shaded}

\_unnamed {[}9 4{]}:

\begin{longtable}[]{@{}llll@{}}
\toprule
:V1 & :V2 & :V3 & :V4\tabularnewline
\midrule
\endhead
2 & 4 & 0.5 & A\tabularnewline
2 & 2 & 1.0 & B\tabularnewline
2 & 8 & 1.0 & B\tabularnewline
2 & 6 & 1.5 & C\tabularnewline
1 & 1 & 0.5 & A\tabularnewline
1 & 7 & 0.5 & A\tabularnewline
1 & 5 & 1.0 & B\tabularnewline
1 & 3 & 1.5 & C\tabularnewline
1 & 9 & 1.5 & C\tabularnewline
\bottomrule
\end{longtable}

\begin{center}\rule{0.5\linewidth}{0.5pt}\end{center}

Custom function can be used to provied ordering key. Here order by
\texttt{:V4} descending, then by product of other columns ascending.

\begin{Shaded}
\begin{Highlighting}[]
\NormalTok{(api/order-by DS [}\AttributeTok{:V4}\NormalTok{ (}\KeywordTok{fn}\NormalTok{ [row] (}\KeywordTok{*}\NormalTok{ (}\AttributeTok{:V1}\NormalTok{ row)}
\NormalTok{                                  (}\AttributeTok{:V2}\NormalTok{ row)}
\NormalTok{                                  (}\AttributeTok{:V3}\NormalTok{ row)))] [}\AttributeTok{:desc} \AttributeTok{:asc}\NormalTok{] \{}\AttributeTok{:select-keys}\NormalTok{ [}\AttributeTok{:V1} \AttributeTok{:V2} \AttributeTok{:V3}\NormalTok{]\})}
\end{Highlighting}
\end{Shaded}

\_unnamed {[}9 4{]}:

\begin{longtable}[]{@{}llll@{}}
\toprule
:V1 & :V2 & :V3 & :V4\tabularnewline
\midrule
\endhead
1 & 1 & 0.5 & A\tabularnewline
1 & 7 & 0.5 & A\tabularnewline
2 & 4 & 0.5 & A\tabularnewline
2 & 2 & 1.0 & B\tabularnewline
1 & 3 & 1.5 & C\tabularnewline
1 & 5 & 1.0 & B\tabularnewline
1 & 9 & 1.5 & C\tabularnewline
2 & 8 & 1.0 & B\tabularnewline
2 & 6 & 1.5 & C\tabularnewline
\bottomrule
\end{longtable}

\begin{center}\rule{0.5\linewidth}{0.5pt}\end{center}

Custom comparator also can be used in case objects are not comparable by
default. Let's define artificial one: if Euclidean distance is lower
than 2, compare along \texttt{z} else along \texttt{x} and \texttt{y}.
We use first three columns for that.

\begin{Shaded}
\begin{Highlighting}[]
\NormalTok{(}\BuiltInTok{defn}\FunctionTok{ dist}
\NormalTok{  [v1 v2]}
\NormalTok{  (}\KeywordTok{->>}\NormalTok{ v2}
\NormalTok{       (}\KeywordTok{map} \KeywordTok{-}\NormalTok{ v1)}
\NormalTok{       (}\KeywordTok{map}\NormalTok{ #(}\KeywordTok{*} \VariableTok\NormalTok{))}
\NormalTok{       (}\KeywordTok{reduce} \KeywordTok{+}\NormalTok{)}
\NormalTok{       (Math/sqrt)))}
\end{Highlighting}
\end{Shaded}

\begin{verbatim}
#'user/dist
\end{verbatim}

\begin{Shaded}
\begin{Highlighting}[]
\NormalTok{(api/order-by DS [}\AttributeTok{:V1} \AttributeTok{:V2} \AttributeTok{:V3}\NormalTok{] (}\KeywordTok{fn}\NormalTok{ [[x1 y1 z1 }\AttributeTok{:as}\NormalTok{ v1] [x2 y2 z2 }\AttributeTok{:as}\NormalTok{ v2]]}
\NormalTok{                                 (}\KeywordTok{let}\NormalTok{ [d (dist v1 v2)]}
\NormalTok{                                   (}\KeywordTok{if}\NormalTok{ (}\KeywordTok{<}\NormalTok{ d }\FloatTok{2.0}\NormalTok{)}
\NormalTok{                                     (}\KeywordTok{compare}\NormalTok{ z1 z2)}
\NormalTok{                                     (}\KeywordTok{compare}\NormalTok{ [x1 y1] [x2 y2])))))}
\end{Highlighting}
\end{Shaded}

\_unnamed {[}9 4{]}:

\begin{longtable}[]{@{}llll@{}}
\toprule
:V1 & :V2 & :V3 & :V4\tabularnewline
\midrule
\endhead
1 & 1 & 0.5 & A\tabularnewline
1 & 5 & 1.0 & B\tabularnewline
1 & 7 & 0.5 & A\tabularnewline
1 & 9 & 1.5 & C\tabularnewline
2 & 2 & 1.0 & B\tabularnewline
2 & 4 & 0.5 & A\tabularnewline
1 & 3 & 1.5 & C\tabularnewline
2 & 6 & 1.5 & C\tabularnewline
2 & 8 & 1.0 & B\tabularnewline
\bottomrule
\end{longtable}

\hypertarget{unique}{%
\subsubsection{Unique}\label{unique}}

Remove rows which contains the same data. By default \texttt{unique-by}
removes duplicates from whole dataset. You can also pass list of columns
or functions (similar as in \texttt{group-by}) to remove duplicates
limited by them. Default strategy is to keep the first row. More
strategies below.

\texttt{unique-by} works on groups

\begin{center}\rule{0.5\linewidth}{0.5pt}\end{center}

Remove duplicates from whole dataset

\begin{Shaded}
\begin{Highlighting}[]
\NormalTok{(api/unique-by DS)}
\end{Highlighting}
\end{Shaded}

\_unnamed {[}9 4{]}:

\begin{longtable}[]{@{}llll@{}}
\toprule
:V1 & :V2 & :V3 & :V4\tabularnewline
\midrule
\endhead
1 & 1 & 0.5 & A\tabularnewline
2 & 2 & 1.0 & B\tabularnewline
1 & 3 & 1.5 & C\tabularnewline
2 & 4 & 0.5 & A\tabularnewline
1 & 5 & 1.0 & B\tabularnewline
2 & 6 & 1.5 & C\tabularnewline
1 & 7 & 0.5 & A\tabularnewline
2 & 8 & 1.0 & B\tabularnewline
1 & 9 & 1.5 & C\tabularnewline
\bottomrule
\end{longtable}

\begin{center}\rule{0.5\linewidth}{0.5pt}\end{center}

Remove duplicates from each group selected by column.

\begin{Shaded}
\begin{Highlighting}[]
\NormalTok{(api/unique-by DS }\AttributeTok{:V1}\NormalTok{)}
\end{Highlighting}
\end{Shaded}

\_unnamed {[}2 4{]}:

\begin{longtable}[]{@{}llll@{}}
\toprule
:V1 & :V2 & :V3 & :V4\tabularnewline
\midrule
\endhead
1 & 1 & 0.5 & A\tabularnewline
2 & 2 & 1.0 & B\tabularnewline
\bottomrule
\end{longtable}

\begin{center}\rule{0.5\linewidth}{0.5pt}\end{center}

Pair of columns

\begin{Shaded}
\begin{Highlighting}[]
\NormalTok{(api/unique-by DS [}\AttributeTok{:V1} \AttributeTok{:V3}\NormalTok{])}
\end{Highlighting}
\end{Shaded}

\_unnamed {[}6 4{]}:

\begin{longtable}[]{@{}llll@{}}
\toprule
:V1 & :V2 & :V3 & :V4\tabularnewline
\midrule
\endhead
1 & 1 & 0.5 & A\tabularnewline
2 & 2 & 1.0 & B\tabularnewline
1 & 3 & 1.5 & C\tabularnewline
2 & 4 & 0.5 & A\tabularnewline
1 & 5 & 1.0 & B\tabularnewline
2 & 6 & 1.5 & C\tabularnewline
\bottomrule
\end{longtable}

\begin{center}\rule{0.5\linewidth}{0.5pt}\end{center}

Also function can be used, split dataset by modulo 3 on columns
\texttt{:V2}

\begin{Shaded}
\begin{Highlighting}[]
\NormalTok{(api/unique-by DS (}\KeywordTok{fn}\NormalTok{ [m] (}\KeywordTok{mod}\NormalTok{ (}\AttributeTok{:V2}\NormalTok{ m) }\DecValTok{3}\NormalTok{)))}
\end{Highlighting}
\end{Shaded}

\_unnamed {[}3 4{]}:

\begin{longtable}[]{@{}llll@{}}
\toprule
:V1 & :V2 & :V3 & :V4\tabularnewline
\midrule
\endhead
1 & 1 & 0.5 & A\tabularnewline
2 & 2 & 1.0 & B\tabularnewline
1 & 3 & 1.5 & C\tabularnewline
\bottomrule
\end{longtable}

\begin{center}\rule{0.5\linewidth}{0.5pt}\end{center}

The same can be achived with \texttt{group-by}

\begin{Shaded}
\begin{Highlighting}[]
\NormalTok{(}\KeywordTok{->}\NormalTok{ DS}
\NormalTok{    (api/group-by (}\KeywordTok{fn}\NormalTok{ [m] (}\KeywordTok{mod}\NormalTok{ (}\AttributeTok{:V2}\NormalTok{ m) }\DecValTok{3}\NormalTok{)))}
\NormalTok{    (api/first)}
\NormalTok{    (api/ungroup))}
\end{Highlighting}
\end{Shaded}

\_unnamed {[}3 4{]}:

\begin{longtable}[]{@{}llll@{}}
\toprule
:V1 & :V2 & :V3 & :V4\tabularnewline
\midrule
\endhead
1 & 3 & 1.5 & C\tabularnewline
1 & 1 & 0.5 & A\tabularnewline
2 & 2 & 1.0 & B\tabularnewline
\bottomrule
\end{longtable}

\begin{center}\rule{0.5\linewidth}{0.5pt}\end{center}

Grouped dataset

\begin{Shaded}
\begin{Highlighting}[]
\NormalTok{(}\KeywordTok{->}\NormalTok{ DS}
\NormalTok{    (api/group-by }\AttributeTok{:V4}\NormalTok{)}
\NormalTok{    (api/unique-by }\AttributeTok{:V1}\NormalTok{)}
\NormalTok{    (api/ungroup))}
\end{Highlighting}
\end{Shaded}

\_unnamed {[}6 4{]}:

\begin{longtable}[]{@{}llll@{}}
\toprule
:V1 & :V2 & :V3 & :V4\tabularnewline
\midrule
\endhead
1 & 1 & 0.5 & A\tabularnewline
2 & 4 & 0.5 & A\tabularnewline
2 & 2 & 1.0 & B\tabularnewline
1 & 5 & 1.0 & B\tabularnewline
1 & 3 & 1.5 & C\tabularnewline
2 & 6 & 1.5 & C\tabularnewline
\bottomrule
\end{longtable}

\hypertarget{strategies}{%
\paragraph{Strategies}\label{strategies}}

There are 4 strategies defined:

\begin{itemize}
\tightlist
\item
  \texttt{:first} - select first row (default)
\item
  \texttt{:last} - select last row
\item
  \texttt{:random} - select random row
\item
  any function - apply function to a columns which are subject of
  uniqueness
\end{itemize}

\begin{center}\rule{0.5\linewidth}{0.5pt}\end{center}

Last

\begin{Shaded}
\begin{Highlighting}[]
\NormalTok{(api/unique-by DS }\AttributeTok{:V1}\NormalTok{ \{}\AttributeTok{:strategy} \AttributeTok{:last}\NormalTok{\})}
\end{Highlighting}
\end{Shaded}

\_unnamed {[}2 4{]}:

\begin{longtable}[]{@{}llll@{}}
\toprule
:V1 & :V2 & :V3 & :V4\tabularnewline
\midrule
\endhead
2 & 8 & 1.0 & B\tabularnewline
1 & 9 & 1.5 & C\tabularnewline
\bottomrule
\end{longtable}

\begin{center}\rule{0.5\linewidth}{0.5pt}\end{center}

Random

\begin{Shaded}
\begin{Highlighting}[]
\NormalTok{(api/unique-by DS }\AttributeTok{:V1}\NormalTok{ \{}\AttributeTok{:strategy} \AttributeTok{:random}\NormalTok{\})}
\end{Highlighting}
\end{Shaded}

\_unnamed {[}2 4{]}:

\begin{longtable}[]{@{}llll@{}}
\toprule
:V1 & :V2 & :V3 & :V4\tabularnewline
\midrule
\endhead
1 & 7 & 0.5 & A\tabularnewline
2 & 8 & 1.0 & B\tabularnewline
\bottomrule
\end{longtable}

\begin{center}\rule{0.5\linewidth}{0.5pt}\end{center}

Pack columns into vector

\begin{Shaded}
\begin{Highlighting}[]
\NormalTok{(api/unique-by DS }\AttributeTok{:V4}\NormalTok{ \{}\AttributeTok{:strategy} \KeywordTok{vec}\NormalTok{\})}
\end{Highlighting}
\end{Shaded}

\_unnamed {[}3 3{]}:

\begin{longtable}[]{@{}lll@{}}
\toprule
:V1 & :V2 & :V3\tabularnewline
\midrule
\endhead
{[}2 1 2{]} & {[}2 5 8{]} & {[}1.0 1.0 1.0{]}\tabularnewline
{[}1 2 1{]} & {[}3 6 9{]} & {[}1.5 1.5 1.5{]}\tabularnewline
{[}1 2 1{]} & {[}1 4 7{]} & {[}0.5 0.5 0.5{]}\tabularnewline
\bottomrule
\end{longtable}

\begin{center}\rule{0.5\linewidth}{0.5pt}\end{center}

Sum columns

\begin{Shaded}
\begin{Highlighting}[]
\NormalTok{(api/unique-by DS }\AttributeTok{:V4}\NormalTok{ \{}\AttributeTok{:strategy}\NormalTok{ (}\KeywordTok{partial} \KeywordTok{reduce} \KeywordTok{+}\NormalTok{)\})}
\end{Highlighting}
\end{Shaded}

\_unnamed {[}3 3{]}:

\begin{longtable}[]{@{}lll@{}}
\toprule
:V1 & :V2 & :V3\tabularnewline
\midrule
\endhead
5 & 15 & 3.0\tabularnewline
4 & 18 & 4.5\tabularnewline
4 & 12 & 1.5\tabularnewline
\bottomrule
\end{longtable}

\begin{center}\rule{0.5\linewidth}{0.5pt}\end{center}

Group by function and apply functions

\begin{Shaded}
\begin{Highlighting}[]
\NormalTok{(api/unique-by DS (}\KeywordTok{fn}\NormalTok{ [m] (}\KeywordTok{mod}\NormalTok{ (}\AttributeTok{:V2}\NormalTok{ m) }\DecValTok{3}\NormalTok{)) \{}\AttributeTok{:strategy} \KeywordTok{vec}\NormalTok{\})}
\end{Highlighting}
\end{Shaded}

\_unnamed {[}3 4{]}:

\begin{longtable}[]{@{}llll@{}}
\toprule
:V1 & :V2 & :V3 & :V4\tabularnewline
\midrule
\endhead
{[}1 2 1{]} & {[}3 6 9{]} & {[}1.5 1.5 1.5{]} & {[}``C'' ``C''
``C''{]}\tabularnewline
{[}1 2 1{]} & {[}1 4 7{]} & {[}0.5 0.5 0.5{]} & {[}``A'' ``A''
``A''{]}\tabularnewline
{[}2 1 2{]} & {[}2 5 8{]} & {[}1.0 1.0 1.0{]} & {[}``B'' ``B''
``B''{]}\tabularnewline
\bottomrule
\end{longtable}

\begin{center}\rule{0.5\linewidth}{0.5pt}\end{center}

Grouped dataset

\begin{Shaded}
\begin{Highlighting}[]
\NormalTok{(}\KeywordTok{->}\NormalTok{ DS}
\NormalTok{    (api/group-by }\AttributeTok{:V1}\NormalTok{)}
\NormalTok{    (api/unique-by (}\KeywordTok{fn}\NormalTok{ [m] (}\KeywordTok{mod}\NormalTok{ (}\AttributeTok{:V2}\NormalTok{ m) }\DecValTok{3}\NormalTok{)) \{}\AttributeTok{:strategy} \KeywordTok{vec}\NormalTok{\})}
\NormalTok{    (api/ungroup \{}\AttributeTok{:add-group-as-column} \AttributeTok{:from-V1}\NormalTok{\}))}
\end{Highlighting}
\end{Shaded}

\_unnamed {[}6 5{]}:

\begin{longtable}[]{@{}lllll@{}}
\toprule
:from-V1 & :V1 & :V2 & :V3 & :V4\tabularnewline
\midrule
\endhead
1 & {[}1 1{]} & {[}3 9{]} & {[}1.5 1.5{]} & {[}``C''
``C''{]}\tabularnewline
1 & {[}1 1{]} & {[}1 7{]} & {[}0.5 0.5{]} & {[}``A''
``A''{]}\tabularnewline
1 & {[}1{]} & {[}5{]} & {[}1.0{]} & {[}``B''{]}\tabularnewline
2 & {[}2{]} & {[}6{]} & {[}1.5{]} & {[}``C''{]}\tabularnewline
2 & {[}2{]} & {[}4{]} & {[}0.5{]} & {[}``A''{]}\tabularnewline
2 & {[}2 2{]} & {[}2 8{]} & {[}1.0 1.0{]} & {[}``B''
``B''{]}\tabularnewline
\bottomrule
\end{longtable}

\hypertarget{missing}{%
\subsubsection{Missing}\label{missing}}

When dataset contains missing values you can select or drop rows with
missing values or replace them using some strategy.

\texttt{column-selector} can be used to limit considered columns

Let's define dataset which contains missing values

\begin{Shaded}
\begin{Highlighting}[]
\NormalTok{(}\BuiltInTok{def}\FunctionTok{ DSm }\NormalTok{(api/dataset \{}\AttributeTok{:V1}\NormalTok{ (}\KeywordTok{take} \DecValTok{9}\NormalTok{ (}\KeywordTok{cycle}\NormalTok{ [}\DecValTok{1} \DecValTok{2} \VariableTok{nil}\NormalTok{]))}
                       \AttributeTok{:V2}\NormalTok{ (}\KeywordTok{range} \DecValTok{1} \DecValTok{10}\NormalTok{)}
                       \AttributeTok{:V3}\NormalTok{ (}\KeywordTok{take} \DecValTok{9}\NormalTok{ (}\KeywordTok{cycle}\NormalTok{ [}\FloatTok{0.5} \FloatTok{1.0} \VariableTok{nil} \FloatTok{1.5}\NormalTok{]))}
                       \AttributeTok{:V4}\NormalTok{ (}\KeywordTok{take} \DecValTok{9}\NormalTok{ (}\KeywordTok{cycle}\NormalTok{ [}\StringTok{"A"} \StringTok{"B"} \StringTok{"C"}\NormalTok{]))\}))}
\end{Highlighting}
\end{Shaded}

\begin{Shaded}
\begin{Highlighting}[]
\NormalTok{DSm}
\end{Highlighting}
\end{Shaded}

\_unnamed {[}9 4{]}:

\begin{longtable}[]{@{}llll@{}}
\toprule
:V1 & :V2 & :V3 & :V4\tabularnewline
\midrule
\endhead
1 & 1 & 0.5 & A\tabularnewline
2 & 2 & 1.0 & B\tabularnewline
& 3 & & C\tabularnewline
1 & 4 & 1.5 & A\tabularnewline
2 & 5 & 0.5 & B\tabularnewline
& 6 & 1.0 & C\tabularnewline
1 & 7 & & A\tabularnewline
2 & 8 & 1.5 & B\tabularnewline
& 9 & 0.5 & C\tabularnewline
\bottomrule
\end{longtable}

\hypertarget{select-2}{%
\paragraph{Select}\label{select-2}}

Select rows with missing values

\begin{Shaded}
\begin{Highlighting}[]
\NormalTok{(api/select-missing DSm)}
\end{Highlighting}
\end{Shaded}

\_unnamed {[}4 4{]}:

\begin{longtable}[]{@{}llll@{}}
\toprule
:V1 & :V2 & :V3 & :V4\tabularnewline
\midrule
\endhead
& 3 & & C\tabularnewline
& 6 & 1.0 & C\tabularnewline
1 & 7 & & A\tabularnewline
& 9 & 0.5 & C\tabularnewline
\bottomrule
\end{longtable}

\begin{center}\rule{0.5\linewidth}{0.5pt}\end{center}

Select rows with missing values in \texttt{:V1}

\begin{Shaded}
\begin{Highlighting}[]
\NormalTok{(api/select-missing DSm }\AttributeTok{:V1}\NormalTok{)}
\end{Highlighting}
\end{Shaded}

\_unnamed {[}3 4{]}:

\begin{longtable}[]{@{}llll@{}}
\toprule
:V1 & :V2 & :V3 & :V4\tabularnewline
\midrule
\endhead
& 3 & & C\tabularnewline
& 6 & 1.0 & C\tabularnewline
& 9 & 0.5 & C\tabularnewline
\bottomrule
\end{longtable}

\begin{center}\rule{0.5\linewidth}{0.5pt}\end{center}

The same with grouped dataset

\begin{Shaded}
\begin{Highlighting}[]
\NormalTok{(}\KeywordTok{->}\NormalTok{ DSm}
\NormalTok{    (api/group-by }\AttributeTok{:V4}\NormalTok{)}
\NormalTok{    (api/select-missing }\AttributeTok{:V3}\NormalTok{)}
\NormalTok{    (api/ungroup))}
\end{Highlighting}
\end{Shaded}

\_unnamed {[}2 4{]}:

\begin{longtable}[]{@{}llll@{}}
\toprule
:V1 & :V2 & :V3 & :V4\tabularnewline
\midrule
\endhead
1 & 7 & & A\tabularnewline
& 3 & & C\tabularnewline
\bottomrule
\end{longtable}

\hypertarget{drop-2}{%
\paragraph{Drop}\label{drop-2}}

Drop rows with missing values

\begin{Shaded}
\begin{Highlighting}[]
\NormalTok{(api/drop-missing DSm)}
\end{Highlighting}
\end{Shaded}

\_unnamed {[}5 4{]}:

\begin{longtable}[]{@{}llll@{}}
\toprule
:V1 & :V2 & :V3 & :V4\tabularnewline
\midrule
\endhead
1 & 1 & 0.5 & A\tabularnewline
2 & 2 & 1.0 & B\tabularnewline
1 & 4 & 1.5 & A\tabularnewline
2 & 5 & 0.5 & B\tabularnewline
2 & 8 & 1.5 & B\tabularnewline
\bottomrule
\end{longtable}

\begin{center}\rule{0.5\linewidth}{0.5pt}\end{center}

Drop rows with missing values in \texttt{:V1}

\begin{Shaded}
\begin{Highlighting}[]
\NormalTok{(api/drop-missing DSm }\AttributeTok{:V1}\NormalTok{)}
\end{Highlighting}
\end{Shaded}

\_unnamed {[}6 4{]}:

\begin{longtable}[]{@{}llll@{}}
\toprule
:V1 & :V2 & :V3 & :V4\tabularnewline
\midrule
\endhead
1 & 1 & 0.5 & A\tabularnewline
2 & 2 & 1.0 & B\tabularnewline
1 & 4 & 1.5 & A\tabularnewline
2 & 5 & 0.5 & B\tabularnewline
1 & 7 & & A\tabularnewline
2 & 8 & 1.5 & B\tabularnewline
\bottomrule
\end{longtable}

\begin{center}\rule{0.5\linewidth}{0.5pt}\end{center}

The same with grouped dataset

\begin{Shaded}
\begin{Highlighting}[]
\NormalTok{(}\KeywordTok{->}\NormalTok{ DSm}
\NormalTok{    (api/group-by }\AttributeTok{:V4}\NormalTok{)}
\NormalTok{    (api/drop-missing }\AttributeTok{:V1}\NormalTok{)}
\NormalTok{    (api/ungroup))}
\end{Highlighting}
\end{Shaded}

\_unnamed {[}6 4{]}:

\begin{longtable}[]{@{}llll@{}}
\toprule
:V1 & :V2 & :V3 & :V4\tabularnewline
\midrule
\endhead
1 & 1 & 0.5 & A\tabularnewline
1 & 4 & 1.5 & A\tabularnewline
1 & 7 & & A\tabularnewline
2 & 2 & 1.0 & B\tabularnewline
2 & 5 & 0.5 & B\tabularnewline
2 & 8 & 1.5 & B\tabularnewline
\bottomrule
\end{longtable}

\hypertarget{replace}{%
\paragraph{Replace}\label{replace}}

Missing values can be replaced using several strategies.
\texttt{replace-missing} accepts:

\begin{itemize}
\tightlist
\item
  dataset
\item
  column selector, default: \texttt{:all}
\item
  strategy, default: \texttt{:mid}
\item
  value (optional)

  \begin{itemize}
  \tightlist
  \item
    single value
  \item
    sequence of values (cycled)
  \item
    function, applied on column(s) with stripped missings
  \end{itemize}
\end{itemize}

Strategies are:

\begin{itemize}
\tightlist
\item
  \texttt{:value} - replace with given value
\item
  \texttt{:up} - copy values up and then down for missing values at the
  end
\item
  \texttt{:down} - copy values down and then up for missing values at
  the beginning
\item
  \texttt{:mid} - copy values around known values
\item
  \texttt{:lerp} - trying to lineary approximate values, works for
  numbers and datetime, otherwise applies \texttt{:mid}
\end{itemize}

Let's define special dataset here:

\begin{Shaded}
\begin{Highlighting}[]
\NormalTok{(}\BuiltInTok{def}\FunctionTok{ DSm2 }\NormalTok{(api/dataset \{}\AttributeTok{:a}\NormalTok{ [}\VariableTok{nil} \VariableTok{nil} \VariableTok{nil} \FloatTok{1.0} \DecValTok{2}  \VariableTok{nil} \VariableTok{nil} \VariableTok{nil} \VariableTok{nil}  \VariableTok{nil} \DecValTok{4}   \VariableTok{nil}  \DecValTok{11} \VariableTok{nil} \VariableTok{nil}\NormalTok{]}
                        \AttributeTok{:b}\NormalTok{ [}\DecValTok{2}   \DecValTok{2}   \DecValTok{2} \VariableTok{nil} \VariableTok{nil} \VariableTok{nil} \VariableTok{nil} \VariableTok{nil} \VariableTok{nil} \DecValTok{13}   \VariableTok{nil}   \DecValTok{3}  \DecValTok{4}  \DecValTok{5} \DecValTok{5}\NormalTok{]\}))}
\end{Highlighting}
\end{Shaded}

\begin{Shaded}
\begin{Highlighting}[]
\NormalTok{DSm2}
\end{Highlighting}
\end{Shaded}

\_unnamed {[}15 2{]}:

\begin{longtable}[]{@{}ll@{}}
\toprule
:a & :b\tabularnewline
\midrule
\endhead
& 2\tabularnewline
& 2\tabularnewline
& 2\tabularnewline
1.0 &\tabularnewline
2.0 &\tabularnewline
&\tabularnewline
&\tabularnewline
&\tabularnewline
&\tabularnewline
& 13\tabularnewline
4.0 &\tabularnewline
& 3\tabularnewline
11.0 & 4\tabularnewline
& 5\tabularnewline
& 5\tabularnewline
\bottomrule
\end{longtable}

\begin{center}\rule{0.5\linewidth}{0.5pt}\end{center}

Replace missing with default strategy for all columns

\begin{Shaded}
\begin{Highlighting}[]
\NormalTok{(api/replace-missing DSm2)}
\end{Highlighting}
\end{Shaded}

\_unnamed {[}15 2{]}:

\begin{longtable}[]{@{}ll@{}}
\toprule
:a & :b\tabularnewline
\midrule
\endhead
1.0 & 2\tabularnewline
1.0 & 2\tabularnewline
1.0 & 2\tabularnewline
1.0 & 2\tabularnewline
2.0 & 2\tabularnewline
2.0 & 2\tabularnewline
2.0 & 13\tabularnewline
2.0 & 13\tabularnewline
4.0 & 13\tabularnewline
4.0 & 13\tabularnewline
4.0 & 13\tabularnewline
4.0 & 3\tabularnewline
11.0 & 4\tabularnewline
11.0 & 5\tabularnewline
11.0 & 5\tabularnewline
\bottomrule
\end{longtable}

\begin{center}\rule{0.5\linewidth}{0.5pt}\end{center}

Replace missing with single value in whole dataset

\begin{Shaded}
\begin{Highlighting}[]
\NormalTok{(api/replace-missing DSm2 }\AttributeTok{:all} \AttributeTok{:value} \DecValTok{999}\NormalTok{)}
\end{Highlighting}
\end{Shaded}

\_unnamed {[}15 2{]}:

\begin{longtable}[]{@{}ll@{}}
\toprule
:a & :b\tabularnewline
\midrule
\endhead
999.0 & 2\tabularnewline
999.0 & 2\tabularnewline
999.0 & 2\tabularnewline
1.0 & 999\tabularnewline
2.0 & 999\tabularnewline
999.0 & 999\tabularnewline
999.0 & 999\tabularnewline
999.0 & 999\tabularnewline
999.0 & 999\tabularnewline
999.0 & 13\tabularnewline
4.0 & 999\tabularnewline
999.0 & 3\tabularnewline
11.0 & 4\tabularnewline
999.0 & 5\tabularnewline
999.0 & 5\tabularnewline
\bottomrule
\end{longtable}

\begin{center}\rule{0.5\linewidth}{0.5pt}\end{center}

Replace missing with single value in \texttt{:a} column

\begin{Shaded}
\begin{Highlighting}[]
\NormalTok{(api/replace-missing DSm2 }\AttributeTok{:a} \AttributeTok{:value} \DecValTok{999}\NormalTok{)}
\end{Highlighting}
\end{Shaded}

\_unnamed {[}15 2{]}:

\begin{longtable}[]{@{}ll@{}}
\toprule
:a & :b\tabularnewline
\midrule
\endhead
999.0 & 2\tabularnewline
999.0 & 2\tabularnewline
999.0 & 2\tabularnewline
1.0 &\tabularnewline
2.0 &\tabularnewline
999.0 &\tabularnewline
999.0 &\tabularnewline
999.0 &\tabularnewline
999.0 &\tabularnewline
999.0 & 13\tabularnewline
4.0 &\tabularnewline
999.0 & 3\tabularnewline
11.0 & 4\tabularnewline
999.0 & 5\tabularnewline
999.0 & 5\tabularnewline
\bottomrule
\end{longtable}

\begin{center}\rule{0.5\linewidth}{0.5pt}\end{center}

Replace missing with sequence in \texttt{:a} column

\begin{Shaded}
\begin{Highlighting}[]
\NormalTok{(api/replace-missing DSm2 }\AttributeTok{:a} \AttributeTok{:value}\NormalTok{ [-}\DecValTok{999} \DecValTok{-998} \DecValTok{-997}\NormalTok{])}
\end{Highlighting}
\end{Shaded}

\_unnamed {[}15 2{]}:

\begin{longtable}[]{@{}ll@{}}
\toprule
:a & :b\tabularnewline
\midrule
\endhead
-999.0 & 2\tabularnewline
-998.0 & 2\tabularnewline
-997.0 & 2\tabularnewline
1.0 &\tabularnewline
2.0 &\tabularnewline
-999.0 &\tabularnewline
-998.0 &\tabularnewline
-997.0 &\tabularnewline
-999.0 &\tabularnewline
-998.0 & 13\tabularnewline
4.0 &\tabularnewline
-997.0 & 3\tabularnewline
11.0 & 4\tabularnewline
-999.0 & 5\tabularnewline
-998.0 & 5\tabularnewline
\bottomrule
\end{longtable}

\begin{center}\rule{0.5\linewidth}{0.5pt}\end{center}

Replace missing with a function (mean)

\begin{Shaded}
\begin{Highlighting}[]
\NormalTok{(api/replace-missing DSm2 }\AttributeTok{:a} \AttributeTok{:value}\NormalTok{ tech.v2.datatype.functional/mean)}
\end{Highlighting}
\end{Shaded}

\_unnamed {[}15 2{]}:

\begin{longtable}[]{@{}ll@{}}
\toprule
:a & :b\tabularnewline
\midrule
\endhead
4.5 & 2\tabularnewline
4.5 & 2\tabularnewline
4.5 & 2\tabularnewline
1.0 &\tabularnewline
2.0 &\tabularnewline
4.5 &\tabularnewline
4.5 &\tabularnewline
4.5 &\tabularnewline
4.5 &\tabularnewline
4.5 & 13\tabularnewline
4.0 &\tabularnewline
4.5 & 3\tabularnewline
11.0 & 4\tabularnewline
4.5 & 5\tabularnewline
4.5 & 5\tabularnewline
\bottomrule
\end{longtable}

\begin{center}\rule{0.5\linewidth}{0.5pt}\end{center}

Using \texttt{:down} strategy, fills gaps with values from above. You
can see that if missings are at the beginning, the are filled with first
value

\begin{Shaded}
\begin{Highlighting}[]
\NormalTok{(api/replace-missing DSm2 [}\AttributeTok{:a} \AttributeTok{:b}\NormalTok{] }\AttributeTok{:down}\NormalTok{)}
\end{Highlighting}
\end{Shaded}

\_unnamed {[}15 2{]}:

\begin{longtable}[]{@{}ll@{}}
\toprule
:a & :b\tabularnewline
\midrule
\endhead
1.0 & 2\tabularnewline
1.0 & 2\tabularnewline
1.0 & 2\tabularnewline
1.0 & 2\tabularnewline
2.0 & 2\tabularnewline
2.0 & 2\tabularnewline
2.0 & 2\tabularnewline
2.0 & 2\tabularnewline
2.0 & 2\tabularnewline
2.0 & 13\tabularnewline
4.0 & 13\tabularnewline
4.0 & 3\tabularnewline
11.0 & 4\tabularnewline
11.0 & 5\tabularnewline
11.0 & 5\tabularnewline
\bottomrule
\end{longtable}

\begin{center}\rule{0.5\linewidth}{0.5pt}\end{center}

To fix above issue you can provide value

\begin{Shaded}
\begin{Highlighting}[]
\NormalTok{(api/replace-missing DSm2 [}\AttributeTok{:a} \AttributeTok{:b}\NormalTok{] }\AttributeTok{:down} \DecValTok{999}\NormalTok{)}
\end{Highlighting}
\end{Shaded}

\_unnamed {[}15 2{]}:

\begin{longtable}[]{@{}ll@{}}
\toprule
:a & :b\tabularnewline
\midrule
\endhead
999.0 & 2\tabularnewline
999.0 & 2\tabularnewline
999.0 & 2\tabularnewline
1.0 & 2\tabularnewline
2.0 & 2\tabularnewline
2.0 & 2\tabularnewline
2.0 & 2\tabularnewline
2.0 & 2\tabularnewline
2.0 & 2\tabularnewline
2.0 & 13\tabularnewline
4.0 & 13\tabularnewline
4.0 & 3\tabularnewline
11.0 & 4\tabularnewline
11.0 & 5\tabularnewline
11.0 & 5\tabularnewline
\bottomrule
\end{longtable}

\begin{center}\rule{0.5\linewidth}{0.5pt}\end{center}

The same applies for \texttt{:up} strategy which is opposite direction.

\begin{Shaded}
\begin{Highlighting}[]
\NormalTok{(api/replace-missing DSm2 [}\AttributeTok{:a} \AttributeTok{:b}\NormalTok{] }\AttributeTok{:up}\NormalTok{)}
\end{Highlighting}
\end{Shaded}

\_unnamed {[}15 2{]}:

\begin{longtable}[]{@{}ll@{}}
\toprule
:a & :b\tabularnewline
\midrule
\endhead
1.0 & 2\tabularnewline
1.0 & 2\tabularnewline
1.0 & 2\tabularnewline
1.0 & 13\tabularnewline
2.0 & 13\tabularnewline
4.0 & 13\tabularnewline
4.0 & 13\tabularnewline
4.0 & 13\tabularnewline
4.0 & 13\tabularnewline
4.0 & 13\tabularnewline
4.0 & 3\tabularnewline
11.0 & 3\tabularnewline
11.0 & 4\tabularnewline
11.0 & 5\tabularnewline
11.0 & 5\tabularnewline
\bottomrule
\end{longtable}

\begin{center}\rule{0.5\linewidth}{0.5pt}\end{center}

We can use a function which is applied after applying \texttt{:up} or
\texttt{:down}

\begin{Shaded}
\begin{Highlighting}[]
\NormalTok{(api/replace-missing DSm2 [}\AttributeTok{:a} \AttributeTok{:b}\NormalTok{] }\AttributeTok{:down}\NormalTok{ tech.v2.datatype.functional/mean)}
\end{Highlighting}
\end{Shaded}

\_unnamed {[}15 2{]}:

\begin{longtable}[]{@{}ll@{}}
\toprule
:a & :b\tabularnewline
\midrule
\endhead
4.5 & 2\tabularnewline
4.5 & 2\tabularnewline
4.5 & 2\tabularnewline
1.0 & 2\tabularnewline
2.0 & 2\tabularnewline
2.0 & 2\tabularnewline
2.0 & 2\tabularnewline
2.0 & 2\tabularnewline
2.0 & 2\tabularnewline
2.0 & 13\tabularnewline
4.0 & 13\tabularnewline
4.0 & 3\tabularnewline
11.0 & 4\tabularnewline
11.0 & 5\tabularnewline
11.0 & 5\tabularnewline
\bottomrule
\end{longtable}

\begin{center}\rule{0.5\linewidth}{0.5pt}\end{center}

Lerp tries to apply linear interpolation of the values

\begin{Shaded}
\begin{Highlighting}[]
\NormalTok{(api/replace-missing DSm2 [}\AttributeTok{:a} \AttributeTok{:b}\NormalTok{] }\AttributeTok{:lerp}\NormalTok{)}
\end{Highlighting}
\end{Shaded}

\_unnamed {[}15 2{]}:

\begin{longtable}[]{@{}ll@{}}
\toprule
:a & :b\tabularnewline
\midrule
\endhead
1.00000000 & 2\tabularnewline
1.00000000 & 2\tabularnewline
1.00000000 & 2\tabularnewline
1.00000000 & 4\tabularnewline
2.00000000 & 5\tabularnewline
2.33333333 & 7\tabularnewline
2.66666667 & 8\tabularnewline
3.00000000 & 10\tabularnewline
3.33333333 & 11\tabularnewline
3.66666667 & 13\tabularnewline
4.00000000 & 8\tabularnewline
7.50000000 & 3\tabularnewline
11.00000000 & 4\tabularnewline
11.00000000 & 5\tabularnewline
11.00000000 & 5\tabularnewline
\bottomrule
\end{longtable}

\begin{center}\rule{0.5\linewidth}{0.5pt}\end{center}

Lerp works also on dates

\begin{Shaded}
\begin{Highlighting}[]
\NormalTok{(}\KeywordTok{comment}\NormalTok{ (}\KeywordTok{->}\NormalTok{ (api/dataset \{}\AttributeTok{:dt}\NormalTok{ [(java.}\KeywordTok{time}\NormalTok{.LocalDateTime/of }\DecValTok{2020} \DecValTok{1} \DecValTok{1} \DecValTok{11} \DecValTok{22} \DecValTok{33}\NormalTok{)}
                                \VariableTok{nil} \VariableTok{nil} \VariableTok{nil} \VariableTok{nil} \VariableTok{nil} \VariableTok{nil} \VariableTok{nil}
\NormalTok{                                (java.}\KeywordTok{time}\NormalTok{.LocalDateTime/of }\DecValTok{2020} \DecValTok{10} \DecValTok{1} \DecValTok{1} \DecValTok{1} \DecValTok{1}\NormalTok{)]\})}
\NormalTok{             (api/replace-missing }\AttributeTok{:lerp}\NormalTok{)))}
\end{Highlighting}
\end{Shaded}

nil

\hypertarget{inject}{%
\paragraph{Inject}\label{inject}}

When your column contains not continuous data range you can fill up with
lacking values. Arguments:

\begin{itemize}
\tightlist
\item
  dataset
\item
  column name
\item
  expected step (\texttt{max-span}, milliseconds in case of datetime
  column)
\item
  (optional) \texttt{missing-strategy} - how to replace missing, default
  \texttt{:down} (set to \texttt{nil} if none)
\item
  (optional) \texttt{missing-value} - optional value for replace missing
\end{itemize}

\begin{center}\rule{0.5\linewidth}{0.5pt}\end{center}

\begin{Shaded}
\begin{Highlighting}[]
\NormalTok{(}\KeywordTok{->}\NormalTok{ (api/dataset \{}\AttributeTok{:a}\NormalTok{ [}\DecValTok{1} \DecValTok{2} \DecValTok{9}\NormalTok{]}
                  \AttributeTok{:b}\NormalTok{ [}\AttributeTok{:a} \AttributeTok{:b} \AttributeTok{:c}\NormalTok{]\})}
\NormalTok{    (api/fill-range-replace }\AttributeTok{:a} \DecValTok{1}\NormalTok{))}
\end{Highlighting}
\end{Shaded}

\_unnamed {[}9 2{]}:

\begin{longtable}[]{@{}ll@{}}
\toprule
:a & :b\tabularnewline
\midrule
\endhead
1.0 & :a\tabularnewline
2.0 & :b\tabularnewline
3.0 & :b\tabularnewline
4.0 & :b\tabularnewline
5.0 & :b\tabularnewline
6.0 & :b\tabularnewline
7.0 & :b\tabularnewline
8.0 & :b\tabularnewline
9.0 & :c\tabularnewline
\bottomrule
\end{longtable}

\hypertarget{joinseparate-columns}{%
\subsubsection{Join/Separate Columns}\label{joinseparate-columns}}

Joining or separating columns are operations which can help to tidy
messy dataset.

\begin{itemize}
\tightlist
\item
  \texttt{join-columns} joins content of the columns (as string
  concatenation or other structure) and stores it in new column
\item
  \texttt{separate-column} splits content of the columns into set of new
  columns
\end{itemize}

\hypertarget{join}{%
\paragraph{Join}\label{join}}

\texttt{join-columns} accepts:

\begin{itemize}
\tightlist
\item
  dataset
\item
  column selector (as in \texttt{select-columns})
\item
  options

  \begin{itemize}
  \tightlist
  \item
    \texttt{:separator} (default \texttt{"-"})
  \item
    \texttt{:drop-columns?} - whether to drop source columns or not
    (default \texttt{true})
  \item
    \texttt{:result-type}

    \begin{itemize}
    \tightlist
    \item
      \texttt{:map} - packs data into map
    \item
      \texttt{:seq} - packs data into sequence
    \item
      \texttt{:string} - join strings with separator (default)
    \item
      or custom function which gets row as a vector
    \end{itemize}
  \item
    \texttt{:missing-subst} - substitution for missing value
  \end{itemize}
\end{itemize}

\begin{center}\rule{0.5\linewidth}{0.5pt}\end{center}

Default usage. Create \texttt{:joined} column out of other columns.

\begin{Shaded}
\begin{Highlighting}[]
\NormalTok{(api/join-columns DSm }\AttributeTok{:joined}\NormalTok{ [}\AttributeTok{:V1} \AttributeTok{:V2} \AttributeTok{:V4}\NormalTok{])}
\end{Highlighting}
\end{Shaded}

\_unnamed {[}9 2{]}:

\begin{longtable}[]{@{}ll@{}}
\toprule
:V3 & :joined\tabularnewline
\midrule
\endhead
0.5 & 1-1-A\tabularnewline
1.0 & 2-2-B\tabularnewline
& 3-C\tabularnewline
1.5 & 1-4-A\tabularnewline
0.5 & 2-5-B\tabularnewline
1.0 & 6-C\tabularnewline
& 1-7-A\tabularnewline
1.5 & 2-8-B\tabularnewline
0.5 & 9-C\tabularnewline
\bottomrule
\end{longtable}

\begin{center}\rule{0.5\linewidth}{0.5pt}\end{center}

Without dropping source columns.

\begin{Shaded}
\begin{Highlighting}[]
\NormalTok{(api/join-columns DSm }\AttributeTok{:joined}\NormalTok{ [}\AttributeTok{:V1} \AttributeTok{:V2} \AttributeTok{:V4}\NormalTok{] \{}\AttributeTok{:drop-columns}\NormalTok{? }\VariableTok{false}\NormalTok{\})}
\end{Highlighting}
\end{Shaded}

\_unnamed {[}9 5{]}:

\begin{longtable}[]{@{}lllll@{}}
\toprule
:V1 & :V2 & :V3 & :V4 & :joined\tabularnewline
\midrule
\endhead
1 & 1 & 0.5 & A & 1-1-A\tabularnewline
2 & 2 & 1.0 & B & 2-2-B\tabularnewline
& 3 & & C & 3-C\tabularnewline
1 & 4 & 1.5 & A & 1-4-A\tabularnewline
2 & 5 & 0.5 & B & 2-5-B\tabularnewline
& 6 & 1.0 & C & 6-C\tabularnewline
1 & 7 & & A & 1-7-A\tabularnewline
2 & 8 & 1.5 & B & 2-8-B\tabularnewline
& 9 & 0.5 & C & 9-C\tabularnewline
\bottomrule
\end{longtable}

\begin{center}\rule{0.5\linewidth}{0.5pt}\end{center}

Let's replace missing value with ``NA'' string.

\begin{Shaded}
\begin{Highlighting}[]
\NormalTok{(api/join-columns DSm }\AttributeTok{:joined}\NormalTok{ [}\AttributeTok{:V1} \AttributeTok{:V2} \AttributeTok{:V4}\NormalTok{] \{}\AttributeTok{:missing-subst} \StringTok{"NA"}\NormalTok{\})}
\end{Highlighting}
\end{Shaded}

\_unnamed {[}9 2{]}:

\begin{longtable}[]{@{}ll@{}}
\toprule
:V3 & :joined\tabularnewline
\midrule
\endhead
0.5 & 1-1-A\tabularnewline
1.0 & 2-2-B\tabularnewline
& NA-3-C\tabularnewline
1.5 & 1-4-A\tabularnewline
0.5 & 2-5-B\tabularnewline
1.0 & NA-6-C\tabularnewline
& 1-7-A\tabularnewline
1.5 & 2-8-B\tabularnewline
0.5 & NA-9-C\tabularnewline
\bottomrule
\end{longtable}

\begin{center}\rule{0.5\linewidth}{0.5pt}\end{center}

We can use custom separator.

\begin{Shaded}
\begin{Highlighting}[]
\NormalTok{(api/join-columns DSm }\AttributeTok{:joined}\NormalTok{ [}\AttributeTok{:V1} \AttributeTok{:V2} \AttributeTok{:V4}\NormalTok{] \{}\AttributeTok{:separator} \StringTok{"/"}
                                             \AttributeTok{:missing-subst} \StringTok{"."}\NormalTok{\})}
\end{Highlighting}
\end{Shaded}

\_unnamed {[}9 2{]}:

\begin{longtable}[]{@{}ll@{}}
\toprule
:V3 & :joined\tabularnewline
\midrule
\endhead
0.5 & 1/1/A\tabularnewline
1.0 & 2/2/B\tabularnewline
& ./3/C\tabularnewline
1.5 & 1/4/A\tabularnewline
0.5 & 2/5/B\tabularnewline
1.0 & ./6/C\tabularnewline
& 1/7/A\tabularnewline
1.5 & 2/8/B\tabularnewline
0.5 & ./9/C\tabularnewline
\bottomrule
\end{longtable}

\begin{center}\rule{0.5\linewidth}{0.5pt}\end{center}

Or even sequence of separators.

\begin{Shaded}
\begin{Highlighting}[]
\NormalTok{(api/join-columns DSm }\AttributeTok{:joined}\NormalTok{ [}\AttributeTok{:V1} \AttributeTok{:V2} \AttributeTok{:V4}\NormalTok{] \{}\AttributeTok{:separator}\NormalTok{ [}\StringTok{"-"} \StringTok{"/"}\NormalTok{]}
                                             \AttributeTok{:missing-subst} \StringTok{"."}\NormalTok{\})}
\end{Highlighting}
\end{Shaded}

\_unnamed {[}9 2{]}:

\begin{longtable}[]{@{}ll@{}}
\toprule
:V3 & :joined\tabularnewline
\midrule
\endhead
0.5 & 1-1/A\tabularnewline
1.0 & 2-2/B\tabularnewline
& .-3/C\tabularnewline
1.5 & 1-4/A\tabularnewline
0.5 & 2-5/B\tabularnewline
1.0 & .-6/C\tabularnewline
& 1-7/A\tabularnewline
1.5 & 2-8/B\tabularnewline
0.5 & .-9/C\tabularnewline
\bottomrule
\end{longtable}

\begin{center}\rule{0.5\linewidth}{0.5pt}\end{center}

The other types of results, map:

\begin{Shaded}
\begin{Highlighting}[]
\NormalTok{(api/join-columns DSm }\AttributeTok{:joined}\NormalTok{ [}\AttributeTok{:V1} \AttributeTok{:V2} \AttributeTok{:V4}\NormalTok{] \{}\AttributeTok{:result-type} \AttributeTok{:map}\NormalTok{\})}
\end{Highlighting}
\end{Shaded}

\_unnamed {[}9 2{]}:

\begin{longtable}[]{@{}ll@{}}
\toprule
:V3 & :joined\tabularnewline
\midrule
\endhead
0.5 & \{:V1 1, :V2 1, :V4 ``A''\}\tabularnewline
1.0 & \{:V1 2, :V2 2, :V4 ``B''\}\tabularnewline
& \{:V1 nil, :V2 3, :V4 ``C''\}\tabularnewline
1.5 & \{:V1 1, :V2 4, :V4 ``A''\}\tabularnewline
0.5 & \{:V1 2, :V2 5, :V4 ``B''\}\tabularnewline
1.0 & \{:V1 nil, :V2 6, :V4 ``C''\}\tabularnewline
& \{:V1 1, :V2 7, :V4 ``A''\}\tabularnewline
1.5 & \{:V1 2, :V2 8, :V4 ``B''\}\tabularnewline
0.5 & \{:V1 nil, :V2 9, :V4 ``C''\}\tabularnewline
\bottomrule
\end{longtable}

\begin{center}\rule{0.5\linewidth}{0.5pt}\end{center}

Sequence

\begin{Shaded}
\begin{Highlighting}[]
\NormalTok{(api/join-columns DSm }\AttributeTok{:joined}\NormalTok{ [}\AttributeTok{:V1} \AttributeTok{:V2} \AttributeTok{:V4}\NormalTok{] \{}\AttributeTok{:result-type} \AttributeTok{:seq}\NormalTok{\})}
\end{Highlighting}
\end{Shaded}

\_unnamed {[}9 2{]}:

\begin{longtable}[]{@{}ll@{}}
\toprule
:V3 & :joined\tabularnewline
\midrule
\endhead
0.5 & (1 1 ``A'')\tabularnewline
1.0 & (2 2 ``B'')\tabularnewline
& (nil 3 ``C'')\tabularnewline
1.5 & (1 4 ``A'')\tabularnewline
0.5 & (2 5 ``B'')\tabularnewline
1.0 & (nil 6 ``C'')\tabularnewline
& (1 7 ``A'')\tabularnewline
1.5 & (2 8 ``B'')\tabularnewline
0.5 & (nil 9 ``C'')\tabularnewline
\bottomrule
\end{longtable}

\begin{center}\rule{0.5\linewidth}{0.5pt}\end{center}

Custom function, calculate hash

\begin{Shaded}
\begin{Highlighting}[]
\NormalTok{(api/join-columns DSm }\AttributeTok{:joined}\NormalTok{ [}\AttributeTok{:V1} \AttributeTok{:V2} \AttributeTok{:V4}\NormalTok{] \{}\AttributeTok{:result-type} \KeywordTok{hash}\NormalTok{\})}
\end{Highlighting}
\end{Shaded}

\_unnamed {[}9 2{]}:

\begin{longtable}[]{@{}ll@{}}
\toprule
:V3 & :joined\tabularnewline
\midrule
\endhead
0.5 & 535226087\tabularnewline
1.0 & 1128801549\tabularnewline
& -1842240303\tabularnewline
1.5 & 2022347171\tabularnewline
0.5 & 1884312041\tabularnewline
1.0 & -1555412370\tabularnewline
& 1640237355\tabularnewline
1.5 & -967279152\tabularnewline
0.5 & 1128367958\tabularnewline
\bottomrule
\end{longtable}

\begin{center}\rule{0.5\linewidth}{0.5pt}\end{center}

Grouped dataset

\begin{Shaded}
\begin{Highlighting}[]
\NormalTok{(}\KeywordTok{->}\NormalTok{ DSm}
\NormalTok{    (api/group-by }\AttributeTok{:V4}\NormalTok{)}
\NormalTok{    (api/join-columns }\AttributeTok{:joined}\NormalTok{ [}\AttributeTok{:V1} \AttributeTok{:V2} \AttributeTok{:V4}\NormalTok{])}
\NormalTok{    (api/ungroup))}
\end{Highlighting}
\end{Shaded}

\_unnamed {[}9 2{]}:

\begin{longtable}[]{@{}ll@{}}
\toprule
:V3 & :joined\tabularnewline
\midrule
\endhead
0.5 & 1-1-A\tabularnewline
1.5 & 1-4-A\tabularnewline
& 1-7-A\tabularnewline
1.0 & 2-2-B\tabularnewline
0.5 & 2-5-B\tabularnewline
1.5 & 2-8-B\tabularnewline
& 3-C\tabularnewline
1.0 & 6-C\tabularnewline
0.5 & 9-C\tabularnewline
\bottomrule
\end{longtable}

\begin{center}\rule{0.5\linewidth}{0.5pt}\end{center}

\hypertarget{tidyr-examples}{%
\subparagraph{Tidyr examples}\label{tidyr-examples}}

\href{https://tidyr.tidyverse.org/reference/unite.html}{source}

\begin{Shaded}
\begin{Highlighting}[]
\NormalTok{(}\BuiltInTok{def}\FunctionTok{ df }\NormalTok{(api/dataset \{}\AttributeTok{:x}\NormalTok{ [}\StringTok{"a"} \StringTok{"a"} \VariableTok{nil} \VariableTok{nil}\NormalTok{]}
                      \AttributeTok{:y}\NormalTok{ [}\StringTok{"b"} \VariableTok{nil} \StringTok{"b"} \VariableTok{nil}\NormalTok{]\}))}
\end{Highlighting}
\end{Shaded}

\begin{verbatim}
#'user/df
\end{verbatim}

\begin{Shaded}
\begin{Highlighting}[]
\NormalTok{df}
\end{Highlighting}
\end{Shaded}

\_unnamed {[}4 2{]}:

\begin{longtable}[]{@{}ll@{}}
\toprule
:x & :y\tabularnewline
\midrule
\endhead
a & b\tabularnewline
a &\tabularnewline
& b\tabularnewline
&\tabularnewline
\bottomrule
\end{longtable}

\begin{center}\rule{0.5\linewidth}{0.5pt}\end{center}

\begin{Shaded}
\begin{Highlighting}[]
\NormalTok{(api/join-columns df }\StringTok{"z"}\NormalTok{ [}\AttributeTok{:x} \AttributeTok{:y}\NormalTok{] \{}\AttributeTok{:drop-columns}\NormalTok{? }\VariableTok{false}
                                  \AttributeTok{:missing-subst} \StringTok{"NA"}
                                  \AttributeTok{:separator} \StringTok{"_"}\NormalTok{\})}
\end{Highlighting}
\end{Shaded}

\_unnamed {[}4 3{]}:

\begin{longtable}[]{@{}lll@{}}
\toprule
:x & :y & z\tabularnewline
\midrule
\endhead
a & b & a\_b\tabularnewline
a & & a\_NA\tabularnewline
& b & NA\_b\tabularnewline
& & NA\_NA\tabularnewline
\bottomrule
\end{longtable}

\begin{center}\rule{0.5\linewidth}{0.5pt}\end{center}

\begin{Shaded}
\begin{Highlighting}[]
\NormalTok{(api/join-columns df }\StringTok{"z"}\NormalTok{ [}\AttributeTok{:x} \AttributeTok{:y}\NormalTok{] \{}\AttributeTok{:drop-columns}\NormalTok{? }\VariableTok{false}
                                  \AttributeTok{:separator} \StringTok{"_"}\NormalTok{\})}
\end{Highlighting}
\end{Shaded}

\_unnamed {[}4 3{]}:

\begin{longtable}[]{@{}lll@{}}
\toprule
:x & :y & z\tabularnewline
\midrule
\endhead
a & b & a\_b\tabularnewline
a & & a\tabularnewline
& b & b\tabularnewline
& &\tabularnewline
\bottomrule
\end{longtable}

\hypertarget{separate}{%
\paragraph{Separate}\label{separate}}

Column can be also separated into several other columns using string as
separator, regex or custom function. Arguments:

\begin{itemize}
\tightlist
\item
  dataset
\item
  source column
\item
  target columns
\item
  separator as:

  \begin{itemize}
  \tightlist
  \item
    string - it's converted to regular expression and passed to
    \texttt{clojure.string/split} function
  \item
    regex
  \item
    or custom function (default: identity)
  \end{itemize}
\item
  options

  \begin{itemize}
  \tightlist
  \item
    \texttt{:drop-columns?} - whether drop source column or not
    (default: \texttt{true})
  \item
    \texttt{:missing-subst} - values which should be treated as missing,
    can be set, sequence, value or function (default: \texttt{""})
  \end{itemize}
\end{itemize}

Custom function (as separator) should return seqence of values for given
value.

\begin{center}\rule{0.5\linewidth}{0.5pt}\end{center}

Separate float into integer and factional values

\begin{Shaded}
\begin{Highlighting}[]
\NormalTok{(api/separate-column DS }\AttributeTok{:V3}\NormalTok{ [}\AttributeTok{:int-part} \AttributeTok{:frac-part}\NormalTok{] (}\KeywordTok{fn}\NormalTok{ [^}\KeywordTok{double}\NormalTok{ v]}
\NormalTok{                                                     [(}\KeywordTok{int}\NormalTok{ (}\KeywordTok{quot}\NormalTok{ v }\FloatTok{1.0}\NormalTok{))}
\NormalTok{                                                      (}\KeywordTok{mod}\NormalTok{ v }\FloatTok{1.0}\NormalTok{)]))}
\end{Highlighting}
\end{Shaded}

\_unnamed {[}9 5{]}:

\begin{longtable}[]{@{}lllll@{}}
\toprule
:V1 & :V2 & :int-part & :frac-part & :V4\tabularnewline
\midrule
\endhead
1 & 1 & 0 & 0.5 & A\tabularnewline
2 & 2 & 1 & 0.0 & B\tabularnewline
1 & 3 & 1 & 0.5 & C\tabularnewline
2 & 4 & 0 & 0.5 & A\tabularnewline
1 & 5 & 1 & 0.0 & B\tabularnewline
2 & 6 & 1 & 0.5 & C\tabularnewline
1 & 7 & 0 & 0.5 & A\tabularnewline
2 & 8 & 1 & 0.0 & B\tabularnewline
1 & 9 & 1 & 0.5 & C\tabularnewline
\bottomrule
\end{longtable}

\begin{center}\rule{0.5\linewidth}{0.5pt}\end{center}

Source column can be kept

\begin{Shaded}
\begin{Highlighting}[]
\NormalTok{(api/separate-column DS }\AttributeTok{:V3}\NormalTok{ [}\AttributeTok{:int-part} \AttributeTok{:frac-part}\NormalTok{] (}\KeywordTok{fn}\NormalTok{ [^}\KeywordTok{double}\NormalTok{ v]}
\NormalTok{                                                     [(}\KeywordTok{int}\NormalTok{ (}\KeywordTok{quot}\NormalTok{ v }\FloatTok{1.0}\NormalTok{))}
\NormalTok{                                                      (}\KeywordTok{mod}\NormalTok{ v }\FloatTok{1.0}\NormalTok{)]) \{}\AttributeTok{:drop-column}\NormalTok{? }\VariableTok{false}\NormalTok{\})}
\end{Highlighting}
\end{Shaded}

\_unnamed {[}9 6{]}:

\begin{longtable}[]{@{}llllll@{}}
\toprule
:V1 & :V2 & :V3 & :int-part & :frac-part & :V4\tabularnewline
\midrule
\endhead
1 & 1 & 0.5 & 0 & 0.5 & A\tabularnewline
2 & 2 & 1.0 & 1 & 0.0 & B\tabularnewline
1 & 3 & 1.5 & 1 & 0.5 & C\tabularnewline
2 & 4 & 0.5 & 0 & 0.5 & A\tabularnewline
1 & 5 & 1.0 & 1 & 0.0 & B\tabularnewline
2 & 6 & 1.5 & 1 & 0.5 & C\tabularnewline
1 & 7 & 0.5 & 0 & 0.5 & A\tabularnewline
2 & 8 & 1.0 & 1 & 0.0 & B\tabularnewline
1 & 9 & 1.5 & 1 & 0.5 & C\tabularnewline
\bottomrule
\end{longtable}

\begin{center}\rule{0.5\linewidth}{0.5pt}\end{center}

We can treat \texttt{0} or \texttt{0.0} as missing value

\begin{Shaded}
\begin{Highlighting}[]
\NormalTok{(api/separate-column DS }\AttributeTok{:V3}\NormalTok{ [}\AttributeTok{:int-part} \AttributeTok{:frac-part}\NormalTok{] (}\KeywordTok{fn}\NormalTok{ [^}\KeywordTok{double}\NormalTok{ v]}
\NormalTok{                                                     [(}\KeywordTok{int}\NormalTok{ (}\KeywordTok{quot}\NormalTok{ v }\FloatTok{1.0}\NormalTok{))}
\NormalTok{                                                      (}\KeywordTok{mod}\NormalTok{ v }\FloatTok{1.0}\NormalTok{)]) \{}\AttributeTok{:missing-subst}\NormalTok{ [}\DecValTok{0} \FloatTok{0.0}\NormalTok{]\})}
\end{Highlighting}
\end{Shaded}

\_unnamed {[}9 5{]}:

\begin{longtable}[]{@{}lllll@{}}
\toprule
:V1 & :V2 & :int-part & :frac-part & :V4\tabularnewline
\midrule
\endhead
1 & 1 & & 0.5 & A\tabularnewline
2 & 2 & 1 & & B\tabularnewline
1 & 3 & 1 & 0.5 & C\tabularnewline
2 & 4 & & 0.5 & A\tabularnewline
1 & 5 & 1 & & B\tabularnewline
2 & 6 & 1 & 0.5 & C\tabularnewline
1 & 7 & & 0.5 & A\tabularnewline
2 & 8 & 1 & & B\tabularnewline
1 & 9 & 1 & 0.5 & C\tabularnewline
\bottomrule
\end{longtable}

\begin{center}\rule{0.5\linewidth}{0.5pt}\end{center}

Works on grouped dataset

\begin{Shaded}
\begin{Highlighting}[]
\NormalTok{(}\KeywordTok{->}\NormalTok{ DS}
\NormalTok{    (api/group-by }\AttributeTok{:V4}\NormalTok{)}
\NormalTok{    (api/separate-column }\AttributeTok{:V3}\NormalTok{ [}\AttributeTok{:int-part} \AttributeTok{:fract-part}\NormalTok{] (}\KeywordTok{fn}\NormalTok{ [^}\KeywordTok{double}\NormalTok{ v]}
\NormalTok{                                                       [(}\KeywordTok{int}\NormalTok{ (}\KeywordTok{quot}\NormalTok{ v }\FloatTok{1.0}\NormalTok{))}
\NormalTok{                                                        (}\KeywordTok{mod}\NormalTok{ v }\FloatTok{1.0}\NormalTok{)]))}
\NormalTok{    (api/ungroup))}
\end{Highlighting}
\end{Shaded}

\_unnamed {[}9 5{]}:

\begin{longtable}[]{@{}lllll@{}}
\toprule
:V1 & :V2 & :int-part & :fract-part & :V4\tabularnewline
\midrule
\endhead
1 & 1 & 0 & 0.5 & A\tabularnewline
2 & 4 & 0 & 0.5 & A\tabularnewline
1 & 7 & 0 & 0.5 & A\tabularnewline
2 & 2 & 1 & 0.0 & B\tabularnewline
1 & 5 & 1 & 0.0 & B\tabularnewline
2 & 8 & 1 & 0.0 & B\tabularnewline
1 & 3 & 1 & 0.5 & C\tabularnewline
2 & 6 & 1 & 0.5 & C\tabularnewline
1 & 9 & 1 & 0.5 & C\tabularnewline
\bottomrule
\end{longtable}

\begin{center}\rule{0.5\linewidth}{0.5pt}\end{center}

Join and separate together.

\begin{Shaded}
\begin{Highlighting}[]
\NormalTok{(}\KeywordTok{->}\NormalTok{ DSm}
\NormalTok{    (api/join-columns }\AttributeTok{:joined}\NormalTok{ [}\AttributeTok{:V1} \AttributeTok{:V2} \AttributeTok{:V4}\NormalTok{] \{}\AttributeTok{:result-type} \AttributeTok{:map}\NormalTok{\})}
\NormalTok{    (api/separate-column }\AttributeTok{:joined}\NormalTok{ [}\AttributeTok{:v1} \AttributeTok{:v2} \AttributeTok{:v4}\NormalTok{] (}\KeywordTok{juxt} \AttributeTok{:V1} \AttributeTok{:V2} \AttributeTok{:V4}\NormalTok{)))}
\end{Highlighting}
\end{Shaded}

\_unnamed {[}9 4{]}:

\begin{longtable}[]{@{}llll@{}}
\toprule
:V3 & :v1 & :v2 & :v4\tabularnewline
\midrule
\endhead
0.5 & 1 & 1 & A\tabularnewline
1.0 & 2 & 2 & B\tabularnewline
& & 3 & C\tabularnewline
1.5 & 1 & 4 & A\tabularnewline
0.5 & 2 & 5 & B\tabularnewline
1.0 & & 6 & C\tabularnewline
& 1 & 7 & A\tabularnewline
1.5 & 2 & 8 & B\tabularnewline
0.5 & & 9 & C\tabularnewline
\bottomrule
\end{longtable}

\begin{Shaded}
\begin{Highlighting}[]
\NormalTok{(}\KeywordTok{->}\NormalTok{ DSm}
\NormalTok{    (api/join-columns }\AttributeTok{:joined}\NormalTok{ [}\AttributeTok{:V1} \AttributeTok{:V2} \AttributeTok{:V4}\NormalTok{] \{}\AttributeTok{:result-type} \AttributeTok{:seq}\NormalTok{\})}
\NormalTok{    (api/separate-column }\AttributeTok{:joined}\NormalTok{ [}\AttributeTok{:v1} \AttributeTok{:v2} \AttributeTok{:v4}\NormalTok{] }\KeywordTok{identity}\NormalTok{))}
\end{Highlighting}
\end{Shaded}

\_unnamed {[}9 4{]}:

\begin{longtable}[]{@{}llll@{}}
\toprule
:V3 & :v1 & :v2 & :v4\tabularnewline
\midrule
\endhead
0.5 & 1 & 1 & A\tabularnewline
1.0 & 2 & 2 & B\tabularnewline
& & 3 & C\tabularnewline
1.5 & 1 & 4 & A\tabularnewline
0.5 & 2 & 5 & B\tabularnewline
1.0 & & 6 & C\tabularnewline
& 1 & 7 & A\tabularnewline
1.5 & 2 & 8 & B\tabularnewline
0.5 & & 9 & C\tabularnewline
\bottomrule
\end{longtable}

\hypertarget{tidyr-examples-1}{%
\subparagraph{Tidyr examples}\label{tidyr-examples-1}}

\href{https://tidyr.tidyverse.org/reference/separate.html}{separate
source}
\href{https://tidyr.tidyverse.org/reference/extract.html}{extract
source}

\begin{Shaded}
\begin{Highlighting}[]
\NormalTok{(}\BuiltInTok{def}\FunctionTok{ df-separate }\NormalTok{(api/dataset \{}\AttributeTok{:x}\NormalTok{ [}\VariableTok{nil} \StringTok{"a.b"} \StringTok{"a.d"} \StringTok{"b.c"}\NormalTok{]\}))}
\NormalTok{(}\BuiltInTok{def}\FunctionTok{ df-separate2 }\NormalTok{(api/dataset \{}\AttributeTok{:x}\NormalTok{ [}\StringTok{"a"} \StringTok{"a b"} \VariableTok{nil} \StringTok{"a b c"}\NormalTok{]\}))}
\NormalTok{(}\BuiltInTok{def}\FunctionTok{ df-separate3 }\NormalTok{(api/dataset \{}\AttributeTok{:x}\NormalTok{ [}\StringTok{"a?b"} \VariableTok{nil} \StringTok{"a.b"} \StringTok{"b:c"}\NormalTok{]\}))}
\NormalTok{(}\BuiltInTok{def}\FunctionTok{ df-extract }\NormalTok{(api/dataset \{}\AttributeTok{:x}\NormalTok{ [}\VariableTok{nil} \StringTok{"a-b"} \StringTok{"a-d"} \StringTok{"b-c"} \StringTok{"d-e"}\NormalTok{]\}))}
\end{Highlighting}
\end{Shaded}

\begin{verbatim}
#'user/df-separate
#'user/df-separate2
#'user/df-separate3
#'user/df-extract
\end{verbatim}

\begin{Shaded}
\begin{Highlighting}[]
\NormalTok{df-separate}
\end{Highlighting}
\end{Shaded}

\_unnamed {[}4 1{]}:

\begin{longtable}[]{@{}l@{}}
\toprule
:x\tabularnewline
\midrule
\endhead
\tabularnewline
a.b\tabularnewline
a.d\tabularnewline
b.c\tabularnewline
\bottomrule
\end{longtable}

\begin{Shaded}
\begin{Highlighting}[]
\NormalTok{df-separate2}
\end{Highlighting}
\end{Shaded}

\_unnamed {[}4 1{]}:

\begin{longtable}[]{@{}l@{}}
\toprule
:x\tabularnewline
\midrule
\endhead
a\tabularnewline
a b\tabularnewline
\tabularnewline
a b c\tabularnewline
\bottomrule
\end{longtable}

\begin{Shaded}
\begin{Highlighting}[]
\NormalTok{df-separate3}
\end{Highlighting}
\end{Shaded}

\_unnamed {[}4 1{]}:

\begin{longtable}[]{@{}l@{}}
\toprule
:x\tabularnewline
\midrule
\endhead
a?b\tabularnewline
\tabularnewline
a.b\tabularnewline
b:c\tabularnewline
\bottomrule
\end{longtable}

\begin{Shaded}
\begin{Highlighting}[]
\NormalTok{df-extract}
\end{Highlighting}
\end{Shaded}

\_unnamed {[}5 1{]}:

\begin{longtable}[]{@{}l@{}}
\toprule
:x\tabularnewline
\midrule
\endhead
\tabularnewline
a-b\tabularnewline
a-d\tabularnewline
b-c\tabularnewline
d-e\tabularnewline
\bottomrule
\end{longtable}

\begin{center}\rule{0.5\linewidth}{0.5pt}\end{center}

\begin{Shaded}
\begin{Highlighting}[]
\NormalTok{(api/separate-column df-separate }\AttributeTok{:x}\NormalTok{ [}\AttributeTok{:A} \AttributeTok{:B}\NormalTok{] }\StringTok{"}\SpecialCharTok{\textbackslash{}\textbackslash{}}\StringTok{."}\NormalTok{)}
\end{Highlighting}
\end{Shaded}

\_unnamed {[}4 2{]}:

\begin{longtable}[]{@{}ll@{}}
\toprule
:A & :B\tabularnewline
\midrule
\endhead
&\tabularnewline
a & b\tabularnewline
a & d\tabularnewline
b & c\tabularnewline
\bottomrule
\end{longtable}

\begin{center}\rule{0.5\linewidth}{0.5pt}\end{center}

You can drop columns after separation by setting \texttt{nil} as a name.
We need second value here.

\begin{Shaded}
\begin{Highlighting}[]
\NormalTok{(api/separate-column df-separate }\AttributeTok{:x}\NormalTok{ [}\VariableTok{nil} \AttributeTok{:B}\NormalTok{] }\StringTok{"}\SpecialCharTok{\textbackslash{}\textbackslash{}}\StringTok{."}\NormalTok{)}
\end{Highlighting}
\end{Shaded}

\_unnamed {[}4 1{]}:

\begin{longtable}[]{@{}l@{}}
\toprule
:B\tabularnewline
\midrule
\endhead
\tabularnewline
b\tabularnewline
d\tabularnewline
c\tabularnewline
\bottomrule
\end{longtable}

\begin{center}\rule{0.5\linewidth}{0.5pt}\end{center}

Extra data is dropped

\begin{Shaded}
\begin{Highlighting}[]
\NormalTok{(api/separate-column df-separate2 }\AttributeTok{:x}\NormalTok{ [}\StringTok{"a"} \StringTok{"b"}\NormalTok{] }\StringTok{" "}\NormalTok{)}
\end{Highlighting}
\end{Shaded}

\_unnamed {[}4 2{]}:

\begin{longtable}[]{@{}ll@{}}
\toprule
a & b\tabularnewline
\midrule
\endhead
a &\tabularnewline
a & b\tabularnewline
&\tabularnewline
a & b\tabularnewline
\bottomrule
\end{longtable}

\begin{center}\rule{0.5\linewidth}{0.5pt}\end{center}

Split with regular expression

\begin{Shaded}
\begin{Highlighting}[]
\NormalTok{(api/separate-column df-separate3 }\AttributeTok{:x}\NormalTok{ [}\StringTok{"a"} \StringTok{"b"}\NormalTok{] }\StringTok{"[?}\SpecialCharTok{\textbackslash{}\textbackslash{}}\StringTok{.:]"}\NormalTok{)}
\end{Highlighting}
\end{Shaded}

\_unnamed {[}4 2{]}:

\begin{longtable}[]{@{}ll@{}}
\toprule
a & b\tabularnewline
\midrule
\endhead
a & b\tabularnewline
&\tabularnewline
a & b\tabularnewline
b & c\tabularnewline
\bottomrule
\end{longtable}

\begin{center}\rule{0.5\linewidth}{0.5pt}\end{center}

Or just regular expression to extract values

\begin{Shaded}
\begin{Highlighting}[]
\NormalTok{(api/separate-column df-separate3 }\AttributeTok{:x}\NormalTok{ [}\StringTok{"a"} \StringTok{"b"}\NormalTok{] }\SpecialStringTok{#"(.).(.)"}\NormalTok{)}
\end{Highlighting}
\end{Shaded}

\_unnamed {[}4 2{]}:

\begin{longtable}[]{@{}ll@{}}
\toprule
a & b\tabularnewline
\midrule
\endhead
a & b\tabularnewline
&\tabularnewline
a & b\tabularnewline
b & c\tabularnewline
\bottomrule
\end{longtable}

\begin{center}\rule{0.5\linewidth}{0.5pt}\end{center}

Extract first value only

\begin{Shaded}
\begin{Highlighting}[]
\NormalTok{(api/separate-column df-extract }\AttributeTok{:x}\NormalTok{ [}\StringTok{"A"}\NormalTok{] }\StringTok{"-"}\NormalTok{)}
\end{Highlighting}
\end{Shaded}

\_unnamed {[}5 1{]}:

\begin{longtable}[]{@{}l@{}}
\toprule
A\tabularnewline
\midrule
\endhead
\tabularnewline
a\tabularnewline
a\tabularnewline
b\tabularnewline
d\tabularnewline
\bottomrule
\end{longtable}

\begin{center}\rule{0.5\linewidth}{0.5pt}\end{center}

Split with regex

\begin{Shaded}
\begin{Highlighting}[]
\NormalTok{(api/separate-column df-extract }\AttributeTok{:x}\NormalTok{ [}\StringTok{"A"} \StringTok{"B"}\NormalTok{] }\SpecialStringTok{#"(\textbackslash{}p\{Alnum\})-(\textbackslash{}p\{Alnum\})"}\NormalTok{)}
\end{Highlighting}
\end{Shaded}

\_unnamed {[}5 2{]}:

\begin{longtable}[]{@{}ll@{}}
\toprule
A & B\tabularnewline
\midrule
\endhead
&\tabularnewline
a & b\tabularnewline
a & d\tabularnewline
b & c\tabularnewline
d & e\tabularnewline
\bottomrule
\end{longtable}

\begin{center}\rule{0.5\linewidth}{0.5pt}\end{center}

Only \texttt{a,b,c,d} strings

\begin{Shaded}
\begin{Highlighting}[]
\NormalTok{(api/separate-column df-extract }\AttributeTok{:x}\NormalTok{ [}\StringTok{"A"} \StringTok{"B"}\NormalTok{] }\SpecialStringTok{#"([a-d]+)-([a-d]+)"}\NormalTok{)}
\end{Highlighting}
\end{Shaded}

\_unnamed {[}5 2{]}:

\begin{longtable}[]{@{}ll@{}}
\toprule
A & B\tabularnewline
\midrule
\endhead
&\tabularnewline
a & b\tabularnewline
a & d\tabularnewline
b & c\tabularnewline
&\tabularnewline
\bottomrule
\end{longtable}

\hypertarget{foldunroll-rows}{%
\subsubsection{Fold/Unroll Rows}\label{foldunroll-rows}}

To pack or unpack the data into single value you can use
\texttt{fold-by} and \texttt{unroll} functions.

\texttt{fold-by} groups dataset and packs columns data from each group
separately into desired datastructure (like vector or sequence).
\texttt{unroll} does the opposite.

\hypertarget{fold-by}{%
\paragraph{Fold-by}\label{fold-by}}

Group-by and pack columns into vector

\begin{Shaded}
\begin{Highlighting}[]
\NormalTok{(api/fold-by DS [}\AttributeTok{:V3} \AttributeTok{:V4} \AttributeTok{:V1}\NormalTok{])}
\end{Highlighting}
\end{Shaded}

\_unnamed {[}6 4{]}:

\begin{longtable}[]{@{}llll@{}}
\toprule
:V4 & :V3 & :V1 & :V2\tabularnewline
\midrule
\endhead
B & 1.0 & 1 & {[}5{]}\tabularnewline
C & 1.5 & 2 & {[}6{]}\tabularnewline
C & 1.5 & 1 & {[}3 9{]}\tabularnewline
A & 0.5 & 1 & {[}1 7{]}\tabularnewline
B & 1.0 & 2 & {[}2 8{]}\tabularnewline
A & 0.5 & 2 & {[}4{]}\tabularnewline
\bottomrule
\end{longtable}

\begin{center}\rule{0.5\linewidth}{0.5pt}\end{center}

You can pack several columns at once.

\begin{Shaded}
\begin{Highlighting}[]
\NormalTok{(api/fold-by DS [}\AttributeTok{:V4}\NormalTok{])}
\end{Highlighting}
\end{Shaded}

\_unnamed {[}3 4{]}:

\begin{longtable}[]{@{}llll@{}}
\toprule
:V4 & :V1 & :V2 & :V3\tabularnewline
\midrule
\endhead
B & {[}2 1 2{]} & {[}2 5 8{]} & {[}1.0 1.0 1.0{]}\tabularnewline
C & {[}1 2 1{]} & {[}3 6 9{]} & {[}1.5 1.5 1.5{]}\tabularnewline
A & {[}1 2 1{]} & {[}1 4 7{]} & {[}0.5 0.5 0.5{]}\tabularnewline
\bottomrule
\end{longtable}

\begin{center}\rule{0.5\linewidth}{0.5pt}\end{center}

You can use custom packing function

\begin{Shaded}
\begin{Highlighting}[]
\NormalTok{(api/fold-by DS [}\AttributeTok{:V4}\NormalTok{] }\KeywordTok{seq}\NormalTok{)}
\end{Highlighting}
\end{Shaded}

\_unnamed {[}3 4{]}:

\begin{longtable}[]{@{}llll@{}}
\toprule
:V4 & :V1 & :V2 & :V3\tabularnewline
\midrule
\endhead
B & (2 1 2) & (2 5 8) & (1.0 1.0 1.0)\tabularnewline
C & (1 2 1) & (3 6 9) & (1.5 1.5 1.5)\tabularnewline
A & (1 2 1) & (1 4 7) & (0.5 0.5 0.5)\tabularnewline
\bottomrule
\end{longtable}

or

\begin{Shaded}
\begin{Highlighting}[]
\NormalTok{(api/fold-by DS [}\AttributeTok{:V4}\NormalTok{] }\KeywordTok{set}\NormalTok{)}
\end{Highlighting}
\end{Shaded}

\_unnamed {[}3 4{]}:

\begin{longtable}[]{@{}llll@{}}
\toprule
:V4 & :V1 & :V2 & :V3\tabularnewline
\midrule
\endhead
B & \#\{1 2\} & \#\{2 5 8\} & \#\{1.0\}\tabularnewline
C & \#\{1 2\} & \#\{6 3 9\} & \#\{1.5\}\tabularnewline
A & \#\{1 2\} & \#\{7 1 4\} & \#\{0.5\}\tabularnewline
\bottomrule
\end{longtable}

\begin{center}\rule{0.5\linewidth}{0.5pt}\end{center}

This works also on grouped dataset

\begin{Shaded}
\begin{Highlighting}[]
\NormalTok{(}\KeywordTok{->}\NormalTok{ DS}
\NormalTok{    (api/group-by }\AttributeTok{:V1}\NormalTok{)}
\NormalTok{    (api/fold-by }\AttributeTok{:V4}\NormalTok{)}
\NormalTok{    (api/ungroup))}
\end{Highlighting}
\end{Shaded}

\_unnamed {[}6 4{]}:

\begin{longtable}[]{@{}llll@{}}
\toprule
:V4 & :V1 & :V2 & :V3\tabularnewline
\midrule
\endhead
B & {[}1{]} & {[}5{]} & {[}1.0{]}\tabularnewline
C & {[}1 1{]} & {[}3 9{]} & {[}1.5 1.5{]}\tabularnewline
A & {[}1 1{]} & {[}1 7{]} & {[}0.5 0.5{]}\tabularnewline
B & {[}2 2{]} & {[}2 8{]} & {[}1.0 1.0{]}\tabularnewline
C & {[}2{]} & {[}6{]} & {[}1.5{]}\tabularnewline
A & {[}2{]} & {[}4{]} & {[}0.5{]}\tabularnewline
\bottomrule
\end{longtable}

\hypertarget{unroll}{%
\paragraph{Unroll}\label{unroll}}

\texttt{unroll} unfolds sequences stored in data, multiplying other ones
when necessary. You can unroll more than one column at once (folded data
should have the same size!).

Options:

\begin{itemize}
\tightlist
\item
  \texttt{:indexes?} if true (or column name), information about index
  of unrolled sequence is added.
\item
  \texttt{:datatypes} list of datatypes which should be applied to
  restored columns, a map
\end{itemize}

\begin{center}\rule{0.5\linewidth}{0.5pt}\end{center}

Unroll one column

\begin{Shaded}
\begin{Highlighting}[]
\NormalTok{(api/unroll (api/fold-by DS [}\AttributeTok{:V4}\NormalTok{]) [}\AttributeTok{:V1}\NormalTok{])}
\end{Highlighting}
\end{Shaded}

\_unnamed {[}9 4{]}:

\begin{longtable}[]{@{}llll@{}}
\toprule
:V4 & :V2 & :V3 & :V1\tabularnewline
\midrule
\endhead
B & {[}2 5 8{]} & {[}1.0 1.0 1.0{]} & 2\tabularnewline
B & {[}2 5 8{]} & {[}1.0 1.0 1.0{]} & 1\tabularnewline
B & {[}2 5 8{]} & {[}1.0 1.0 1.0{]} & 2\tabularnewline
C & {[}3 6 9{]} & {[}1.5 1.5 1.5{]} & 1\tabularnewline
C & {[}3 6 9{]} & {[}1.5 1.5 1.5{]} & 2\tabularnewline
C & {[}3 6 9{]} & {[}1.5 1.5 1.5{]} & 1\tabularnewline
A & {[}1 4 7{]} & {[}0.5 0.5 0.5{]} & 1\tabularnewline
A & {[}1 4 7{]} & {[}0.5 0.5 0.5{]} & 2\tabularnewline
A & {[}1 4 7{]} & {[}0.5 0.5 0.5{]} & 1\tabularnewline
\bottomrule
\end{longtable}

\begin{center}\rule{0.5\linewidth}{0.5pt}\end{center}

Unroll all folded columns

\begin{Shaded}
\begin{Highlighting}[]
\NormalTok{(api/unroll (api/fold-by DS [}\AttributeTok{:V4}\NormalTok{]) [}\AttributeTok{:V1} \AttributeTok{:V2} \AttributeTok{:V3}\NormalTok{])}
\end{Highlighting}
\end{Shaded}

\_unnamed {[}9 4{]}:

\begin{longtable}[]{@{}llll@{}}
\toprule
:V4 & :V1 & :V2 & :V3\tabularnewline
\midrule
\endhead
B & 2 & 2 & 1.000\tabularnewline
B & 1 & 5 & 1.000\tabularnewline
B & 2 & 8 & 1.000\tabularnewline
C & 1 & 3 & 1.500\tabularnewline
C & 2 & 6 & 1.500\tabularnewline
C & 1 & 9 & 1.500\tabularnewline
A & 1 & 1 & 0.5000\tabularnewline
A & 2 & 4 & 0.5000\tabularnewline
A & 1 & 7 & 0.5000\tabularnewline
\bottomrule
\end{longtable}

\begin{center}\rule{0.5\linewidth}{0.5pt}\end{center}

Unroll one by one leads to cartesian product

\begin{Shaded}
\begin{Highlighting}[]
\NormalTok{(}\KeywordTok{->}\NormalTok{ DS}
\NormalTok{    (api/fold-by [}\AttributeTok{:V4} \AttributeTok{:V1}\NormalTok{])}
\NormalTok{    (api/unroll [}\AttributeTok{:V2}\NormalTok{])}
\NormalTok{    (api/unroll [}\AttributeTok{:V3}\NormalTok{]))}
\end{Highlighting}
\end{Shaded}

\_unnamed {[}15 4{]}:

\begin{longtable}[]{@{}llll@{}}
\toprule
:V4 & :V1 & :V2 & :V3\tabularnewline
\midrule
\endhead
C & 2 & 6 & 1.500\tabularnewline
A & 1 & 1 & 0.5000\tabularnewline
A & 1 & 1 & 0.5000\tabularnewline
A & 1 & 7 & 0.5000\tabularnewline
A & 1 & 7 & 0.5000\tabularnewline
B & 1 & 5 & 1.000\tabularnewline
C & 1 & 3 & 1.500\tabularnewline
C & 1 & 3 & 1.500\tabularnewline
C & 1 & 9 & 1.500\tabularnewline
C & 1 & 9 & 1.500\tabularnewline
A & 2 & 4 & 0.5000\tabularnewline
B & 2 & 2 & 1.000\tabularnewline
B & 2 & 2 & 1.000\tabularnewline
B & 2 & 8 & 1.000\tabularnewline
B & 2 & 8 & 1.000\tabularnewline
\bottomrule
\end{longtable}

\begin{center}\rule{0.5\linewidth}{0.5pt}\end{center}

You can add indexes

\begin{Shaded}
\begin{Highlighting}[]
\NormalTok{(api/unroll (api/fold-by DS [}\AttributeTok{:V1}\NormalTok{]) [}\AttributeTok{:V4} \AttributeTok{:V2} \AttributeTok{:V3}\NormalTok{] \{}\AttributeTok{:indexes}\NormalTok{? }\VariableTok{true}\NormalTok{\})}
\end{Highlighting}
\end{Shaded}

\_unnamed {[}9 5{]}:

\begin{longtable}[]{@{}lllll@{}}
\toprule
:V1 & :indexes & :V2 & :V3 & :V4\tabularnewline
\midrule
\endhead
1 & 0 & 1 & 0.5000 & A\tabularnewline
1 & 1 & 3 & 1.500 & C\tabularnewline
1 & 2 & 5 & 1.000 & B\tabularnewline
1 & 3 & 7 & 0.5000 & A\tabularnewline
1 & 4 & 9 & 1.500 & C\tabularnewline
2 & 0 & 2 & 1.000 & B\tabularnewline
2 & 1 & 4 & 0.5000 & A\tabularnewline
2 & 2 & 6 & 1.500 & C\tabularnewline
2 & 3 & 8 & 1.000 & B\tabularnewline
\bottomrule
\end{longtable}

\begin{Shaded}
\begin{Highlighting}[]
\NormalTok{(api/unroll (api/fold-by DS [}\AttributeTok{:V1}\NormalTok{]) [}\AttributeTok{:V4} \AttributeTok{:V2} \AttributeTok{:V3}\NormalTok{] \{}\AttributeTok{:indexes}\NormalTok{? }\StringTok{"vector idx"}\NormalTok{\})}
\end{Highlighting}
\end{Shaded}

\_unnamed {[}9 5{]}:

\begin{longtable}[]{@{}lllll@{}}
\toprule
:V1 & vector idx & :V2 & :V3 & :V4\tabularnewline
\midrule
\endhead
1 & 0 & 1 & 0.5000 & A\tabularnewline
1 & 1 & 3 & 1.500 & C\tabularnewline
1 & 2 & 5 & 1.000 & B\tabularnewline
1 & 3 & 7 & 0.5000 & A\tabularnewline
1 & 4 & 9 & 1.500 & C\tabularnewline
2 & 0 & 2 & 1.000 & B\tabularnewline
2 & 1 & 4 & 0.5000 & A\tabularnewline
2 & 2 & 6 & 1.500 & C\tabularnewline
2 & 3 & 8 & 1.000 & B\tabularnewline
\bottomrule
\end{longtable}

\begin{center}\rule{0.5\linewidth}{0.5pt}\end{center}

You can also force datatypes

\begin{Shaded}
\begin{Highlighting}[]
\NormalTok{(}\KeywordTok{->}\NormalTok{ DS}
\NormalTok{    (api/fold-by [}\AttributeTok{:V1}\NormalTok{])}
\NormalTok{    (api/unroll [}\AttributeTok{:V4} \AttributeTok{:V2} \AttributeTok{:V3}\NormalTok{] \{}\AttributeTok{:datatypes}\NormalTok{ \{}\AttributeTok{:V4} \AttributeTok{:string}
                                           \AttributeTok{:V2} \AttributeTok{:int16}
                                           \AttributeTok{:V3} \AttributeTok{:float32}\NormalTok{\}\})}
\NormalTok{    (api/info }\AttributeTok{:columns}\NormalTok{))}
\end{Highlighting}
\end{Shaded}

\_unnamed :column info {[}4 4{]}:

\begin{longtable}[]{@{}llll@{}}
\toprule
:name & :size & :datatype & :categorical?\tabularnewline
\midrule
\endhead
:V1 & 9 & :int64 &\tabularnewline
:V2 & 9 & :int16 &\tabularnewline
:V3 & 9 & :float32 &\tabularnewline
:V4 & 9 & :string & true\tabularnewline
\bottomrule
\end{longtable}

\begin{center}\rule{0.5\linewidth}{0.5pt}\end{center}

This works also on grouped dataset

\begin{Shaded}
\begin{Highlighting}[]
\NormalTok{(}\KeywordTok{->}\NormalTok{ DS}
\NormalTok{    (api/group-by }\AttributeTok{:V1}\NormalTok{)}
\NormalTok{    (api/fold-by [}\AttributeTok{:V1} \AttributeTok{:V4}\NormalTok{])}
\NormalTok{    (api/unroll }\AttributeTok{:V3}\NormalTok{ \{}\AttributeTok{:indexes}\NormalTok{? }\VariableTok{true}\NormalTok{\})}
\NormalTok{    (api/ungroup))}
\end{Highlighting}
\end{Shaded}

\_unnamed {[}9 5{]}:

\begin{longtable}[]{@{}lllll@{}}
\toprule
:V4 & :V1 & :V2 & :indexes & :V3\tabularnewline
\midrule
\endhead
A & 1 & {[}1 7{]} & 0 & 0.5000\tabularnewline
A & 1 & {[}1 7{]} & 1 & 0.5000\tabularnewline
B & 1 & {[}5{]} & 0 & 1.000\tabularnewline
C & 1 & {[}3 9{]} & 0 & 1.500\tabularnewline
C & 1 & {[}3 9{]} & 1 & 1.500\tabularnewline
C & 2 & {[}6{]} & 0 & 1.500\tabularnewline
A & 2 & {[}4{]} & 0 & 0.5000\tabularnewline
B & 2 & {[}2 8{]} & 0 & 1.000\tabularnewline
B & 2 & {[}2 8{]} & 1 & 1.000\tabularnewline
\bottomrule
\end{longtable}

\hypertarget{reshape}{%
\subsubsection{Reshape}\label{reshape}}

Reshaping data provides two types of operations:

\begin{itemize}
\tightlist
\item
  \texttt{pivot-\textgreater{}longer} - converting columns to rows
\item
  \texttt{pivot-\textgreater{}wider} - converting rows to columns
\end{itemize}

Both functions are inspired on
\href{https://tidyr.tidyverse.org/articles/pivot.html}{tidyr} R package
and provide almost the same functionality.

All examples are taken from mentioned above documentation.

Both functions work only on regular dataset.

\hypertarget{longer}{%
\paragraph{Longer}\label{longer}}

\texttt{pivot-\textgreater{}longer} converts columns to rows. Column
names are treated as data.

Arguments:

\begin{itemize}
\tightlist
\item
  dataset
\item
  columns selector
\item
  options:

  \begin{itemize}
  \tightlist
  \item
    \texttt{:target-columns} - names of the columns created or columns
    pattern (see below) (default: \texttt{:\$column})
  \item
    \texttt{:value-column-name} - name of the column for values
    (default: \texttt{:\$value})
  \item
    \texttt{:splitter} - string, regular expression or function which
    splits source column names into data
  \item
    \texttt{:drop-missing?} - remove rows with missing? (default:
    \texttt{:true})
  \item
    \texttt{:datatypes} - map of target columns data types
  \end{itemize}
\end{itemize}

\texttt{:target-columns} - can be:

\begin{itemize}
\tightlist
\item
  column name - source columns names are put there as a data
\item
  column names as seqence - source columns names after split are put
  separately into \texttt{:target-columns} as data
\item
  pattern - is a sequence of names, where some of the names are
  \texttt{nil}. \texttt{nil} is replaced by a name taken from splitter
  and such column is used for values.
\end{itemize}

\begin{center}\rule{0.5\linewidth}{0.5pt}\end{center}

Create rows from all columns but \texttt{"religion"}.

\begin{Shaded}
\begin{Highlighting}[]
\NormalTok{(}\BuiltInTok{def}\FunctionTok{ relig-income }\NormalTok{(api/dataset }\StringTok{"data/relig_income.csv"}\NormalTok{))}
\end{Highlighting}
\end{Shaded}

\begin{Shaded}
\begin{Highlighting}[]
\NormalTok{relig-income}
\end{Highlighting}
\end{Shaded}

data/relig\_income.csv {[}18 11{]}:

\begin{longtable}[]{@{}lllllllllll@{}}
\toprule
\begin{minipage}[b]{0.14\columnwidth}\raggedright
religion\strut
\end{minipage} & \begin{minipage}[b]{0.04\columnwidth}\raggedright
\textless{}\$10k\strut
\end{minipage} & \begin{minipage}[b]{0.05\columnwidth}\raggedright
\$10-20k\strut
\end{minipage} & \begin{minipage}[b]{0.05\columnwidth}\raggedright
\$20-30k\strut
\end{minipage} & \begin{minipage}[b]{0.05\columnwidth}\raggedright
\$30-40k\strut
\end{minipage} & \begin{minipage}[b]{0.05\columnwidth}\raggedright
\$40-50k\strut
\end{minipage} & \begin{minipage}[b]{0.05\columnwidth}\raggedright
\$50-75k\strut
\end{minipage} & \begin{minipage}[b]{0.06\columnwidth}\raggedright
\$75-100k\strut
\end{minipage} & \begin{minipage}[b]{0.06\columnwidth}\raggedright
\$100-150k\strut
\end{minipage} & \begin{minipage}[b]{0.04\columnwidth}\raggedright
\textgreater{}150k\strut
\end{minipage} & \begin{minipage}[b]{0.11\columnwidth}\raggedright
Don't know/refused\strut
\end{minipage}\tabularnewline
\midrule
\endhead
\begin{minipage}[t]{0.14\columnwidth}\raggedright
Agnostic\strut
\end{minipage} & \begin{minipage}[t]{0.04\columnwidth}\raggedright
27\strut
\end{minipage} & \begin{minipage}[t]{0.05\columnwidth}\raggedright
34\strut
\end{minipage} & \begin{minipage}[t]{0.05\columnwidth}\raggedright
60\strut
\end{minipage} & \begin{minipage}[t]{0.05\columnwidth}\raggedright
81\strut
\end{minipage} & \begin{minipage}[t]{0.05\columnwidth}\raggedright
76\strut
\end{minipage} & \begin{minipage}[t]{0.05\columnwidth}\raggedright
137\strut
\end{minipage} & \begin{minipage}[t]{0.06\columnwidth}\raggedright
122\strut
\end{minipage} & \begin{minipage}[t]{0.06\columnwidth}\raggedright
109\strut
\end{minipage} & \begin{minipage}[t]{0.04\columnwidth}\raggedright
84\strut
\end{minipage} & \begin{minipage}[t]{0.11\columnwidth}\raggedright
96\strut
\end{minipage}\tabularnewline
\begin{minipage}[t]{0.14\columnwidth}\raggedright
Atheist\strut
\end{minipage} & \begin{minipage}[t]{0.04\columnwidth}\raggedright
12\strut
\end{minipage} & \begin{minipage}[t]{0.05\columnwidth}\raggedright
27\strut
\end{minipage} & \begin{minipage}[t]{0.05\columnwidth}\raggedright
37\strut
\end{minipage} & \begin{minipage}[t]{0.05\columnwidth}\raggedright
52\strut
\end{minipage} & \begin{minipage}[t]{0.05\columnwidth}\raggedright
35\strut
\end{minipage} & \begin{minipage}[t]{0.05\columnwidth}\raggedright
70\strut
\end{minipage} & \begin{minipage}[t]{0.06\columnwidth}\raggedright
73\strut
\end{minipage} & \begin{minipage}[t]{0.06\columnwidth}\raggedright
59\strut
\end{minipage} & \begin{minipage}[t]{0.04\columnwidth}\raggedright
74\strut
\end{minipage} & \begin{minipage}[t]{0.11\columnwidth}\raggedright
76\strut
\end{minipage}\tabularnewline
\begin{minipage}[t]{0.14\columnwidth}\raggedright
Buddhist\strut
\end{minipage} & \begin{minipage}[t]{0.04\columnwidth}\raggedright
27\strut
\end{minipage} & \begin{minipage}[t]{0.05\columnwidth}\raggedright
21\strut
\end{minipage} & \begin{minipage}[t]{0.05\columnwidth}\raggedright
30\strut
\end{minipage} & \begin{minipage}[t]{0.05\columnwidth}\raggedright
34\strut
\end{minipage} & \begin{minipage}[t]{0.05\columnwidth}\raggedright
33\strut
\end{minipage} & \begin{minipage}[t]{0.05\columnwidth}\raggedright
58\strut
\end{minipage} & \begin{minipage}[t]{0.06\columnwidth}\raggedright
62\strut
\end{minipage} & \begin{minipage}[t]{0.06\columnwidth}\raggedright
39\strut
\end{minipage} & \begin{minipage}[t]{0.04\columnwidth}\raggedright
53\strut
\end{minipage} & \begin{minipage}[t]{0.11\columnwidth}\raggedright
54\strut
\end{minipage}\tabularnewline
\begin{minipage}[t]{0.14\columnwidth}\raggedright
Catholic\strut
\end{minipage} & \begin{minipage}[t]{0.04\columnwidth}\raggedright
418\strut
\end{minipage} & \begin{minipage}[t]{0.05\columnwidth}\raggedright
617\strut
\end{minipage} & \begin{minipage}[t]{0.05\columnwidth}\raggedright
732\strut
\end{minipage} & \begin{minipage}[t]{0.05\columnwidth}\raggedright
670\strut
\end{minipage} & \begin{minipage}[t]{0.05\columnwidth}\raggedright
638\strut
\end{minipage} & \begin{minipage}[t]{0.05\columnwidth}\raggedright
1116\strut
\end{minipage} & \begin{minipage}[t]{0.06\columnwidth}\raggedright
949\strut
\end{minipage} & \begin{minipage}[t]{0.06\columnwidth}\raggedright
792\strut
\end{minipage} & \begin{minipage}[t]{0.04\columnwidth}\raggedright
633\strut
\end{minipage} & \begin{minipage}[t]{0.11\columnwidth}\raggedright
1489\strut
\end{minipage}\tabularnewline
\begin{minipage}[t]{0.14\columnwidth}\raggedright
Don't know/refused\strut
\end{minipage} & \begin{minipage}[t]{0.04\columnwidth}\raggedright
15\strut
\end{minipage} & \begin{minipage}[t]{0.05\columnwidth}\raggedright
14\strut
\end{minipage} & \begin{minipage}[t]{0.05\columnwidth}\raggedright
15\strut
\end{minipage} & \begin{minipage}[t]{0.05\columnwidth}\raggedright
11\strut
\end{minipage} & \begin{minipage}[t]{0.05\columnwidth}\raggedright
10\strut
\end{minipage} & \begin{minipage}[t]{0.05\columnwidth}\raggedright
35\strut
\end{minipage} & \begin{minipage}[t]{0.06\columnwidth}\raggedright
21\strut
\end{minipage} & \begin{minipage}[t]{0.06\columnwidth}\raggedright
17\strut
\end{minipage} & \begin{minipage}[t]{0.04\columnwidth}\raggedright
18\strut
\end{minipage} & \begin{minipage}[t]{0.11\columnwidth}\raggedright
116\strut
\end{minipage}\tabularnewline
\begin{minipage}[t]{0.14\columnwidth}\raggedright
Evangelical Prot\strut
\end{minipage} & \begin{minipage}[t]{0.04\columnwidth}\raggedright
575\strut
\end{minipage} & \begin{minipage}[t]{0.05\columnwidth}\raggedright
869\strut
\end{minipage} & \begin{minipage}[t]{0.05\columnwidth}\raggedright
1064\strut
\end{minipage} & \begin{minipage}[t]{0.05\columnwidth}\raggedright
982\strut
\end{minipage} & \begin{minipage}[t]{0.05\columnwidth}\raggedright
881\strut
\end{minipage} & \begin{minipage}[t]{0.05\columnwidth}\raggedright
1486\strut
\end{minipage} & \begin{minipage}[t]{0.06\columnwidth}\raggedright
949\strut
\end{minipage} & \begin{minipage}[t]{0.06\columnwidth}\raggedright
723\strut
\end{minipage} & \begin{minipage}[t]{0.04\columnwidth}\raggedright
414\strut
\end{minipage} & \begin{minipage}[t]{0.11\columnwidth}\raggedright
1529\strut
\end{minipage}\tabularnewline
\begin{minipage}[t]{0.14\columnwidth}\raggedright
Hindu\strut
\end{minipage} & \begin{minipage}[t]{0.04\columnwidth}\raggedright
1\strut
\end{minipage} & \begin{minipage}[t]{0.05\columnwidth}\raggedright
9\strut
\end{minipage} & \begin{minipage}[t]{0.05\columnwidth}\raggedright
7\strut
\end{minipage} & \begin{minipage}[t]{0.05\columnwidth}\raggedright
9\strut
\end{minipage} & \begin{minipage}[t]{0.05\columnwidth}\raggedright
11\strut
\end{minipage} & \begin{minipage}[t]{0.05\columnwidth}\raggedright
34\strut
\end{minipage} & \begin{minipage}[t]{0.06\columnwidth}\raggedright
47\strut
\end{minipage} & \begin{minipage}[t]{0.06\columnwidth}\raggedright
48\strut
\end{minipage} & \begin{minipage}[t]{0.04\columnwidth}\raggedright
54\strut
\end{minipage} & \begin{minipage}[t]{0.11\columnwidth}\raggedright
37\strut
\end{minipage}\tabularnewline
\begin{minipage}[t]{0.14\columnwidth}\raggedright
Historically Black Prot\strut
\end{minipage} & \begin{minipage}[t]{0.04\columnwidth}\raggedright
228\strut
\end{minipage} & \begin{minipage}[t]{0.05\columnwidth}\raggedright
244\strut
\end{minipage} & \begin{minipage}[t]{0.05\columnwidth}\raggedright
236\strut
\end{minipage} & \begin{minipage}[t]{0.05\columnwidth}\raggedright
238\strut
\end{minipage} & \begin{minipage}[t]{0.05\columnwidth}\raggedright
197\strut
\end{minipage} & \begin{minipage}[t]{0.05\columnwidth}\raggedright
223\strut
\end{minipage} & \begin{minipage}[t]{0.06\columnwidth}\raggedright
131\strut
\end{minipage} & \begin{minipage}[t]{0.06\columnwidth}\raggedright
81\strut
\end{minipage} & \begin{minipage}[t]{0.04\columnwidth}\raggedright
78\strut
\end{minipage} & \begin{minipage}[t]{0.11\columnwidth}\raggedright
339\strut
\end{minipage}\tabularnewline
\begin{minipage}[t]{0.14\columnwidth}\raggedright
Jehovah's Witness\strut
\end{minipage} & \begin{minipage}[t]{0.04\columnwidth}\raggedright
20\strut
\end{minipage} & \begin{minipage}[t]{0.05\columnwidth}\raggedright
27\strut
\end{minipage} & \begin{minipage}[t]{0.05\columnwidth}\raggedright
24\strut
\end{minipage} & \begin{minipage}[t]{0.05\columnwidth}\raggedright
24\strut
\end{minipage} & \begin{minipage}[t]{0.05\columnwidth}\raggedright
21\strut
\end{minipage} & \begin{minipage}[t]{0.05\columnwidth}\raggedright
30\strut
\end{minipage} & \begin{minipage}[t]{0.06\columnwidth}\raggedright
15\strut
\end{minipage} & \begin{minipage}[t]{0.06\columnwidth}\raggedright
11\strut
\end{minipage} & \begin{minipage}[t]{0.04\columnwidth}\raggedright
6\strut
\end{minipage} & \begin{minipage}[t]{0.11\columnwidth}\raggedright
37\strut
\end{minipage}\tabularnewline
\begin{minipage}[t]{0.14\columnwidth}\raggedright
Jewish\strut
\end{minipage} & \begin{minipage}[t]{0.04\columnwidth}\raggedright
19\strut
\end{minipage} & \begin{minipage}[t]{0.05\columnwidth}\raggedright
19\strut
\end{minipage} & \begin{minipage}[t]{0.05\columnwidth}\raggedright
25\strut
\end{minipage} & \begin{minipage}[t]{0.05\columnwidth}\raggedright
25\strut
\end{minipage} & \begin{minipage}[t]{0.05\columnwidth}\raggedright
30\strut
\end{minipage} & \begin{minipage}[t]{0.05\columnwidth}\raggedright
95\strut
\end{minipage} & \begin{minipage}[t]{0.06\columnwidth}\raggedright
69\strut
\end{minipage} & \begin{minipage}[t]{0.06\columnwidth}\raggedright
87\strut
\end{minipage} & \begin{minipage}[t]{0.04\columnwidth}\raggedright
151\strut
\end{minipage} & \begin{minipage}[t]{0.11\columnwidth}\raggedright
162\strut
\end{minipage}\tabularnewline
\begin{minipage}[t]{0.14\columnwidth}\raggedright
Mainline Prot\strut
\end{minipage} & \begin{minipage}[t]{0.04\columnwidth}\raggedright
289\strut
\end{minipage} & \begin{minipage}[t]{0.05\columnwidth}\raggedright
495\strut
\end{minipage} & \begin{minipage}[t]{0.05\columnwidth}\raggedright
619\strut
\end{minipage} & \begin{minipage}[t]{0.05\columnwidth}\raggedright
655\strut
\end{minipage} & \begin{minipage}[t]{0.05\columnwidth}\raggedright
651\strut
\end{minipage} & \begin{minipage}[t]{0.05\columnwidth}\raggedright
1107\strut
\end{minipage} & \begin{minipage}[t]{0.06\columnwidth}\raggedright
939\strut
\end{minipage} & \begin{minipage}[t]{0.06\columnwidth}\raggedright
753\strut
\end{minipage} & \begin{minipage}[t]{0.04\columnwidth}\raggedright
634\strut
\end{minipage} & \begin{minipage}[t]{0.11\columnwidth}\raggedright
1328\strut
\end{minipage}\tabularnewline
\begin{minipage}[t]{0.14\columnwidth}\raggedright
Mormon\strut
\end{minipage} & \begin{minipage}[t]{0.04\columnwidth}\raggedright
29\strut
\end{minipage} & \begin{minipage}[t]{0.05\columnwidth}\raggedright
40\strut
\end{minipage} & \begin{minipage}[t]{0.05\columnwidth}\raggedright
48\strut
\end{minipage} & \begin{minipage}[t]{0.05\columnwidth}\raggedright
51\strut
\end{minipage} & \begin{minipage}[t]{0.05\columnwidth}\raggedright
56\strut
\end{minipage} & \begin{minipage}[t]{0.05\columnwidth}\raggedright
112\strut
\end{minipage} & \begin{minipage}[t]{0.06\columnwidth}\raggedright
85\strut
\end{minipage} & \begin{minipage}[t]{0.06\columnwidth}\raggedright
49\strut
\end{minipage} & \begin{minipage}[t]{0.04\columnwidth}\raggedright
42\strut
\end{minipage} & \begin{minipage}[t]{0.11\columnwidth}\raggedright
69\strut
\end{minipage}\tabularnewline
\begin{minipage}[t]{0.14\columnwidth}\raggedright
Muslim\strut
\end{minipage} & \begin{minipage}[t]{0.04\columnwidth}\raggedright
6\strut
\end{minipage} & \begin{minipage}[t]{0.05\columnwidth}\raggedright
7\strut
\end{minipage} & \begin{minipage}[t]{0.05\columnwidth}\raggedright
9\strut
\end{minipage} & \begin{minipage}[t]{0.05\columnwidth}\raggedright
10\strut
\end{minipage} & \begin{minipage}[t]{0.05\columnwidth}\raggedright
9\strut
\end{minipage} & \begin{minipage}[t]{0.05\columnwidth}\raggedright
23\strut
\end{minipage} & \begin{minipage}[t]{0.06\columnwidth}\raggedright
16\strut
\end{minipage} & \begin{minipage}[t]{0.06\columnwidth}\raggedright
8\strut
\end{minipage} & \begin{minipage}[t]{0.04\columnwidth}\raggedright
6\strut
\end{minipage} & \begin{minipage}[t]{0.11\columnwidth}\raggedright
22\strut
\end{minipage}\tabularnewline
\begin{minipage}[t]{0.14\columnwidth}\raggedright
Orthodox\strut
\end{minipage} & \begin{minipage}[t]{0.04\columnwidth}\raggedright
13\strut
\end{minipage} & \begin{minipage}[t]{0.05\columnwidth}\raggedright
17\strut
\end{minipage} & \begin{minipage}[t]{0.05\columnwidth}\raggedright
23\strut
\end{minipage} & \begin{minipage}[t]{0.05\columnwidth}\raggedright
32\strut
\end{minipage} & \begin{minipage}[t]{0.05\columnwidth}\raggedright
32\strut
\end{minipage} & \begin{minipage}[t]{0.05\columnwidth}\raggedright
47\strut
\end{minipage} & \begin{minipage}[t]{0.06\columnwidth}\raggedright
38\strut
\end{minipage} & \begin{minipage}[t]{0.06\columnwidth}\raggedright
42\strut
\end{minipage} & \begin{minipage}[t]{0.04\columnwidth}\raggedright
46\strut
\end{minipage} & \begin{minipage}[t]{0.11\columnwidth}\raggedright
73\strut
\end{minipage}\tabularnewline
\begin{minipage}[t]{0.14\columnwidth}\raggedright
Other Christian\strut
\end{minipage} & \begin{minipage}[t]{0.04\columnwidth}\raggedright
9\strut
\end{minipage} & \begin{minipage}[t]{0.05\columnwidth}\raggedright
7\strut
\end{minipage} & \begin{minipage}[t]{0.05\columnwidth}\raggedright
11\strut
\end{minipage} & \begin{minipage}[t]{0.05\columnwidth}\raggedright
13\strut
\end{minipage} & \begin{minipage}[t]{0.05\columnwidth}\raggedright
13\strut
\end{minipage} & \begin{minipage}[t]{0.05\columnwidth}\raggedright
14\strut
\end{minipage} & \begin{minipage}[t]{0.06\columnwidth}\raggedright
18\strut
\end{minipage} & \begin{minipage}[t]{0.06\columnwidth}\raggedright
14\strut
\end{minipage} & \begin{minipage}[t]{0.04\columnwidth}\raggedright
12\strut
\end{minipage} & \begin{minipage}[t]{0.11\columnwidth}\raggedright
18\strut
\end{minipage}\tabularnewline
\begin{minipage}[t]{0.14\columnwidth}\raggedright
Other Faiths\strut
\end{minipage} & \begin{minipage}[t]{0.04\columnwidth}\raggedright
20\strut
\end{minipage} & \begin{minipage}[t]{0.05\columnwidth}\raggedright
33\strut
\end{minipage} & \begin{minipage}[t]{0.05\columnwidth}\raggedright
40\strut
\end{minipage} & \begin{minipage}[t]{0.05\columnwidth}\raggedright
46\strut
\end{minipage} & \begin{minipage}[t]{0.05\columnwidth}\raggedright
49\strut
\end{minipage} & \begin{minipage}[t]{0.05\columnwidth}\raggedright
63\strut
\end{minipage} & \begin{minipage}[t]{0.06\columnwidth}\raggedright
46\strut
\end{minipage} & \begin{minipage}[t]{0.06\columnwidth}\raggedright
40\strut
\end{minipage} & \begin{minipage}[t]{0.04\columnwidth}\raggedright
41\strut
\end{minipage} & \begin{minipage}[t]{0.11\columnwidth}\raggedright
71\strut
\end{minipage}\tabularnewline
\begin{minipage}[t]{0.14\columnwidth}\raggedright
Other World Religions\strut
\end{minipage} & \begin{minipage}[t]{0.04\columnwidth}\raggedright
5\strut
\end{minipage} & \begin{minipage}[t]{0.05\columnwidth}\raggedright
2\strut
\end{minipage} & \begin{minipage}[t]{0.05\columnwidth}\raggedright
3\strut
\end{minipage} & \begin{minipage}[t]{0.05\columnwidth}\raggedright
4\strut
\end{minipage} & \begin{minipage}[t]{0.05\columnwidth}\raggedright
2\strut
\end{minipage} & \begin{minipage}[t]{0.05\columnwidth}\raggedright
7\strut
\end{minipage} & \begin{minipage}[t]{0.06\columnwidth}\raggedright
3\strut
\end{minipage} & \begin{minipage}[t]{0.06\columnwidth}\raggedright
4\strut
\end{minipage} & \begin{minipage}[t]{0.04\columnwidth}\raggedright
4\strut
\end{minipage} & \begin{minipage}[t]{0.11\columnwidth}\raggedright
8\strut
\end{minipage}\tabularnewline
\begin{minipage}[t]{0.14\columnwidth}\raggedright
Unaffiliated\strut
\end{minipage} & \begin{minipage}[t]{0.04\columnwidth}\raggedright
217\strut
\end{minipage} & \begin{minipage}[t]{0.05\columnwidth}\raggedright
299\strut
\end{minipage} & \begin{minipage}[t]{0.05\columnwidth}\raggedright
374\strut
\end{minipage} & \begin{minipage}[t]{0.05\columnwidth}\raggedright
365\strut
\end{minipage} & \begin{minipage}[t]{0.05\columnwidth}\raggedright
341\strut
\end{minipage} & \begin{minipage}[t]{0.05\columnwidth}\raggedright
528\strut
\end{minipage} & \begin{minipage}[t]{0.06\columnwidth}\raggedright
407\strut
\end{minipage} & \begin{minipage}[t]{0.06\columnwidth}\raggedright
321\strut
\end{minipage} & \begin{minipage}[t]{0.04\columnwidth}\raggedright
258\strut
\end{minipage} & \begin{minipage}[t]{0.11\columnwidth}\raggedright
597\strut
\end{minipage}\tabularnewline
\bottomrule
\end{longtable}

\begin{Shaded}
\begin{Highlighting}[]
\NormalTok{(api/pivot->longer relig-income (}\KeywordTok{complement}\NormalTok{ #\{}\StringTok{"religion"}\NormalTok{\}))}
\end{Highlighting}
\end{Shaded}

data/relig\_income.csv {[}180 3{]}:

\begin{longtable}[]{@{}lll@{}}
\toprule
religion & :\(column | :\)value &\tabularnewline
\midrule
\endhead
Agnostic & \textless{}\$10k & 27\tabularnewline
Atheist & \textless{}\$10k & 12\tabularnewline
Buddhist & \textless{}\$10k & 27\tabularnewline
Catholic & \textless{}\$10k & 418\tabularnewline
Don't know/refused & \textless{}\$10k & 15\tabularnewline
Evangelical Prot & \textless{}\$10k & 575\tabularnewline
Hindu & \textless{}\$10k & 1\tabularnewline
Historically Black Prot & \textless{}\$10k & 228\tabularnewline
Jehovah's Witness & \textless{}\$10k & 20\tabularnewline
Jewish & \textless{}\$10k & 19\tabularnewline
Mainline Prot & \textless{}\$10k & 289\tabularnewline
Mormon & \textless{}\$10k & 29\tabularnewline
Muslim & \textless{}\$10k & 6\tabularnewline
Orthodox & \textless{}\$10k & 13\tabularnewline
Other Christian & \textless{}\$10k & 9\tabularnewline
Other Faiths & \textless{}\$10k & 20\tabularnewline
Other World Religions & \textless{}\$10k & 5\tabularnewline
Unaffiliated & \textless{}\$10k & 217\tabularnewline
Agnostic & Don't know/refused & 96\tabularnewline
Atheist & Don't know/refused & 76\tabularnewline
Buddhist & Don't know/refused & 54\tabularnewline
Catholic & Don't know/refused & 1489\tabularnewline
Don't know/refused & Don't know/refused & 116\tabularnewline
Evangelical Prot & Don't know/refused & 1529\tabularnewline
Hindu & Don't know/refused & 37\tabularnewline
\bottomrule
\end{longtable}

\begin{center}\rule{0.5\linewidth}{0.5pt}\end{center}

Convert only columns starting with \texttt{"wk"} and pack them into
\texttt{:week} column, values go to \texttt{:rank} column

\begin{Shaded}
\begin{Highlighting}[]
\NormalTok{(}\BuiltInTok{def}\FunctionTok{ bilboard }\NormalTok{(}\KeywordTok{->}\NormalTok{ (api/dataset }\StringTok{"data/billboard.csv.gz"}\NormalTok{)}
\NormalTok{                  (api/drop-columns }\AttributeTok{:type/boolean}\NormalTok{))) }\CommentTok{;; drop some boolean columns, tidyr just skips them}
\end{Highlighting}
\end{Shaded}

\begin{Shaded}
\begin{Highlighting}[]
\NormalTok{(}\KeywordTok{->>}\NormalTok{ bilboard}
\NormalTok{     (api/column-names)}
\NormalTok{     (}\KeywordTok{take} \DecValTok{13}\NormalTok{)}
\NormalTok{     (api/select-columns bilboard))}
\end{Highlighting}
\end{Shaded}

data/billboard.csv.gz {[}317 13{]}:

\begin{longtable}[]{@{}lllllllllllll@{}}
\toprule
artist & track & date.entered & wk1 & wk2 & wk3 & wk4 & wk5 & wk6 & wk7
& wk8 & wk9 & wk10\tabularnewline
\midrule
\endhead
2 Pac & Baby Don't Cry (Keep\ldots{} & 2000-02-26 & 87 & 82 & 72 & 77 &
87 & 94 & 99 & & &\tabularnewline
2Ge+her & The Hardest Part Of \ldots{} & 2000-09-02 & 91 & 87 & 92 & & &
& & & &\tabularnewline
3 Doors Down & Kryptonite & 2000-04-08 & 81 & 70 & 68 & 67 & 66 & 57 &
54 & 53 & 51 & 51\tabularnewline
3 Doors Down & Loser & 2000-10-21 & 76 & 76 & 72 & 69 & 67 & 65 & 55 &
59 & 62 & 61\tabularnewline
504 Boyz & Wobble Wobble & 2000-04-15 & 57 & 34 & 25 & 17 & 17 & 31 & 36
& 49 & 53 & 57\tabularnewline
98\^{}0 & Give Me Just One Nig\ldots{} & 2000-08-19 & 51 & 39 & 34 & 26
& 26 & 19 & 2 & 2 & 3 & 6\tabularnewline
A*Teens & Dancing Queen & 2000-07-08 & 97 & 97 & 96 & 95 & 100 & & & &
&\tabularnewline
Aaliyah & I Don't Wanna & 2000-01-29 & 84 & 62 & 51 & 41 & 38 & 35 & 35
& 38 & 38 & 36\tabularnewline
Aaliyah & Try Again & 2000-03-18 & 59 & 53 & 38 & 28 & 21 & 18 & 16 & 14
& 12 & 10\tabularnewline
Adams, Yolanda & Open My Heart & 2000-08-26 & 76 & 76 & 74 & 69 & 68 &
67 & 61 & 58 & 57 & 59\tabularnewline
Adkins, Trace & More & 2000-04-29 & 84 & 84 & 75 & 73 & 73 & 69 & 68 &
65 & 73 & 83\tabularnewline
Aguilera, Christina & Come On Over Baby (A\ldots{} & 2000-08-05 & 57 &
47 & 45 & 29 & 23 & 18 & 11 & 9 & 9 & 11\tabularnewline
Aguilera, Christina & I Turn To You & 2000-04-15 & 50 & 39 & 30 & 28 &
21 & 19 & 20 & 17 & 17 & 17\tabularnewline
Aguilera, Christina & What A Girl Wants & 1999-11-27 & 71 & 51 & 28 & 18
& 13 & 13 & 11 & 1 & 1 & 2\tabularnewline
Alice Deejay & Better Off Alone & 2000-04-08 & 79 & 65 & 53 & 48 & 45 &
36 & 34 & 29 & 27 & 30\tabularnewline
Allan, Gary & Smoke Rings In The D\ldots{} & 2000-01-22 & 80 & 78 & 76 &
77 & 92 & & & & &\tabularnewline
Amber & Sexual & 1999-07-17 & 99 & 99 & 96 & 96 & 100 & 93 & 93 & 96 &
&\tabularnewline
Anastacia & I'm Outta Love & 2000-04-01 & 92 & & & 95 & & & & &
&\tabularnewline
Anthony, Marc & My Baby You & 2000-09-16 & 82 & 76 & 76 & 70 & 82 & 81 &
74 & 80 & 76 & 76\tabularnewline
Anthony, Marc & You Sang To Me & 2000-02-26 & 77 & 54 & 50 & 43 & 30 &
27 & 21 & 18 & 15 & 13\tabularnewline
Avant & My First Love & 2000-11-04 & 70 & 62 & 56 & 43 & 39 & 33 & 26 &
26 & 26 & 31\tabularnewline
Avant & Separated & 2000-04-29 & 62 & 32 & 30 & 23 & 26 & 30 & 35 & 32 &
32 & 25\tabularnewline
BBMak & Back Here & 2000-04-29 & 99 & 86 & 60 & 52 & 38 & 34 & 28 & 21 &
18 & 18\tabularnewline
Backstreet Boys, The & Shape Of My Heart & 2000-10-14 & 39 & 25 & 24 &
15 & 12 & 12 & 10 & 9 & 10 & 12\tabularnewline
Backstreet Boys, The & Show Me The Meaning \ldots{} & 2000-01-01 & 74 &
62 & 55 & 25 & 16 & 14 & 12 & 10 & 12 & 9\tabularnewline
\bottomrule
\end{longtable}

\begin{Shaded}
\begin{Highlighting}[]
\NormalTok{(api/pivot->longer bilboard #(clojure.string/starts-with? }\VariableTok{%} \StringTok{"wk"}\NormalTok{) \{}\AttributeTok{:target-columns} \AttributeTok{:week}
                                                                   \AttributeTok{:value-column-name} \AttributeTok{:rank}\NormalTok{\})}
\end{Highlighting}
\end{Shaded}

data/billboard.csv.gz {[}5307 5{]}:

\begin{longtable}[]{@{}lllll@{}}
\toprule
artist & track & date.entered & :week & :rank\tabularnewline
\midrule
\endhead
3 Doors Down & Kryptonite & 2000-04-08 & wk35 & 4\tabularnewline
Braxton, Toni & He Wasn't Man Enough & 2000-03-18 & wk35 &
34\tabularnewline
Creed & Higher & 1999-09-11 & wk35 & 22\tabularnewline
Creed & With Arms Wide Open & 2000-05-13 & wk35 & 5\tabularnewline
Hill, Faith & Breathe & 1999-11-06 & wk35 & 8\tabularnewline
Joe & I Wanna Know & 2000-01-01 & wk35 & 5\tabularnewline
Lonestar & Amazed & 1999-06-05 & wk35 & 14\tabularnewline
Vertical Horizon & Everything You Want & 2000-01-22 & wk35 &
27\tabularnewline
matchbox twenty & Bent & 2000-04-29 & wk35 & 33\tabularnewline
Creed & Higher & 1999-09-11 & wk55 & 21\tabularnewline
Lonestar & Amazed & 1999-06-05 & wk55 & 22\tabularnewline
3 Doors Down & Kryptonite & 2000-04-08 & wk19 & 18\tabularnewline
3 Doors Down & Loser & 2000-10-21 & wk19 & 73\tabularnewline
98\^{}0 & Give Me Just One Nig\ldots{} & 2000-08-19 & wk19 &
93\tabularnewline
Aaliyah & I Don't Wanna & 2000-01-29 & wk19 & 83\tabularnewline
Aaliyah & Try Again & 2000-03-18 & wk19 & 3\tabularnewline
Adams, Yolanda & Open My Heart & 2000-08-26 & wk19 & 79\tabularnewline
Aguilera, Christina & Come On Over Baby (A\ldots{} & 2000-08-05 & wk19 &
23\tabularnewline
Aguilera, Christina & I Turn To You & 2000-04-15 & wk19 &
29\tabularnewline
Aguilera, Christina & What A Girl Wants & 1999-11-27 & wk19 &
18\tabularnewline
Alice Deejay & Better Off Alone & 2000-04-08 & wk19 & 79\tabularnewline
Amber & Sexual & 1999-07-17 & wk19 & 95\tabularnewline
Anthony, Marc & My Baby You & 2000-09-16 & wk19 & 91\tabularnewline
Anthony, Marc & You Sang To Me & 2000-02-26 & wk19 & 9\tabularnewline
Avant & My First Love & 2000-11-04 & wk19 & 81\tabularnewline
\bottomrule
\end{longtable}

\begin{center}\rule{0.5\linewidth}{0.5pt}\end{center}

We can create numerical column out of column names

\begin{Shaded}
\begin{Highlighting}[]
\NormalTok{(api/pivot->longer bilboard #(clojure.string/starts-with? }\VariableTok{%} \StringTok{"wk"}\NormalTok{) \{}\AttributeTok{:target-columns} \AttributeTok{:week}
                                                                   \AttributeTok{:value-column-name} \AttributeTok{:rank}
                                                                   \AttributeTok{:splitter} \SpecialStringTok{#"wk(.*)"}
                                                                   \AttributeTok{:datatypes}\NormalTok{ \{}\AttributeTok{:week} \AttributeTok{:int16}\NormalTok{\}\})}
\end{Highlighting}
\end{Shaded}

data/billboard.csv.gz {[}5307 5{]}:

\begin{longtable}[]{@{}lllll@{}}
\toprule
artist & track & date.entered & :week & :rank\tabularnewline
\midrule
\endhead
3 Doors Down & Kryptonite & 2000-04-08 & 46 & 21\tabularnewline
Creed & Higher & 1999-09-11 & 46 & 7\tabularnewline
Creed & With Arms Wide Open & 2000-05-13 & 46 & 37\tabularnewline
Hill, Faith & Breathe & 1999-11-06 & 46 & 31\tabularnewline
Lonestar & Amazed & 1999-06-05 & 46 & 5\tabularnewline
3 Doors Down & Kryptonite & 2000-04-08 & 51 & 42\tabularnewline
Creed & Higher & 1999-09-11 & 51 & 14\tabularnewline
Hill, Faith & Breathe & 1999-11-06 & 51 & 49\tabularnewline
Lonestar & Amazed & 1999-06-05 & 51 & 12\tabularnewline
2 Pac & Baby Don't Cry (Keep\ldots{} & 2000-02-26 & 6 &
94\tabularnewline
3 Doors Down & Kryptonite & 2000-04-08 & 6 & 57\tabularnewline
3 Doors Down & Loser & 2000-10-21 & 6 & 65\tabularnewline
504 Boyz & Wobble Wobble & 2000-04-15 & 6 & 31\tabularnewline
98\^{}0 & Give Me Just One Nig\ldots{} & 2000-08-19 & 6 &
19\tabularnewline
Aaliyah & I Don't Wanna & 2000-01-29 & 6 & 35\tabularnewline
Aaliyah & Try Again & 2000-03-18 & 6 & 18\tabularnewline
Adams, Yolanda & Open My Heart & 2000-08-26 & 6 & 67\tabularnewline
Adkins, Trace & More & 2000-04-29 & 6 & 69\tabularnewline
Aguilera, Christina & Come On Over Baby (A\ldots{} & 2000-08-05 & 6 &
18\tabularnewline
Aguilera, Christina & I Turn To You & 2000-04-15 & 6 & 19\tabularnewline
Aguilera, Christina & What A Girl Wants & 1999-11-27 & 6 &
13\tabularnewline
Alice Deejay & Better Off Alone & 2000-04-08 & 6 & 36\tabularnewline
Amber & Sexual & 1999-07-17 & 6 & 93\tabularnewline
Anthony, Marc & My Baby You & 2000-09-16 & 6 & 81\tabularnewline
Anthony, Marc & You Sang To Me & 2000-02-26 & 6 & 27\tabularnewline
\bottomrule
\end{longtable}

\begin{center}\rule{0.5\linewidth}{0.5pt}\end{center}

When column names contain observation data, such column names can be
splitted and data can be restored into separate columns.

\begin{Shaded}
\begin{Highlighting}[]
\NormalTok{(}\BuiltInTok{def}\FunctionTok{ who }\NormalTok{(api/dataset }\StringTok{"data/who.csv.gz"}\NormalTok{))}
\end{Highlighting}
\end{Shaded}

\begin{Shaded}
\begin{Highlighting}[]
\NormalTok{(}\KeywordTok{->>}\NormalTok{ who}
\NormalTok{     (api/column-names)}
\NormalTok{     (}\KeywordTok{take} \DecValTok{10}\NormalTok{)}
\NormalTok{     (api/select-columns who))}
\end{Highlighting}
\end{Shaded}

data/who.csv.gz {[}7240 10{]}:

\begin{longtable}[]{@{}llllllllll@{}}
\toprule
\begin{minipage}[b]{0.08\columnwidth}\raggedright
country\strut
\end{minipage} & \begin{minipage}[b]{0.04\columnwidth}\raggedright
iso2\strut
\end{minipage} & \begin{minipage}[b]{0.04\columnwidth}\raggedright
iso3\strut
\end{minipage} & \begin{minipage}[b]{0.04\columnwidth}\raggedright
year\strut
\end{minipage} & \begin{minipage}[b]{0.08\columnwidth}\raggedright
new\_sp\_m014\strut
\end{minipage} & \begin{minipage}[b]{0.09\columnwidth}\raggedright
new\_sp\_m1524\strut
\end{minipage} & \begin{minipage}[b]{0.09\columnwidth}\raggedright
new\_sp\_m2534\strut
\end{minipage} & \begin{minipage}[b]{0.09\columnwidth}\raggedright
new\_sp\_m3544\strut
\end{minipage} & \begin{minipage}[b]{0.09\columnwidth}\raggedright
new\_sp\_m4554\strut
\end{minipage} & \begin{minipage}[b]{0.09\columnwidth}\raggedright
new\_sp\_m5564\strut
\end{minipage}\tabularnewline
\midrule
\endhead
\begin{minipage}[t]{0.08\columnwidth}\raggedright
Afghanistan\strut
\end{minipage} & \begin{minipage}[t]{0.04\columnwidth}\raggedright
AF\strut
\end{minipage} & \begin{minipage}[t]{0.04\columnwidth}\raggedright
AFG\strut
\end{minipage} & \begin{minipage}[t]{0.04\columnwidth}\raggedright
1980\strut
\end{minipage} & \begin{minipage}[t]{0.08\columnwidth}\raggedright
\strut
\end{minipage} & \begin{minipage}[t]{0.09\columnwidth}\raggedright
\strut
\end{minipage} & \begin{minipage}[t]{0.09\columnwidth}\raggedright
\strut
\end{minipage} & \begin{minipage}[t]{0.09\columnwidth}\raggedright
\strut
\end{minipage} & \begin{minipage}[t]{0.09\columnwidth}\raggedright
\strut
\end{minipage} & \begin{minipage}[t]{0.09\columnwidth}\raggedright
\strut
\end{minipage}\tabularnewline
\begin{minipage}[t]{0.08\columnwidth}\raggedright
Afghanistan\strut
\end{minipage} & \begin{minipage}[t]{0.04\columnwidth}\raggedright
AF\strut
\end{minipage} & \begin{minipage}[t]{0.04\columnwidth}\raggedright
AFG\strut
\end{minipage} & \begin{minipage}[t]{0.04\columnwidth}\raggedright
1981\strut
\end{minipage} & \begin{minipage}[t]{0.08\columnwidth}\raggedright
\strut
\end{minipage} & \begin{minipage}[t]{0.09\columnwidth}\raggedright
\strut
\end{minipage} & \begin{minipage}[t]{0.09\columnwidth}\raggedright
\strut
\end{minipage} & \begin{minipage}[t]{0.09\columnwidth}\raggedright
\strut
\end{minipage} & \begin{minipage}[t]{0.09\columnwidth}\raggedright
\strut
\end{minipage} & \begin{minipage}[t]{0.09\columnwidth}\raggedright
\strut
\end{minipage}\tabularnewline
\begin{minipage}[t]{0.08\columnwidth}\raggedright
Afghanistan\strut
\end{minipage} & \begin{minipage}[t]{0.04\columnwidth}\raggedright
AF\strut
\end{minipage} & \begin{minipage}[t]{0.04\columnwidth}\raggedright
AFG\strut
\end{minipage} & \begin{minipage}[t]{0.04\columnwidth}\raggedright
1982\strut
\end{minipage} & \begin{minipage}[t]{0.08\columnwidth}\raggedright
\strut
\end{minipage} & \begin{minipage}[t]{0.09\columnwidth}\raggedright
\strut
\end{minipage} & \begin{minipage}[t]{0.09\columnwidth}\raggedright
\strut
\end{minipage} & \begin{minipage}[t]{0.09\columnwidth}\raggedright
\strut
\end{minipage} & \begin{minipage}[t]{0.09\columnwidth}\raggedright
\strut
\end{minipage} & \begin{minipage}[t]{0.09\columnwidth}\raggedright
\strut
\end{minipage}\tabularnewline
\begin{minipage}[t]{0.08\columnwidth}\raggedright
Afghanistan\strut
\end{minipage} & \begin{minipage}[t]{0.04\columnwidth}\raggedright
AF\strut
\end{minipage} & \begin{minipage}[t]{0.04\columnwidth}\raggedright
AFG\strut
\end{minipage} & \begin{minipage}[t]{0.04\columnwidth}\raggedright
1983\strut
\end{minipage} & \begin{minipage}[t]{0.08\columnwidth}\raggedright
\strut
\end{minipage} & \begin{minipage}[t]{0.09\columnwidth}\raggedright
\strut
\end{minipage} & \begin{minipage}[t]{0.09\columnwidth}\raggedright
\strut
\end{minipage} & \begin{minipage}[t]{0.09\columnwidth}\raggedright
\strut
\end{minipage} & \begin{minipage}[t]{0.09\columnwidth}\raggedright
\strut
\end{minipage} & \begin{minipage}[t]{0.09\columnwidth}\raggedright
\strut
\end{minipage}\tabularnewline
\begin{minipage}[t]{0.08\columnwidth}\raggedright
Afghanistan\strut
\end{minipage} & \begin{minipage}[t]{0.04\columnwidth}\raggedright
AF\strut
\end{minipage} & \begin{minipage}[t]{0.04\columnwidth}\raggedright
AFG\strut
\end{minipage} & \begin{minipage}[t]{0.04\columnwidth}\raggedright
1984\strut
\end{minipage} & \begin{minipage}[t]{0.08\columnwidth}\raggedright
\strut
\end{minipage} & \begin{minipage}[t]{0.09\columnwidth}\raggedright
\strut
\end{minipage} & \begin{minipage}[t]{0.09\columnwidth}\raggedright
\strut
\end{minipage} & \begin{minipage}[t]{0.09\columnwidth}\raggedright
\strut
\end{minipage} & \begin{minipage}[t]{0.09\columnwidth}\raggedright
\strut
\end{minipage} & \begin{minipage}[t]{0.09\columnwidth}\raggedright
\strut
\end{minipage}\tabularnewline
\begin{minipage}[t]{0.08\columnwidth}\raggedright
Afghanistan\strut
\end{minipage} & \begin{minipage}[t]{0.04\columnwidth}\raggedright
AF\strut
\end{minipage} & \begin{minipage}[t]{0.04\columnwidth}\raggedright
AFG\strut
\end{minipage} & \begin{minipage}[t]{0.04\columnwidth}\raggedright
1985\strut
\end{minipage} & \begin{minipage}[t]{0.08\columnwidth}\raggedright
\strut
\end{minipage} & \begin{minipage}[t]{0.09\columnwidth}\raggedright
\strut
\end{minipage} & \begin{minipage}[t]{0.09\columnwidth}\raggedright
\strut
\end{minipage} & \begin{minipage}[t]{0.09\columnwidth}\raggedright
\strut
\end{minipage} & \begin{minipage}[t]{0.09\columnwidth}\raggedright
\strut
\end{minipage} & \begin{minipage}[t]{0.09\columnwidth}\raggedright
\strut
\end{minipage}\tabularnewline
\begin{minipage}[t]{0.08\columnwidth}\raggedright
Afghanistan\strut
\end{minipage} & \begin{minipage}[t]{0.04\columnwidth}\raggedright
AF\strut
\end{minipage} & \begin{minipage}[t]{0.04\columnwidth}\raggedright
AFG\strut
\end{minipage} & \begin{minipage}[t]{0.04\columnwidth}\raggedright
1986\strut
\end{minipage} & \begin{minipage}[t]{0.08\columnwidth}\raggedright
\strut
\end{minipage} & \begin{minipage}[t]{0.09\columnwidth}\raggedright
\strut
\end{minipage} & \begin{minipage}[t]{0.09\columnwidth}\raggedright
\strut
\end{minipage} & \begin{minipage}[t]{0.09\columnwidth}\raggedright
\strut
\end{minipage} & \begin{minipage}[t]{0.09\columnwidth}\raggedright
\strut
\end{minipage} & \begin{minipage}[t]{0.09\columnwidth}\raggedright
\strut
\end{minipage}\tabularnewline
\begin{minipage}[t]{0.08\columnwidth}\raggedright
Afghanistan\strut
\end{minipage} & \begin{minipage}[t]{0.04\columnwidth}\raggedright
AF\strut
\end{minipage} & \begin{minipage}[t]{0.04\columnwidth}\raggedright
AFG\strut
\end{minipage} & \begin{minipage}[t]{0.04\columnwidth}\raggedright
1987\strut
\end{minipage} & \begin{minipage}[t]{0.08\columnwidth}\raggedright
\strut
\end{minipage} & \begin{minipage}[t]{0.09\columnwidth}\raggedright
\strut
\end{minipage} & \begin{minipage}[t]{0.09\columnwidth}\raggedright
\strut
\end{minipage} & \begin{minipage}[t]{0.09\columnwidth}\raggedright
\strut
\end{minipage} & \begin{minipage}[t]{0.09\columnwidth}\raggedright
\strut
\end{minipage} & \begin{minipage}[t]{0.09\columnwidth}\raggedright
\strut
\end{minipage}\tabularnewline
\begin{minipage}[t]{0.08\columnwidth}\raggedright
Afghanistan\strut
\end{minipage} & \begin{minipage}[t]{0.04\columnwidth}\raggedright
AF\strut
\end{minipage} & \begin{minipage}[t]{0.04\columnwidth}\raggedright
AFG\strut
\end{minipage} & \begin{minipage}[t]{0.04\columnwidth}\raggedright
1988\strut
\end{minipage} & \begin{minipage}[t]{0.08\columnwidth}\raggedright
\strut
\end{minipage} & \begin{minipage}[t]{0.09\columnwidth}\raggedright
\strut
\end{minipage} & \begin{minipage}[t]{0.09\columnwidth}\raggedright
\strut
\end{minipage} & \begin{minipage}[t]{0.09\columnwidth}\raggedright
\strut
\end{minipage} & \begin{minipage}[t]{0.09\columnwidth}\raggedright
\strut
\end{minipage} & \begin{minipage}[t]{0.09\columnwidth}\raggedright
\strut
\end{minipage}\tabularnewline
\begin{minipage}[t]{0.08\columnwidth}\raggedright
Afghanistan\strut
\end{minipage} & \begin{minipage}[t]{0.04\columnwidth}\raggedright
AF\strut
\end{minipage} & \begin{minipage}[t]{0.04\columnwidth}\raggedright
AFG\strut
\end{minipage} & \begin{minipage}[t]{0.04\columnwidth}\raggedright
1989\strut
\end{minipage} & \begin{minipage}[t]{0.08\columnwidth}\raggedright
\strut
\end{minipage} & \begin{minipage}[t]{0.09\columnwidth}\raggedright
\strut
\end{minipage} & \begin{minipage}[t]{0.09\columnwidth}\raggedright
\strut
\end{minipage} & \begin{minipage}[t]{0.09\columnwidth}\raggedright
\strut
\end{minipage} & \begin{minipage}[t]{0.09\columnwidth}\raggedright
\strut
\end{minipage} & \begin{minipage}[t]{0.09\columnwidth}\raggedright
\strut
\end{minipage}\tabularnewline
\begin{minipage}[t]{0.08\columnwidth}\raggedright
Afghanistan\strut
\end{minipage} & \begin{minipage}[t]{0.04\columnwidth}\raggedright
AF\strut
\end{minipage} & \begin{minipage}[t]{0.04\columnwidth}\raggedright
AFG\strut
\end{minipage} & \begin{minipage}[t]{0.04\columnwidth}\raggedright
1990\strut
\end{minipage} & \begin{minipage}[t]{0.08\columnwidth}\raggedright
\strut
\end{minipage} & \begin{minipage}[t]{0.09\columnwidth}\raggedright
\strut
\end{minipage} & \begin{minipage}[t]{0.09\columnwidth}\raggedright
\strut
\end{minipage} & \begin{minipage}[t]{0.09\columnwidth}\raggedright
\strut
\end{minipage} & \begin{minipage}[t]{0.09\columnwidth}\raggedright
\strut
\end{minipage} & \begin{minipage}[t]{0.09\columnwidth}\raggedright
\strut
\end{minipage}\tabularnewline
\begin{minipage}[t]{0.08\columnwidth}\raggedright
Afghanistan\strut
\end{minipage} & \begin{minipage}[t]{0.04\columnwidth}\raggedright
AF\strut
\end{minipage} & \begin{minipage}[t]{0.04\columnwidth}\raggedright
AFG\strut
\end{minipage} & \begin{minipage}[t]{0.04\columnwidth}\raggedright
1991\strut
\end{minipage} & \begin{minipage}[t]{0.08\columnwidth}\raggedright
\strut
\end{minipage} & \begin{minipage}[t]{0.09\columnwidth}\raggedright
\strut
\end{minipage} & \begin{minipage}[t]{0.09\columnwidth}\raggedright
\strut
\end{minipage} & \begin{minipage}[t]{0.09\columnwidth}\raggedright
\strut
\end{minipage} & \begin{minipage}[t]{0.09\columnwidth}\raggedright
\strut
\end{minipage} & \begin{minipage}[t]{0.09\columnwidth}\raggedright
\strut
\end{minipage}\tabularnewline
\begin{minipage}[t]{0.08\columnwidth}\raggedright
Afghanistan\strut
\end{minipage} & \begin{minipage}[t]{0.04\columnwidth}\raggedright
AF\strut
\end{minipage} & \begin{minipage}[t]{0.04\columnwidth}\raggedright
AFG\strut
\end{minipage} & \begin{minipage}[t]{0.04\columnwidth}\raggedright
1992\strut
\end{minipage} & \begin{minipage}[t]{0.08\columnwidth}\raggedright
\strut
\end{minipage} & \begin{minipage}[t]{0.09\columnwidth}\raggedright
\strut
\end{minipage} & \begin{minipage}[t]{0.09\columnwidth}\raggedright
\strut
\end{minipage} & \begin{minipage}[t]{0.09\columnwidth}\raggedright
\strut
\end{minipage} & \begin{minipage}[t]{0.09\columnwidth}\raggedright
\strut
\end{minipage} & \begin{minipage}[t]{0.09\columnwidth}\raggedright
\strut
\end{minipage}\tabularnewline
\begin{minipage}[t]{0.08\columnwidth}\raggedright
Afghanistan\strut
\end{minipage} & \begin{minipage}[t]{0.04\columnwidth}\raggedright
AF\strut
\end{minipage} & \begin{minipage}[t]{0.04\columnwidth}\raggedright
AFG\strut
\end{minipage} & \begin{minipage}[t]{0.04\columnwidth}\raggedright
1993\strut
\end{minipage} & \begin{minipage}[t]{0.08\columnwidth}\raggedright
\strut
\end{minipage} & \begin{minipage}[t]{0.09\columnwidth}\raggedright
\strut
\end{minipage} & \begin{minipage}[t]{0.09\columnwidth}\raggedright
\strut
\end{minipage} & \begin{minipage}[t]{0.09\columnwidth}\raggedright
\strut
\end{minipage} & \begin{minipage}[t]{0.09\columnwidth}\raggedright
\strut
\end{minipage} & \begin{minipage}[t]{0.09\columnwidth}\raggedright
\strut
\end{minipage}\tabularnewline
\begin{minipage}[t]{0.08\columnwidth}\raggedright
Afghanistan\strut
\end{minipage} & \begin{minipage}[t]{0.04\columnwidth}\raggedright
AF\strut
\end{minipage} & \begin{minipage}[t]{0.04\columnwidth}\raggedright
AFG\strut
\end{minipage} & \begin{minipage}[t]{0.04\columnwidth}\raggedright
1994\strut
\end{minipage} & \begin{minipage}[t]{0.08\columnwidth}\raggedright
\strut
\end{minipage} & \begin{minipage}[t]{0.09\columnwidth}\raggedright
\strut
\end{minipage} & \begin{minipage}[t]{0.09\columnwidth}\raggedright
\strut
\end{minipage} & \begin{minipage}[t]{0.09\columnwidth}\raggedright
\strut
\end{minipage} & \begin{minipage}[t]{0.09\columnwidth}\raggedright
\strut
\end{minipage} & \begin{minipage}[t]{0.09\columnwidth}\raggedright
\strut
\end{minipage}\tabularnewline
\begin{minipage}[t]{0.08\columnwidth}\raggedright
Afghanistan\strut
\end{minipage} & \begin{minipage}[t]{0.04\columnwidth}\raggedright
AF\strut
\end{minipage} & \begin{minipage}[t]{0.04\columnwidth}\raggedright
AFG\strut
\end{minipage} & \begin{minipage}[t]{0.04\columnwidth}\raggedright
1995\strut
\end{minipage} & \begin{minipage}[t]{0.08\columnwidth}\raggedright
\strut
\end{minipage} & \begin{minipage}[t]{0.09\columnwidth}\raggedright
\strut
\end{minipage} & \begin{minipage}[t]{0.09\columnwidth}\raggedright
\strut
\end{minipage} & \begin{minipage}[t]{0.09\columnwidth}\raggedright
\strut
\end{minipage} & \begin{minipage}[t]{0.09\columnwidth}\raggedright
\strut
\end{minipage} & \begin{minipage}[t]{0.09\columnwidth}\raggedright
\strut
\end{minipage}\tabularnewline
\begin{minipage}[t]{0.08\columnwidth}\raggedright
Afghanistan\strut
\end{minipage} & \begin{minipage}[t]{0.04\columnwidth}\raggedright
AF\strut
\end{minipage} & \begin{minipage}[t]{0.04\columnwidth}\raggedright
AFG\strut
\end{minipage} & \begin{minipage}[t]{0.04\columnwidth}\raggedright
1996\strut
\end{minipage} & \begin{minipage}[t]{0.08\columnwidth}\raggedright
\strut
\end{minipage} & \begin{minipage}[t]{0.09\columnwidth}\raggedright
\strut
\end{minipage} & \begin{minipage}[t]{0.09\columnwidth}\raggedright
\strut
\end{minipage} & \begin{minipage}[t]{0.09\columnwidth}\raggedright
\strut
\end{minipage} & \begin{minipage}[t]{0.09\columnwidth}\raggedright
\strut
\end{minipage} & \begin{minipage}[t]{0.09\columnwidth}\raggedright
\strut
\end{minipage}\tabularnewline
\begin{minipage}[t]{0.08\columnwidth}\raggedright
Afghanistan\strut
\end{minipage} & \begin{minipage}[t]{0.04\columnwidth}\raggedright
AF\strut
\end{minipage} & \begin{minipage}[t]{0.04\columnwidth}\raggedright
AFG\strut
\end{minipage} & \begin{minipage}[t]{0.04\columnwidth}\raggedright
1997\strut
\end{minipage} & \begin{minipage}[t]{0.08\columnwidth}\raggedright
0\strut
\end{minipage} & \begin{minipage}[t]{0.09\columnwidth}\raggedright
10\strut
\end{minipage} & \begin{minipage}[t]{0.09\columnwidth}\raggedright
6\strut
\end{minipage} & \begin{minipage}[t]{0.09\columnwidth}\raggedright
3\strut
\end{minipage} & \begin{minipage}[t]{0.09\columnwidth}\raggedright
5\strut
\end{minipage} & \begin{minipage}[t]{0.09\columnwidth}\raggedright
2\strut
\end{minipage}\tabularnewline
\begin{minipage}[t]{0.08\columnwidth}\raggedright
Afghanistan\strut
\end{minipage} & \begin{minipage}[t]{0.04\columnwidth}\raggedright
AF\strut
\end{minipage} & \begin{minipage}[t]{0.04\columnwidth}\raggedright
AFG\strut
\end{minipage} & \begin{minipage}[t]{0.04\columnwidth}\raggedright
1998\strut
\end{minipage} & \begin{minipage}[t]{0.08\columnwidth}\raggedright
30\strut
\end{minipage} & \begin{minipage}[t]{0.09\columnwidth}\raggedright
129\strut
\end{minipage} & \begin{minipage}[t]{0.09\columnwidth}\raggedright
128\strut
\end{minipage} & \begin{minipage}[t]{0.09\columnwidth}\raggedright
90\strut
\end{minipage} & \begin{minipage}[t]{0.09\columnwidth}\raggedright
89\strut
\end{minipage} & \begin{minipage}[t]{0.09\columnwidth}\raggedright
64\strut
\end{minipage}\tabularnewline
\begin{minipage}[t]{0.08\columnwidth}\raggedright
Afghanistan\strut
\end{minipage} & \begin{minipage}[t]{0.04\columnwidth}\raggedright
AF\strut
\end{minipage} & \begin{minipage}[t]{0.04\columnwidth}\raggedright
AFG\strut
\end{minipage} & \begin{minipage}[t]{0.04\columnwidth}\raggedright
1999\strut
\end{minipage} & \begin{minipage}[t]{0.08\columnwidth}\raggedright
8\strut
\end{minipage} & \begin{minipage}[t]{0.09\columnwidth}\raggedright
55\strut
\end{minipage} & \begin{minipage}[t]{0.09\columnwidth}\raggedright
55\strut
\end{minipage} & \begin{minipage}[t]{0.09\columnwidth}\raggedright
47\strut
\end{minipage} & \begin{minipage}[t]{0.09\columnwidth}\raggedright
34\strut
\end{minipage} & \begin{minipage}[t]{0.09\columnwidth}\raggedright
21\strut
\end{minipage}\tabularnewline
\begin{minipage}[t]{0.08\columnwidth}\raggedright
Afghanistan\strut
\end{minipage} & \begin{minipage}[t]{0.04\columnwidth}\raggedright
AF\strut
\end{minipage} & \begin{minipage}[t]{0.04\columnwidth}\raggedright
AFG\strut
\end{minipage} & \begin{minipage}[t]{0.04\columnwidth}\raggedright
2000\strut
\end{minipage} & \begin{minipage}[t]{0.08\columnwidth}\raggedright
52\strut
\end{minipage} & \begin{minipage}[t]{0.09\columnwidth}\raggedright
228\strut
\end{minipage} & \begin{minipage}[t]{0.09\columnwidth}\raggedright
183\strut
\end{minipage} & \begin{minipage}[t]{0.09\columnwidth}\raggedright
149\strut
\end{minipage} & \begin{minipage}[t]{0.09\columnwidth}\raggedright
129\strut
\end{minipage} & \begin{minipage}[t]{0.09\columnwidth}\raggedright
94\strut
\end{minipage}\tabularnewline
\begin{minipage}[t]{0.08\columnwidth}\raggedright
Afghanistan\strut
\end{minipage} & \begin{minipage}[t]{0.04\columnwidth}\raggedright
AF\strut
\end{minipage} & \begin{minipage}[t]{0.04\columnwidth}\raggedright
AFG\strut
\end{minipage} & \begin{minipage}[t]{0.04\columnwidth}\raggedright
2001\strut
\end{minipage} & \begin{minipage}[t]{0.08\columnwidth}\raggedright
129\strut
\end{minipage} & \begin{minipage}[t]{0.09\columnwidth}\raggedright
379\strut
\end{minipage} & \begin{minipage}[t]{0.09\columnwidth}\raggedright
349\strut
\end{minipage} & \begin{minipage}[t]{0.09\columnwidth}\raggedright
274\strut
\end{minipage} & \begin{minipage}[t]{0.09\columnwidth}\raggedright
204\strut
\end{minipage} & \begin{minipage}[t]{0.09\columnwidth}\raggedright
139\strut
\end{minipage}\tabularnewline
\begin{minipage}[t]{0.08\columnwidth}\raggedright
Afghanistan\strut
\end{minipage} & \begin{minipage}[t]{0.04\columnwidth}\raggedright
AF\strut
\end{minipage} & \begin{minipage}[t]{0.04\columnwidth}\raggedright
AFG\strut
\end{minipage} & \begin{minipage}[t]{0.04\columnwidth}\raggedright
2002\strut
\end{minipage} & \begin{minipage}[t]{0.08\columnwidth}\raggedright
90\strut
\end{minipage} & \begin{minipage}[t]{0.09\columnwidth}\raggedright
476\strut
\end{minipage} & \begin{minipage}[t]{0.09\columnwidth}\raggedright
481\strut
\end{minipage} & \begin{minipage}[t]{0.09\columnwidth}\raggedright
368\strut
\end{minipage} & \begin{minipage}[t]{0.09\columnwidth}\raggedright
246\strut
\end{minipage} & \begin{minipage}[t]{0.09\columnwidth}\raggedright
241\strut
\end{minipage}\tabularnewline
\begin{minipage}[t]{0.08\columnwidth}\raggedright
Afghanistan\strut
\end{minipage} & \begin{minipage}[t]{0.04\columnwidth}\raggedright
AF\strut
\end{minipage} & \begin{minipage}[t]{0.04\columnwidth}\raggedright
AFG\strut
\end{minipage} & \begin{minipage}[t]{0.04\columnwidth}\raggedright
2003\strut
\end{minipage} & \begin{minipage}[t]{0.08\columnwidth}\raggedright
127\strut
\end{minipage} & \begin{minipage}[t]{0.09\columnwidth}\raggedright
511\strut
\end{minipage} & \begin{minipage}[t]{0.09\columnwidth}\raggedright
436\strut
\end{minipage} & \begin{minipage}[t]{0.09\columnwidth}\raggedright
284\strut
\end{minipage} & \begin{minipage}[t]{0.09\columnwidth}\raggedright
256\strut
\end{minipage} & \begin{minipage}[t]{0.09\columnwidth}\raggedright
288\strut
\end{minipage}\tabularnewline
\begin{minipage}[t]{0.08\columnwidth}\raggedright
Afghanistan\strut
\end{minipage} & \begin{minipage}[t]{0.04\columnwidth}\raggedright
AF\strut
\end{minipage} & \begin{minipage}[t]{0.04\columnwidth}\raggedright
AFG\strut
\end{minipage} & \begin{minipage}[t]{0.04\columnwidth}\raggedright
2004\strut
\end{minipage} & \begin{minipage}[t]{0.08\columnwidth}\raggedright
139\strut
\end{minipage} & \begin{minipage}[t]{0.09\columnwidth}\raggedright
537\strut
\end{minipage} & \begin{minipage}[t]{0.09\columnwidth}\raggedright
568\strut
\end{minipage} & \begin{minipage}[t]{0.09\columnwidth}\raggedright
360\strut
\end{minipage} & \begin{minipage}[t]{0.09\columnwidth}\raggedright
358\strut
\end{minipage} & \begin{minipage}[t]{0.09\columnwidth}\raggedright
386\strut
\end{minipage}\tabularnewline
\bottomrule
\end{longtable}

\begin{Shaded}
\begin{Highlighting}[]
\NormalTok{(api/pivot->longer who #(clojure.string/starts-with? }\VariableTok{%} \StringTok{"new"}\NormalTok{) \{}\AttributeTok{:target-columns}\NormalTok{ [}\AttributeTok{:diagnosis} \AttributeTok{:gender} \AttributeTok{:age}\NormalTok{]}
                                                               \AttributeTok{:splitter} \SpecialStringTok{#"new_?(.*)_(.)(.*)"}
                                                               \AttributeTok{:value-column-name} \AttributeTok{:count}\NormalTok{\})}
\end{Highlighting}
\end{Shaded}

data/who.csv.gz {[}76046 8{]}:

\begin{longtable}[]{@{}llllllll@{}}
\toprule
country & iso2 & iso3 & year & :diagnosis & :gender & :age &
:count\tabularnewline
\midrule
\endhead
Albania & AL & ALB & 2013 & rel & m & 1524 & 60\tabularnewline
Algeria & DZ & DZA & 2013 & rel & m & 1524 & 1021\tabularnewline
Andorra & AD & AND & 2013 & rel & m & 1524 & 0\tabularnewline
Angola & AO & AGO & 2013 & rel & m & 1524 & 2992\tabularnewline
Anguilla & AI & AIA & 2013 & rel & m & 1524 & 0\tabularnewline
Antigua and Barbuda & AG & ATG & 2013 & rel & m & 1524 &
1\tabularnewline
Argentina & AR & ARG & 2013 & rel & m & 1524 & 1124\tabularnewline
Armenia & AM & ARM & 2013 & rel & m & 1524 & 116\tabularnewline
Australia & AU & AUS & 2013 & rel & m & 1524 & 105\tabularnewline
Austria & AT & AUT & 2013 & rel & m & 1524 & 44\tabularnewline
Azerbaijan & AZ & AZE & 2013 & rel & m & 1524 & 958\tabularnewline
Bahamas & BS & BHS & 2013 & rel & m & 1524 & 2\tabularnewline
Bahrain & BH & BHR & 2013 & rel & m & 1524 & 13\tabularnewline
Bangladesh & BD & BGD & 2013 & rel & m & 1524 & 14705\tabularnewline
Barbados & BB & BRB & 2013 & rel & m & 1524 & 0\tabularnewline
Belarus & BY & BLR & 2013 & rel & m & 1524 & 162\tabularnewline
Belgium & BE & BEL & 2013 & rel & m & 1524 & 63\tabularnewline
Belize & BZ & BLZ & 2013 & rel & m & 1524 & 8\tabularnewline
Benin & BJ & BEN & 2013 & rel & m & 1524 & 301\tabularnewline
Bermuda & BM & BMU & 2013 & rel & m & 1524 & 0\tabularnewline
Bhutan & BT & BTN & 2013 & rel & m & 1524 & 180\tabularnewline
Bolivia (Plurinational State of) & BO & BOL & 2013 & rel & m & 1524 &
1470\tabularnewline
Bonaire, Saint Eustatius and Saba & BQ & BES & 2013 & rel & m & 1524 &
0\tabularnewline
Bosnia and Herzegovina & BA & BIH & 2013 & rel & m & 1524 &
57\tabularnewline
Botswana & BW & BWA & 2013 & rel & m & 1524 & 423\tabularnewline
\bottomrule
\end{longtable}

\begin{center}\rule{0.5\linewidth}{0.5pt}\end{center}

When data contains multiple observations per row, we can use splitter
and pattern for target columns to create new columns and put values
there. In following dataset we have two obseravations \texttt{dob} and
\texttt{gender} for two childs. We want to put child infomation into the
column and leave dob and gender for values.

\begin{Shaded}
\begin{Highlighting}[]
\NormalTok{(}\BuiltInTok{def}\FunctionTok{ family }\NormalTok{(api/dataset }\StringTok{"data/family.csv"}\NormalTok{))}
\end{Highlighting}
\end{Shaded}

\begin{Shaded}
\begin{Highlighting}[]
\NormalTok{family}
\end{Highlighting}
\end{Shaded}

data/family.csv {[}5 5{]}:

\begin{longtable}[]{@{}lllll@{}}
\toprule
family & dob\_child1 & dob\_child2 & gender\_child1 &
gender\_child2\tabularnewline
\midrule
\endhead
1 & 1998-11-26 & 2000-01-29 & 1 & 2\tabularnewline
2 & 1996-06-22 & & 2 &\tabularnewline
3 & 2002-07-11 & 2004-04-05 & 2 & 2\tabularnewline
4 & 2004-10-10 & 2009-08-27 & 1 & 1\tabularnewline
5 & 2000-12-05 & 2005-02-28 & 2 & 1\tabularnewline
\bottomrule
\end{longtable}

\begin{Shaded}
\begin{Highlighting}[]
\NormalTok{(api/pivot->longer family (}\KeywordTok{complement}\NormalTok{ #\{}\StringTok{"family"}\NormalTok{\}) \{}\AttributeTok{:target-columns}\NormalTok{ [}\VariableTok{nil} \AttributeTok{:child}\NormalTok{]}
                                                    \AttributeTok{:splitter} \StringTok{"_"}
                                                    \AttributeTok{:datatypes}\NormalTok{ \{}\StringTok{"gender"} \AttributeTok{:int16}\NormalTok{\}\})}
\end{Highlighting}
\end{Shaded}

data/family.csv {[}9 4{]}:

\begin{longtable}[]{@{}llll@{}}
\toprule
family & :child & dob & gender\tabularnewline
\midrule
\endhead
1 & child1 & 1998-11-26 & 1\tabularnewline
2 & child1 & 1996-06-22 & 2\tabularnewline
3 & child1 & 2002-07-11 & 2\tabularnewline
4 & child1 & 2004-10-10 & 1\tabularnewline
5 & child1 & 2000-12-05 & 2\tabularnewline
1 & child2 & 2000-01-29 & 2\tabularnewline
3 & child2 & 2004-04-05 & 2\tabularnewline
4 & child2 & 2009-08-27 & 1\tabularnewline
5 & child2 & 2005-02-28 & 1\tabularnewline
\bottomrule
\end{longtable}

\begin{center}\rule{0.5\linewidth}{0.5pt}\end{center}

Similar here, we have two observations: \texttt{x} and \texttt{y} in
four groups.

\begin{Shaded}
\begin{Highlighting}[]
\NormalTok{(}\BuiltInTok{def}\FunctionTok{ anscombe }\NormalTok{(api/dataset }\StringTok{"data/anscombe.csv"}\NormalTok{))}
\end{Highlighting}
\end{Shaded}

\begin{Shaded}
\begin{Highlighting}[]
\NormalTok{anscombe}
\end{Highlighting}
\end{Shaded}

data/anscombe.csv {[}11 8{]}:

\begin{longtable}[]{@{}llllllll@{}}
\toprule
x1 & x2 & x3 & x4 & y1 & y2 & y3 & y4\tabularnewline
\midrule
\endhead
10 & 10 & 10 & 8 & 8.04 & 9.14 & 7.46 & 6.58\tabularnewline
8 & 8 & 8 & 8 & 6.95 & 8.14 & 6.77 & 5.76\tabularnewline
13 & 13 & 13 & 8 & 7.58 & 8.74 & 12.74 & 7.71\tabularnewline
9 & 9 & 9 & 8 & 8.81 & 8.77 & 7.11 & 8.84\tabularnewline
11 & 11 & 11 & 8 & 8.33 & 9.26 & 7.81 & 8.47\tabularnewline
14 & 14 & 14 & 8 & 9.96 & 8.10 & 8.84 & 7.04\tabularnewline
6 & 6 & 6 & 8 & 7.24 & 6.13 & 6.08 & 5.25\tabularnewline
4 & 4 & 4 & 19 & 4.26 & 3.10 & 5.39 & 12.50\tabularnewline
12 & 12 & 12 & 8 & 10.84 & 9.13 & 8.15 & 5.56\tabularnewline
7 & 7 & 7 & 8 & 4.82 & 7.26 & 6.42 & 7.91\tabularnewline
5 & 5 & 5 & 8 & 5.68 & 4.74 & 5.73 & 6.89\tabularnewline
\bottomrule
\end{longtable}

\begin{Shaded}
\begin{Highlighting}[]
\NormalTok{(api/pivot->longer anscombe }\AttributeTok{:all}\NormalTok{ \{}\AttributeTok{:splitter} \SpecialStringTok{#"(.)(.)"}
                                  \AttributeTok{:target-columns}\NormalTok{ [}\VariableTok{nil} \AttributeTok{:set}\NormalTok{]\})}
\end{Highlighting}
\end{Shaded}

data/anscombe.csv {[}44 3{]}:

\begin{longtable}[]{@{}lll@{}}
\toprule
:set & x & y\tabularnewline
\midrule
\endhead
1 & 10 & 8.04\tabularnewline
1 & 8 & 6.95\tabularnewline
1 & 13 & 7.58\tabularnewline
1 & 9 & 8.81\tabularnewline
1 & 11 & 8.33\tabularnewline
1 & 14 & 9.96\tabularnewline
1 & 6 & 7.24\tabularnewline
1 & 4 & 4.26\tabularnewline
1 & 12 & 10.84\tabularnewline
1 & 7 & 4.82\tabularnewline
1 & 5 & 5.68\tabularnewline
2 & 10 & 9.14\tabularnewline
2 & 8 & 8.14\tabularnewline
2 & 13 & 8.74\tabularnewline
2 & 9 & 8.77\tabularnewline
2 & 11 & 9.26\tabularnewline
2 & 14 & 8.10\tabularnewline
2 & 6 & 6.13\tabularnewline
2 & 4 & 3.10\tabularnewline
2 & 12 & 9.13\tabularnewline
2 & 7 & 7.26\tabularnewline
2 & 5 & 4.74\tabularnewline
3 & 10 & 7.46\tabularnewline
3 & 8 & 6.77\tabularnewline
3 & 13 & 12.74\tabularnewline
\bottomrule
\end{longtable}

\begin{center}\rule{0.5\linewidth}{0.5pt}\end{center}

\begin{Shaded}
\begin{Highlighting}[]
\NormalTok{(}\BuiltInTok{def}\FunctionTok{ pnl }\NormalTok{(api/dataset \{}\AttributeTok{:x}\NormalTok{ [}\DecValTok{1} \DecValTok{2} \DecValTok{3} \DecValTok{4}\NormalTok{]}
                       \AttributeTok{:a}\NormalTok{ [}\DecValTok{1} \DecValTok{1} \DecValTok{0} \DecValTok{0}\NormalTok{]}
                       \AttributeTok{:b}\NormalTok{ [}\DecValTok{0} \DecValTok{1} \DecValTok{1} \DecValTok{1}\NormalTok{]}
                       \AttributeTok{:y1}\NormalTok{ (}\KeywordTok{repeatedly} \DecValTok{4} \KeywordTok{rand}\NormalTok{)}
                       \AttributeTok{:y2}\NormalTok{ (}\KeywordTok{repeatedly} \DecValTok{4} \KeywordTok{rand}\NormalTok{)}
                       \AttributeTok{:z1}\NormalTok{ [}\DecValTok{3} \DecValTok{3} \DecValTok{3} \DecValTok{3}\NormalTok{]}
                       \AttributeTok{:z2}\NormalTok{ [-}\DecValTok{2} \DecValTok{-2} \DecValTok{-2} \DecValTok{-2}\NormalTok{]\}))}
\end{Highlighting}
\end{Shaded}

\begin{Shaded}
\begin{Highlighting}[]
\NormalTok{pnl}
\end{Highlighting}
\end{Shaded}

\_unnamed {[}4 7{]}:

\begin{longtable}[]{@{}lllllll@{}}
\toprule
:x & :a & :b & :y1 & :y2 & :z1 & :z2\tabularnewline
\midrule
\endhead
1 & 1 & 0 & 0.03172287 & 0.89715528 & 3 & -2\tabularnewline
2 & 1 & 1 & 0.94363662 & 0.74683685 & 3 & -2\tabularnewline
3 & 0 & 1 & 0.95758169 & 0.69157817 & 3 & -2\tabularnewline
4 & 0 & 1 & 0.13274742 & 0.26250552 & 3 & -2\tabularnewline
\bottomrule
\end{longtable}

\begin{Shaded}
\begin{Highlighting}[]
\NormalTok{(api/pivot->longer pnl [}\AttributeTok{:y1} \AttributeTok{:y2} \AttributeTok{:z1} \AttributeTok{:z2}\NormalTok{] \{}\AttributeTok{:target-columns}\NormalTok{ [}\VariableTok{nil} \AttributeTok{:times}\NormalTok{]}
                                          \AttributeTok{:splitter} \SpecialStringTok{#":(.)(.)"}\NormalTok{\})}
\end{Highlighting}
\end{Shaded}

\_unnamed {[}8 6{]}:

\begin{longtable}[]{@{}llllll@{}}
\toprule
:x & :a & :b & :times & y & z\tabularnewline
\midrule
\endhead
1 & 1 & 0 & 1 & 0.03172287 & 3\tabularnewline
2 & 1 & 1 & 1 & 0.94363662 & 3\tabularnewline
3 & 0 & 1 & 1 & 0.95758169 & 3\tabularnewline
4 & 0 & 1 & 1 & 0.13274742 & 3\tabularnewline
1 & 1 & 0 & 2 & 0.89715528 & -2\tabularnewline
2 & 1 & 1 & 2 & 0.74683685 & -2\tabularnewline
3 & 0 & 1 & 2 & 0.69157817 & -2\tabularnewline
4 & 0 & 1 & 2 & 0.26250552 & -2\tabularnewline
\bottomrule
\end{longtable}

\hypertarget{wider}{%
\paragraph{Wider}\label{wider}}

\texttt{pivot-\textgreater{}wider} converts rows to columns.

Arguments:

\begin{itemize}
\tightlist
\item
  dataset
\item
  \texttt{columns-selector} - values from selected columns are converted
  to new columns
\item
  \texttt{value-columns} - what are values
\end{itemize}

When multiple columns are used as columns selector, names are joined
using \texttt{:concat-columns-with} option.
\texttt{:concat-columns-with} can be a string or function (default:
"\_"). Function accepts sequence of names.

When \texttt{columns-selector} creates non unique set of values, they
are folded using \texttt{:fold-fn} (default: \texttt{vec}) option.

When \texttt{value-columns} is a sequence, multiple observations as
columns are created appending value column names into new columns.
Column names are joined using \texttt{:concat-value-with} option.
\texttt{:concat-value-with} can be a string or function (default:
``-''). Function accepts current column name and value.

\begin{center}\rule{0.5\linewidth}{0.5pt}\end{center}

Use \texttt{station} as a name source for columns and \texttt{seen} for
values

\begin{Shaded}
\begin{Highlighting}[]
\NormalTok{(}\BuiltInTok{def}\FunctionTok{ fish }\NormalTok{(api/dataset }\StringTok{"data/fish_encounters.csv"}\NormalTok{))}
\end{Highlighting}
\end{Shaded}

\begin{Shaded}
\begin{Highlighting}[]
\NormalTok{fish}
\end{Highlighting}
\end{Shaded}

data/fish\_encounters.csv {[}114 3{]}:

\begin{longtable}[]{@{}lll@{}}
\toprule
fish & station & seen\tabularnewline
\midrule
\endhead
4842 & Release & 1\tabularnewline
4842 & I80\_1 & 1\tabularnewline
4842 & Lisbon & 1\tabularnewline
4842 & Rstr & 1\tabularnewline
4842 & Base\_TD & 1\tabularnewline
4842 & BCE & 1\tabularnewline
4842 & BCW & 1\tabularnewline
4842 & BCE2 & 1\tabularnewline
4842 & BCW2 & 1\tabularnewline
4842 & MAE & 1\tabularnewline
4842 & MAW & 1\tabularnewline
4843 & Release & 1\tabularnewline
4843 & I80\_1 & 1\tabularnewline
4843 & Lisbon & 1\tabularnewline
4843 & Rstr & 1\tabularnewline
4843 & Base\_TD & 1\tabularnewline
4843 & BCE & 1\tabularnewline
4843 & BCW & 1\tabularnewline
4843 & BCE2 & 1\tabularnewline
4843 & BCW2 & 1\tabularnewline
4843 & MAE & 1\tabularnewline
4843 & MAW & 1\tabularnewline
4844 & Release & 1\tabularnewline
4844 & I80\_1 & 1\tabularnewline
4844 & Lisbon & 1\tabularnewline
\bottomrule
\end{longtable}

\begin{Shaded}
\begin{Highlighting}[]
\NormalTok{(api/pivot->wider fish }\StringTok{"station"} \StringTok{"seen"}\NormalTok{ \{}\AttributeTok{:drop-missing}\NormalTok{? }\VariableTok{false}\NormalTok{\})}
\end{Highlighting}
\end{Shaded}

data/fish\_encounters.csv {[}19 12{]}:

\begin{longtable}[]{@{}llllllllllll@{}}
\toprule
fish & Rstr & Base\_TD & I80\_1 & Release & MAE & BCE2 & MAW & BCW2 &
BCE & Lisbon & BCW\tabularnewline
\midrule
\endhead
4842 & 1 & 1 & 1 & 1 & 1 & 1 & 1 & 1 & 1 & 1 & 1\tabularnewline
4843 & 1 & 1 & 1 & 1 & 1 & 1 & 1 & 1 & 1 & 1 & 1\tabularnewline
4844 & 1 & 1 & 1 & 1 & 1 & 1 & 1 & 1 & 1 & 1 & 1\tabularnewline
4850 & 1 & 1 & 1 & 1 & & & & & 1 & & 1\tabularnewline
4857 & 1 & 1 & 1 & 1 & & 1 & & 1 & 1 & 1 & 1\tabularnewline
4858 & 1 & 1 & 1 & 1 & 1 & 1 & 1 & 1 & 1 & 1 & 1\tabularnewline
4861 & 1 & 1 & 1 & 1 & 1 & 1 & 1 & 1 & 1 & 1 & 1\tabularnewline
4862 & 1 & 1 & 1 & 1 & & 1 & & 1 & 1 & 1 & 1\tabularnewline
4864 & & & 1 & 1 & & & & & & &\tabularnewline
4865 & & & 1 & 1 & & & & & & 1 &\tabularnewline
4845 & 1 & 1 & 1 & 1 & & & & & & 1 &\tabularnewline
4847 & & & 1 & 1 & & & & & & 1 &\tabularnewline
4848 & 1 & & 1 & 1 & & & & & & 1 &\tabularnewline
4849 & & & 1 & 1 & & & & & & &\tabularnewline
4851 & & & 1 & 1 & & & & & & &\tabularnewline
4854 & & & 1 & 1 & & & & & & &\tabularnewline
4855 & 1 & 1 & 1 & 1 & & & & & & 1 &\tabularnewline
4859 & 1 & 1 & 1 & 1 & & & & & & 1 &\tabularnewline
4863 & & & 1 & 1 & & & & & & &\tabularnewline
\bottomrule
\end{longtable}

\begin{center}\rule{0.5\linewidth}{0.5pt}\end{center}

If selected columns contain multiple values, such values should be
folded.

\begin{Shaded}
\begin{Highlighting}[]
\NormalTok{(}\BuiltInTok{def}\FunctionTok{ warpbreaks }\NormalTok{(api/dataset }\StringTok{"data/warpbreaks.csv"}\NormalTok{))}
\end{Highlighting}
\end{Shaded}

\begin{Shaded}
\begin{Highlighting}[]
\NormalTok{warpbreaks}
\end{Highlighting}
\end{Shaded}

data/warpbreaks.csv {[}54 3{]}:

\begin{longtable}[]{@{}lll@{}}
\toprule
breaks & wool & tension\tabularnewline
\midrule
\endhead
26 & A & L\tabularnewline
30 & A & L\tabularnewline
54 & A & L\tabularnewline
25 & A & L\tabularnewline
70 & A & L\tabularnewline
52 & A & L\tabularnewline
51 & A & L\tabularnewline
26 & A & L\tabularnewline
67 & A & L\tabularnewline
18 & A & M\tabularnewline
21 & A & M\tabularnewline
29 & A & M\tabularnewline
17 & A & M\tabularnewline
12 & A & M\tabularnewline
18 & A & M\tabularnewline
35 & A & M\tabularnewline
30 & A & M\tabularnewline
36 & A & M\tabularnewline
36 & A & H\tabularnewline
21 & A & H\tabularnewline
24 & A & H\tabularnewline
18 & A & H\tabularnewline
10 & A & H\tabularnewline
43 & A & H\tabularnewline
28 & A & H\tabularnewline
\bottomrule
\end{longtable}

Let's see how many values are for each type of \texttt{wool} and
\texttt{tension} groups

\begin{Shaded}
\begin{Highlighting}[]
\NormalTok{(}\KeywordTok{->}\NormalTok{ warpbreaks}
\NormalTok{    (api/group-by [}\StringTok{"wool"} \StringTok{"tension"}\NormalTok{])}
\NormalTok{    (api/aggregate \{}\AttributeTok{:n}\NormalTok{ api/row-count\}))}
\end{Highlighting}
\end{Shaded}

\_unnamed {[}6 3{]}:

\begin{longtable}[]{@{}lll@{}}
\toprule
wool & tension & :n\tabularnewline
\midrule
\endhead
A & H & 9\tabularnewline
B & H & 9\tabularnewline
A & L & 9\tabularnewline
A & M & 9\tabularnewline
B & L & 9\tabularnewline
B & M & 9\tabularnewline
\bottomrule
\end{longtable}

\begin{Shaded}
\begin{Highlighting}[]
\NormalTok{(}\KeywordTok{->}\NormalTok{ warpbreaks}
\NormalTok{    (api/reorder-columns [}\StringTok{"wool"} \StringTok{"tension"} \StringTok{"breaks"}\NormalTok{])}
\NormalTok{    (api/pivot->wider }\StringTok{"wool"} \StringTok{"breaks"}\NormalTok{ \{}\AttributeTok{:fold-fn} \KeywordTok{vec}\NormalTok{\}))}
\end{Highlighting}
\end{Shaded}

data/warpbreaks.csv {[}3 3{]}:

\begin{longtable}[]{@{}lll@{}}
\toprule
tension & B & A\tabularnewline
\midrule
\endhead
M & {[}42 26 19 16 39 28 21 39 29{]} & {[}18 21 29 17 12 18 35 30
36{]}\tabularnewline
H & {[}20 21 24 17 13 15 15 16 28{]} & {[}36 21 24 18 10 43 28 15
26{]}\tabularnewline
L & {[}27 14 29 19 29 31 41 20 44{]} & {[}26 30 54 25 70 52 51 26
67{]}\tabularnewline
\bottomrule
\end{longtable}

We can also calculate mean (aggreate values)

\begin{Shaded}
\begin{Highlighting}[]
\NormalTok{(}\KeywordTok{->}\NormalTok{ warpbreaks}
\NormalTok{    (api/reorder-columns [}\StringTok{"wool"} \StringTok{"tension"} \StringTok{"breaks"}\NormalTok{])}
\NormalTok{    (api/pivot->wider }\StringTok{"wool"} \StringTok{"breaks"}\NormalTok{ \{}\AttributeTok{:fold-fn}\NormalTok{ tech.v2.datatype.functional/mean\}))}
\end{Highlighting}
\end{Shaded}

data/warpbreaks.csv {[}3 3{]}:

\begin{longtable}[]{@{}lll@{}}
\toprule
tension & B & A\tabularnewline
\midrule
\endhead
H & 18.77777778 & 24.55555556\tabularnewline
M & 28.77777778 & 24.00000000\tabularnewline
L & 28.22222222 & 44.55555556\tabularnewline
\bottomrule
\end{longtable}

\begin{center}\rule{0.5\linewidth}{0.5pt}\end{center}

Multiple source columns, joined with default separator.

\begin{Shaded}
\begin{Highlighting}[]
\NormalTok{(}\BuiltInTok{def}\FunctionTok{ production }\NormalTok{(api/dataset }\StringTok{"data/production.csv"}\NormalTok{))}
\end{Highlighting}
\end{Shaded}

\begin{Shaded}
\begin{Highlighting}[]
\NormalTok{production}
\end{Highlighting}
\end{Shaded}

data/production.csv {[}45 4{]}:

\begin{longtable}[]{@{}llll@{}}
\toprule
product & country & year & production\tabularnewline
\midrule
\endhead
A & AI & 2000 & 1.63727158\tabularnewline
A & AI & 2001 & 0.15870784\tabularnewline
A & AI & 2002 & -1.56797745\tabularnewline
A & AI & 2003 & -0.44455509\tabularnewline
A & AI & 2004 & -0.07133701\tabularnewline
A & AI & 2005 & 1.61183090\tabularnewline
A & AI & 2006 & -0.70434682\tabularnewline
A & AI & 2007 & -1.53550542\tabularnewline
A & AI & 2008 & 0.83907155\tabularnewline
A & AI & 2009 & -0.37424110\tabularnewline
A & AI & 2010 & -0.71158926\tabularnewline
A & AI & 2011 & 1.12805634\tabularnewline
A & AI & 2012 & 1.45718247\tabularnewline
A & AI & 2013 & -1.55934101\tabularnewline
A & AI & 2014 & -0.11695838\tabularnewline
B & AI & 2000 & -0.02617661\tabularnewline
B & AI & 2001 & -0.68863576\tabularnewline
B & AI & 2002 & 0.06248741\tabularnewline
B & AI & 2003 & -0.72339686\tabularnewline
B & AI & 2004 & 0.47248952\tabularnewline
B & AI & 2005 & -0.94173861\tabularnewline
B & AI & 2006 & -0.34782108\tabularnewline
B & AI & 2007 & 0.52425284\tabularnewline
B & AI & 2008 & 1.83230937\tabularnewline
B & AI & 2009 & 0.10706491\tabularnewline
\bottomrule
\end{longtable}

\begin{Shaded}
\begin{Highlighting}[]
\NormalTok{(api/pivot->wider production [}\StringTok{"product"} \StringTok{"country"}\NormalTok{] }\StringTok{"production"}\NormalTok{)}
\end{Highlighting}
\end{Shaded}

data/production.csv {[}15 4{]}:

\begin{longtable}[]{@{}llll@{}}
\toprule
year & A\_AI & B\_EI & B\_AI\tabularnewline
\midrule
\endhead
2000 & 1.63727158 & 1.40470848 & -0.02617661\tabularnewline
2001 & 0.15870784 & -0.59618369 & -0.68863576\tabularnewline
2002 & -1.56797745 & -0.26568579 & 0.06248741\tabularnewline
2003 & -0.44455509 & 0.65257808 & -0.72339686\tabularnewline
2004 & -0.07133701 & 0.62564999 & 0.47248952\tabularnewline
2005 & 1.61183090 & -1.34530299 & -0.94173861\tabularnewline
2006 & -0.70434682 & -0.97184975 & -0.34782108\tabularnewline
2007 & -1.53550542 & -1.69715821 & 0.52425284\tabularnewline
2008 & 0.83907155 & 0.04556128 & 1.83230937\tabularnewline
2009 & -0.37424110 & 1.19315043 & 0.10706491\tabularnewline
2010 & -0.71158926 & -1.60557503 & -0.32903664\tabularnewline
2011 & 1.12805634 & -0.77235497 & -1.78319121\tabularnewline
2012 & 1.45718247 & -2.50262738 & 0.61125798\tabularnewline
2013 & -1.55934101 & -1.62753769 & -0.78526092\tabularnewline
2014 & -0.11695838 & 0.03329645 & 0.97843635\tabularnewline
\bottomrule
\end{longtable}

Joined with custom function

\begin{Shaded}
\begin{Highlighting}[]
\NormalTok{(api/pivot->wider production [}\StringTok{"product"} \StringTok{"country"}\NormalTok{] }\StringTok{"production"}\NormalTok{ \{}\AttributeTok{:concat-columns-with} \KeywordTok{vec}\NormalTok{\})}
\end{Highlighting}
\end{Shaded}

data/production.csv {[}15 4{]}:

\begin{longtable}[]{@{}llll@{}}
\toprule
year & {[}``A'' ``AI''{]} & {[}``B'' ``EI''{]} & {[}``B''
``AI''{]}\tabularnewline
\midrule
\endhead
2000 & 1.63727158 & 1.40470848 & -0.02617661\tabularnewline
2001 & 0.15870784 & -0.59618369 & -0.68863576\tabularnewline
2002 & -1.56797745 & -0.26568579 & 0.06248741\tabularnewline
2003 & -0.44455509 & 0.65257808 & -0.72339686\tabularnewline
2004 & -0.07133701 & 0.62564999 & 0.47248952\tabularnewline
2005 & 1.61183090 & -1.34530299 & -0.94173861\tabularnewline
2006 & -0.70434682 & -0.97184975 & -0.34782108\tabularnewline
2007 & -1.53550542 & -1.69715821 & 0.52425284\tabularnewline
2008 & 0.83907155 & 0.04556128 & 1.83230937\tabularnewline
2009 & -0.37424110 & 1.19315043 & 0.10706491\tabularnewline
2010 & -0.71158926 & -1.60557503 & -0.32903664\tabularnewline
2011 & 1.12805634 & -0.77235497 & -1.78319121\tabularnewline
2012 & 1.45718247 & -2.50262738 & 0.61125798\tabularnewline
2013 & -1.55934101 & -1.62753769 & -0.78526092\tabularnewline
2014 & -0.11695838 & 0.03329645 & 0.97843635\tabularnewline
\bottomrule
\end{longtable}

\begin{center}\rule{0.5\linewidth}{0.5pt}\end{center}

Multiple value columns

\begin{Shaded}
\begin{Highlighting}[]
\NormalTok{(}\BuiltInTok{def}\FunctionTok{ income }\NormalTok{(api/dataset }\StringTok{"data/us_rent_income.csv"}\NormalTok{))}
\end{Highlighting}
\end{Shaded}

\begin{Shaded}
\begin{Highlighting}[]
\NormalTok{income}
\end{Highlighting}
\end{Shaded}

data/us\_rent\_income.csv {[}104 5{]}:

\begin{longtable}[]{@{}lllll@{}}
\toprule
GEOID & NAME & variable & estimate & moe\tabularnewline
\midrule
\endhead
1 & Alabama & income & 24476 & 136\tabularnewline
1 & Alabama & rent & 747 & 3\tabularnewline
2 & Alaska & income & 32940 & 508\tabularnewline
2 & Alaska & rent & 1200 & 13\tabularnewline
4 & Arizona & income & 27517 & 148\tabularnewline
4 & Arizona & rent & 972 & 4\tabularnewline
5 & Arkansas & income & 23789 & 165\tabularnewline
5 & Arkansas & rent & 709 & 5\tabularnewline
6 & California & income & 29454 & 109\tabularnewline
6 & California & rent & 1358 & 3\tabularnewline
8 & Colorado & income & 32401 & 109\tabularnewline
8 & Colorado & rent & 1125 & 5\tabularnewline
9 & Connecticut & income & 35326 & 195\tabularnewline
9 & Connecticut & rent & 1123 & 5\tabularnewline
10 & Delaware & income & 31560 & 247\tabularnewline
10 & Delaware & rent & 1076 & 10\tabularnewline
11 & District of Columbia & income & 43198 & 681\tabularnewline
11 & District of Columbia & rent & 1424 & 17\tabularnewline
12 & Florida & income & 25952 & 70\tabularnewline
12 & Florida & rent & 1077 & 3\tabularnewline
13 & Georgia & income & 27024 & 106\tabularnewline
13 & Georgia & rent & 927 & 3\tabularnewline
15 & Hawaii & income & 32453 & 218\tabularnewline
15 & Hawaii & rent & 1507 & 18\tabularnewline
16 & Idaho & income & 25298 & 208\tabularnewline
\bottomrule
\end{longtable}

\begin{Shaded}
\begin{Highlighting}[]
\NormalTok{(api/pivot->wider income }\StringTok{"variable"}\NormalTok{ [}\StringTok{"estimate"} \StringTok{"moe"}\NormalTok{] \{}\AttributeTok{:drop-missing}\NormalTok{? }\VariableTok{false}\NormalTok{\})}
\end{Highlighting}
\end{Shaded}

data/us\_rent\_income.csv {[}52 6{]}:

\begin{longtable}[]{@{}llllll@{}}
\toprule
GEOID & NAME & rent-estimate & rent-moe & income-estimate &
income-moe\tabularnewline
\midrule
\endhead
1 & Alabama & 747 & 3 & 24476 & 136\tabularnewline
2 & Alaska & 1200 & 13 & 32940 & 508\tabularnewline
4 & Arizona & 972 & 4 & 27517 & 148\tabularnewline
5 & Arkansas & 709 & 5 & 23789 & 165\tabularnewline
6 & California & 1358 & 3 & 29454 & 109\tabularnewline
8 & Colorado & 1125 & 5 & 32401 & 109\tabularnewline
9 & Connecticut & 1123 & 5 & 35326 & 195\tabularnewline
10 & Delaware & 1076 & 10 & 31560 & 247\tabularnewline
11 & District of Columbia & 1424 & 17 & 43198 & 681\tabularnewline
12 & Florida & 1077 & 3 & 25952 & 70\tabularnewline
13 & Georgia & 927 & 3 & 27024 & 106\tabularnewline
15 & Hawaii & 1507 & 18 & 32453 & 218\tabularnewline
16 & Idaho & 792 & 7 & 25298 & 208\tabularnewline
17 & Illinois & 952 & 3 & 30684 & 83\tabularnewline
18 & Indiana & 782 & 3 & 27247 & 117\tabularnewline
19 & Iowa & 740 & 4 & 30002 & 143\tabularnewline
20 & Kansas & 801 & 5 & 29126 & 208\tabularnewline
21 & Kentucky & 713 & 4 & 24702 & 159\tabularnewline
22 & Louisiana & 825 & 4 & 25086 & 155\tabularnewline
23 & Maine & 808 & 7 & 26841 & 187\tabularnewline
24 & Maryland & 1311 & 5 & 37147 & 152\tabularnewline
25 & Massachusetts & 1173 & 5 & 34498 & 199\tabularnewline
26 & Michigan & 824 & 3 & 26987 & 82\tabularnewline
27 & Minnesota & 906 & 4 & 32734 & 189\tabularnewline
28 & Mississippi & 740 & 5 & 22766 & 194\tabularnewline
\bottomrule
\end{longtable}

Value concatenated by custom function

\begin{Shaded}
\begin{Highlighting}[]
\NormalTok{(api/pivot->wider income }\StringTok{"variable"}\NormalTok{ [}\StringTok{"estimate"} \StringTok{"moe"}\NormalTok{] \{}\AttributeTok{:concat-columns-with} \KeywordTok{vec}
                                                        \AttributeTok{:concat-value-with} \KeywordTok{vector}
                                                        \AttributeTok{:drop-missing}\NormalTok{? }\VariableTok{false}\NormalTok{\})}
\end{Highlighting}
\end{Shaded}

data/us\_rent\_income.csv {[}52 6{]}:

\begin{longtable}[]{@{}llllll@{}}
\toprule
\begin{minipage}[b]{0.05\columnwidth}\raggedright
GEOID\strut
\end{minipage} & \begin{minipage}[b]{0.17\columnwidth}\raggedright
NAME\strut
\end{minipage} & \begin{minipage}[b]{0.16\columnwidth}\raggedright
{[}``rent'' ``estimate''{]}\strut
\end{minipage} & \begin{minipage}[b]{0.12\columnwidth}\raggedright
{[}``rent'' ``moe''{]}\strut
\end{minipage} & \begin{minipage}[b]{0.18\columnwidth}\raggedright
{[}``income'' ``estimate''{]}\strut
\end{minipage} & \begin{minipage}[b]{0.14\columnwidth}\raggedright
{[}``income'' ``moe''{]}\strut
\end{minipage}\tabularnewline
\midrule
\endhead
\begin{minipage}[t]{0.05\columnwidth}\raggedright
1\strut
\end{minipage} & \begin{minipage}[t]{0.17\columnwidth}\raggedright
Alabama\strut
\end{minipage} & \begin{minipage}[t]{0.16\columnwidth}\raggedright
747\strut
\end{minipage} & \begin{minipage}[t]{0.12\columnwidth}\raggedright
3\strut
\end{minipage} & \begin{minipage}[t]{0.18\columnwidth}\raggedright
24476\strut
\end{minipage} & \begin{minipage}[t]{0.14\columnwidth}\raggedright
136\strut
\end{minipage}\tabularnewline
\begin{minipage}[t]{0.05\columnwidth}\raggedright
2\strut
\end{minipage} & \begin{minipage}[t]{0.17\columnwidth}\raggedright
Alaska\strut
\end{minipage} & \begin{minipage}[t]{0.16\columnwidth}\raggedright
1200\strut
\end{minipage} & \begin{minipage}[t]{0.12\columnwidth}\raggedright
13\strut
\end{minipage} & \begin{minipage}[t]{0.18\columnwidth}\raggedright
32940\strut
\end{minipage} & \begin{minipage}[t]{0.14\columnwidth}\raggedright
508\strut
\end{minipage}\tabularnewline
\begin{minipage}[t]{0.05\columnwidth}\raggedright
4\strut
\end{minipage} & \begin{minipage}[t]{0.17\columnwidth}\raggedright
Arizona\strut
\end{minipage} & \begin{minipage}[t]{0.16\columnwidth}\raggedright
972\strut
\end{minipage} & \begin{minipage}[t]{0.12\columnwidth}\raggedright
4\strut
\end{minipage} & \begin{minipage}[t]{0.18\columnwidth}\raggedright
27517\strut
\end{minipage} & \begin{minipage}[t]{0.14\columnwidth}\raggedright
148\strut
\end{minipage}\tabularnewline
\begin{minipage}[t]{0.05\columnwidth}\raggedright
5\strut
\end{minipage} & \begin{minipage}[t]{0.17\columnwidth}\raggedright
Arkansas\strut
\end{minipage} & \begin{minipage}[t]{0.16\columnwidth}\raggedright
709\strut
\end{minipage} & \begin{minipage}[t]{0.12\columnwidth}\raggedright
5\strut
\end{minipage} & \begin{minipage}[t]{0.18\columnwidth}\raggedright
23789\strut
\end{minipage} & \begin{minipage}[t]{0.14\columnwidth}\raggedright
165\strut
\end{minipage}\tabularnewline
\begin{minipage}[t]{0.05\columnwidth}\raggedright
6\strut
\end{minipage} & \begin{minipage}[t]{0.17\columnwidth}\raggedright
California\strut
\end{minipage} & \begin{minipage}[t]{0.16\columnwidth}\raggedright
1358\strut
\end{minipage} & \begin{minipage}[t]{0.12\columnwidth}\raggedright
3\strut
\end{minipage} & \begin{minipage}[t]{0.18\columnwidth}\raggedright
29454\strut
\end{minipage} & \begin{minipage}[t]{0.14\columnwidth}\raggedright
109\strut
\end{minipage}\tabularnewline
\begin{minipage}[t]{0.05\columnwidth}\raggedright
8\strut
\end{minipage} & \begin{minipage}[t]{0.17\columnwidth}\raggedright
Colorado\strut
\end{minipage} & \begin{minipage}[t]{0.16\columnwidth}\raggedright
1125\strut
\end{minipage} & \begin{minipage}[t]{0.12\columnwidth}\raggedright
5\strut
\end{minipage} & \begin{minipage}[t]{0.18\columnwidth}\raggedright
32401\strut
\end{minipage} & \begin{minipage}[t]{0.14\columnwidth}\raggedright
109\strut
\end{minipage}\tabularnewline
\begin{minipage}[t]{0.05\columnwidth}\raggedright
9\strut
\end{minipage} & \begin{minipage}[t]{0.17\columnwidth}\raggedright
Connecticut\strut
\end{minipage} & \begin{minipage}[t]{0.16\columnwidth}\raggedright
1123\strut
\end{minipage} & \begin{minipage}[t]{0.12\columnwidth}\raggedright
5\strut
\end{minipage} & \begin{minipage}[t]{0.18\columnwidth}\raggedright
35326\strut
\end{minipage} & \begin{minipage}[t]{0.14\columnwidth}\raggedright
195\strut
\end{minipage}\tabularnewline
\begin{minipage}[t]{0.05\columnwidth}\raggedright
10\strut
\end{minipage} & \begin{minipage}[t]{0.17\columnwidth}\raggedright
Delaware\strut
\end{minipage} & \begin{minipage}[t]{0.16\columnwidth}\raggedright
1076\strut
\end{minipage} & \begin{minipage}[t]{0.12\columnwidth}\raggedright
10\strut
\end{minipage} & \begin{minipage}[t]{0.18\columnwidth}\raggedright
31560\strut
\end{minipage} & \begin{minipage}[t]{0.14\columnwidth}\raggedright
247\strut
\end{minipage}\tabularnewline
\begin{minipage}[t]{0.05\columnwidth}\raggedright
11\strut
\end{minipage} & \begin{minipage}[t]{0.17\columnwidth}\raggedright
District of Columbia\strut
\end{minipage} & \begin{minipage}[t]{0.16\columnwidth}\raggedright
1424\strut
\end{minipage} & \begin{minipage}[t]{0.12\columnwidth}\raggedright
17\strut
\end{minipage} & \begin{minipage}[t]{0.18\columnwidth}\raggedright
43198\strut
\end{minipage} & \begin{minipage}[t]{0.14\columnwidth}\raggedright
681\strut
\end{minipage}\tabularnewline
\begin{minipage}[t]{0.05\columnwidth}\raggedright
12\strut
\end{minipage} & \begin{minipage}[t]{0.17\columnwidth}\raggedright
Florida\strut
\end{minipage} & \begin{minipage}[t]{0.16\columnwidth}\raggedright
1077\strut
\end{minipage} & \begin{minipage}[t]{0.12\columnwidth}\raggedright
3\strut
\end{minipage} & \begin{minipage}[t]{0.18\columnwidth}\raggedright
25952\strut
\end{minipage} & \begin{minipage}[t]{0.14\columnwidth}\raggedright
70\strut
\end{minipage}\tabularnewline
\begin{minipage}[t]{0.05\columnwidth}\raggedright
13\strut
\end{minipage} & \begin{minipage}[t]{0.17\columnwidth}\raggedright
Georgia\strut
\end{minipage} & \begin{minipage}[t]{0.16\columnwidth}\raggedright
927\strut
\end{minipage} & \begin{minipage}[t]{0.12\columnwidth}\raggedright
3\strut
\end{minipage} & \begin{minipage}[t]{0.18\columnwidth}\raggedright
27024\strut
\end{minipage} & \begin{minipage}[t]{0.14\columnwidth}\raggedright
106\strut
\end{minipage}\tabularnewline
\begin{minipage}[t]{0.05\columnwidth}\raggedright
15\strut
\end{minipage} & \begin{minipage}[t]{0.17\columnwidth}\raggedright
Hawaii\strut
\end{minipage} & \begin{minipage}[t]{0.16\columnwidth}\raggedright
1507\strut
\end{minipage} & \begin{minipage}[t]{0.12\columnwidth}\raggedright
18\strut
\end{minipage} & \begin{minipage}[t]{0.18\columnwidth}\raggedright
32453\strut
\end{minipage} & \begin{minipage}[t]{0.14\columnwidth}\raggedright
218\strut
\end{minipage}\tabularnewline
\begin{minipage}[t]{0.05\columnwidth}\raggedright
16\strut
\end{minipage} & \begin{minipage}[t]{0.17\columnwidth}\raggedright
Idaho\strut
\end{minipage} & \begin{minipage}[t]{0.16\columnwidth}\raggedright
792\strut
\end{minipage} & \begin{minipage}[t]{0.12\columnwidth}\raggedright
7\strut
\end{minipage} & \begin{minipage}[t]{0.18\columnwidth}\raggedright
25298\strut
\end{minipage} & \begin{minipage}[t]{0.14\columnwidth}\raggedright
208\strut
\end{minipage}\tabularnewline
\begin{minipage}[t]{0.05\columnwidth}\raggedright
17\strut
\end{minipage} & \begin{minipage}[t]{0.17\columnwidth}\raggedright
Illinois\strut
\end{minipage} & \begin{minipage}[t]{0.16\columnwidth}\raggedright
952\strut
\end{minipage} & \begin{minipage}[t]{0.12\columnwidth}\raggedright
3\strut
\end{minipage} & \begin{minipage}[t]{0.18\columnwidth}\raggedright
30684\strut
\end{minipage} & \begin{minipage}[t]{0.14\columnwidth}\raggedright
83\strut
\end{minipage}\tabularnewline
\begin{minipage}[t]{0.05\columnwidth}\raggedright
18\strut
\end{minipage} & \begin{minipage}[t]{0.17\columnwidth}\raggedright
Indiana\strut
\end{minipage} & \begin{minipage}[t]{0.16\columnwidth}\raggedright
782\strut
\end{minipage} & \begin{minipage}[t]{0.12\columnwidth}\raggedright
3\strut
\end{minipage} & \begin{minipage}[t]{0.18\columnwidth}\raggedright
27247\strut
\end{minipage} & \begin{minipage}[t]{0.14\columnwidth}\raggedright
117\strut
\end{minipage}\tabularnewline
\begin{minipage}[t]{0.05\columnwidth}\raggedright
19\strut
\end{minipage} & \begin{minipage}[t]{0.17\columnwidth}\raggedright
Iowa\strut
\end{minipage} & \begin{minipage}[t]{0.16\columnwidth}\raggedright
740\strut
\end{minipage} & \begin{minipage}[t]{0.12\columnwidth}\raggedright
4\strut
\end{minipage} & \begin{minipage}[t]{0.18\columnwidth}\raggedright
30002\strut
\end{minipage} & \begin{minipage}[t]{0.14\columnwidth}\raggedright
143\strut
\end{minipage}\tabularnewline
\begin{minipage}[t]{0.05\columnwidth}\raggedright
20\strut
\end{minipage} & \begin{minipage}[t]{0.17\columnwidth}\raggedright
Kansas\strut
\end{minipage} & \begin{minipage}[t]{0.16\columnwidth}\raggedright
801\strut
\end{minipage} & \begin{minipage}[t]{0.12\columnwidth}\raggedright
5\strut
\end{minipage} & \begin{minipage}[t]{0.18\columnwidth}\raggedright
29126\strut
\end{minipage} & \begin{minipage}[t]{0.14\columnwidth}\raggedright
208\strut
\end{minipage}\tabularnewline
\begin{minipage}[t]{0.05\columnwidth}\raggedright
21\strut
\end{minipage} & \begin{minipage}[t]{0.17\columnwidth}\raggedright
Kentucky\strut
\end{minipage} & \begin{minipage}[t]{0.16\columnwidth}\raggedright
713\strut
\end{minipage} & \begin{minipage}[t]{0.12\columnwidth}\raggedright
4\strut
\end{minipage} & \begin{minipage}[t]{0.18\columnwidth}\raggedright
24702\strut
\end{minipage} & \begin{minipage}[t]{0.14\columnwidth}\raggedright
159\strut
\end{minipage}\tabularnewline
\begin{minipage}[t]{0.05\columnwidth}\raggedright
22\strut
\end{minipage} & \begin{minipage}[t]{0.17\columnwidth}\raggedright
Louisiana\strut
\end{minipage} & \begin{minipage}[t]{0.16\columnwidth}\raggedright
825\strut
\end{minipage} & \begin{minipage}[t]{0.12\columnwidth}\raggedright
4\strut
\end{minipage} & \begin{minipage}[t]{0.18\columnwidth}\raggedright
25086\strut
\end{minipage} & \begin{minipage}[t]{0.14\columnwidth}\raggedright
155\strut
\end{minipage}\tabularnewline
\begin{minipage}[t]{0.05\columnwidth}\raggedright
23\strut
\end{minipage} & \begin{minipage}[t]{0.17\columnwidth}\raggedright
Maine\strut
\end{minipage} & \begin{minipage}[t]{0.16\columnwidth}\raggedright
808\strut
\end{minipage} & \begin{minipage}[t]{0.12\columnwidth}\raggedright
7\strut
\end{minipage} & \begin{minipage}[t]{0.18\columnwidth}\raggedright
26841\strut
\end{minipage} & \begin{minipage}[t]{0.14\columnwidth}\raggedright
187\strut
\end{minipage}\tabularnewline
\begin{minipage}[t]{0.05\columnwidth}\raggedright
24\strut
\end{minipage} & \begin{minipage}[t]{0.17\columnwidth}\raggedright
Maryland\strut
\end{minipage} & \begin{minipage}[t]{0.16\columnwidth}\raggedright
1311\strut
\end{minipage} & \begin{minipage}[t]{0.12\columnwidth}\raggedright
5\strut
\end{minipage} & \begin{minipage}[t]{0.18\columnwidth}\raggedright
37147\strut
\end{minipage} & \begin{minipage}[t]{0.14\columnwidth}\raggedright
152\strut
\end{minipage}\tabularnewline
\begin{minipage}[t]{0.05\columnwidth}\raggedright
25\strut
\end{minipage} & \begin{minipage}[t]{0.17\columnwidth}\raggedright
Massachusetts\strut
\end{minipage} & \begin{minipage}[t]{0.16\columnwidth}\raggedright
1173\strut
\end{minipage} & \begin{minipage}[t]{0.12\columnwidth}\raggedright
5\strut
\end{minipage} & \begin{minipage}[t]{0.18\columnwidth}\raggedright
34498\strut
\end{minipage} & \begin{minipage}[t]{0.14\columnwidth}\raggedright
199\strut
\end{minipage}\tabularnewline
\begin{minipage}[t]{0.05\columnwidth}\raggedright
26\strut
\end{minipage} & \begin{minipage}[t]{0.17\columnwidth}\raggedright
Michigan\strut
\end{minipage} & \begin{minipage}[t]{0.16\columnwidth}\raggedright
824\strut
\end{minipage} & \begin{minipage}[t]{0.12\columnwidth}\raggedright
3\strut
\end{minipage} & \begin{minipage}[t]{0.18\columnwidth}\raggedright
26987\strut
\end{minipage} & \begin{minipage}[t]{0.14\columnwidth}\raggedright
82\strut
\end{minipage}\tabularnewline
\begin{minipage}[t]{0.05\columnwidth}\raggedright
27\strut
\end{minipage} & \begin{minipage}[t]{0.17\columnwidth}\raggedright
Minnesota\strut
\end{minipage} & \begin{minipage}[t]{0.16\columnwidth}\raggedright
906\strut
\end{minipage} & \begin{minipage}[t]{0.12\columnwidth}\raggedright
4\strut
\end{minipage} & \begin{minipage}[t]{0.18\columnwidth}\raggedright
32734\strut
\end{minipage} & \begin{minipage}[t]{0.14\columnwidth}\raggedright
189\strut
\end{minipage}\tabularnewline
\begin{minipage}[t]{0.05\columnwidth}\raggedright
28\strut
\end{minipage} & \begin{minipage}[t]{0.17\columnwidth}\raggedright
Mississippi\strut
\end{minipage} & \begin{minipage}[t]{0.16\columnwidth}\raggedright
740\strut
\end{minipage} & \begin{minipage}[t]{0.12\columnwidth}\raggedright
5\strut
\end{minipage} & \begin{minipage}[t]{0.18\columnwidth}\raggedright
22766\strut
\end{minipage} & \begin{minipage}[t]{0.14\columnwidth}\raggedright
194\strut
\end{minipage}\tabularnewline
\bottomrule
\end{longtable}

\begin{center}\rule{0.5\linewidth}{0.5pt}\end{center}

Reshape contact data

\begin{Shaded}
\begin{Highlighting}[]
\NormalTok{(}\BuiltInTok{def}\FunctionTok{ contacts }\NormalTok{(api/dataset }\StringTok{"data/contacts.csv"}\NormalTok{))}
\end{Highlighting}
\end{Shaded}

\begin{Shaded}
\begin{Highlighting}[]
\NormalTok{contacts}
\end{Highlighting}
\end{Shaded}

data/contacts.csv {[}6 3{]}:

\begin{longtable}[]{@{}lll@{}}
\toprule
field & value & person\_id\tabularnewline
\midrule
\endhead
name & Jiena McLellan & 1\tabularnewline
company & Toyota & 1\tabularnewline
name & John Smith & 2\tabularnewline
company & google & 2\tabularnewline
email & \href{mailto:john@google.com}{\nolinkurl{john@google.com}} &
2\tabularnewline
name & Huxley Ratcliffe & 3\tabularnewline
\bottomrule
\end{longtable}

\begin{Shaded}
\begin{Highlighting}[]
\NormalTok{(api/pivot->wider contacts }\StringTok{"field"} \StringTok{"value"}\NormalTok{ \{}\AttributeTok{:drop-missing}\NormalTok{? }\VariableTok{false}\NormalTok{\})}
\end{Highlighting}
\end{Shaded}

data/contacts.csv {[}3 4{]}:

\begin{longtable}[]{@{}llll@{}}
\toprule
person\_id & email & name & company\tabularnewline
\midrule
\endhead
1 & & Jiena McLellan & Toyota\tabularnewline
2 & \href{mailto:john@google.com}{\nolinkurl{john@google.com}} & John
Smith & google\tabularnewline
3 & & Huxley Ratcliffe &\tabularnewline
\bottomrule
\end{longtable}

\hypertarget{reshaping}{%
\paragraph{Reshaping}\label{reshaping}}

A couple of \texttt{tidyr} examples of more complex reshaping.

\begin{center}\rule{0.5\linewidth}{0.5pt}\end{center}

\href{https://tidyr.tidyverse.org/articles/pivot.html\#world-bank}{World
bank}

\begin{Shaded}
\begin{Highlighting}[]
\NormalTok{(}\BuiltInTok{def}\FunctionTok{ world-bank-pop }\NormalTok{(api/dataset }\StringTok{"data/world_bank_pop.csv.gz"}\NormalTok{))}
\end{Highlighting}
\end{Shaded}

\begin{Shaded}
\begin{Highlighting}[]
\NormalTok{(}\KeywordTok{->>}\NormalTok{ world-bank-pop}
\NormalTok{     (api/column-names)}
\NormalTok{     (}\KeywordTok{take} \DecValTok{8}\NormalTok{)}
\NormalTok{     (api/select-columns world-bank-pop))}
\end{Highlighting}
\end{Shaded}

data/world\_bank\_pop.csv.gz {[}1056 8{]}:

\begin{longtable}[]{@{}llllllll@{}}
\toprule
\begin{minipage}[b]{0.06\columnwidth}\raggedright
country\strut
\end{minipage} & \begin{minipage}[b]{0.08\columnwidth}\raggedright
indicator\strut
\end{minipage} & \begin{minipage}[b]{0.11\columnwidth}\raggedright
2000\strut
\end{minipage} & \begin{minipage}[b]{0.11\columnwidth}\raggedright
2001\strut
\end{minipage} & \begin{minipage}[b]{0.11\columnwidth}\raggedright
2002\strut
\end{minipage} & \begin{minipage}[b]{0.11\columnwidth}\raggedright
2003\strut
\end{minipage} & \begin{minipage}[b]{0.11\columnwidth}\raggedright
2004\strut
\end{minipage} & \begin{minipage}[b]{0.11\columnwidth}\raggedright
2005\strut
\end{minipage}\tabularnewline
\midrule
\endhead
\begin{minipage}[t]{0.06\columnwidth}\raggedright
ABW\strut
\end{minipage} & \begin{minipage}[t]{0.08\columnwidth}\raggedright
SP.URB.TOTL\strut
\end{minipage} & \begin{minipage}[t]{0.11\columnwidth}\raggedright
4.24440000E+04\strut
\end{minipage} & \begin{minipage}[t]{0.11\columnwidth}\raggedright
4.30480000E+04\strut
\end{minipage} & \begin{minipage}[t]{0.11\columnwidth}\raggedright
4.36700000E+04\strut
\end{minipage} & \begin{minipage}[t]{0.11\columnwidth}\raggedright
4.42460000E+04\strut
\end{minipage} & \begin{minipage}[t]{0.11\columnwidth}\raggedright
4.46690000E+04\strut
\end{minipage} & \begin{minipage}[t]{0.11\columnwidth}\raggedright
4.48890000E+04\strut
\end{minipage}\tabularnewline
\begin{minipage}[t]{0.06\columnwidth}\raggedright
ABW\strut
\end{minipage} & \begin{minipage}[t]{0.08\columnwidth}\raggedright
SP.URB.GROW\strut
\end{minipage} & \begin{minipage}[t]{0.11\columnwidth}\raggedright
1.18263237E+00\strut
\end{minipage} & \begin{minipage}[t]{0.11\columnwidth}\raggedright
1.41302122E+00\strut
\end{minipage} & \begin{minipage}[t]{0.11\columnwidth}\raggedright
1.43455953E+00\strut
\end{minipage} & \begin{minipage}[t]{0.11\columnwidth}\raggedright
1.31036044E+00\strut
\end{minipage} & \begin{minipage}[t]{0.11\columnwidth}\raggedright
9.51477684E-01\strut
\end{minipage} & \begin{minipage}[t]{0.11\columnwidth}\raggedright
4.91302715E-01\strut
\end{minipage}\tabularnewline
\begin{minipage}[t]{0.06\columnwidth}\raggedright
ABW\strut
\end{minipage} & \begin{minipage}[t]{0.08\columnwidth}\raggedright
SP.POP.TOTL\strut
\end{minipage} & \begin{minipage}[t]{0.11\columnwidth}\raggedright
9.08530000E+04\strut
\end{minipage} & \begin{minipage}[t]{0.11\columnwidth}\raggedright
9.28980000E+04\strut
\end{minipage} & \begin{minipage}[t]{0.11\columnwidth}\raggedright
9.49920000E+04\strut
\end{minipage} & \begin{minipage}[t]{0.11\columnwidth}\raggedright
9.70170000E+04\strut
\end{minipage} & \begin{minipage}[t]{0.11\columnwidth}\raggedright
9.87370000E+04\strut
\end{minipage} & \begin{minipage}[t]{0.11\columnwidth}\raggedright
1.00031000E+05\strut
\end{minipage}\tabularnewline
\begin{minipage}[t]{0.06\columnwidth}\raggedright
ABW\strut
\end{minipage} & \begin{minipage}[t]{0.08\columnwidth}\raggedright
SP.POP.GROW\strut
\end{minipage} & \begin{minipage}[t]{0.11\columnwidth}\raggedright
2.05502678E+00\strut
\end{minipage} & \begin{minipage}[t]{0.11\columnwidth}\raggedright
2.22593013E+00\strut
\end{minipage} & \begin{minipage}[t]{0.11\columnwidth}\raggedright
2.22905605E+00\strut
\end{minipage} & \begin{minipage}[t]{0.11\columnwidth}\raggedright
2.10935434E+00\strut
\end{minipage} & \begin{minipage}[t]{0.11\columnwidth}\raggedright
1.75735287E+00\strut
\end{minipage} & \begin{minipage}[t]{0.11\columnwidth}\raggedright
1.30203884E+00\strut
\end{minipage}\tabularnewline
\begin{minipage}[t]{0.06\columnwidth}\raggedright
AFG\strut
\end{minipage} & \begin{minipage}[t]{0.08\columnwidth}\raggedright
SP.URB.TOTL\strut
\end{minipage} & \begin{minipage}[t]{0.11\columnwidth}\raggedright
4.43629900E+06\strut
\end{minipage} & \begin{minipage}[t]{0.11\columnwidth}\raggedright
4.64805500E+06\strut
\end{minipage} & \begin{minipage}[t]{0.11\columnwidth}\raggedright
4.89295100E+06\strut
\end{minipage} & \begin{minipage}[t]{0.11\columnwidth}\raggedright
5.15568600E+06\strut
\end{minipage} & \begin{minipage}[t]{0.11\columnwidth}\raggedright
5.42677000E+06\strut
\end{minipage} & \begin{minipage}[t]{0.11\columnwidth}\raggedright
5.69182300E+06\strut
\end{minipage}\tabularnewline
\begin{minipage}[t]{0.06\columnwidth}\raggedright
AFG\strut
\end{minipage} & \begin{minipage}[t]{0.08\columnwidth}\raggedright
SP.URB.GROW\strut
\end{minipage} & \begin{minipage}[t]{0.11\columnwidth}\raggedright
3.91222846E+00\strut
\end{minipage} & \begin{minipage}[t]{0.11\columnwidth}\raggedright
4.66283822E+00\strut
\end{minipage} & \begin{minipage}[t]{0.11\columnwidth}\raggedright
5.13467454E+00\strut
\end{minipage} & \begin{minipage}[t]{0.11\columnwidth}\raggedright
5.23045853E+00\strut
\end{minipage} & \begin{minipage}[t]{0.11\columnwidth}\raggedright
5.12439302E+00\strut
\end{minipage} & \begin{minipage}[t]{0.11\columnwidth}\raggedright
4.76864700E+00\strut
\end{minipage}\tabularnewline
\begin{minipage}[t]{0.06\columnwidth}\raggedright
AFG\strut
\end{minipage} & \begin{minipage}[t]{0.08\columnwidth}\raggedright
SP.POP.TOTL\strut
\end{minipage} & \begin{minipage}[t]{0.11\columnwidth}\raggedright
2.00937560E+07\strut
\end{minipage} & \begin{minipage}[t]{0.11\columnwidth}\raggedright
2.09664630E+07\strut
\end{minipage} & \begin{minipage}[t]{0.11\columnwidth}\raggedright
2.19799230E+07\strut
\end{minipage} & \begin{minipage}[t]{0.11\columnwidth}\raggedright
2.30648510E+07\strut
\end{minipage} & \begin{minipage}[t]{0.11\columnwidth}\raggedright
2.41189790E+07\strut
\end{minipage} & \begin{minipage}[t]{0.11\columnwidth}\raggedright
2.50707980E+07\strut
\end{minipage}\tabularnewline
\begin{minipage}[t]{0.06\columnwidth}\raggedright
AFG\strut
\end{minipage} & \begin{minipage}[t]{0.08\columnwidth}\raggedright
SP.POP.GROW\strut
\end{minipage} & \begin{minipage}[t]{0.11\columnwidth}\raggedright
3.49465874E+00\strut
\end{minipage} & \begin{minipage}[t]{0.11\columnwidth}\raggedright
4.25150411E+00\strut
\end{minipage} & \begin{minipage}[t]{0.11\columnwidth}\raggedright
4.72052846E+00\strut
\end{minipage} & \begin{minipage}[t]{0.11\columnwidth}\raggedright
4.81804112E+00\strut
\end{minipage} & \begin{minipage}[t]{0.11\columnwidth}\raggedright
4.46891840E+00\strut
\end{minipage} & \begin{minipage}[t]{0.11\columnwidth}\raggedright
3.87047016E+00\strut
\end{minipage}\tabularnewline
\begin{minipage}[t]{0.06\columnwidth}\raggedright
AGO\strut
\end{minipage} & \begin{minipage}[t]{0.08\columnwidth}\raggedright
SP.URB.TOTL\strut
\end{minipage} & \begin{minipage}[t]{0.11\columnwidth}\raggedright
8.23476600E+06\strut
\end{minipage} & \begin{minipage}[t]{0.11\columnwidth}\raggedright
8.70800000E+06\strut
\end{minipage} & \begin{minipage}[t]{0.11\columnwidth}\raggedright
9.21878700E+06\strut
\end{minipage} & \begin{minipage}[t]{0.11\columnwidth}\raggedright
9.76519700E+06\strut
\end{minipage} & \begin{minipage}[t]{0.11\columnwidth}\raggedright
1.03435060E+07\strut
\end{minipage} & \begin{minipage}[t]{0.11\columnwidth}\raggedright
1.09494240E+07\strut
\end{minipage}\tabularnewline
\begin{minipage}[t]{0.06\columnwidth}\raggedright
AGO\strut
\end{minipage} & \begin{minipage}[t]{0.08\columnwidth}\raggedright
SP.URB.GROW\strut
\end{minipage} & \begin{minipage}[t]{0.11\columnwidth}\raggedright
5.43749411E+00\strut
\end{minipage} & \begin{minipage}[t]{0.11\columnwidth}\raggedright
5.58771954E+00\strut
\end{minipage} & \begin{minipage}[t]{0.11\columnwidth}\raggedright
5.70013237E+00\strut
\end{minipage} & \begin{minipage}[t]{0.11\columnwidth}\raggedright
5.75812711E+00\strut
\end{minipage} & \begin{minipage}[t]{0.11\columnwidth}\raggedright
5.75341450E+00\strut
\end{minipage} & \begin{minipage}[t]{0.11\columnwidth}\raggedright
5.69279690E+00\strut
\end{minipage}\tabularnewline
\begin{minipage}[t]{0.06\columnwidth}\raggedright
AGO\strut
\end{minipage} & \begin{minipage}[t]{0.08\columnwidth}\raggedright
SP.POP.TOTL\strut
\end{minipage} & \begin{minipage}[t]{0.11\columnwidth}\raggedright
1.64409240E+07\strut
\end{minipage} & \begin{minipage}[t]{0.11\columnwidth}\raggedright
1.69832660E+07\strut
\end{minipage} & \begin{minipage}[t]{0.11\columnwidth}\raggedright
1.75726490E+07\strut
\end{minipage} & \begin{minipage}[t]{0.11\columnwidth}\raggedright
1.82033690E+07\strut
\end{minipage} & \begin{minipage}[t]{0.11\columnwidth}\raggedright
1.88657160E+07\strut
\end{minipage} & \begin{minipage}[t]{0.11\columnwidth}\raggedright
1.95525420E+07\strut
\end{minipage}\tabularnewline
\begin{minipage}[t]{0.06\columnwidth}\raggedright
AGO\strut
\end{minipage} & \begin{minipage}[t]{0.08\columnwidth}\raggedright
SP.POP.GROW\strut
\end{minipage} & \begin{minipage}[t]{0.11\columnwidth}\raggedright
3.03294342E+00\strut
\end{minipage} & \begin{minipage}[t]{0.11\columnwidth}\raggedright
3.24549139E+00\strut
\end{minipage} & \begin{minipage}[t]{0.11\columnwidth}\raggedright
3.41151529E+00\strut
\end{minipage} & \begin{minipage}[t]{0.11\columnwidth}\raggedright
3.52630277E+00\strut
\end{minipage} & \begin{minipage}[t]{0.11\columnwidth}\raggedright
3.57396197E+00\strut
\end{minipage} & \begin{minipage}[t]{0.11\columnwidth}\raggedright
3.57589970E+00\strut
\end{minipage}\tabularnewline
\begin{minipage}[t]{0.06\columnwidth}\raggedright
ALB\strut
\end{minipage} & \begin{minipage}[t]{0.08\columnwidth}\raggedright
SP.URB.TOTL\strut
\end{minipage} & \begin{minipage}[t]{0.11\columnwidth}\raggedright
1.28939100E+06\strut
\end{minipage} & \begin{minipage}[t]{0.11\columnwidth}\raggedright
1.29858400E+06\strut
\end{minipage} & \begin{minipage}[t]{0.11\columnwidth}\raggedright
1.32722000E+06\strut
\end{minipage} & \begin{minipage}[t]{0.11\columnwidth}\raggedright
1.35484800E+06\strut
\end{minipage} & \begin{minipage}[t]{0.11\columnwidth}\raggedright
1.38182800E+06\strut
\end{minipage} & \begin{minipage}[t]{0.11\columnwidth}\raggedright
1.40729800E+06\strut
\end{minipage}\tabularnewline
\begin{minipage}[t]{0.06\columnwidth}\raggedright
ALB\strut
\end{minipage} & \begin{minipage}[t]{0.08\columnwidth}\raggedright
SP.URB.GROW\strut
\end{minipage} & \begin{minipage}[t]{0.11\columnwidth}\raggedright
7.42478629E-01\strut
\end{minipage} & \begin{minipage}[t]{0.11\columnwidth}\raggedright
7.10442618E-01\strut
\end{minipage} & \begin{minipage}[t]{0.11\columnwidth}\raggedright
2.18120890E+00\strut
\end{minipage} & \begin{minipage}[t]{0.11\columnwidth}\raggedright
2.06027418E+00\strut
\end{minipage} & \begin{minipage}[t]{0.11\columnwidth}\raggedright
1.97179894E+00\strut
\end{minipage} & \begin{minipage}[t]{0.11\columnwidth}\raggedright
1.82642936E+00\strut
\end{minipage}\tabularnewline
\begin{minipage}[t]{0.06\columnwidth}\raggedright
ALB\strut
\end{minipage} & \begin{minipage}[t]{0.08\columnwidth}\raggedright
SP.POP.TOTL\strut
\end{minipage} & \begin{minipage}[t]{0.11\columnwidth}\raggedright
3.08902700E+06\strut
\end{minipage} & \begin{minipage}[t]{0.11\columnwidth}\raggedright
3.06017300E+06\strut
\end{minipage} & \begin{minipage}[t]{0.11\columnwidth}\raggedright
3.05101000E+06\strut
\end{minipage} & \begin{minipage}[t]{0.11\columnwidth}\raggedright
3.03961600E+06\strut
\end{minipage} & \begin{minipage}[t]{0.11\columnwidth}\raggedright
3.02693900E+06\strut
\end{minipage} & \begin{minipage}[t]{0.11\columnwidth}\raggedright
3.01148700E+06\strut
\end{minipage}\tabularnewline
\begin{minipage}[t]{0.06\columnwidth}\raggedright
ALB\strut
\end{minipage} & \begin{minipage}[t]{0.08\columnwidth}\raggedright
SP.POP.GROW\strut
\end{minipage} & \begin{minipage}[t]{0.11\columnwidth}\raggedright
-6.37356834E-01\strut
\end{minipage} & \begin{minipage}[t]{0.11\columnwidth}\raggedright
-9.38470428E-01\strut
\end{minipage} & \begin{minipage}[t]{0.11\columnwidth}\raggedright
-2.99876697E-01\strut
\end{minipage} & \begin{minipage}[t]{0.11\columnwidth}\raggedright
-3.74149169E-01\strut
\end{minipage} & \begin{minipage}[t]{0.11\columnwidth}\raggedright
-4.17931378E-01\strut
\end{minipage} & \begin{minipage}[t]{0.11\columnwidth}\raggedright
-5.11790116E-01\strut
\end{minipage}\tabularnewline
\begin{minipage}[t]{0.06\columnwidth}\raggedright
AND\strut
\end{minipage} & \begin{minipage}[t]{0.08\columnwidth}\raggedright
SP.URB.TOTL\strut
\end{minipage} & \begin{minipage}[t]{0.11\columnwidth}\raggedright
6.04170000E+04\strut
\end{minipage} & \begin{minipage}[t]{0.11\columnwidth}\raggedright
6.19910000E+04\strut
\end{minipage} & \begin{minipage}[t]{0.11\columnwidth}\raggedright
6.41940000E+04\strut
\end{minipage} & \begin{minipage}[t]{0.11\columnwidth}\raggedright
6.67470000E+04\strut
\end{minipage} & \begin{minipage}[t]{0.11\columnwidth}\raggedright
6.91920000E+04\strut
\end{minipage} & \begin{minipage}[t]{0.11\columnwidth}\raggedright
7.12050000E+04\strut
\end{minipage}\tabularnewline
\begin{minipage}[t]{0.06\columnwidth}\raggedright
AND\strut
\end{minipage} & \begin{minipage}[t]{0.08\columnwidth}\raggedright
SP.URB.GROW\strut
\end{minipage} & \begin{minipage}[t]{0.11\columnwidth}\raggedright
1.27931383E+00\strut
\end{minipage} & \begin{minipage}[t]{0.11\columnwidth}\raggedright
2.57186909E+00\strut
\end{minipage} & \begin{minipage}[t]{0.11\columnwidth}\raggedright
3.49205352E+00\strut
\end{minipage} & \begin{minipage}[t]{0.11\columnwidth}\raggedright
3.89996041E+00\strut
\end{minipage} & \begin{minipage}[t]{0.11\columnwidth}\raggedright
3.59758966E+00\strut
\end{minipage} & \begin{minipage}[t]{0.11\columnwidth}\raggedright
2.86777917E+00\strut
\end{minipage}\tabularnewline
\begin{minipage}[t]{0.06\columnwidth}\raggedright
AND\strut
\end{minipage} & \begin{minipage}[t]{0.08\columnwidth}\raggedright
SP.POP.TOTL\strut
\end{minipage} & \begin{minipage}[t]{0.11\columnwidth}\raggedright
6.53900000E+04\strut
\end{minipage} & \begin{minipage}[t]{0.11\columnwidth}\raggedright
6.73410000E+04\strut
\end{minipage} & \begin{minipage}[t]{0.11\columnwidth}\raggedright
7.00490000E+04\strut
\end{minipage} & \begin{minipage}[t]{0.11\columnwidth}\raggedright
7.31820000E+04\strut
\end{minipage} & \begin{minipage}[t]{0.11\columnwidth}\raggedright
7.62440000E+04\strut
\end{minipage} & \begin{minipage}[t]{0.11\columnwidth}\raggedright
7.88670000E+04\strut
\end{minipage}\tabularnewline
\begin{minipage}[t]{0.06\columnwidth}\raggedright
AND\strut
\end{minipage} & \begin{minipage}[t]{0.08\columnwidth}\raggedright
SP.POP.GROW\strut
\end{minipage} & \begin{minipage}[t]{0.11\columnwidth}\raggedright
1.57216555E+00\strut
\end{minipage} & \begin{minipage}[t]{0.11\columnwidth}\raggedright
2.93999221E+00\strut
\end{minipage} & \begin{minipage}[t]{0.11\columnwidth}\raggedright
3.94257335E+00\strut
\end{minipage} & \begin{minipage}[t]{0.11\columnwidth}\raggedright
4.37544919E+00\strut
\end{minipage} & \begin{minipage}[t]{0.11\columnwidth}\raggedright
4.09892348E+00\strut
\end{minipage} & \begin{minipage}[t]{0.11\columnwidth}\raggedright
3.38241655E+00\strut
\end{minipage}\tabularnewline
\begin{minipage}[t]{0.06\columnwidth}\raggedright
ARB\strut
\end{minipage} & \begin{minipage}[t]{0.08\columnwidth}\raggedright
SP.URB.TOTL\strut
\end{minipage} & \begin{minipage}[t]{0.11\columnwidth}\raggedright
1.49981223E+08\strut
\end{minipage} & \begin{minipage}[t]{0.11\columnwidth}\raggedright
1.53924351E+08\strut
\end{minipage} & \begin{minipage}[t]{0.11\columnwidth}\raggedright
1.57985738E+08\strut
\end{minipage} & \begin{minipage}[t]{0.11\columnwidth}\raggedright
1.62267754E+08\strut
\end{minipage} & \begin{minipage}[t]{0.11\columnwidth}\raggedright
1.66820459E+08\strut
\end{minipage} & \begin{minipage}[t]{0.11\columnwidth}\raggedright
1.71813698E+08\strut
\end{minipage}\tabularnewline
\begin{minipage}[t]{0.06\columnwidth}\raggedright
ARB\strut
\end{minipage} & \begin{minipage}[t]{0.08\columnwidth}\raggedright
SP.URB.GROW\strut
\end{minipage} & \begin{minipage}[t]{0.11\columnwidth}\raggedright
2.59956290E+00\strut
\end{minipage} & \begin{minipage}[t]{0.11\columnwidth}\raggedright
2.62908111E+00\strut
\end{minipage} & \begin{minipage}[t]{0.11\columnwidth}\raggedright
2.63856042E+00\strut
\end{minipage} & \begin{minipage}[t]{0.11\columnwidth}\raggedright
2.71038136E+00\strut
\end{minipage} & \begin{minipage}[t]{0.11\columnwidth}\raggedright
2.80567450E+00\strut
\end{minipage} & \begin{minipage}[t]{0.11\columnwidth}\raggedright
2.99318143E+00\strut
\end{minipage}\tabularnewline
\begin{minipage}[t]{0.06\columnwidth}\raggedright
ARB\strut
\end{minipage} & \begin{minipage}[t]{0.08\columnwidth}\raggedright
SP.POP.TOTL\strut
\end{minipage} & \begin{minipage}[t]{0.11\columnwidth}\raggedright
2.83832016E+08\strut
\end{minipage} & \begin{minipage}[t]{0.11\columnwidth}\raggedright
2.89850357E+08\strut
\end{minipage} & \begin{minipage}[t]{0.11\columnwidth}\raggedright
2.96026575E+08\strut
\end{minipage} & \begin{minipage}[t]{0.11\columnwidth}\raggedright
3.02434519E+08\strut
\end{minipage} & \begin{minipage}[t]{0.11\columnwidth}\raggedright
3.09162029E+08\strut
\end{minipage} & \begin{minipage}[t]{0.11\columnwidth}\raggedright
3.16264728E+08\strut
\end{minipage}\tabularnewline
\begin{minipage}[t]{0.06\columnwidth}\raggedright
ARB\strut
\end{minipage} & \begin{minipage}[t]{0.08\columnwidth}\raggedright
SP.POP.GROW\strut
\end{minipage} & \begin{minipage}[t]{0.11\columnwidth}\raggedright
2.11148598E+00\strut
\end{minipage} & \begin{minipage}[t]{0.11\columnwidth}\raggedright
2.12038835E+00\strut
\end{minipage} & \begin{minipage}[t]{0.11\columnwidth}\raggedright
2.13082988E+00\strut
\end{minipage} & \begin{minipage}[t]{0.11\columnwidth}\raggedright
2.16465160E+00\strut
\end{minipage} & \begin{minipage}[t]{0.11\columnwidth}\raggedright
2.22445177E+00\strut
\end{minipage} & \begin{minipage}[t]{0.11\columnwidth}\raggedright
2.29740341E+00\strut
\end{minipage}\tabularnewline
\begin{minipage}[t]{0.06\columnwidth}\raggedright
ARE\strut
\end{minipage} & \begin{minipage}[t]{0.08\columnwidth}\raggedright
SP.URB.TOTL\strut
\end{minipage} & \begin{minipage}[t]{0.11\columnwidth}\raggedright
2.53138600E+06\strut
\end{minipage} & \begin{minipage}[t]{0.11\columnwidth}\raggedright
2.68261100E+06\strut
\end{minipage} & \begin{minipage}[t]{0.11\columnwidth}\raggedright
2.84320800E+06\strut
\end{minipage} & \begin{minipage}[t]{0.11\columnwidth}\raggedright
3.04862700E+06\strut
\end{minipage} & \begin{minipage}[t]{0.11\columnwidth}\raggedright
3.34683000E+06\strut
\end{minipage} & \begin{minipage}[t]{0.11\columnwidth}\raggedright
3.76723900E+06\strut
\end{minipage}\tabularnewline
\bottomrule
\end{longtable}

Step 1 - convert years column into values

\begin{Shaded}
\begin{Highlighting}[]
\NormalTok{(}\BuiltInTok{def}\FunctionTok{ pop2 }\NormalTok{(api/pivot->longer world-bank-pop (}\KeywordTok{map} \KeywordTok{str}\NormalTok{ (}\KeywordTok{range} \DecValTok{2000} \DecValTok{2018}\NormalTok{)) \{}\AttributeTok{:drop-missing}\NormalTok{? }\VariableTok{false}
                                                                         \AttributeTok{:target-columns}\NormalTok{ [}\StringTok{"year"}\NormalTok{]}
                                                                         \AttributeTok{:value-column-name} \StringTok{"value"}\NormalTok{\}))}
\end{Highlighting}
\end{Shaded}

\begin{Shaded}
\begin{Highlighting}[]
\NormalTok{pop2}
\end{Highlighting}
\end{Shaded}

data/world\_bank\_pop.csv.gz {[}19008 4{]}:

\begin{longtable}[]{@{}llll@{}}
\toprule
country & indicator & year & value\tabularnewline
\midrule
\endhead
ABW & SP.URB.TOTL & 2013 & 4.43600000E+04\tabularnewline
ABW & SP.URB.GROW & 2013 & 6.69503994E-01\tabularnewline
ABW & SP.POP.TOTL & 2013 & 1.03187000E+05\tabularnewline
ABW & SP.POP.GROW & 2013 & 5.92914005E-01\tabularnewline
AFG & SP.URB.TOTL & 2013 & 7.73396400E+06\tabularnewline
AFG & SP.URB.GROW & 2013 & 4.19297967E+00\tabularnewline
AFG & SP.POP.TOTL & 2013 & 3.17316880E+07\tabularnewline
AFG & SP.POP.GROW & 2013 & 3.31522413E+00\tabularnewline
AGO & SP.URB.TOTL & 2013 & 1.61194910E+07\tabularnewline
AGO & SP.URB.GROW & 2013 & 4.72272270E+00\tabularnewline
AGO & SP.POP.TOTL & 2013 & 2.59983400E+07\tabularnewline
AGO & SP.POP.GROW & 2013 & 3.53182419E+00\tabularnewline
ALB & SP.URB.TOTL & 2013 & 1.60350500E+06\tabularnewline
ALB & SP.URB.GROW & 2013 & 1.74363937E+00\tabularnewline
ALB & SP.POP.TOTL & 2013 & 2.89509200E+06\tabularnewline
ALB & SP.POP.GROW & 2013 & -1.83211385E-01\tabularnewline
AND & SP.URB.TOTL & 2013 & 7.15270000E+04\tabularnewline
AND & SP.URB.GROW & 2013 & -2.11923331E+00\tabularnewline
AND & SP.POP.TOTL & 2013 & 8.07880000E+04\tabularnewline
AND & SP.POP.GROW & 2013 & -2.01331401E+00\tabularnewline
ARB & SP.URB.TOTL & 2013 & 2.18605128E+08\tabularnewline
ARB & SP.URB.GROW & 2013 & 2.78289395E+00\tabularnewline
ARB & SP.POP.TOTL & 2013 & 3.81702086E+08\tabularnewline
ARB & SP.POP.GROW & 2013 & 2.24884429E+00\tabularnewline
ARE & SP.URB.TOTL & 2013 & 7.66126800E+06\tabularnewline
\bottomrule
\end{longtable}

Step 2 - separate \texttt{"indicate"} column

\begin{Shaded}
\begin{Highlighting}[]
\NormalTok{(}\BuiltInTok{def}\FunctionTok{ pop3 }\NormalTok{(api/separate-column pop2}
                               \StringTok{"indicator"}\NormalTok{ [}\StringTok{"area"} \StringTok{"variable"}\NormalTok{]}
\NormalTok{                               #(}\KeywordTok{rest}\NormalTok{ (clojure.string/split }\VariableTok{%} \SpecialStringTok{#"\textbackslash{}."}\NormalTok{))))}
\end{Highlighting}
\end{Shaded}

\begin{Shaded}
\begin{Highlighting}[]
\NormalTok{pop3}
\end{Highlighting}
\end{Shaded}

data/world\_bank\_pop.csv.gz {[}19008 5{]}:

\begin{longtable}[]{@{}lllll@{}}
\toprule
country & area & variable & year & value\tabularnewline
\midrule
\endhead
ABW & URB & TOTL & 2013 & 4.43600000E+04\tabularnewline
ABW & URB & GROW & 2013 & 6.69503994E-01\tabularnewline
ABW & POP & TOTL & 2013 & 1.03187000E+05\tabularnewline
ABW & POP & GROW & 2013 & 5.92914005E-01\tabularnewline
AFG & URB & TOTL & 2013 & 7.73396400E+06\tabularnewline
AFG & URB & GROW & 2013 & 4.19297967E+00\tabularnewline
AFG & POP & TOTL & 2013 & 3.17316880E+07\tabularnewline
AFG & POP & GROW & 2013 & 3.31522413E+00\tabularnewline
AGO & URB & TOTL & 2013 & 1.61194910E+07\tabularnewline
AGO & URB & GROW & 2013 & 4.72272270E+00\tabularnewline
AGO & POP & TOTL & 2013 & 2.59983400E+07\tabularnewline
AGO & POP & GROW & 2013 & 3.53182419E+00\tabularnewline
ALB & URB & TOTL & 2013 & 1.60350500E+06\tabularnewline
ALB & URB & GROW & 2013 & 1.74363937E+00\tabularnewline
ALB & POP & TOTL & 2013 & 2.89509200E+06\tabularnewline
ALB & POP & GROW & 2013 & -1.83211385E-01\tabularnewline
AND & URB & TOTL & 2013 & 7.15270000E+04\tabularnewline
AND & URB & GROW & 2013 & -2.11923331E+00\tabularnewline
AND & POP & TOTL & 2013 & 8.07880000E+04\tabularnewline
AND & POP & GROW & 2013 & -2.01331401E+00\tabularnewline
ARB & URB & TOTL & 2013 & 2.18605128E+08\tabularnewline
ARB & URB & GROW & 2013 & 2.78289395E+00\tabularnewline
ARB & POP & TOTL & 2013 & 3.81702086E+08\tabularnewline
ARB & POP & GROW & 2013 & 2.24884429E+00\tabularnewline
ARE & URB & TOTL & 2013 & 7.66126800E+06\tabularnewline
\bottomrule
\end{longtable}

Step 3 - Make columns based on \texttt{"variable"} values.

\begin{Shaded}
\begin{Highlighting}[]
\NormalTok{(api/pivot->wider pop3 }\StringTok{"variable"} \StringTok{"value"}\NormalTok{ \{}\AttributeTok{:drop-missing}\NormalTok{? }\VariableTok{false}\NormalTok{\})}
\end{Highlighting}
\end{Shaded}

data/world\_bank\_pop.csv.gz {[}9504 5{]}:

\begin{longtable}[]{@{}lllll@{}}
\toprule
country & area & year & GROW & TOTL\tabularnewline
\midrule
\endhead
ABW & URB & 2013 & 0.66950399 & 4.43600000E+04\tabularnewline
ABW & POP & 2013 & 0.59291401 & 1.03187000E+05\tabularnewline
AFG & URB & 2013 & 4.19297967 & 7.73396400E+06\tabularnewline
AFG & POP & 2013 & 3.31522413 & 3.17316880E+07\tabularnewline
AGO & URB & 2013 & 4.72272270 & 1.61194910E+07\tabularnewline
AGO & POP & 2013 & 3.53182419 & 2.59983400E+07\tabularnewline
ALB & URB & 2013 & 1.74363937 & 1.60350500E+06\tabularnewline
ALB & POP & 2013 & -0.18321138 & 2.89509200E+06\tabularnewline
AND & URB & 2013 & -2.11923331 & 7.15270000E+04\tabularnewline
AND & POP & 2013 & -2.01331401 & 8.07880000E+04\tabularnewline
ARB & URB & 2013 & 2.78289395 & 2.18605128E+08\tabularnewline
ARB & POP & 2013 & 2.24884429 & 3.81702086E+08\tabularnewline
ARE & URB & 2013 & 1.55515587 & 7.66126800E+06\tabularnewline
ARE & POP & 2013 & 1.18180499 & 9.00626300E+06\tabularnewline
ARG & URB & 2013 & 1.18764913 & 3.88172560E+07\tabularnewline
ARG & POP & 2013 & 1.04727675 & 4.25399250E+07\tabularnewline
ARM & URB & 2013 & 0.28102719 & 1.82765600E+06\tabularnewline
ARM & POP & 2013 & 0.40125198 & 2.89350900E+06\tabularnewline
ASM & URB & 2013 & 0.05797582 & 4.83100000E+04\tabularnewline
ASM & POP & 2013 & 0.13931989 & 5.53070000E+04\tabularnewline
ATG & URB & 2013 & 0.38383110 & 2.47980000E+04\tabularnewline
ATG & POP & 2013 & 1.07605830 & 9.78240000E+04\tabularnewline
AUS & URB & 2013 & 1.87536404 & 1.97902080E+07\tabularnewline
AUS & POP & 2013 & 1.75833808 & 2.31459010E+07\tabularnewline
AUT & URB & 2013 & 0.91956020 & 4.86199100E+06\tabularnewline
\bottomrule
\end{longtable}

\begin{center}\rule{0.5\linewidth}{0.5pt}\end{center}

\begin{center}\rule{0.5\linewidth}{0.5pt}\end{center}

\href{https://tidyr.tidyverse.org/articles/pivot.html\#multi-choice}{Multi-choice}

\begin{Shaded}
\begin{Highlighting}[]
\NormalTok{(}\BuiltInTok{def}\FunctionTok{ multi }\NormalTok{(api/dataset \{}\AttributeTok{:id}\NormalTok{ [}\DecValTok{1} \DecValTok{2} \DecValTok{3} \DecValTok{4}\NormalTok{]}
                         \AttributeTok{:choice1}\NormalTok{ [}\StringTok{"A"} \StringTok{"C"} \StringTok{"D"} \StringTok{"B"}\NormalTok{]}
                         \AttributeTok{:choice2}\NormalTok{ [}\StringTok{"B"} \StringTok{"B"} \VariableTok{nil} \StringTok{"D"}\NormalTok{]}
                         \AttributeTok{:choice3}\NormalTok{ [}\StringTok{"C"} \VariableTok{nil} \VariableTok{nil} \VariableTok{nil}\NormalTok{]\}))}
\end{Highlighting}
\end{Shaded}

\begin{Shaded}
\begin{Highlighting}[]
\NormalTok{multi}
\end{Highlighting}
\end{Shaded}

\_unnamed {[}4 4{]}:

\begin{longtable}[]{@{}llll@{}}
\toprule
:id & :choice1 & :choice2 & :choice3\tabularnewline
\midrule
\endhead
1 & A & B & C\tabularnewline
2 & C & B &\tabularnewline
3 & D & &\tabularnewline
4 & B & D &\tabularnewline
\bottomrule
\end{longtable}

Step 1 - convert all choices into rows and add artificial column to all
values which are not missing.

\begin{Shaded}
\begin{Highlighting}[]
\NormalTok{(}\BuiltInTok{def}\FunctionTok{ multi2 }\NormalTok{(}\KeywordTok{->}\NormalTok{ multi}
\NormalTok{                (api/pivot->longer (}\KeywordTok{complement}\NormalTok{ #\{}\AttributeTok{:id}\NormalTok{\}))}
\NormalTok{                (api/add-or-replace-column }\AttributeTok{:checked} \VariableTok{true}\NormalTok{)))}
\end{Highlighting}
\end{Shaded}

\begin{Shaded}
\begin{Highlighting}[]
\NormalTok{multi2}
\end{Highlighting}
\end{Shaded}

\_unnamed {[}8 4{]}:

\begin{longtable}[]{@{}llll@{}}
\toprule
:id & :\(column | :\)value & :checked &\tabularnewline
\midrule
\endhead
1 & :choice1 & A & true\tabularnewline
2 & :choice1 & C & true\tabularnewline
3 & :choice1 & D & true\tabularnewline
4 & :choice1 & B & true\tabularnewline
1 & :choice2 & B & true\tabularnewline
2 & :choice2 & B & true\tabularnewline
4 & :choice2 & D & true\tabularnewline
1 & :choice3 & C & true\tabularnewline
\bottomrule
\end{longtable}

Step 2 - Convert back to wide form with actual choices as columns

\begin{Shaded}
\begin{Highlighting}[]
\NormalTok{(}\KeywordTok{->}\NormalTok{ multi2}
\NormalTok{    (api/drop-columns :$column)}
\NormalTok{    (api/pivot->wider :$value }\AttributeTok{:checked}\NormalTok{ \{}\AttributeTok{:drop-missing}\NormalTok{? }\VariableTok{false}\NormalTok{\})}
\NormalTok{    (api/order-by }\AttributeTok{:id}\NormalTok{))}
\end{Highlighting}
\end{Shaded}

\_unnamed {[}4 5{]}:

\begin{longtable}[]{@{}lllll@{}}
\toprule
:id & A & B & C & D\tabularnewline
\midrule
\endhead
1 & true & true & true &\tabularnewline
2 & & true & true &\tabularnewline
3 & & & & true\tabularnewline
4 & & true & & true\tabularnewline
\bottomrule
\end{longtable}

\begin{center}\rule{0.5\linewidth}{0.5pt}\end{center}

\begin{center}\rule{0.5\linewidth}{0.5pt}\end{center}

\href{https://tidyr.tidyverse.org/articles/pivot.html\#by-hand}{Construction}

\begin{Shaded}
\begin{Highlighting}[]
\NormalTok{(}\BuiltInTok{def}\FunctionTok{ construction }\NormalTok{(api/dataset }\StringTok{"data/construction.csv"}\NormalTok{))}
\NormalTok{(}\BuiltInTok{def}\FunctionTok{ construction-unit-map }\NormalTok{\{}\StringTok{"1 unit"} \StringTok{"1"}
                            \StringTok{"2 to 4 units"} \StringTok{"2-4"}
                            \StringTok{"5 units or more"} \StringTok{"5+"}\NormalTok{\})}
\end{Highlighting}
\end{Shaded}

\begin{Shaded}
\begin{Highlighting}[]
\NormalTok{construction}
\end{Highlighting}
\end{Shaded}

data/construction.csv {[}9 9{]}:

\begin{longtable}[]{@{}lllllllll@{}}
\toprule
Year & Month & 1 unit & 2 to 4 units & 5 units or more & Northeast &
Midwest & South & West\tabularnewline
\midrule
\endhead
2018 & January & 859 & & 348 & 114 & 169 & 596 & 339\tabularnewline
2018 & February & 882 & & 400 & 138 & 160 & 655 & 336\tabularnewline
2018 & March & 862 & & 356 & 150 & 154 & 595 & 330\tabularnewline
2018 & April & 797 & & 447 & 144 & 196 & 613 & 304\tabularnewline
2018 & May & 875 & & 364 & 90 & 169 & 673 & 319\tabularnewline
2018 & June & 867 & & 342 & 76 & 170 & 610 & 360\tabularnewline
2018 & July & 829 & & 360 & 108 & 183 & 594 & 310\tabularnewline
2018 & August & 939 & & 286 & 90 & 205 & 649 & 286\tabularnewline
2018 & September & 835 & & 304 & 117 & 175 & 560 & 296\tabularnewline
\bottomrule
\end{longtable}

Conversion 1 - Group two column types

\begin{Shaded}
\begin{Highlighting}[]
\NormalTok{(}\KeywordTok{->}\NormalTok{ construction}
\NormalTok{    (api/pivot->longer }\SpecialStringTok{#"^[125NWS].*|Midwest"}\NormalTok{ \{}\AttributeTok{:target-columns}\NormalTok{ [}\AttributeTok{:units} \AttributeTok{:region}\NormalTok{]}
                                               \AttributeTok{:splitter}\NormalTok{ (}\KeywordTok{fn}\NormalTok{ [col-name]}
\NormalTok{                                                           (}\KeywordTok{if}\NormalTok{ (}\KeywordTok{re-matches} \SpecialStringTok{#"^[125].*"}\NormalTok{ col-name)}
\NormalTok{                                                             [(construction-unit-map col-name) }\VariableTok{nil}\NormalTok{]}
\NormalTok{                                                             [}\VariableTok{nil}\NormalTok{ col-name]))}
                                               \AttributeTok{:value-column-name} \AttributeTok{:n}
                                               \AttributeTok{:drop-missing}\NormalTok{? }\VariableTok{false}\NormalTok{\}))}
\end{Highlighting}
\end{Shaded}

data/construction.csv {[}63 5{]}:

\begin{longtable}[]{@{}lllll@{}}
\toprule
Year & Month & :units & :region & :n\tabularnewline
\midrule
\endhead
2018 & January & 1 & & 859\tabularnewline
2018 & February & 1 & & 882\tabularnewline
2018 & March & 1 & & 862\tabularnewline
2018 & April & 1 & & 797\tabularnewline
2018 & May & 1 & & 875\tabularnewline
2018 & June & 1 & & 867\tabularnewline
2018 & July & 1 & & 829\tabularnewline
2018 & August & 1 & & 939\tabularnewline
2018 & September & 1 & & 835\tabularnewline
2018 & January & 2-4 & &\tabularnewline
2018 & February & 2-4 & &\tabularnewline
2018 & March & 2-4 & &\tabularnewline
2018 & April & 2-4 & &\tabularnewline
2018 & May & 2-4 & &\tabularnewline
2018 & June & 2-4 & &\tabularnewline
2018 & July & 2-4 & &\tabularnewline
2018 & August & 2-4 & &\tabularnewline
2018 & September & 2-4 & &\tabularnewline
2018 & January & 5+ & & 348\tabularnewline
2018 & February & 5+ & & 400\tabularnewline
2018 & March & 5+ & & 356\tabularnewline
2018 & April & 5+ & & 447\tabularnewline
2018 & May & 5+ & & 364\tabularnewline
2018 & June & 5+ & & 342\tabularnewline
2018 & July & 5+ & & 360\tabularnewline
\bottomrule
\end{longtable}

Conversion 2 - Convert to longer form and back and rename columns

\begin{Shaded}
\begin{Highlighting}[]
\NormalTok{(}\KeywordTok{->}\NormalTok{ construction}
\NormalTok{    (api/pivot->longer }\SpecialStringTok{#"^[125NWS].*|Midwest"}\NormalTok{ \{}\AttributeTok{:target-columns}\NormalTok{ [}\AttributeTok{:units} \AttributeTok{:region}\NormalTok{]}
                                               \AttributeTok{:splitter}\NormalTok{ (}\KeywordTok{fn}\NormalTok{ [col-name]}
\NormalTok{                                                           (}\KeywordTok{if}\NormalTok{ (}\KeywordTok{re-matches} \SpecialStringTok{#"^[125].*"}\NormalTok{ col-name)}
\NormalTok{                                                             [(construction-unit-map col-name) }\VariableTok{nil}\NormalTok{]}
\NormalTok{                                                             [}\VariableTok{nil}\NormalTok{ col-name]))}
                                               \AttributeTok{:value-column-name} \AttributeTok{:n}
                                               \AttributeTok{:drop-missing}\NormalTok{? }\VariableTok{false}\NormalTok{\})}
\NormalTok{    (api/pivot->wider [}\AttributeTok{:units} \AttributeTok{:region}\NormalTok{] }\AttributeTok{:n}\NormalTok{ \{}\AttributeTok{:drop-missing}\NormalTok{? }\VariableTok{false}\NormalTok{\})}
\NormalTok{    (api/rename-columns (}\KeywordTok{zipmap}\NormalTok{ (}\KeywordTok{vals}\NormalTok{ construction-unit-map)}
\NormalTok{                                (}\KeywordTok{keys}\NormalTok{ construction-unit-map))))}
\end{Highlighting}
\end{Shaded}

data/construction.csv {[}9 9{]}:

\begin{longtable}[]{@{}lllllllll@{}}
\toprule
Year & Month & Midwest & 5 units or more & 2 to 4 units & Northeast &
South & 1 unit & West\tabularnewline
\midrule
\endhead
2018 & January & 169 & 348 & & 114 & 596 & 859 & 339\tabularnewline
2018 & February & 160 & 400 & & 138 & 655 & 882 & 336\tabularnewline
2018 & March & 154 & 356 & & 150 & 595 & 862 & 330\tabularnewline
2018 & April & 196 & 447 & & 144 & 613 & 797 & 304\tabularnewline
2018 & May & 169 & 364 & & 90 & 673 & 875 & 319\tabularnewline
2018 & June & 170 & 342 & & 76 & 610 & 867 & 360\tabularnewline
2018 & July & 183 & 360 & & 108 & 594 & 829 & 310\tabularnewline
2018 & August & 205 & 286 & & 90 & 649 & 939 & 286\tabularnewline
2018 & September & 175 & 304 & & 117 & 560 & 835 & 296\tabularnewline
\bottomrule
\end{longtable}

\begin{center}\rule{0.5\linewidth}{0.5pt}\end{center}

Various operations on stocks, examples taken from
\href{https://tidyr.tidyverse.org/reference/gather.html}{gather} and
\href{https://tidyr.tidyverse.org/reference/spread.html}{spread}
manuals.

\begin{Shaded}
\begin{Highlighting}[]
\NormalTok{(}\BuiltInTok{def}\FunctionTok{ stocks-tidyr }\NormalTok{(api/dataset }\StringTok{"data/stockstidyr.csv"}\NormalTok{))}
\end{Highlighting}
\end{Shaded}

\begin{Shaded}
\begin{Highlighting}[]
\NormalTok{stocks-tidyr}
\end{Highlighting}
\end{Shaded}

data/stockstidyr.csv {[}10 4{]}:

\begin{longtable}[]{@{}llll@{}}
\toprule
time & X & Y & Z\tabularnewline
\midrule
\endhead
2009-01-01 & 1.30989806 & -1.89040193 & -1.77946880\tabularnewline
2009-01-02 & -0.29993804 & -1.82473090 & 2.39892513\tabularnewline
2009-01-03 & 0.53647501 & -1.03606860 & -3.98697977\tabularnewline
2009-01-04 & -1.88390802 & -0.52178390 & -2.83065490\tabularnewline
2009-01-05 & -0.96052361 & -2.21683349 & 1.43715171\tabularnewline
2009-01-06 & -1.18528966 & -2.89350924 & 3.39784140\tabularnewline
2009-01-07 & -0.85207056 & -2.16794818 & -1.20108258\tabularnewline
2009-01-08 & 0.25234172 & -0.32854117 & -1.53160473\tabularnewline
2009-01-09 & 0.40257136 & 1.96407898 & -6.80878830\tabularnewline
2009-01-10 & -0.64383500 & 2.68618382 & -2.55909321\tabularnewline
\bottomrule
\end{longtable}

Convert to longer form

\begin{Shaded}
\begin{Highlighting}[]
\NormalTok{(}\BuiltInTok{def}\FunctionTok{ stocks-long }\NormalTok{(api/pivot->longer stocks-tidyr [}\StringTok{"X"} \StringTok{"Y"} \StringTok{"Z"}\NormalTok{] \{}\AttributeTok{:value-column-name} \AttributeTok{:price}
                                                                \AttributeTok{:target-columns} \AttributeTok{:stocks}\NormalTok{\}))}
\end{Highlighting}
\end{Shaded}

\begin{Shaded}
\begin{Highlighting}[]
\NormalTok{stocks-long}
\end{Highlighting}
\end{Shaded}

data/stockstidyr.csv {[}30 3{]}:

\begin{longtable}[]{@{}lll@{}}
\toprule
time & :stocks & :price\tabularnewline
\midrule
\endhead
2009-01-01 & X & 1.30989806\tabularnewline
2009-01-02 & X & -0.29993804\tabularnewline
2009-01-03 & X & 0.53647501\tabularnewline
2009-01-04 & X & -1.88390802\tabularnewline
2009-01-05 & X & -0.96052361\tabularnewline
2009-01-06 & X & -1.18528966\tabularnewline
2009-01-07 & X & -0.85207056\tabularnewline
2009-01-08 & X & 0.25234172\tabularnewline
2009-01-09 & X & 0.40257136\tabularnewline
2009-01-10 & X & -0.64383500\tabularnewline
2009-01-01 & Y & -1.89040193\tabularnewline
2009-01-02 & Y & -1.82473090\tabularnewline
2009-01-03 & Y & -1.03606860\tabularnewline
2009-01-04 & Y & -0.52178390\tabularnewline
2009-01-05 & Y & -2.21683349\tabularnewline
2009-01-06 & Y & -2.89350924\tabularnewline
2009-01-07 & Y & -2.16794818\tabularnewline
2009-01-08 & Y & -0.32854117\tabularnewline
2009-01-09 & Y & 1.96407898\tabularnewline
2009-01-10 & Y & 2.68618382\tabularnewline
2009-01-01 & Z & -1.77946880\tabularnewline
2009-01-02 & Z & 2.39892513\tabularnewline
2009-01-03 & Z & -3.98697977\tabularnewline
2009-01-04 & Z & -2.83065490\tabularnewline
2009-01-05 & Z & 1.43715171\tabularnewline
\bottomrule
\end{longtable}

Convert back to wide form

\begin{Shaded}
\begin{Highlighting}[]
\NormalTok{(api/pivot->wider stocks-long }\AttributeTok{:stocks} \AttributeTok{:price}\NormalTok{)}
\end{Highlighting}
\end{Shaded}

data/stockstidyr.csv {[}10 4{]}:

\begin{longtable}[]{@{}llll@{}}
\toprule
time & Z & X & Y\tabularnewline
\midrule
\endhead
2009-01-01 & -1.77946880 & 1.30989806 & -1.89040193\tabularnewline
2009-01-02 & 2.39892513 & -0.29993804 & -1.82473090\tabularnewline
2009-01-03 & -3.98697977 & 0.53647501 & -1.03606860\tabularnewline
2009-01-04 & -2.83065490 & -1.88390802 & -0.52178390\tabularnewline
2009-01-05 & 1.43715171 & -0.96052361 & -2.21683349\tabularnewline
2009-01-06 & 3.39784140 & -1.18528966 & -2.89350924\tabularnewline
2009-01-07 & -1.20108258 & -0.85207056 & -2.16794818\tabularnewline
2009-01-08 & -1.53160473 & 0.25234172 & -0.32854117\tabularnewline
2009-01-09 & -6.80878830 & 0.40257136 & 1.96407898\tabularnewline
2009-01-10 & -2.55909321 & -0.64383500 & 2.68618382\tabularnewline
\bottomrule
\end{longtable}

Convert to wide form on time column (let's limit values to a couple of
rows)

\begin{Shaded}
\begin{Highlighting}[]
\NormalTok{(}\KeywordTok{->}\NormalTok{ stocks-long}
\NormalTok{    (api/select-rows (}\KeywordTok{range} \DecValTok{0} \DecValTok{30} \DecValTok{4}\NormalTok{))}
\NormalTok{    (api/pivot->wider }\StringTok{"time"} \AttributeTok{:price}\NormalTok{ \{}\AttributeTok{:drop-missing}\NormalTok{? }\VariableTok{false}\NormalTok{\}))}
\end{Highlighting}
\end{Shaded}

data/stockstidyr.csv {[}3 6{]}:

\begin{longtable}[]{@{}llllll@{}}
\toprule
:stocks & 2009-01-05 & 2009-01-07 & 2009-01-01 & 2009-01-03 &
2009-01-09\tabularnewline
\midrule
\endhead
X & -0.96052361 & & 1.30989806 & & 0.40257136\tabularnewline
Z & 1.43715171 & & -1.77946880 & & -6.80878830\tabularnewline
Y & & -2.16794818 & & -1.0360686 &\tabularnewline
\bottomrule
\end{longtable}

\hypertarget{joinconcat-datasets}{%
\subsubsection{Join/Concat Datasets}\label{joinconcat-datasets}}

Dataset join and concatenation functions.

Joins accept left-side and right-side datasets and columns selector.
Options are the same as in \texttt{tech.ml.dataset} functions.

The difference between \texttt{tech.ml.dataset} join functions are:
arguments order (first datasets) and possibility to join on multiple
columns.

Additionally set operations are defined: \texttt{intersect} and
\texttt{difference}.

To concat two datasets rowwise you can choose:

\begin{itemize}
\tightlist
\item
  \texttt{concat} - concats rows for matching columns, the number of
  columns should be equal.
\item
  \texttt{union} - like concat but returns unique values
\item
  \texttt{bind} - concats rows add missing, empty columns
\end{itemize}

To add two datasets columnwise use \texttt{bind}. The number of rows
should be equal.

Datasets used in examples:

\begin{Shaded}
\begin{Highlighting}[]
\NormalTok{(}\BuiltInTok{def}\FunctionTok{ ds1 }\NormalTok{(api/dataset \{}\AttributeTok{:a}\NormalTok{ [}\DecValTok{1} \DecValTok{2} \DecValTok{1} \DecValTok{2} \DecValTok{3} \DecValTok{4} \VariableTok{nil} \VariableTok{nil} \DecValTok{4}\NormalTok{]}
                       \AttributeTok{:b}\NormalTok{ (}\KeywordTok{range} \DecValTok{101} \DecValTok{110}\NormalTok{)}
                       \AttributeTok{:c}\NormalTok{ (}\KeywordTok{map} \KeywordTok{str} \StringTok{"abs tract"}\NormalTok{)\}))}
\NormalTok{(}\BuiltInTok{def}\FunctionTok{ ds2 }\NormalTok{(api/dataset \{}\AttributeTok{:a}\NormalTok{ [}\VariableTok{nil} \DecValTok{1} \DecValTok{2} \DecValTok{5} \DecValTok{4} \DecValTok{3} \DecValTok{2} \DecValTok{1} \VariableTok{nil}\NormalTok{]}
                       \AttributeTok{:b}\NormalTok{ (}\KeywordTok{range} \DecValTok{110} \DecValTok{101} \DecValTok{-1}\NormalTok{)}
                       \AttributeTok{:c}\NormalTok{ (}\KeywordTok{map} \KeywordTok{str} \StringTok{"datatable"}\NormalTok{)}
                       \AttributeTok{:d}\NormalTok{ (}\KeywordTok{symbol} \StringTok{"X"}\NormalTok{)\}))}
\end{Highlighting}
\end{Shaded}

\begin{Shaded}
\begin{Highlighting}[]
\NormalTok{ds1}
\NormalTok{ds2}
\end{Highlighting}
\end{Shaded}

\_unnamed {[}9 3{]}:

\begin{longtable}[]{@{}lll@{}}
\toprule
:a & :b & :c\tabularnewline
\midrule
\endhead
1 & 101 & a\tabularnewline
2 & 102 & b\tabularnewline
1 & 103 & s\tabularnewline
2 & 104 &\tabularnewline
3 & 105 & t\tabularnewline
4 & 106 & r\tabularnewline
& 107 & a\tabularnewline
& 108 & c\tabularnewline
4 & 109 & t\tabularnewline
\bottomrule
\end{longtable}

\_unnamed {[}9 4{]}:

\begin{longtable}[]{@{}llll@{}}
\toprule
:a & :b & :c & :d\tabularnewline
\midrule
\endhead
& 110 & d & X\tabularnewline
1 & 109 & a & X\tabularnewline
2 & 108 & t & X\tabularnewline
5 & 107 & a & X\tabularnewline
4 & 106 & t & X\tabularnewline
3 & 105 & a & X\tabularnewline
2 & 104 & b & X\tabularnewline
1 & 103 & l & X\tabularnewline
& 102 & e & X\tabularnewline
\bottomrule
\end{longtable}

\hypertarget{left}{%
\paragraph{Left}\label{left}}

\begin{Shaded}
\begin{Highlighting}[]
\NormalTok{(api/left-join ds1 ds2 }\AttributeTok{:b}\NormalTok{)}
\end{Highlighting}
\end{Shaded}

left-outer-join {[}9 7{]}:

\begin{longtable}[]{@{}lllllll@{}}
\toprule
:b & :a & :c & :right.b & :right.a & :right.c & :d\tabularnewline
\midrule
\endhead
109 & 4 & t & 109 & 1 & a & X\tabularnewline
108 & & c & 108 & 2 & t & X\tabularnewline
107 & & a & 107 & 5 & a & X\tabularnewline
106 & 4 & r & 106 & 4 & t & X\tabularnewline
105 & 3 & t & 105 & 3 & a & X\tabularnewline
104 & 2 & & 104 & 2 & b & X\tabularnewline
103 & 1 & s & 103 & 1 & l & X\tabularnewline
102 & 2 & b & 102 & & e & X\tabularnewline
101 & 1 & a & & & &\tabularnewline
\bottomrule
\end{longtable}

\begin{center}\rule{0.5\linewidth}{0.5pt}\end{center}

\begin{Shaded}
\begin{Highlighting}[]
\NormalTok{(api/left-join ds2 ds1 }\AttributeTok{:b}\NormalTok{)}
\end{Highlighting}
\end{Shaded}

left-outer-join {[}9 7{]}:

\begin{longtable}[]{@{}lllllll@{}}
\toprule
:b & :a & :c & :d & :right.b & :right.a & :right.c\tabularnewline
\midrule
\endhead
102 & & e & X & 102 & 2 & b\tabularnewline
103 & 1 & l & X & 103 & 1 & s\tabularnewline
104 & 2 & b & X & 104 & 2 &\tabularnewline
105 & 3 & a & X & 105 & 3 & t\tabularnewline
106 & 4 & t & X & 106 & 4 & r\tabularnewline
107 & 5 & a & X & 107 & & a\tabularnewline
108 & 2 & t & X & 108 & & c\tabularnewline
109 & 1 & a & X & 109 & 4 & t\tabularnewline
110 & & d & X & & &\tabularnewline
\bottomrule
\end{longtable}

\begin{center}\rule{0.5\linewidth}{0.5pt}\end{center}

\begin{Shaded}
\begin{Highlighting}[]
\NormalTok{(api/left-join ds1 ds2 [}\AttributeTok{:a} \AttributeTok{:b}\NormalTok{])}
\end{Highlighting}
\end{Shaded}

left-outer-join {[}9 7{]}:

\begin{longtable}[]{@{}lllllll@{}}
\toprule
:a & :b & :c & :right.a & :right.b & :right.c & :d\tabularnewline
\midrule
\endhead
4 & 106 & r & 4 & 106 & t & X\tabularnewline
3 & 105 & t & 3 & 105 & a & X\tabularnewline
2 & 104 & & 2 & 104 & b & X\tabularnewline
1 & 103 & s & 1 & 103 & l & X\tabularnewline
2 & 102 & b & & & &\tabularnewline
& 108 & c & & & &\tabularnewline
& 107 & a & & & &\tabularnewline
1 & 101 & a & & & &\tabularnewline
4 & 109 & t & & & &\tabularnewline
\bottomrule
\end{longtable}

\begin{center}\rule{0.5\linewidth}{0.5pt}\end{center}

\begin{Shaded}
\begin{Highlighting}[]
\NormalTok{(api/left-join ds2 ds1 [}\AttributeTok{:a} \AttributeTok{:b}\NormalTok{])}
\end{Highlighting}
\end{Shaded}

left-outer-join {[}9 7{]}:

\begin{longtable}[]{@{}lllllll@{}}
\toprule
:a & :b & :c & :d & :right.a & :right.b & :right.c\tabularnewline
\midrule
\endhead
1 & 103 & l & X & 1 & 103 & s\tabularnewline
2 & 104 & b & X & 2 & 104 &\tabularnewline
3 & 105 & a & X & 3 & 105 & t\tabularnewline
4 & 106 & t & X & 4 & 106 & r\tabularnewline
2 & 108 & t & X & & &\tabularnewline
1 & 109 & a & X & & &\tabularnewline
5 & 107 & a & X & & &\tabularnewline
& 110 & d & X & & &\tabularnewline
& 102 & e & X & & &\tabularnewline
\bottomrule
\end{longtable}

\hypertarget{right}{%
\paragraph{Right}\label{right}}

\begin{Shaded}
\begin{Highlighting}[]
\NormalTok{(api/right-join ds1 ds2 }\AttributeTok{:b}\NormalTok{)}
\end{Highlighting}
\end{Shaded}

right-outer-join {[}9 7{]}:

\begin{longtable}[]{@{}lllllll@{}}
\toprule
:b & :a & :c & :right.b & :right.a & :right.c & :d\tabularnewline
\midrule
\endhead
109 & 4 & t & 109 & 1 & a & X\tabularnewline
108 & & c & 108 & 2 & t & X\tabularnewline
107 & & a & 107 & 5 & a & X\tabularnewline
106 & 4 & r & 106 & 4 & t & X\tabularnewline
105 & 3 & t & 105 & 3 & a & X\tabularnewline
104 & 2 & & 104 & 2 & b & X\tabularnewline
103 & 1 & s & 103 & 1 & l & X\tabularnewline
102 & 2 & b & 102 & & e & X\tabularnewline
& & & 110 & & d & X\tabularnewline
\bottomrule
\end{longtable}

\begin{center}\rule{0.5\linewidth}{0.5pt}\end{center}

\begin{Shaded}
\begin{Highlighting}[]
\NormalTok{(api/right-join ds2 ds1 }\AttributeTok{:b}\NormalTok{)}
\end{Highlighting}
\end{Shaded}

right-outer-join {[}9 7{]}:

\begin{longtable}[]{@{}lllllll@{}}
\toprule
:b & :a & :c & :d & :right.b & :right.a & :right.c\tabularnewline
\midrule
\endhead
102 & & e & X & 102 & 2 & b\tabularnewline
103 & 1 & l & X & 103 & 1 & s\tabularnewline
104 & 2 & b & X & 104 & 2 &\tabularnewline
105 & 3 & a & X & 105 & 3 & t\tabularnewline
106 & 4 & t & X & 106 & 4 & r\tabularnewline
107 & 5 & a & X & 107 & & a\tabularnewline
108 & 2 & t & X & 108 & & c\tabularnewline
109 & 1 & a & X & 109 & 4 & t\tabularnewline
& & & & 101 & 1 & a\tabularnewline
\bottomrule
\end{longtable}

\begin{center}\rule{0.5\linewidth}{0.5pt}\end{center}

\begin{Shaded}
\begin{Highlighting}[]
\NormalTok{(api/right-join ds1 ds2 [}\AttributeTok{:a} \AttributeTok{:b}\NormalTok{])}
\end{Highlighting}
\end{Shaded}

right-outer-join {[}9 7{]}:

\begin{longtable}[]{@{}lllllll@{}}
\toprule
:a & :b & :c & :right.a & :right.b & :right.c & :d\tabularnewline
\midrule
\endhead
4 & 106 & r & 4 & 106 & t & X\tabularnewline
3 & 105 & t & 3 & 105 & a & X\tabularnewline
2 & 104 & & 2 & 104 & b & X\tabularnewline
1 & 103 & s & 1 & 103 & l & X\tabularnewline
& & & & 110 & d & X\tabularnewline
& & & 1 & 109 & a & X\tabularnewline
& & & 2 & 108 & t & X\tabularnewline
& & & 5 & 107 & a & X\tabularnewline
& & & & 102 & e & X\tabularnewline
\bottomrule
\end{longtable}

\begin{center}\rule{0.5\linewidth}{0.5pt}\end{center}

\begin{Shaded}
\begin{Highlighting}[]
\NormalTok{(api/right-join ds2 ds1 [}\AttributeTok{:a} \AttributeTok{:b}\NormalTok{])}
\end{Highlighting}
\end{Shaded}

right-outer-join {[}9 7{]}:

\begin{longtable}[]{@{}lllllll@{}}
\toprule
:a & :b & :c & :d & :right.a & :right.b & :right.c\tabularnewline
\midrule
\endhead
1 & 103 & l & X & 1 & 103 & s\tabularnewline
2 & 104 & b & X & 2 & 104 &\tabularnewline
3 & 105 & a & X & 3 & 105 & t\tabularnewline
4 & 106 & t & X & 4 & 106 & r\tabularnewline
& & & & 1 & 101 & a\tabularnewline
& & & & 2 & 102 & b\tabularnewline
& & & & & 107 & a\tabularnewline
& & & & & 108 & c\tabularnewline
& & & & 4 & 109 & t\tabularnewline
\bottomrule
\end{longtable}

\hypertarget{inner}{%
\paragraph{Inner}\label{inner}}

\begin{Shaded}
\begin{Highlighting}[]
\NormalTok{(api/inner-join ds1 ds2 }\AttributeTok{:b}\NormalTok{)}
\end{Highlighting}
\end{Shaded}

inner-join {[}8 6{]}:

\begin{longtable}[]{@{}llllll@{}}
\toprule
:b & :a & :c & :right.a & :right.c & :d\tabularnewline
\midrule
\endhead
109 & 4 & t & 1 & a & X\tabularnewline
108 & & c & 2 & t & X\tabularnewline
107 & & a & 5 & a & X\tabularnewline
106 & 4 & r & 4 & t & X\tabularnewline
105 & 3 & t & 3 & a & X\tabularnewline
104 & 2 & & 2 & b & X\tabularnewline
103 & 1 & s & 1 & l & X\tabularnewline
102 & 2 & b & & e & X\tabularnewline
\bottomrule
\end{longtable}

\begin{center}\rule{0.5\linewidth}{0.5pt}\end{center}

\begin{Shaded}
\begin{Highlighting}[]
\NormalTok{(api/inner-join ds2 ds1 }\AttributeTok{:b}\NormalTok{)}
\end{Highlighting}
\end{Shaded}

inner-join {[}8 6{]}:

\begin{longtable}[]{@{}llllll@{}}
\toprule
:b & :a & :c & :d & :right.a & :right.c\tabularnewline
\midrule
\endhead
102 & & e & X & 2 & b\tabularnewline
103 & 1 & l & X & 1 & s\tabularnewline
104 & 2 & b & X & 2 &\tabularnewline
105 & 3 & a & X & 3 & t\tabularnewline
106 & 4 & t & X & 4 & r\tabularnewline
107 & 5 & a & X & & a\tabularnewline
108 & 2 & t & X & & c\tabularnewline
109 & 1 & a & X & 4 & t\tabularnewline
\bottomrule
\end{longtable}

\begin{center}\rule{0.5\linewidth}{0.5pt}\end{center}

\begin{Shaded}
\begin{Highlighting}[]
\NormalTok{(api/inner-join ds1 ds2 [}\AttributeTok{:a} \AttributeTok{:b}\NormalTok{])}
\end{Highlighting}
\end{Shaded}

inner-join {[}4 7{]}:

\begin{longtable}[]{@{}lllllll@{}}
\toprule
:a & :b & :c & :right.a & :right.b & :right.c & :d\tabularnewline
\midrule
\endhead
4 & 106 & r & 4 & 106 & t & X\tabularnewline
3 & 105 & t & 3 & 105 & a & X\tabularnewline
2 & 104 & & 2 & 104 & b & X\tabularnewline
1 & 103 & s & 1 & 103 & l & X\tabularnewline
\bottomrule
\end{longtable}

\begin{center}\rule{0.5\linewidth}{0.5pt}\end{center}

\begin{Shaded}
\begin{Highlighting}[]
\NormalTok{(api/inner-join ds2 ds1 [}\AttributeTok{:a} \AttributeTok{:b}\NormalTok{])}
\end{Highlighting}
\end{Shaded}

inner-join {[}4 7{]}:

\begin{longtable}[]{@{}lllllll@{}}
\toprule
:a & :b & :c & :d & :right.a & :right.b & :right.c\tabularnewline
\midrule
\endhead
1 & 103 & l & X & 1 & 103 & s\tabularnewline
2 & 104 & b & X & 2 & 104 &\tabularnewline
3 & 105 & a & X & 3 & 105 & t\tabularnewline
4 & 106 & t & X & 4 & 106 & r\tabularnewline
\bottomrule
\end{longtable}

\hypertarget{full}{%
\paragraph{Full}\label{full}}

Join keeping all rows

\begin{Shaded}
\begin{Highlighting}[]
\NormalTok{(api/full-join ds1 ds2 }\AttributeTok{:b}\NormalTok{)}
\end{Highlighting}
\end{Shaded}

full-join {[}10 7{]}:

\begin{longtable}[]{@{}lllllll@{}}
\toprule
:b & :a & :c & :right.b & :right.a & :right.c & :d\tabularnewline
\midrule
\endhead
109 & 4 & t & 109 & 1 & a & X\tabularnewline
108 & & c & 108 & 2 & t & X\tabularnewline
107 & & a & 107 & 5 & a & X\tabularnewline
106 & 4 & r & 106 & 4 & t & X\tabularnewline
105 & 3 & t & 105 & 3 & a & X\tabularnewline
104 & 2 & & 104 & 2 & b & X\tabularnewline
103 & 1 & s & 103 & 1 & l & X\tabularnewline
102 & 2 & b & 102 & & e & X\tabularnewline
101 & 1 & a & & & &\tabularnewline
& & & 110 & & d & X\tabularnewline
\bottomrule
\end{longtable}

\begin{center}\rule{0.5\linewidth}{0.5pt}\end{center}

\begin{Shaded}
\begin{Highlighting}[]
\NormalTok{(api/full-join ds2 ds1 }\AttributeTok{:b}\NormalTok{)}
\end{Highlighting}
\end{Shaded}

full-join {[}10 7{]}:

\begin{longtable}[]{@{}lllllll@{}}
\toprule
:b & :a & :c & :d & :right.b & :right.a & :right.c\tabularnewline
\midrule
\endhead
102 & & e & X & 102 & 2 & b\tabularnewline
103 & 1 & l & X & 103 & 1 & s\tabularnewline
104 & 2 & b & X & 104 & 2 &\tabularnewline
105 & 3 & a & X & 105 & 3 & t\tabularnewline
106 & 4 & t & X & 106 & 4 & r\tabularnewline
107 & 5 & a & X & 107 & & a\tabularnewline
108 & 2 & t & X & 108 & & c\tabularnewline
109 & 1 & a & X & 109 & 4 & t\tabularnewline
110 & & d & X & & &\tabularnewline
& & & & 101 & 1 & a\tabularnewline
\bottomrule
\end{longtable}

\begin{center}\rule{0.5\linewidth}{0.5pt}\end{center}

\begin{Shaded}
\begin{Highlighting}[]
\NormalTok{(api/full-join ds1 ds2 [}\AttributeTok{:a} \AttributeTok{:b}\NormalTok{])}
\end{Highlighting}
\end{Shaded}

full-join {[}14 7{]}:

\begin{longtable}[]{@{}lllllll@{}}
\toprule
:a & :b & :c & :right.a & :right.b & :right.c & :d\tabularnewline
\midrule
\endhead
4 & 106 & r & 4 & 106 & t & X\tabularnewline
3 & 105 & t & 3 & 105 & a & X\tabularnewline
2 & 104 & & 2 & 104 & b & X\tabularnewline
1 & 103 & s & 1 & 103 & l & X\tabularnewline
2 & 102 & b & & & &\tabularnewline
& 108 & c & & & &\tabularnewline
& 107 & a & & & &\tabularnewline
1 & 101 & a & & & &\tabularnewline
4 & 109 & t & & & &\tabularnewline
& & & & 110 & d & X\tabularnewline
& & & 1 & 109 & a & X\tabularnewline
& & & 2 & 108 & t & X\tabularnewline
& & & 5 & 107 & a & X\tabularnewline
& & & & 102 & e & X\tabularnewline
\bottomrule
\end{longtable}

\begin{center}\rule{0.5\linewidth}{0.5pt}\end{center}

\begin{Shaded}
\begin{Highlighting}[]
\NormalTok{(api/full-join ds2 ds1 [}\AttributeTok{:a} \AttributeTok{:b}\NormalTok{])}
\end{Highlighting}
\end{Shaded}

full-join {[}14 7{]}:

\begin{longtable}[]{@{}lllllll@{}}
\toprule
:a & :b & :c & :d & :right.a & :right.b & :right.c\tabularnewline
\midrule
\endhead
1 & 103 & l & X & 1 & 103 & s\tabularnewline
2 & 104 & b & X & 2 & 104 &\tabularnewline
3 & 105 & a & X & 3 & 105 & t\tabularnewline
4 & 106 & t & X & 4 & 106 & r\tabularnewline
2 & 108 & t & X & & &\tabularnewline
1 & 109 & a & X & & &\tabularnewline
5 & 107 & a & X & & &\tabularnewline
& 110 & d & X & & &\tabularnewline
& 102 & e & X & & &\tabularnewline
& & & & 1 & 101 & a\tabularnewline
& & & & 2 & 102 & b\tabularnewline
& & & & & 107 & a\tabularnewline
& & & & & 108 & c\tabularnewline
& & & & 4 & 109 & t\tabularnewline
\bottomrule
\end{longtable}

\hypertarget{semi}{%
\paragraph{Semi}\label{semi}}

Return rows from ds1 matching ds2

\begin{Shaded}
\begin{Highlighting}[]
\NormalTok{(api/semi-join ds1 ds2 }\AttributeTok{:b}\NormalTok{)}
\end{Highlighting}
\end{Shaded}

semi-join {[}5 3{]}:

\begin{longtable}[]{@{}lll@{}}
\toprule
:b & :a & :c\tabularnewline
\midrule
\endhead
109 & 4 & t\tabularnewline
106 & 4 & r\tabularnewline
105 & 3 & t\tabularnewline
104 & 2 &\tabularnewline
103 & 1 & s\tabularnewline
\bottomrule
\end{longtable}

\begin{center}\rule{0.5\linewidth}{0.5pt}\end{center}

\begin{Shaded}
\begin{Highlighting}[]
\NormalTok{(api/semi-join ds2 ds1 }\AttributeTok{:b}\NormalTok{)}
\end{Highlighting}
\end{Shaded}

semi-join {[}5 4{]}:

\begin{longtable}[]{@{}llll@{}}
\toprule
:b & :a & :c & :d\tabularnewline
\midrule
\endhead
103 & 1 & l & X\tabularnewline
104 & 2 & b & X\tabularnewline
105 & 3 & a & X\tabularnewline
106 & 4 & t & X\tabularnewline
109 & 1 & a & X\tabularnewline
\bottomrule
\end{longtable}

\begin{center}\rule{0.5\linewidth}{0.5pt}\end{center}

\begin{Shaded}
\begin{Highlighting}[]
\NormalTok{(api/semi-join ds1 ds2 [}\AttributeTok{:a} \AttributeTok{:b}\NormalTok{])}
\end{Highlighting}
\end{Shaded}

semi-join {[}4 3{]}:

\begin{longtable}[]{@{}lll@{}}
\toprule
:a & :b & :c\tabularnewline
\midrule
\endhead
4 & 106 & r\tabularnewline
3 & 105 & t\tabularnewline
2 & 104 &\tabularnewline
1 & 103 & s\tabularnewline
\bottomrule
\end{longtable}

\begin{center}\rule{0.5\linewidth}{0.5pt}\end{center}

\begin{Shaded}
\begin{Highlighting}[]
\NormalTok{(api/semi-join ds2 ds1 [}\AttributeTok{:a} \AttributeTok{:b}\NormalTok{])}
\end{Highlighting}
\end{Shaded}

semi-join {[}4 4{]}:

\begin{longtable}[]{@{}llll@{}}
\toprule
:a & :b & :c & :d\tabularnewline
\midrule
\endhead
1 & 103 & l & X\tabularnewline
2 & 104 & b & X\tabularnewline
3 & 105 & a & X\tabularnewline
4 & 106 & t & X\tabularnewline
\bottomrule
\end{longtable}

\hypertarget{anti}{%
\paragraph{Anti}\label{anti}}

Return rows from ds1 not matching ds2

\begin{Shaded}
\begin{Highlighting}[]
\NormalTok{(api/anti-join ds1 ds2 }\AttributeTok{:b}\NormalTok{)}
\end{Highlighting}
\end{Shaded}

anti-join {[}4 3{]}:

\begin{longtable}[]{@{}lll@{}}
\toprule
:b & :a & :c\tabularnewline
\midrule
\endhead
108 & & c\tabularnewline
107 & & a\tabularnewline
102 & 2 & b\tabularnewline
101 & 1 & a\tabularnewline
\bottomrule
\end{longtable}

\begin{center}\rule{0.5\linewidth}{0.5pt}\end{center}

\begin{Shaded}
\begin{Highlighting}[]
\NormalTok{(api/anti-join ds2 ds1 }\AttributeTok{:b}\NormalTok{)}
\end{Highlighting}
\end{Shaded}

anti-join {[}4 4{]}:

\begin{longtable}[]{@{}llll@{}}
\toprule
:b & :a & :c & :d\tabularnewline
\midrule
\endhead
102 & & e & X\tabularnewline
107 & 5 & a & X\tabularnewline
108 & 2 & t & X\tabularnewline
110 & & d & X\tabularnewline
\bottomrule
\end{longtable}

\begin{center}\rule{0.5\linewidth}{0.5pt}\end{center}

\begin{Shaded}
\begin{Highlighting}[]
\NormalTok{(api/anti-join ds1 ds2 [}\AttributeTok{:a} \AttributeTok{:b}\NormalTok{])}
\end{Highlighting}
\end{Shaded}

anti-join {[}5 3{]}:

\begin{longtable}[]{@{}lll@{}}
\toprule
:a & :b & :c\tabularnewline
\midrule
\endhead
2 & 102 & b\tabularnewline
& 108 & c\tabularnewline
& 107 & a\tabularnewline
1 & 101 & a\tabularnewline
4 & 109 & t\tabularnewline
\bottomrule
\end{longtable}

\begin{center}\rule{0.5\linewidth}{0.5pt}\end{center}

\begin{Shaded}
\begin{Highlighting}[]
\NormalTok{(api/anti-join ds2 ds1 [}\AttributeTok{:a} \AttributeTok{:b}\NormalTok{])}
\end{Highlighting}
\end{Shaded}

anti-join {[}5 4{]}:

\begin{longtable}[]{@{}llll@{}}
\toprule
:a & :b & :c & :d\tabularnewline
\midrule
\endhead
2 & 108 & t & X\tabularnewline
1 & 109 & a & X\tabularnewline
5 & 107 & a & X\tabularnewline
& 110 & d & X\tabularnewline
& 102 & e & X\tabularnewline
\bottomrule
\end{longtable}

\hypertarget{asof}{%
\paragraph{asof}\label{asof}}

\begin{Shaded}
\begin{Highlighting}[]
\NormalTok{(}\BuiltInTok{def}\FunctionTok{ left-ds }\NormalTok{(api/dataset \{}\AttributeTok{:a}\NormalTok{ [}\DecValTok{1} \DecValTok{5} \DecValTok{10}\NormalTok{]}
                           \AttributeTok{:left-val}\NormalTok{ [}\StringTok{"a"} \StringTok{"b"} \StringTok{"c"}\NormalTok{]\}))}
\NormalTok{(}\BuiltInTok{def}\FunctionTok{ right-ds }\NormalTok{(api/dataset \{}\AttributeTok{:a}\NormalTok{ [}\DecValTok{1} \DecValTok{2} \DecValTok{3} \DecValTok{6} \DecValTok{7}\NormalTok{]}
                            \AttributeTok{:right-val}\NormalTok{ [}\AttributeTok{:a} \AttributeTok{:b} \AttributeTok{:c} \AttributeTok{:d} \AttributeTok{:e}\NormalTok{]\}))}
\end{Highlighting}
\end{Shaded}

\begin{Shaded}
\begin{Highlighting}[]
\NormalTok{left-ds}
\NormalTok{right-ds}
\end{Highlighting}
\end{Shaded}

\_unnamed {[}3 2{]}:

\begin{longtable}[]{@{}ll@{}}
\toprule
:a & :left-val\tabularnewline
\midrule
\endhead
1 & a\tabularnewline
5 & b\tabularnewline
10 & c\tabularnewline
\bottomrule
\end{longtable}

\_unnamed {[}5 2{]}:

\begin{longtable}[]{@{}ll@{}}
\toprule
:a & :right-val\tabularnewline
\midrule
\endhead
1 & :a\tabularnewline
2 & :b\tabularnewline
3 & :c\tabularnewline
6 & :d\tabularnewline
7 & :e\tabularnewline
\bottomrule
\end{longtable}

\begin{Shaded}
\begin{Highlighting}[]
\NormalTok{(api/asof-join left-ds right-ds }\AttributeTok{:a}\NormalTok{)}
\end{Highlighting}
\end{Shaded}

asof-\textless{}= {[}3 4{]}:

\begin{longtable}[]{@{}llll@{}}
\toprule
:a & :left-val & :right.a & :right-val\tabularnewline
\midrule
\endhead
1 & a & 1 & :a\tabularnewline
5 & b & 6 & :d\tabularnewline
10 & c & &\tabularnewline
\bottomrule
\end{longtable}

\begin{Shaded}
\begin{Highlighting}[]
\NormalTok{(api/asof-join left-ds right-ds }\AttributeTok{:a}\NormalTok{ \{}\AttributeTok{:asof-op} \AttributeTok{:nearest}\NormalTok{\})}
\end{Highlighting}
\end{Shaded}

asof-nearest {[}3 4{]}:

\begin{longtable}[]{@{}llll@{}}
\toprule
:a & :left-val & :right.a & :right-val\tabularnewline
\midrule
\endhead
1 & a & 1 & :a\tabularnewline
5 & b & 6 & :d\tabularnewline
10 & c & 7 & :e\tabularnewline
\bottomrule
\end{longtable}

\begin{Shaded}
\begin{Highlighting}[]
\NormalTok{(api/asof-join left-ds right-ds }\AttributeTok{:a}\NormalTok{ \{}\AttributeTok{:asof-op}\NormalTok{ :>=\})}
\end{Highlighting}
\end{Shaded}

asof-\textgreater{}= {[}3 4{]}:

\begin{longtable}[]{@{}llll@{}}
\toprule
:a & :left-val & :right.a & :right-val\tabularnewline
\midrule
\endhead
1 & a & 1 & :a\tabularnewline
5 & b & 3 & :c\tabularnewline
10 & c & 7 & :e\tabularnewline
\bottomrule
\end{longtable}

\hypertarget{concat}{%
\paragraph{Concat}\label{concat}}

\texttt{contact} joins rows from other datasets

\begin{Shaded}
\begin{Highlighting}[]
\NormalTok{(api/concat ds1)}
\end{Highlighting}
\end{Shaded}

null {[}9 3{]}:

\begin{longtable}[]{@{}lll@{}}
\toprule
:a & :b & :c\tabularnewline
\midrule
\endhead
1 & 101 & a\tabularnewline
2 & 102 & b\tabularnewline
1 & 103 & s\tabularnewline
2 & 104 &\tabularnewline
3 & 105 & t\tabularnewline
4 & 106 & r\tabularnewline
& 107 & a\tabularnewline
& 108 & c\tabularnewline
4 & 109 & t\tabularnewline
\bottomrule
\end{longtable}

\begin{center}\rule{0.5\linewidth}{0.5pt}\end{center}

\begin{Shaded}
\begin{Highlighting}[]
\NormalTok{(api/concat ds1 (api/drop-columns ds2 }\AttributeTok{:d}\NormalTok{))}
\end{Highlighting}
\end{Shaded}

null {[}18 3{]}:

\begin{longtable}[]{@{}lll@{}}
\toprule
:a & :b & :c\tabularnewline
\midrule
\endhead
1 & 101 & a\tabularnewline
2 & 102 & b\tabularnewline
1 & 103 & s\tabularnewline
2 & 104 &\tabularnewline
3 & 105 & t\tabularnewline
4 & 106 & r\tabularnewline
& 107 & a\tabularnewline
& 108 & c\tabularnewline
4 & 109 & t\tabularnewline
& 110 & d\tabularnewline
1 & 109 & a\tabularnewline
2 & 108 & t\tabularnewline
5 & 107 & a\tabularnewline
4 & 106 & t\tabularnewline
3 & 105 & a\tabularnewline
2 & 104 & b\tabularnewline
1 & 103 & l\tabularnewline
& 102 & e\tabularnewline
\bottomrule
\end{longtable}

\begin{center}\rule{0.5\linewidth}{0.5pt}\end{center}

\begin{Shaded}
\begin{Highlighting}[]
\NormalTok{(}\KeywordTok{apply}\NormalTok{ api/concat (}\KeywordTok{repeatedly} \DecValTok{3}\NormalTok{ #(api/random DS)))}
\end{Highlighting}
\end{Shaded}

null {[}27 4{]}:

\begin{longtable}[]{@{}llll@{}}
\toprule
:V1 & :V2 & :V3 & :V4\tabularnewline
\midrule
\endhead
2 & 6 & 1.5 & C\tabularnewline
2 & 2 & 1.0 & B\tabularnewline
2 & 4 & 0.5 & A\tabularnewline
1 & 5 & 1.0 & B\tabularnewline
1 & 1 & 0.5 & A\tabularnewline
1 & 9 & 1.5 & C\tabularnewline
2 & 8 & 1.0 & B\tabularnewline
1 & 9 & 1.5 & C\tabularnewline
1 & 7 & 0.5 & A\tabularnewline
1 & 5 & 1.0 & B\tabularnewline
1 & 5 & 1.0 & B\tabularnewline
2 & 4 & 0.5 & A\tabularnewline
2 & 8 & 1.0 & B\tabularnewline
1 & 3 & 1.5 & C\tabularnewline
2 & 4 & 0.5 & A\tabularnewline
1 & 7 & 0.5 & A\tabularnewline
1 & 7 & 0.5 & A\tabularnewline
1 & 1 & 0.5 & A\tabularnewline
1 & 1 & 0.5 & A\tabularnewline
1 & 3 & 1.5 & C\tabularnewline
2 & 2 & 1.0 & B\tabularnewline
2 & 8 & 1.0 & B\tabularnewline
1 & 5 & 1.0 & B\tabularnewline
1 & 1 & 0.5 & A\tabularnewline
2 & 2 & 1.0 & B\tabularnewline
\bottomrule
\end{longtable}

\hypertarget{union}{%
\paragraph{Union}\label{union}}

The same as \texttt{concat} but returns unique rows

\begin{Shaded}
\begin{Highlighting}[]
\NormalTok{(}\KeywordTok{apply}\NormalTok{ api/union (api/drop-columns ds2 }\AttributeTok{:d}\NormalTok{) (}\KeywordTok{repeat} \DecValTok{10}\NormalTok{ ds1))}
\end{Highlighting}
\end{Shaded}

union {[}18 3{]}:

\begin{longtable}[]{@{}lll@{}}
\toprule
:a & :b & :c\tabularnewline
\midrule
\endhead
& 110 & d\tabularnewline
1 & 109 & a\tabularnewline
2 & 108 & t\tabularnewline
5 & 107 & a\tabularnewline
4 & 106 & t\tabularnewline
3 & 105 & a\tabularnewline
2 & 104 & b\tabularnewline
1 & 103 & l\tabularnewline
& 102 & e\tabularnewline
1 & 101 & a\tabularnewline
2 & 102 & b\tabularnewline
1 & 103 & s\tabularnewline
2 & 104 &\tabularnewline
3 & 105 & t\tabularnewline
4 & 106 & r\tabularnewline
& 107 & a\tabularnewline
& 108 & c\tabularnewline
4 & 109 & t\tabularnewline
\bottomrule
\end{longtable}

\begin{center}\rule{0.5\linewidth}{0.5pt}\end{center}

\begin{Shaded}
\begin{Highlighting}[]
\NormalTok{(}\KeywordTok{apply}\NormalTok{ api/union (}\KeywordTok{repeatedly} \DecValTok{10}\NormalTok{ #(api/random DS)))}
\end{Highlighting}
\end{Shaded}

union {[}9 4{]}:

\begin{longtable}[]{@{}llll@{}}
\toprule
:V1 & :V2 & :V3 & :V4\tabularnewline
\midrule
\endhead
1 & 7 & 0.5 & A\tabularnewline
2 & 8 & 1.0 & B\tabularnewline
2 & 4 & 0.5 & A\tabularnewline
2 & 2 & 1.0 & B\tabularnewline
1 & 9 & 1.5 & C\tabularnewline
1 & 1 & 0.5 & A\tabularnewline
1 & 3 & 1.5 & C\tabularnewline
2 & 6 & 1.5 & C\tabularnewline
1 & 5 & 1.0 & B\tabularnewline
\bottomrule
\end{longtable}

\hypertarget{bind}{%
\paragraph{Bind}\label{bind}}

\texttt{bind} adds empty columns during concat

\begin{Shaded}
\begin{Highlighting}[]
\NormalTok{(api/bind ds1 ds2)}
\end{Highlighting}
\end{Shaded}

null {[}18 4{]}:

\begin{longtable}[]{@{}llll@{}}
\toprule
:a & :b & :c & :d\tabularnewline
\midrule
\endhead
1 & 101 & a &\tabularnewline
2 & 102 & b &\tabularnewline
1 & 103 & s &\tabularnewline
2 & 104 & &\tabularnewline
3 & 105 & t &\tabularnewline
4 & 106 & r &\tabularnewline
& 107 & a &\tabularnewline
& 108 & c &\tabularnewline
4 & 109 & t &\tabularnewline
& 110 & d & X\tabularnewline
1 & 109 & a & X\tabularnewline
2 & 108 & t & X\tabularnewline
5 & 107 & a & X\tabularnewline
4 & 106 & t & X\tabularnewline
3 & 105 & a & X\tabularnewline
2 & 104 & b & X\tabularnewline
1 & 103 & l & X\tabularnewline
& 102 & e & X\tabularnewline
\bottomrule
\end{longtable}

\begin{center}\rule{0.5\linewidth}{0.5pt}\end{center}

\begin{Shaded}
\begin{Highlighting}[]
\NormalTok{(api/bind ds2 ds1)}
\end{Highlighting}
\end{Shaded}

null {[}18 4{]}:

\begin{longtable}[]{@{}llll@{}}
\toprule
:a & :b & :c & :d\tabularnewline
\midrule
\endhead
& 110 & d & X\tabularnewline
1 & 109 & a & X\tabularnewline
2 & 108 & t & X\tabularnewline
5 & 107 & a & X\tabularnewline
4 & 106 & t & X\tabularnewline
3 & 105 & a & X\tabularnewline
2 & 104 & b & X\tabularnewline
1 & 103 & l & X\tabularnewline
& 102 & e & X\tabularnewline
1 & 101 & a &\tabularnewline
2 & 102 & b &\tabularnewline
1 & 103 & s &\tabularnewline
2 & 104 & &\tabularnewline
3 & 105 & t &\tabularnewline
4 & 106 & r &\tabularnewline
& 107 & a &\tabularnewline
& 108 & c &\tabularnewline
4 & 109 & t &\tabularnewline
\bottomrule
\end{longtable}

\hypertarget{append}{%
\paragraph{Append}\label{append}}

\texttt{append} concats columns

\begin{Shaded}
\begin{Highlighting}[]
\NormalTok{(api/append ds1 ds2)}
\end{Highlighting}
\end{Shaded}

\_unnamed {[}9 7{]}:

\begin{longtable}[]{@{}lllllll@{}}
\toprule
:a & :b & :c & :a & :b & :c & :d\tabularnewline
\midrule
\endhead
1 & 101 & a & & 110 & d & X\tabularnewline
2 & 102 & b & 1 & 109 & a & X\tabularnewline
1 & 103 & s & 2 & 108 & t & X\tabularnewline
2 & 104 & & 5 & 107 & a & X\tabularnewline
3 & 105 & t & 4 & 106 & t & X\tabularnewline
4 & 106 & r & 3 & 105 & a & X\tabularnewline
& 107 & a & 2 & 104 & b & X\tabularnewline
& 108 & c & 1 & 103 & l & X\tabularnewline
4 & 109 & t & & 102 & e & X\tabularnewline
\bottomrule
\end{longtable}

\hypertarget{intersection}{%
\paragraph{Intersection}\label{intersection}}

\begin{Shaded}
\begin{Highlighting}[]
\NormalTok{(api/intersect (api/select-columns ds1 }\AttributeTok{:b}\NormalTok{)}
\NormalTok{               (api/select-columns ds2 }\AttributeTok{:b}\NormalTok{))}
\end{Highlighting}
\end{Shaded}

intersection {[}8 1{]}:

\begin{longtable}[]{@{}l@{}}
\toprule
:b\tabularnewline
\midrule
\endhead
109\tabularnewline
108\tabularnewline
107\tabularnewline
106\tabularnewline
105\tabularnewline
104\tabularnewline
103\tabularnewline
102\tabularnewline
\bottomrule
\end{longtable}

\hypertarget{difference}{%
\paragraph{Difference}\label{difference}}

\begin{Shaded}
\begin{Highlighting}[]
\NormalTok{(api/difference (api/select-columns ds1 }\AttributeTok{:b}\NormalTok{)}
\NormalTok{                (api/select-columns ds2 }\AttributeTok{:b}\NormalTok{))}
\end{Highlighting}
\end{Shaded}

difference {[}1 1{]}:

\begin{longtable}[]{@{}l@{}}
\toprule
:b\tabularnewline
\midrule
\endhead
101\tabularnewline
\bottomrule
\end{longtable}

\begin{center}\rule{0.5\linewidth}{0.5pt}\end{center}

\begin{Shaded}
\begin{Highlighting}[]
\NormalTok{(api/difference (api/select-columns ds2 }\AttributeTok{:b}\NormalTok{)}
\NormalTok{                (api/select-columns ds1 }\AttributeTok{:b}\NormalTok{))}
\end{Highlighting}
\end{Shaded}

difference {[}1 1{]}:

\begin{longtable}[]{@{}l@{}}
\toprule
:b\tabularnewline
\midrule
\endhead
110\tabularnewline
\bottomrule
\end{longtable}

\hypertarget{functions}{%
\subsection{Functions}\label{functions}}

This API doesn't provide any statistical, numerical or date/time
functions. Use below namespaces:

\begin{longtable}[]{@{}ll@{}}
\toprule
Namespace & functions\tabularnewline
\midrule
\endhead
\texttt{tech.v2.datatype.functional} & primitive oprations, reducers,
statistics\tabularnewline
\texttt{tech.v2.datatype.datetime} & date/time converters\tabularnewline
\texttt{tech.v2.datatype.datetime.operations} & date/time
functions\tabularnewline
\texttt{tech.ml.dataset.pipeline} & pipeline operations\tabularnewline
\bottomrule
\end{longtable}

\hypertarget{other-examples}{%
\subsection{Other examples}\label{other-examples}}

\hypertarget{stocks}{%
\subsubsection{Stocks}\label{stocks}}

\begin{Shaded}
\begin{Highlighting}[]
\NormalTok{(}\BuiltInTok{defonce}\FunctionTok{ stocks }\NormalTok{(api/dataset }\StringTok{"https://raw.githubusercontent.com/techascent/tech.ml.dataset/master/test/data/stocks.csv"}\NormalTok{ \{}\AttributeTok{:key-fn} \KeywordTok{keyword}\NormalTok{\}))}
\end{Highlighting}
\end{Shaded}

\begin{Shaded}
\begin{Highlighting}[]
\NormalTok{stocks}
\end{Highlighting}
\end{Shaded}

\url{https://raw.githubusercontent.com/techascent/tech.ml.dataset/master/test/data/stocks.csv}
{[}560 3{]}:

\begin{longtable}[]{@{}lll@{}}
\toprule
:symbol & :date & :price\tabularnewline
\midrule
\endhead
MSFT & 2000-01-01 & 39.81\tabularnewline
MSFT & 2000-02-01 & 36.35\tabularnewline
MSFT & 2000-03-01 & 43.22\tabularnewline
MSFT & 2000-04-01 & 28.37\tabularnewline
MSFT & 2000-05-01 & 25.45\tabularnewline
MSFT & 2000-06-01 & 32.54\tabularnewline
MSFT & 2000-07-01 & 28.40\tabularnewline
MSFT & 2000-08-01 & 28.40\tabularnewline
MSFT & 2000-09-01 & 24.53\tabularnewline
MSFT & 2000-10-01 & 28.02\tabularnewline
MSFT & 2000-11-01 & 23.34\tabularnewline
MSFT & 2000-12-01 & 17.65\tabularnewline
MSFT & 2001-01-01 & 24.84\tabularnewline
MSFT & 2001-02-01 & 24.00\tabularnewline
MSFT & 2001-03-01 & 22.25\tabularnewline
MSFT & 2001-04-01 & 27.56\tabularnewline
MSFT & 2001-05-01 & 28.14\tabularnewline
MSFT & 2001-06-01 & 29.70\tabularnewline
MSFT & 2001-07-01 & 26.93\tabularnewline
MSFT & 2001-08-01 & 23.21\tabularnewline
MSFT & 2001-09-01 & 20.82\tabularnewline
MSFT & 2001-10-01 & 23.65\tabularnewline
MSFT & 2001-11-01 & 26.12\tabularnewline
MSFT & 2001-12-01 & 26.95\tabularnewline
MSFT & 2002-01-01 & 25.92\tabularnewline
\bottomrule
\end{longtable}

\begin{Shaded}
\begin{Highlighting}[]
\NormalTok{(}\KeywordTok{->}\NormalTok{ stocks}
\NormalTok{    (api/group-by (}\KeywordTok{fn}\NormalTok{ [row]}
\NormalTok{                    \{}\AttributeTok{:symbol}\NormalTok{ (}\AttributeTok{:symbol}\NormalTok{ row)}
                     \AttributeTok{:year}\NormalTok{ (tech.v2.datatype.datetime.operations/get-years (}\AttributeTok{:date}\NormalTok{ row))\}))}
\NormalTok{    (api/aggregate #(tech.v2.datatype.functional/mean (}\VariableTok{%} \AttributeTok{:price}\NormalTok{)))}
\NormalTok{    (api/order-by [}\AttributeTok{:symbol} \AttributeTok{:year}\NormalTok{]))}
\end{Highlighting}
\end{Shaded}

\_unnamed {[}51 3{]}:

\begin{longtable}[]{@{}lll@{}}
\toprule
:symbol & :year & :summary\tabularnewline
\midrule
\endhead
AAPL & 2000 & 21.74833333\tabularnewline
AAPL & 2001 & 10.17583333\tabularnewline
AAPL & 2002 & 9.40833333\tabularnewline
AAPL & 2003 & 9.34750000\tabularnewline
AAPL & 2004 & 18.72333333\tabularnewline
AAPL & 2005 & 48.17166667\tabularnewline
AAPL & 2006 & 72.04333333\tabularnewline
AAPL & 2007 & 133.35333333\tabularnewline
AAPL & 2008 & 138.48083333\tabularnewline
AAPL & 2009 & 150.39333333\tabularnewline
AAPL & 2010 & 206.56666667\tabularnewline
AMZN & 2000 & 43.93083333\tabularnewline
AMZN & 2001 & 11.73916667\tabularnewline
AMZN & 2002 & 16.72333333\tabularnewline
AMZN & 2003 & 39.01666667\tabularnewline
AMZN & 2004 & 43.26750000\tabularnewline
AMZN & 2005 & 40.18750000\tabularnewline
AMZN & 2006 & 36.25166667\tabularnewline
AMZN & 2007 & 69.95250000\tabularnewline
AMZN & 2008 & 69.01500000\tabularnewline
AMZN & 2009 & 90.73083333\tabularnewline
AMZN & 2010 & 124.21000000\tabularnewline
GOOG & 2004 & 159.47600000\tabularnewline
GOOG & 2005 & 286.47250000\tabularnewline
GOOG & 2006 & 415.25666667\tabularnewline
\bottomrule
\end{longtable}

\begin{Shaded}
\begin{Highlighting}[]
\NormalTok{(}\KeywordTok{->}\NormalTok{ stocks}
\NormalTok{    (api/group-by (}\KeywordTok{juxt} \AttributeTok{:symbol}\NormalTok{ #(tech.v2.datatype.datetime.operations/get-years (}\VariableTok \AttributeTok{:price}\NormalTok{)))}
\NormalTok{    (api/rename-columns \{:$group-name}\DecValTok{-0} \AttributeTok{:symbol}
\NormalTok{                         :$group-name}\DecValTok{-1} \AttributeTok{:year}\NormalTok{\}))}
\end{Highlighting}
\end{Shaded}

\_unnamed {[}51 3{]}:

\begin{longtable}[]{@{}lll@{}}
\toprule
:symbol & :year & :summary\tabularnewline
\midrule
\endhead
AMZN & 2007 & 69.95250000\tabularnewline
AMZN & 2008 & 69.01500000\tabularnewline
AMZN & 2009 & 90.73083333\tabularnewline
AMZN & 2010 & 124.21000000\tabularnewline
AMZN & 2000 & 43.93083333\tabularnewline
AMZN & 2001 & 11.73916667\tabularnewline
AMZN & 2002 & 16.72333333\tabularnewline
AMZN & 2003 & 39.01666667\tabularnewline
AMZN & 2004 & 43.26750000\tabularnewline
AMZN & 2005 & 40.18750000\tabularnewline
AMZN & 2006 & 36.25166667\tabularnewline
IBM & 2001 & 96.96833333\tabularnewline
IBM & 2002 & 75.12500000\tabularnewline
IBM & 2000 & 96.91416667\tabularnewline
MSFT & 2006 & 24.75833333\tabularnewline
MSFT & 2005 & 23.84583333\tabularnewline
MSFT & 2004 & 22.67416667\tabularnewline
MSFT & 2003 & 20.93416667\tabularnewline
AAPL & 2001 & 10.17583333\tabularnewline
MSFT & 2010 & 28.50666667\tabularnewline
AAPL & 2002 & 9.40833333\tabularnewline
MSFT & 2009 & 22.87250000\tabularnewline
MSFT & 2008 & 25.20833333\tabularnewline
AAPL & 2000 & 21.74833333\tabularnewline
MSFT & 2007 & 29.28416667\tabularnewline
\bottomrule
\end{longtable}

\hypertarget{data.table}{%
\subsubsection{data.table}\label{data.table}}

Below you can find comparizon between functionality of
\texttt{data.table} and Clojure dataset API. I leave it without
comments, please refer original document explaining details:

\href{https://rdatatable.gitlab.io/data.table/articles/datatable-intro.html}{Introduction
to \texttt{data.table}}

R

\begin{Shaded}
\begin{Highlighting}[]
\KeywordTok{library}\NormalTok{(data.table)}
\KeywordTok{library}\NormalTok{(knitr)}

\NormalTok{flights <-}\StringTok{ }\KeywordTok{fread}\NormalTok{(}\StringTok{"https://raw.githubusercontent.com/Rdatatable/data.table/master/vignettes/flights14.csv"}\NormalTok{)}

\KeywordTok{kable}\NormalTok{(}\KeywordTok{head}\NormalTok{(flights))}
\end{Highlighting}
\end{Shaded}

\begin{longtable}[]{@{}rrrrrlllrrr@{}}
\toprule
year & month & day & dep\_delay & arr\_delay & carrier & origin & dest &
air\_time & distance & hour\tabularnewline
\midrule
\endhead
2014 & 1 & 1 & 14 & 13 & AA & JFK & LAX & 359 & 2475 & 9\tabularnewline
2014 & 1 & 1 & -3 & 13 & AA & JFK & LAX & 363 & 2475 & 11\tabularnewline
2014 & 1 & 1 & 2 & 9 & AA & JFK & LAX & 351 & 2475 & 19\tabularnewline
2014 & 1 & 1 & -8 & -26 & AA & LGA & PBI & 157 & 1035 & 7\tabularnewline
2014 & 1 & 1 & 2 & 1 & AA & JFK & LAX & 350 & 2475 & 13\tabularnewline
2014 & 1 & 1 & 4 & 0 & AA & EWR & LAX & 339 & 2454 & 18\tabularnewline
\bottomrule
\end{longtable}

\begin{center}\rule{0.5\linewidth}{0.5pt}\end{center}

Clojure

\begin{Shaded}
\begin{Highlighting}[]
\NormalTok{(}\KeywordTok{require}\NormalTok{ '[tech.v2.datatype.functional }\AttributeTok{:as}\NormalTok{ dfn]}
\NormalTok{         '[tech.v2.datatype }\AttributeTok{:as}\NormalTok{ dtype])}

\NormalTok{(}\BuiltInTok{defonce}\FunctionTok{ flights }\NormalTok{(api/dataset }\StringTok{"https://raw.githubusercontent.com/Rdatatable/data.table/master/vignettes/flights14.csv"}\NormalTok{))}
\end{Highlighting}
\end{Shaded}

\begin{Shaded}
\begin{Highlighting}[]
\NormalTok{(api/head flights }\DecValTok{6}\NormalTok{)}
\end{Highlighting}
\end{Shaded}

\url{https://raw.githubusercontent.com/Rdatatable/data.table/master/vignettes/flights14.csv}
{[}6 11{]}:

\begin{longtable}[]{@{}lllllllllll@{}}
\toprule
year & month & day & dep\_delay & arr\_delay & carrier & origin & dest &
air\_time & distance & hour\tabularnewline
\midrule
\endhead
2014 & 1 & 1 & 14 & 13 & AA & JFK & LAX & 359 & 2475 & 9\tabularnewline
2014 & 1 & 1 & -3 & 13 & AA & JFK & LAX & 363 & 2475 & 11\tabularnewline
2014 & 1 & 1 & 2 & 9 & AA & JFK & LAX & 351 & 2475 & 19\tabularnewline
2014 & 1 & 1 & -8 & -26 & AA & LGA & PBI & 157 & 1035 & 7\tabularnewline
2014 & 1 & 1 & 2 & 1 & AA & JFK & LAX & 350 & 2475 & 13\tabularnewline
2014 & 1 & 1 & 4 & 0 & AA & EWR & LAX & 339 & 2454 & 18\tabularnewline
\bottomrule
\end{longtable}

\hypertarget{basics}{%
\paragraph{Basics}\label{basics}}

\hypertarget{shape-of-loaded-data}{%
\subparagraph{Shape of loaded data}\label{shape-of-loaded-data}}

R

\begin{Shaded}
\begin{Highlighting}[]
\KeywordTok{dim}\NormalTok{(flights)}
\end{Highlighting}
\end{Shaded}

\begin{verbatim}
[1] 253316     11
\end{verbatim}

\begin{center}\rule{0.5\linewidth}{0.5pt}\end{center}

Clojure

\begin{Shaded}
\begin{Highlighting}[]
\NormalTok{(api/shape flights)}
\end{Highlighting}
\end{Shaded}

\begin{verbatim}
[253316 11]
\end{verbatim}

\hypertarget{what-is-data.table}{%
\subparagraph{\texorpdfstring{What is
\texttt{data.table}?}{What is data.table?}}\label{what-is-data.table}}

R

\begin{Shaded}
\begin{Highlighting}[]
\NormalTok{DT =}\StringTok{ }\KeywordTok{data.table}\NormalTok{(}
  \DataTypeTok{ID =} \KeywordTok{c}\NormalTok{(}\StringTok{"b"}\NormalTok{,}\StringTok{"b"}\NormalTok{,}\StringTok{"b"}\NormalTok{,}\StringTok{"a"}\NormalTok{,}\StringTok{"a"}\NormalTok{,}\StringTok{"c"}\NormalTok{),}
  \DataTypeTok{a =} \DecValTok{1}\OperatorTok{:}\DecValTok{6}\NormalTok{,}
  \DataTypeTok{b =} \DecValTok{7}\OperatorTok{:}\DecValTok{12}\NormalTok{,}
  \DataTypeTok{c =} \DecValTok{13}\OperatorTok{:}\DecValTok{18}
\NormalTok{)}

\KeywordTok{kable}\NormalTok{(DT)}
\end{Highlighting}
\end{Shaded}

\begin{longtable}[]{@{}lrrr@{}}
\toprule
ID & a & b & c\tabularnewline
\midrule
\endhead
b & 1 & 7 & 13\tabularnewline
b & 2 & 8 & 14\tabularnewline
b & 3 & 9 & 15\tabularnewline
a & 4 & 10 & 16\tabularnewline
a & 5 & 11 & 17\tabularnewline
c & 6 & 12 & 18\tabularnewline
\bottomrule
\end{longtable}

\begin{Shaded}
\begin{Highlighting}[]
\KeywordTok{class}\NormalTok{(DT}\OperatorTok{$}\NormalTok{ID)}
\end{Highlighting}
\end{Shaded}

\begin{verbatim}
[1] "character"
\end{verbatim}

\begin{center}\rule{0.5\linewidth}{0.5pt}\end{center}

Clojure

\begin{Shaded}
\begin{Highlighting}[]
\NormalTok{(}\BuiltInTok{def}\FunctionTok{ DT }\NormalTok{(api/dataset \{}\AttributeTok{:ID}\NormalTok{ [}\StringTok{"b"} \StringTok{"b"} \StringTok{"b"} \StringTok{"a"} \StringTok{"a"} \StringTok{"c"}\NormalTok{]}
                      \AttributeTok{:a}\NormalTok{ (}\KeywordTok{range} \DecValTok{1} \DecValTok{7}\NormalTok{)}
                      \AttributeTok{:b}\NormalTok{ (}\KeywordTok{range} \DecValTok{7} \DecValTok{13}\NormalTok{)}
                      \AttributeTok{:c}\NormalTok{ (}\KeywordTok{range} \DecValTok{13} \DecValTok{19}\NormalTok{)\}))}
\end{Highlighting}
\end{Shaded}

\begin{Shaded}
\begin{Highlighting}[]
\NormalTok{DT}
\end{Highlighting}
\end{Shaded}

\_unnamed {[}6 4{]}:

\begin{longtable}[]{@{}llll@{}}
\toprule
:ID & :a & :b & :c\tabularnewline
\midrule
\endhead
b & 1 & 7 & 13\tabularnewline
b & 2 & 8 & 14\tabularnewline
b & 3 & 9 & 15\tabularnewline
a & 4 & 10 & 16\tabularnewline
a & 5 & 11 & 17\tabularnewline
c & 6 & 12 & 18\tabularnewline
\bottomrule
\end{longtable}

\begin{Shaded}
\begin{Highlighting}[]
\NormalTok{(}\KeywordTok{->} \AttributeTok{:ID}\NormalTok{ DT }\KeywordTok{meta} \AttributeTok{:datatype}\NormalTok{)}
\end{Highlighting}
\end{Shaded}

\begin{verbatim}
:string
\end{verbatim}

\hypertarget{get-all-the-flights-with-jfk-as-the-origin-airport-in-the-month-of-june.}{%
\subparagraph{Get all the flights with ``JFK'' as the origin airport in
the month of
June.}\label{get-all-the-flights-with-jfk-as-the-origin-airport-in-the-month-of-june.}}

R

\begin{Shaded}
\begin{Highlighting}[]
\NormalTok{ans <-}\StringTok{ }\NormalTok{flights[origin }\OperatorTok{==}\StringTok{ "JFK"} \OperatorTok{&}\StringTok{ }\NormalTok{month }\OperatorTok{==}\StringTok{ }\NormalTok{6L]}
\KeywordTok{kable}\NormalTok{(}\KeywordTok{head}\NormalTok{(ans))}
\end{Highlighting}
\end{Shaded}

\begin{longtable}[]{@{}rrrrrlllrrr@{}}
\toprule
year & month & day & dep\_delay & arr\_delay & carrier & origin & dest &
air\_time & distance & hour\tabularnewline
\midrule
\endhead
2014 & 6 & 1 & -9 & -5 & AA & JFK & LAX & 324 & 2475 & 8\tabularnewline
2014 & 6 & 1 & -10 & -13 & AA & JFK & LAX & 329 & 2475 &
12\tabularnewline
2014 & 6 & 1 & 18 & -1 & AA & JFK & LAX & 326 & 2475 & 7\tabularnewline
2014 & 6 & 1 & -6 & -16 & AA & JFK & LAX & 320 & 2475 &
10\tabularnewline
2014 & 6 & 1 & -4 & -45 & AA & JFK & LAX & 326 & 2475 &
18\tabularnewline
2014 & 6 & 1 & -6 & -23 & AA & JFK & LAX & 329 & 2475 &
14\tabularnewline
\bottomrule
\end{longtable}

\begin{center}\rule{0.5\linewidth}{0.5pt}\end{center}

Clojure

\begin{Shaded}
\begin{Highlighting}[]
\NormalTok{(}\KeywordTok{->}\NormalTok{ flights}
\NormalTok{    (api/select-rows (}\KeywordTok{fn}\NormalTok{ [row] (}\KeywordTok{and}\NormalTok{ (}\KeywordTok{=}\NormalTok{ (}\KeywordTok{get}\NormalTok{ row }\StringTok{"origin"}\NormalTok{) }\StringTok{"JFK"}\NormalTok{)}
\NormalTok{                                   (}\KeywordTok{=}\NormalTok{ (}\KeywordTok{get}\NormalTok{ row }\StringTok{"month"}\NormalTok{) }\DecValTok{6}\NormalTok{))))}
\NormalTok{    (api/head }\DecValTok{6}\NormalTok{))}
\end{Highlighting}
\end{Shaded}

\url{https://raw.githubusercontent.com/Rdatatable/data.table/master/vignettes/flights14.csv}
{[}6 11{]}:

\begin{longtable}[]{@{}lllllllllll@{}}
\toprule
year & month & day & dep\_delay & arr\_delay & carrier & origin & dest &
air\_time & distance & hour\tabularnewline
\midrule
\endhead
2014 & 6 & 1 & -9 & -5 & AA & JFK & LAX & 324 & 2475 & 8\tabularnewline
2014 & 6 & 1 & -10 & -13 & AA & JFK & LAX & 329 & 2475 &
12\tabularnewline
2014 & 6 & 1 & 18 & -1 & AA & JFK & LAX & 326 & 2475 & 7\tabularnewline
2014 & 6 & 1 & -6 & -16 & AA & JFK & LAX & 320 & 2475 &
10\tabularnewline
2014 & 6 & 1 & -4 & -45 & AA & JFK & LAX & 326 & 2475 &
18\tabularnewline
2014 & 6 & 1 & -6 & -23 & AA & JFK & LAX & 329 & 2475 &
14\tabularnewline
\bottomrule
\end{longtable}

\hypertarget{get-the-first-two-rows-from-flights.}{%
\subparagraph{\texorpdfstring{Get the first two rows from
\texttt{flights}.}{Get the first two rows from flights.}}\label{get-the-first-two-rows-from-flights.}}

R

\begin{Shaded}
\begin{Highlighting}[]
\NormalTok{ans <-}\StringTok{ }\NormalTok{flights[}\DecValTok{1}\OperatorTok{:}\DecValTok{2}\NormalTok{]}
\KeywordTok{kable}\NormalTok{(ans)}
\end{Highlighting}
\end{Shaded}

\begin{longtable}[]{@{}rrrrrlllrrr@{}}
\toprule
year & month & day & dep\_delay & arr\_delay & carrier & origin & dest &
air\_time & distance & hour\tabularnewline
\midrule
\endhead
2014 & 1 & 1 & 14 & 13 & AA & JFK & LAX & 359 & 2475 & 9\tabularnewline
2014 & 1 & 1 & -3 & 13 & AA & JFK & LAX & 363 & 2475 & 11\tabularnewline
\bottomrule
\end{longtable}

\begin{center}\rule{0.5\linewidth}{0.5pt}\end{center}

Clojure

\begin{Shaded}
\begin{Highlighting}[]
\NormalTok{(api/select-rows flights (}\KeywordTok{range} \DecValTok{2}\NormalTok{))}
\end{Highlighting}
\end{Shaded}

\url{https://raw.githubusercontent.com/Rdatatable/data.table/master/vignettes/flights14.csv}
{[}2 11{]}:

\begin{longtable}[]{@{}lllllllllll@{}}
\toprule
year & month & day & dep\_delay & arr\_delay & carrier & origin & dest &
air\_time & distance & hour\tabularnewline
\midrule
\endhead
2014 & 1 & 1 & 14 & 13 & AA & JFK & LAX & 359 & 2475 & 9\tabularnewline
2014 & 1 & 1 & -3 & 13 & AA & JFK & LAX & 363 & 2475 & 11\tabularnewline
\bottomrule
\end{longtable}

\hypertarget{sort-flights-first-by-column-origin-in-ascending-order-and-then-by-dest-in-descending-order}{%
\subparagraph{\texorpdfstring{Sort \texttt{flights} first by column
\texttt{origin} in ascending order, and then by \texttt{dest} in
descending
order}{Sort flights first by column origin in ascending order, and then by dest in descending order}}\label{sort-flights-first-by-column-origin-in-ascending-order-and-then-by-dest-in-descending-order}}

R

\begin{Shaded}
\begin{Highlighting}[]
\NormalTok{ans <-}\StringTok{ }\NormalTok{flights[}\KeywordTok{order}\NormalTok{(origin, }\OperatorTok{-}\NormalTok{dest)]}
\KeywordTok{kable}\NormalTok{(}\KeywordTok{head}\NormalTok{(ans))}
\end{Highlighting}
\end{Shaded}

\begin{longtable}[]{@{}rrrrrlllrrr@{}}
\toprule
year & month & day & dep\_delay & arr\_delay & carrier & origin & dest &
air\_time & distance & hour\tabularnewline
\midrule
\endhead
2014 & 1 & 5 & 6 & 49 & EV & EWR & XNA & 195 & 1131 & 8\tabularnewline
2014 & 1 & 6 & 7 & 13 & EV & EWR & XNA & 190 & 1131 & 8\tabularnewline
2014 & 1 & 7 & -6 & -13 & EV & EWR & XNA & 179 & 1131 & 8\tabularnewline
2014 & 1 & 8 & -7 & -12 & EV & EWR & XNA & 184 & 1131 & 8\tabularnewline
2014 & 1 & 9 & 16 & 7 & EV & EWR & XNA & 181 & 1131 & 8\tabularnewline
2014 & 1 & 13 & 66 & 66 & EV & EWR & XNA & 188 & 1131 & 9\tabularnewline
\bottomrule
\end{longtable}

\begin{center}\rule{0.5\linewidth}{0.5pt}\end{center}

Clojure

\begin{Shaded}
\begin{Highlighting}[]
\NormalTok{(}\KeywordTok{->}\NormalTok{ flights}
\NormalTok{    (api/order-by [}\StringTok{"origin"} \StringTok{"dest"}\NormalTok{] [}\AttributeTok{:asc} \AttributeTok{:desc}\NormalTok{])}
\NormalTok{    (api/head }\DecValTok{6}\NormalTok{))}
\end{Highlighting}
\end{Shaded}

\url{https://raw.githubusercontent.com/Rdatatable/data.table/master/vignettes/flights14.csv}
{[}6 11{]}:

\begin{longtable}[]{@{}lllllllllll@{}}
\toprule
year & month & day & dep\_delay & arr\_delay & carrier & origin & dest &
air\_time & distance & hour\tabularnewline
\midrule
\endhead
2014 & 6 & 3 & -6 & -38 & EV & EWR & XNA & 154 & 1131 & 6\tabularnewline
2014 & 1 & 20 & -9 & -17 & EV & EWR & XNA & 177 & 1131 &
8\tabularnewline
2014 & 3 & 19 & -6 & 10 & EV & EWR & XNA & 201 & 1131 & 6\tabularnewline
2014 & 2 & 3 & 231 & 268 & EV & EWR & XNA & 184 & 1131 &
12\tabularnewline
2014 & 4 & 25 & -8 & -32 & EV & EWR & XNA & 159 & 1131 &
6\tabularnewline
2014 & 2 & 19 & 21 & 10 & EV & EWR & XNA & 176 & 1131 & 8\tabularnewline
\bottomrule
\end{longtable}

\hypertarget{select-arr_delay-column-but-return-it-as-a-vector}{%
\subparagraph{\texorpdfstring{Select \texttt{arr\_delay} column, but
return it as a
vector}{Select arr\_delay column, but return it as a vector}}\label{select-arr_delay-column-but-return-it-as-a-vector}}

R

\begin{Shaded}
\begin{Highlighting}[]
\NormalTok{ans <-}\StringTok{ }\NormalTok{flights[, arr_delay]}
\KeywordTok{head}\NormalTok{(ans)}
\end{Highlighting}
\end{Shaded}

\begin{verbatim}
[1]  13  13   9 -26   1   0
\end{verbatim}

\begin{center}\rule{0.5\linewidth}{0.5pt}\end{center}

Clojure

\begin{Shaded}
\begin{Highlighting}[]
\NormalTok{(}\KeywordTok{take} \DecValTok{6}\NormalTok{ (flights }\StringTok{"arr_delay"}\NormalTok{))}
\end{Highlighting}
\end{Shaded}

\begin{verbatim}
(13 13 9 -26 1 0)
\end{verbatim}

\hypertarget{select-arr_delay-column-but-return-as-a-data.table-instead}{%
\subparagraph{\texorpdfstring{Select \texttt{arr\_delay} column, but
return as a data.table
instead}{Select arr\_delay column, but return as a data.table instead}}\label{select-arr_delay-column-but-return-as-a-data.table-instead}}

R

\begin{Shaded}
\begin{Highlighting}[]
\NormalTok{ans <-}\StringTok{ }\NormalTok{flights[, }\KeywordTok{list}\NormalTok{(arr_delay)]}
\KeywordTok{kable}\NormalTok{(}\KeywordTok{head}\NormalTok{(ans))}
\end{Highlighting}
\end{Shaded}

\begin{longtable}[]{@{}r@{}}
\toprule
arr\_delay\tabularnewline
\midrule
\endhead
13\tabularnewline
13\tabularnewline
9\tabularnewline
-26\tabularnewline
1\tabularnewline
0\tabularnewline
\bottomrule
\end{longtable}

\begin{center}\rule{0.5\linewidth}{0.5pt}\end{center}

Clojure

\begin{Shaded}
\begin{Highlighting}[]
\NormalTok{(}\KeywordTok{->}\NormalTok{ flights}
\NormalTok{    (api/select-columns }\StringTok{"arr_delay"}\NormalTok{)}
\NormalTok{    (api/head }\DecValTok{6}\NormalTok{))}
\end{Highlighting}
\end{Shaded}

\url{https://raw.githubusercontent.com/Rdatatable/data.table/master/vignettes/flights14.csv}
{[}6 1{]}:

\begin{longtable}[]{@{}l@{}}
\toprule
arr\_delay\tabularnewline
\midrule
\endhead
13\tabularnewline
13\tabularnewline
9\tabularnewline
-26\tabularnewline
1\tabularnewline
0\tabularnewline
\bottomrule
\end{longtable}

\hypertarget{select-both-arr_delay-and-dep_delay-columns}{%
\subparagraph{\texorpdfstring{Select both \texttt{arr\_delay} and
\texttt{dep\_delay}
columns}{Select both arr\_delay and dep\_delay columns}}\label{select-both-arr_delay-and-dep_delay-columns}}

R

\begin{Shaded}
\begin{Highlighting}[]
\NormalTok{ans <-}\StringTok{ }\NormalTok{flights[, .(arr_delay, dep_delay)]}
\KeywordTok{kable}\NormalTok{(}\KeywordTok{head}\NormalTok{(ans))}
\end{Highlighting}
\end{Shaded}

\begin{longtable}[]{@{}rr@{}}
\toprule
arr\_delay & dep\_delay\tabularnewline
\midrule
\endhead
13 & 14\tabularnewline
13 & -3\tabularnewline
9 & 2\tabularnewline
-26 & -8\tabularnewline
1 & 2\tabularnewline
0 & 4\tabularnewline
\bottomrule
\end{longtable}

\begin{center}\rule{0.5\linewidth}{0.5pt}\end{center}

Clojure

\begin{Shaded}
\begin{Highlighting}[]
\NormalTok{(}\KeywordTok{->}\NormalTok{ flights}
\NormalTok{    (api/select-columns [}\StringTok{"arr_delay"} \StringTok{"dep_delay"}\NormalTok{])}
\NormalTok{    (api/head }\DecValTok{6}\NormalTok{))}
\end{Highlighting}
\end{Shaded}

\url{https://raw.githubusercontent.com/Rdatatable/data.table/master/vignettes/flights14.csv}
{[}6 2{]}:

\begin{longtable}[]{@{}ll@{}}
\toprule
dep\_delay & arr\_delay\tabularnewline
\midrule
\endhead
14 & 13\tabularnewline
-3 & 13\tabularnewline
2 & 9\tabularnewline
-8 & -26\tabularnewline
2 & 1\tabularnewline
4 & 0\tabularnewline
\bottomrule
\end{longtable}

\hypertarget{select-both-arr_delay-and-dep_delay-columns-and-rename-them-to-delay_arr-and-delay_dep}{%
\subparagraph{\texorpdfstring{Select both \texttt{arr\_delay} and
\texttt{dep\_delay} columns and rename them to \texttt{delay\_arr} and
\texttt{delay\_dep}}{Select both arr\_delay and dep\_delay columns and rename them to delay\_arr and delay\_dep}}\label{select-both-arr_delay-and-dep_delay-columns-and-rename-them-to-delay_arr-and-delay_dep}}

R

\begin{Shaded}
\begin{Highlighting}[]
\NormalTok{ans <-}\StringTok{ }\NormalTok{flights[, .(}\DataTypeTok{delay_arr =}\NormalTok{ arr_delay, }\DataTypeTok{delay_dep =}\NormalTok{ dep_delay)]}
\KeywordTok{kable}\NormalTok{(}\KeywordTok{head}\NormalTok{(ans))}
\end{Highlighting}
\end{Shaded}

\begin{longtable}[]{@{}rr@{}}
\toprule
delay\_arr & delay\_dep\tabularnewline
\midrule
\endhead
13 & 14\tabularnewline
13 & -3\tabularnewline
9 & 2\tabularnewline
-26 & -8\tabularnewline
1 & 2\tabularnewline
0 & 4\tabularnewline
\bottomrule
\end{longtable}

\begin{center}\rule{0.5\linewidth}{0.5pt}\end{center}

Clojure

\begin{Shaded}
\begin{Highlighting}[]
\NormalTok{(}\KeywordTok{->}\NormalTok{ flights}
\NormalTok{    (api/select-columns \{}\StringTok{"arr_delay"} \StringTok{"delay_arr"}
                         \StringTok{"dep_delay"} \StringTok{"delay_arr"}\NormalTok{\})}
\NormalTok{    (api/head }\DecValTok{6}\NormalTok{))}
\end{Highlighting}
\end{Shaded}

\url{https://raw.githubusercontent.com/Rdatatable/data.table/master/vignettes/flights14.csv}
{[}6 2{]}:

\begin{longtable}[]{@{}ll@{}}
\toprule
delay\_arr & delay\_arr\tabularnewline
\midrule
\endhead
14 & 13\tabularnewline
-3 & 13\tabularnewline
2 & 9\tabularnewline
-8 & -26\tabularnewline
2 & 1\tabularnewline
4 & 0\tabularnewline
\bottomrule
\end{longtable}

\hypertarget{how-many-trips-have-had-total-delay-0}{%
\subparagraph{How many trips have had total delay \textless{}
0?}\label{how-many-trips-have-had-total-delay-0}}

R

\begin{Shaded}
\begin{Highlighting}[]
\NormalTok{ans <-}\StringTok{ }\NormalTok{flights[, }\KeywordTok{sum}\NormalTok{( (arr_delay }\OperatorTok{+}\StringTok{ }\NormalTok{dep_delay) }\OperatorTok{<}\StringTok{ }\DecValTok{0}\NormalTok{ )]}
\NormalTok{ans}
\end{Highlighting}
\end{Shaded}

\begin{verbatim}
[1] 141814
\end{verbatim}

\begin{center}\rule{0.5\linewidth}{0.5pt}\end{center}

Clojure

\begin{Shaded}
\begin{Highlighting}[]
\NormalTok{(}\KeywordTok{->>}\NormalTok{ (dfn/+ (flights }\StringTok{"arr_delay"}\NormalTok{) (flights }\StringTok{"dep_delay"}\NormalTok{))}
\NormalTok{     (dfn/argfilter #(}\KeywordTok{<} \VariableTok{%} \FloatTok{0.0}\NormalTok{))}
\NormalTok{     (dtype/ecount))}
\end{Highlighting}
\end{Shaded}

\begin{verbatim}
141814
\end{verbatim}

or pure Clojure functions (much, much slower)

\begin{Shaded}
\begin{Highlighting}[]
\NormalTok{(}\KeywordTok{->>}\NormalTok{ (}\KeywordTok{map} \KeywordTok{+}\NormalTok{ (flights }\StringTok{"arr_delay"}\NormalTok{) (flights }\StringTok{"dep_delay"}\NormalTok{))}
\NormalTok{     (}\KeywordTok{filter} \KeywordTok{neg?}\NormalTok{)}
\NormalTok{     (}\KeywordTok{count}\NormalTok{))}
\end{Highlighting}
\end{Shaded}

\begin{verbatim}
141814
\end{verbatim}

\hypertarget{calculate-the-average-arrival-and-departure-delay-for-all-flights-with-jfk-as-the-origin-airport-in-the-month-of-june}{%
\subparagraph{Calculate the average arrival and departure delay for all
flights with ``JFK'' as the origin airport in the month of
June}\label{calculate-the-average-arrival-and-departure-delay-for-all-flights-with-jfk-as-the-origin-airport-in-the-month-of-june}}

R

\begin{Shaded}
\begin{Highlighting}[]
\NormalTok{ans <-}\StringTok{ }\NormalTok{flights[origin }\OperatorTok{==}\StringTok{ "JFK"} \OperatorTok{&}\StringTok{ }\NormalTok{month }\OperatorTok{==}\StringTok{ }\NormalTok{6L,}
\NormalTok{               .(}\DataTypeTok{m_arr =} \KeywordTok{mean}\NormalTok{(arr_delay), }\DataTypeTok{m_dep =} \KeywordTok{mean}\NormalTok{(dep_delay))]}
\KeywordTok{kable}\NormalTok{(ans)}
\end{Highlighting}
\end{Shaded}

\begin{longtable}[]{@{}rr@{}}
\toprule
m\_arr & m\_dep\tabularnewline
\midrule
\endhead
5.839349 & 9.807884\tabularnewline
\bottomrule
\end{longtable}

\begin{center}\rule{0.5\linewidth}{0.5pt}\end{center}

Clojure

\begin{Shaded}
\begin{Highlighting}[]
\NormalTok{(}\KeywordTok{->}\NormalTok{ flights}
\NormalTok{    (api/select-rows (}\KeywordTok{fn}\NormalTok{ [row] (}\KeywordTok{and}\NormalTok{ (}\KeywordTok{=}\NormalTok{ (}\KeywordTok{get}\NormalTok{ row }\StringTok{"origin"}\NormalTok{) }\StringTok{"JFK"}\NormalTok{)}
\NormalTok{                                   (}\KeywordTok{=}\NormalTok{ (}\KeywordTok{get}\NormalTok{ row }\StringTok{"month"}\NormalTok{) }\DecValTok{6}\NormalTok{))))}
\NormalTok{    (api/aggregate \{}\AttributeTok{:m}\NormalTok{_arr #(dfn/mean (}\VariableTok \StringTok{"dep_delay"}\NormalTok{))\}))}
\end{Highlighting}
\end{Shaded}

\_unnamed {[}1 2{]}:

\begin{longtable}[]{@{}ll@{}}
\toprule
:m\_arr & :m\_dep\tabularnewline
\midrule
\endhead
5.83934932 & 9.80788411\tabularnewline
\bottomrule
\end{longtable}

\hypertarget{how-many-trips-have-been-made-in-2014-from-jfk-airport-in-the-month-of-june}{%
\subparagraph{How many trips have been made in 2014 from ``JFK'' airport
in the month of
June?}\label{how-many-trips-have-been-made-in-2014-from-jfk-airport-in-the-month-of-june}}

R

\begin{Shaded}
\begin{Highlighting}[]
\NormalTok{ans <-}\StringTok{ }\NormalTok{flights[origin }\OperatorTok{==}\StringTok{ "JFK"} \OperatorTok{&}\StringTok{ }\NormalTok{month }\OperatorTok{==}\StringTok{ }\NormalTok{6L, }\KeywordTok{length}\NormalTok{(dest)]}
\NormalTok{ans}
\end{Highlighting}
\end{Shaded}

\begin{verbatim}
[1] 8422
\end{verbatim}

or

\begin{Shaded}
\begin{Highlighting}[]
\NormalTok{ans <-}\StringTok{ }\NormalTok{flights[origin }\OperatorTok{==}\StringTok{ "JFK"} \OperatorTok{&}\StringTok{ }\NormalTok{month }\OperatorTok{==}\StringTok{ }\NormalTok{6L, .N]}
\NormalTok{ans}
\end{Highlighting}
\end{Shaded}

\begin{verbatim}
[1] 8422
\end{verbatim}

\begin{center}\rule{0.5\linewidth}{0.5pt}\end{center}

Clojure

\begin{Shaded}
\begin{Highlighting}[]
\NormalTok{(}\KeywordTok{->}\NormalTok{ flights}
\NormalTok{    (api/select-rows (}\KeywordTok{fn}\NormalTok{ [row] (}\KeywordTok{and}\NormalTok{ (}\KeywordTok{=}\NormalTok{ (}\KeywordTok{get}\NormalTok{ row }\StringTok{"origin"}\NormalTok{) }\StringTok{"JFK"}\NormalTok{)}
\NormalTok{                                   (}\KeywordTok{=}\NormalTok{ (}\KeywordTok{get}\NormalTok{ row }\StringTok{"month"}\NormalTok{) }\DecValTok{6}\NormalTok{))))}
\NormalTok{    (api/row-count))}
\end{Highlighting}
\end{Shaded}

\begin{verbatim}
8422
\end{verbatim}

\hypertarget{deselect-columns-using---or}{%
\subparagraph{deselect columns using - or
!}\label{deselect-columns-using---or}}

R

\begin{Shaded}
\begin{Highlighting}[]
\NormalTok{ans <-}\StringTok{ }\NormalTok{flights[, }\OperatorTok{!}\KeywordTok{c}\NormalTok{(}\StringTok{"arr_delay"}\NormalTok{, }\StringTok{"dep_delay"}\NormalTok{)]}
\KeywordTok{kable}\NormalTok{(}\KeywordTok{head}\NormalTok{(ans))}
\end{Highlighting}
\end{Shaded}

\begin{longtable}[]{@{}rrrlllrrr@{}}
\toprule
year & month & day & carrier & origin & dest & air\_time & distance &
hour\tabularnewline
\midrule
\endhead
2014 & 1 & 1 & AA & JFK & LAX & 359 & 2475 & 9\tabularnewline
2014 & 1 & 1 & AA & JFK & LAX & 363 & 2475 & 11\tabularnewline
2014 & 1 & 1 & AA & JFK & LAX & 351 & 2475 & 19\tabularnewline
2014 & 1 & 1 & AA & LGA & PBI & 157 & 1035 & 7\tabularnewline
2014 & 1 & 1 & AA & JFK & LAX & 350 & 2475 & 13\tabularnewline
2014 & 1 & 1 & AA & EWR & LAX & 339 & 2454 & 18\tabularnewline
\bottomrule
\end{longtable}

or

\begin{Shaded}
\begin{Highlighting}[]
\NormalTok{ans <-}\StringTok{ }\NormalTok{flights[, }\OperatorTok{-}\KeywordTok{c}\NormalTok{(}\StringTok{"arr_delay"}\NormalTok{, }\StringTok{"dep_delay"}\NormalTok{)]}
\KeywordTok{kable}\NormalTok{(}\KeywordTok{head}\NormalTok{(ans))}
\end{Highlighting}
\end{Shaded}

\begin{longtable}[]{@{}rrrlllrrr@{}}
\toprule
year & month & day & carrier & origin & dest & air\_time & distance &
hour\tabularnewline
\midrule
\endhead
2014 & 1 & 1 & AA & JFK & LAX & 359 & 2475 & 9\tabularnewline
2014 & 1 & 1 & AA & JFK & LAX & 363 & 2475 & 11\tabularnewline
2014 & 1 & 1 & AA & JFK & LAX & 351 & 2475 & 19\tabularnewline
2014 & 1 & 1 & AA & LGA & PBI & 157 & 1035 & 7\tabularnewline
2014 & 1 & 1 & AA & JFK & LAX & 350 & 2475 & 13\tabularnewline
2014 & 1 & 1 & AA & EWR & LAX & 339 & 2454 & 18\tabularnewline
\bottomrule
\end{longtable}

\begin{center}\rule{0.5\linewidth}{0.5pt}\end{center}

Clojure

\begin{Shaded}
\begin{Highlighting}[]
\NormalTok{(}\KeywordTok{->}\NormalTok{ flights}
\NormalTok{    (api/select-columns (}\KeywordTok{complement}\NormalTok{ #\{}\StringTok{"arr_delay"} \StringTok{"dep_delay"}\NormalTok{\}))}
\NormalTok{    (api/head }\DecValTok{6}\NormalTok{))}
\end{Highlighting}
\end{Shaded}

\url{https://raw.githubusercontent.com/Rdatatable/data.table/master/vignettes/flights14.csv}
{[}6 9{]}:

\begin{longtable}[]{@{}lllllllll@{}}
\toprule
year & month & day & carrier & origin & dest & air\_time & distance &
hour\tabularnewline
\midrule
\endhead
2014 & 1 & 1 & AA & JFK & LAX & 359 & 2475 & 9\tabularnewline
2014 & 1 & 1 & AA & JFK & LAX & 363 & 2475 & 11\tabularnewline
2014 & 1 & 1 & AA & JFK & LAX & 351 & 2475 & 19\tabularnewline
2014 & 1 & 1 & AA & LGA & PBI & 157 & 1035 & 7\tabularnewline
2014 & 1 & 1 & AA & JFK & LAX & 350 & 2475 & 13\tabularnewline
2014 & 1 & 1 & AA & EWR & LAX & 339 & 2454 & 18\tabularnewline
\bottomrule
\end{longtable}

\hypertarget{aggregations}{%
\paragraph{Aggregations}\label{aggregations}}

\hypertarget{how-can-we-get-the-number-of-trips-corresponding-to-each-origin-airport}{%
\subparagraph{How can we get the number of trips corresponding to each
origin
airport?}\label{how-can-we-get-the-number-of-trips-corresponding-to-each-origin-airport}}

R

\begin{Shaded}
\begin{Highlighting}[]
\NormalTok{ans <-}\StringTok{ }\NormalTok{flights[, .(.N), by =}\StringTok{ }\NormalTok{.(origin)]}
\KeywordTok{kable}\NormalTok{(ans)}
\end{Highlighting}
\end{Shaded}

\begin{longtable}[]{@{}lr@{}}
\toprule
origin & N\tabularnewline
\midrule
\endhead
JFK & 81483\tabularnewline
LGA & 84433\tabularnewline
EWR & 87400\tabularnewline
\bottomrule
\end{longtable}

\begin{center}\rule{0.5\linewidth}{0.5pt}\end{center}

Clojure

\begin{Shaded}
\begin{Highlighting}[]
\NormalTok{(}\KeywordTok{->}\NormalTok{ flights}
\NormalTok{    (api/group-by [}\StringTok{"origin"}\NormalTok{])}
\NormalTok{    (api/aggregate \{}\AttributeTok{:N}\NormalTok{ api/row-count\}))}
\end{Highlighting}
\end{Shaded}

\_unnamed {[}3 2{]}:

\begin{longtable}[]{@{}ll@{}}
\toprule
origin & :N\tabularnewline
\midrule
\endhead
LGA & 84433\tabularnewline
EWR & 87400\tabularnewline
JFK & 81483\tabularnewline
\bottomrule
\end{longtable}

\hypertarget{how-can-we-calculate-the-number-of-trips-for-each-origin-airport-for-carrier-code-aa}{%
\subparagraph{How can we calculate the number of trips for each origin
airport for carrier code
``AA''?}\label{how-can-we-calculate-the-number-of-trips-for-each-origin-airport-for-carrier-code-aa}}

R

\begin{Shaded}
\begin{Highlighting}[]
\NormalTok{ans <-}\StringTok{ }\NormalTok{flights[carrier }\OperatorTok{==}\StringTok{ "AA"}\NormalTok{, .N, by =}\StringTok{ }\NormalTok{origin]}
\KeywordTok{kable}\NormalTok{(ans)}
\end{Highlighting}
\end{Shaded}

\begin{longtable}[]{@{}lr@{}}
\toprule
origin & N\tabularnewline
\midrule
\endhead
JFK & 11923\tabularnewline
LGA & 11730\tabularnewline
EWR & 2649\tabularnewline
\bottomrule
\end{longtable}

\begin{center}\rule{0.5\linewidth}{0.5pt}\end{center}

Clojure

\begin{Shaded}
\begin{Highlighting}[]
\NormalTok{(}\KeywordTok{->}\NormalTok{ flights}
\NormalTok{    (api/select-rows #(}\KeywordTok{=}\NormalTok{ (}\KeywordTok{get} \VariableTok{%} \StringTok{"carrier"}\NormalTok{) }\StringTok{"AA"}\NormalTok{))}
\NormalTok{    (api/group-by [}\StringTok{"origin"}\NormalTok{])}
\NormalTok{    (api/aggregate \{}\AttributeTok{:N}\NormalTok{ api/row-count\}))}
\end{Highlighting}
\end{Shaded}

\_unnamed {[}3 2{]}:

\begin{longtable}[]{@{}ll@{}}
\toprule
origin & :N\tabularnewline
\midrule
\endhead
LGA & 11730\tabularnewline
EWR & 2649\tabularnewline
JFK & 11923\tabularnewline
\bottomrule
\end{longtable}

\hypertarget{how-can-we-get-the-total-number-of-trips-for-each-origin-dest-pair-for-carrier-code-aa}{%
\subparagraph{\texorpdfstring{How can we get the total number of trips
for each \texttt{origin}, \texttt{dest} pair for carrier code
``AA''?}{How can we get the total number of trips for each origin, dest pair for carrier code ``AA''?}}\label{how-can-we-get-the-total-number-of-trips-for-each-origin-dest-pair-for-carrier-code-aa}}

R

\begin{Shaded}
\begin{Highlighting}[]
\NormalTok{ans <-}\StringTok{ }\NormalTok{flights[carrier }\OperatorTok{==}\StringTok{ "AA"}\NormalTok{, .N, by =}\StringTok{ }\NormalTok{.(origin, dest)]}
\KeywordTok{kable}\NormalTok{(}\KeywordTok{head}\NormalTok{(ans))}
\end{Highlighting}
\end{Shaded}

\begin{longtable}[]{@{}llr@{}}
\toprule
origin & dest & N\tabularnewline
\midrule
\endhead
JFK & LAX & 3387\tabularnewline
LGA & PBI & 245\tabularnewline
EWR & LAX & 62\tabularnewline
JFK & MIA & 1876\tabularnewline
JFK & SEA & 298\tabularnewline
EWR & MIA & 848\tabularnewline
\bottomrule
\end{longtable}

\begin{center}\rule{0.5\linewidth}{0.5pt}\end{center}

Clojure

\begin{Shaded}
\begin{Highlighting}[]
\NormalTok{(}\KeywordTok{->}\NormalTok{ flights}
\NormalTok{    (api/select-rows #(}\KeywordTok{=}\NormalTok{ (}\KeywordTok{get} \VariableTok{%} \StringTok{"carrier"}\NormalTok{) }\StringTok{"AA"}\NormalTok{))}
\NormalTok{    (api/group-by [}\StringTok{"origin"} \StringTok{"dest"}\NormalTok{])}
\NormalTok{    (api/aggregate \{}\AttributeTok{:N}\NormalTok{ api/row-count\})}
\NormalTok{    (api/head }\DecValTok{6}\NormalTok{))}
\end{Highlighting}
\end{Shaded}

\_unnamed {[}6 3{]}:

\begin{longtable}[]{@{}lll@{}}
\toprule
origin & dest & :N\tabularnewline
\midrule
\endhead
JFK & MIA & 1876\tabularnewline
LGA & PBI & 245\tabularnewline
JFK & SEA & 298\tabularnewline
LGA & DFW & 3785\tabularnewline
JFK & AUS & 297\tabularnewline
JFK & STT & 229\tabularnewline
\bottomrule
\end{longtable}

\hypertarget{how-can-we-get-the-average-arrival-and-departure-delay-for-each-origdest-pair-for-each-month-for-carrier-code-aa}{%
\subparagraph{\texorpdfstring{How can we get the average arrival and
departure delay for each \texttt{orig},\texttt{dest} pair for each month
for carrier code
``AA''?}{How can we get the average arrival and departure delay for each orig,dest pair for each month for carrier code ``AA''?}}\label{how-can-we-get-the-average-arrival-and-departure-delay-for-each-origdest-pair-for-each-month-for-carrier-code-aa}}

R

\begin{Shaded}
\begin{Highlighting}[]
\NormalTok{ans <-}\StringTok{ }\NormalTok{flights[carrier }\OperatorTok{==}\StringTok{ "AA"}\NormalTok{,}
\NormalTok{        .(}\KeywordTok{mean}\NormalTok{(arr_delay), }\KeywordTok{mean}\NormalTok{(dep_delay)),}
\NormalTok{        by =}\StringTok{ }\NormalTok{.(origin, dest, month)]}
\KeywordTok{kable}\NormalTok{(}\KeywordTok{head}\NormalTok{(ans,}\DecValTok{10}\NormalTok{))}
\end{Highlighting}
\end{Shaded}

\begin{longtable}[]{@{}llrrr@{}}
\toprule
origin & dest & month & V1 & V2\tabularnewline
\midrule
\endhead
JFK & LAX & 1 & 6.590361 & 14.2289157\tabularnewline
LGA & PBI & 1 & -7.758621 & 0.3103448\tabularnewline
EWR & LAX & 1 & 1.366667 & 7.5000000\tabularnewline
JFK & MIA & 1 & 15.720670 & 18.7430168\tabularnewline
JFK & SEA & 1 & 14.357143 & 30.7500000\tabularnewline
EWR & MIA & 1 & 11.011236 & 12.1235955\tabularnewline
JFK & SFO & 1 & 19.252252 & 28.6396396\tabularnewline
JFK & BOS & 1 & 12.919643 & 15.2142857\tabularnewline
JFK & ORD & 1 & 31.586207 & 40.1724138\tabularnewline
JFK & IAH & 1 & 28.857143 & 14.2857143\tabularnewline
\bottomrule
\end{longtable}

\begin{center}\rule{0.5\linewidth}{0.5pt}\end{center}

Clojure

\begin{Shaded}
\begin{Highlighting}[]
\NormalTok{(}\KeywordTok{->}\NormalTok{ flights}
\NormalTok{    (api/select-rows #(}\KeywordTok{=}\NormalTok{ (}\KeywordTok{get} \VariableTok \StringTok{"arr_delay"}\NormalTok{))}
\NormalTok{                    #(dfn/mean (}\VariableTok{%} \StringTok{"dep_delay"}\NormalTok{))])}
\NormalTok{    (api/head }\DecValTok{10}\NormalTok{))}
\end{Highlighting}
\end{Shaded}

\_unnamed {[}10 5{]}:

\begin{longtable}[]{@{}lllll@{}}
\toprule
month & origin & dest & :summary-0 & :summary-1\tabularnewline
\midrule
\endhead
9 & LGA & DFW & -8.78772379 & -0.25575448\tabularnewline
10 & LGA & DFW & 3.50000000 & 4.55276382\tabularnewline
1 & JFK & AUS & 25.20000000 & 27.60000000\tabularnewline
4 & JFK & AUS & 4.36666667 & -0.13333333\tabularnewline
5 & JFK & AUS & 6.76666667 & 14.73333333\tabularnewline
2 & JFK & AUS & 26.26923077 & 21.50000000\tabularnewline
3 & JFK & AUS & 8.19354839 & 2.70967742\tabularnewline
8 & JFK & AUS & 20.41935484 & 20.77419355\tabularnewline
1 & EWR & LAX & 1.36666667 & 7.50000000\tabularnewline
9 & JFK & AUS & 16.26666667 & 14.36666667\tabularnewline
\bottomrule
\end{longtable}

\hypertarget{so-how-can-we-directly-order-by-all-the-grouping-variables}{%
\subparagraph{So how can we directly order by all the grouping
variables?}\label{so-how-can-we-directly-order-by-all-the-grouping-variables}}

R

\begin{Shaded}
\begin{Highlighting}[]
\NormalTok{ans <-}\StringTok{ }\NormalTok{flights[carrier }\OperatorTok{==}\StringTok{ "AA"}\NormalTok{,}
\NormalTok{        .(}\KeywordTok{mean}\NormalTok{(arr_delay), }\KeywordTok{mean}\NormalTok{(dep_delay)),}
\NormalTok{        keyby =}\StringTok{ }\NormalTok{.(origin, dest, month)]}
\KeywordTok{kable}\NormalTok{(}\KeywordTok{head}\NormalTok{(ans,}\DecValTok{10}\NormalTok{))}
\end{Highlighting}
\end{Shaded}

\begin{longtable}[]{@{}llrrr@{}}
\toprule
origin & dest & month & V1 & V2\tabularnewline
\midrule
\endhead
EWR & DFW & 1 & 6.427673 & 10.012579\tabularnewline
EWR & DFW & 2 & 10.536765 & 11.345588\tabularnewline
EWR & DFW & 3 & 12.865031 & 8.079755\tabularnewline
EWR & DFW & 4 & 17.792683 & 12.920732\tabularnewline
EWR & DFW & 5 & 18.487805 & 18.682927\tabularnewline
EWR & DFW & 6 & 37.005952 & 38.744048\tabularnewline
EWR & DFW & 7 & 20.250000 & 21.154762\tabularnewline
EWR & DFW & 8 & 16.936046 & 22.069767\tabularnewline
EWR & DFW & 9 & 5.865031 & 13.055215\tabularnewline
EWR & DFW & 10 & 18.813665 & 18.894410\tabularnewline
\bottomrule
\end{longtable}

\begin{center}\rule{0.5\linewidth}{0.5pt}\end{center}

Clojure

\begin{Shaded}
\begin{Highlighting}[]
\NormalTok{(}\KeywordTok{->}\NormalTok{ flights}
\NormalTok{    (api/select-rows #(}\KeywordTok{=}\NormalTok{ (}\KeywordTok{get} \VariableTok \StringTok{"arr_delay"}\NormalTok{))}
\NormalTok{                    #(dfn/mean (}\VariableTok{%} \StringTok{"dep_delay"}\NormalTok{))])}
\NormalTok{    (api/order-by [}\StringTok{"origin"} \StringTok{"dest"} \StringTok{"month"}\NormalTok{])}
\NormalTok{    (api/head }\DecValTok{10}\NormalTok{))}
\end{Highlighting}
\end{Shaded}

\_unnamed {[}10 5{]}:

\begin{longtable}[]{@{}lllll@{}}
\toprule
month & origin & dest & :summary-0 & :summary-1\tabularnewline
\midrule
\endhead
1 & EWR & DFW & 6.42767296 & 10.01257862\tabularnewline
2 & EWR & DFW & 10.53676471 & 11.34558824\tabularnewline
3 & EWR & DFW & 12.86503067 & 8.07975460\tabularnewline
4 & EWR & DFW & 17.79268293 & 12.92073171\tabularnewline
5 & EWR & DFW & 18.48780488 & 18.68292683\tabularnewline
6 & EWR & DFW & 37.00595238 & 38.74404762\tabularnewline
7 & EWR & DFW & 20.25000000 & 21.15476190\tabularnewline
8 & EWR & DFW & 16.93604651 & 22.06976744\tabularnewline
9 & EWR & DFW & 5.86503067 & 13.05521472\tabularnewline
10 & EWR & DFW & 18.81366460 & 18.89440994\tabularnewline
\bottomrule
\end{longtable}

\hypertarget{can-by-accept-expressions-as-well-or-does-it-just-take-columns}{%
\subparagraph{\texorpdfstring{Can \texttt{by} accept expressions as well
or does it just take
columns?}{Can by accept expressions as well or does it just take columns?}}\label{can-by-accept-expressions-as-well-or-does-it-just-take-columns}}

R

\begin{Shaded}
\begin{Highlighting}[]
\NormalTok{ans <-}\StringTok{ }\NormalTok{flights[, .N, .(dep_delay}\OperatorTok{>}\DecValTok{0}\NormalTok{, arr_delay}\OperatorTok{>}\DecValTok{0}\NormalTok{)]}
\KeywordTok{kable}\NormalTok{(ans)}
\end{Highlighting}
\end{Shaded}

\begin{longtable}[]{@{}llr@{}}
\toprule
dep\_delay & arr\_delay & N\tabularnewline
\midrule
\endhead
TRUE & TRUE & 72836\tabularnewline
FALSE & TRUE & 34583\tabularnewline
FALSE & FALSE & 119304\tabularnewline
TRUE & FALSE & 26593\tabularnewline
\bottomrule
\end{longtable}

\begin{center}\rule{0.5\linewidth}{0.5pt}\end{center}

Clojure

\begin{Shaded}
\begin{Highlighting}[]
\NormalTok{(}\KeywordTok{->}\NormalTok{ flights}
\NormalTok{    (api/group-by (}\KeywordTok{fn}\NormalTok{ [row]}
\NormalTok{                    \{}\AttributeTok{:dep}\NormalTok{_delay (}\KeywordTok{pos?}\NormalTok{ (}\KeywordTok{get}\NormalTok{ row }\StringTok{"dep_delay"}\NormalTok{))}
                     \AttributeTok{:arr}\NormalTok{_delay (}\KeywordTok{pos?}\NormalTok{ (}\KeywordTok{get}\NormalTok{ row }\StringTok{"arr_delay"}\NormalTok{))\}))}
\NormalTok{    (api/aggregate \{}\AttributeTok{:N}\NormalTok{ api/row-count\}))}
\end{Highlighting}
\end{Shaded}

\_unnamed {[}4 3{]}:

\begin{longtable}[]{@{}lll@{}}
\toprule
:dep\_delay & :arr\_delay & :N\tabularnewline
\midrule
\endhead
true & false & 26593\tabularnewline
false & true & 34583\tabularnewline
false & false & 119304\tabularnewline
true & true & 72836\tabularnewline
\bottomrule
\end{longtable}

\hypertarget{do-we-have-to-compute-mean-for-each-column-individually}{%
\subparagraph{\texorpdfstring{Do we have to compute \texttt{mean()} for
each column
individually?}{Do we have to compute mean() for each column individually?}}\label{do-we-have-to-compute-mean-for-each-column-individually}}

R

\begin{Shaded}
\begin{Highlighting}[]
\KeywordTok{kable}\NormalTok{(DT)}
\end{Highlighting}
\end{Shaded}

\begin{longtable}[]{@{}lrrr@{}}
\toprule
ID & a & b & c\tabularnewline
\midrule
\endhead
b & 1 & 7 & 13\tabularnewline
b & 2 & 8 & 14\tabularnewline
b & 3 & 9 & 15\tabularnewline
a & 4 & 10 & 16\tabularnewline
a & 5 & 11 & 17\tabularnewline
c & 6 & 12 & 18\tabularnewline
\bottomrule
\end{longtable}

\begin{Shaded}
\begin{Highlighting}[]
\NormalTok{DT[, }\KeywordTok{print}\NormalTok{(.SD), by =}\StringTok{ }\NormalTok{ID]}
\end{Highlighting}
\end{Shaded}

\begin{verbatim}
   a b  c
1: 1 7 13
2: 2 8 14
3: 3 9 15
   a  b  c
1: 4 10 16
2: 5 11 17
   a  b  c
1: 6 12 18
\end{verbatim}

\begin{verbatim}
Empty data.table (0 rows and 1 cols): ID
\end{verbatim}

\begin{Shaded}
\begin{Highlighting}[]
\KeywordTok{kable}\NormalTok{(DT[, }\KeywordTok{lapply}\NormalTok{(.SD, mean), }\DataTypeTok{by =}\NormalTok{ ID])}
\end{Highlighting}
\end{Shaded}

\begin{longtable}[]{@{}lrrr@{}}
\toprule
ID & a & b & c\tabularnewline
\midrule
\endhead
b & 2.0 & 8.0 & 14.0\tabularnewline
a & 4.5 & 10.5 & 16.5\tabularnewline
c & 6.0 & 12.0 & 18.0\tabularnewline
\bottomrule
\end{longtable}

\begin{center}\rule{0.5\linewidth}{0.5pt}\end{center}

Clojure

\begin{Shaded}
\begin{Highlighting}[]
\NormalTok{DT}

\NormalTok{(api/group-by DT }\AttributeTok{:ID}\NormalTok{ \{}\AttributeTok{:result-type} \AttributeTok{:as-map}\NormalTok{\})}
\end{Highlighting}
\end{Shaded}

\_unnamed {[}6 4{]}:

\begin{longtable}[]{@{}llll@{}}
\toprule
:ID & :a & :b & :c\tabularnewline
\midrule
\endhead
b & 1 & 7 & 13\tabularnewline
b & 2 & 8 & 14\tabularnewline
b & 3 & 9 & 15\tabularnewline
a & 4 & 10 & 16\tabularnewline
a & 5 & 11 & 17\tabularnewline
c & 6 & 12 & 18\tabularnewline
\bottomrule
\end{longtable}

\{``a'' Group: a {[}2 4{]}:

\begin{longtable}[]{@{}llll@{}}
\toprule
:ID & :a & :b & :c\tabularnewline
\midrule
\endhead
a & 4 & 10 & 16\tabularnewline
a & 5 & 11 & 17\tabularnewline
\bottomrule
\end{longtable}

, ``b'' Group: b {[}3 4{]}:

\begin{longtable}[]{@{}llll@{}}
\toprule
:ID & :a & :b & :c\tabularnewline
\midrule
\endhead
b & 1 & 7 & 13\tabularnewline
b & 2 & 8 & 14\tabularnewline
b & 3 & 9 & 15\tabularnewline
\bottomrule
\end{longtable}

, ``c'' Group: c {[}1 4{]}:

\begin{longtable}[]{@{}llll@{}}
\toprule
:ID & :a & :b & :c\tabularnewline
\midrule
\endhead
c & 6 & 12 & 18\tabularnewline
\bottomrule
\end{longtable}

\}

\begin{Shaded}
\begin{Highlighting}[]
\NormalTok{(}\KeywordTok{->}\NormalTok{ DT}
\NormalTok{    (api/group-by [}\AttributeTok{:ID}\NormalTok{])}
\NormalTok{    (api/aggregate-columns (}\KeywordTok{complement}\NormalTok{ #\{}\AttributeTok{:ID}\NormalTok{\}) dfn/mean))}
\end{Highlighting}
\end{Shaded}

\_unnamed {[}3 4{]}:

\begin{longtable}[]{@{}llll@{}}
\toprule
:ID & :a & :b & :c\tabularnewline
\midrule
\endhead
a & 4.5 & 10.5 & 16.5\tabularnewline
b & 2.0 & 8.0 & 14.0\tabularnewline
c & 6.0 & 12.0 & 18.0\tabularnewline
\bottomrule
\end{longtable}

\hypertarget{how-can-we-specify-just-the-columns-we-would-like-to-compute-the-mean-on}{%
\subparagraph{\texorpdfstring{How can we specify just the columns we
would like to compute the \texttt{mean()}
on?}{How can we specify just the columns we would like to compute the mean() on?}}\label{how-can-we-specify-just-the-columns-we-would-like-to-compute-the-mean-on}}

R

\begin{Shaded}
\begin{Highlighting}[]
\KeywordTok{kable}\NormalTok{(}\KeywordTok{head}\NormalTok{(flights[carrier }\OperatorTok{==}\StringTok{ "AA"}\NormalTok{,                         }\CommentTok{## Only on trips with carrier "AA"}
                   \KeywordTok{lapply}\NormalTok{(.SD, mean),                       }\CommentTok{## compute the mean}
                   \DataTypeTok{by =}\NormalTok{ .(origin, dest, month),             }\CommentTok{## for every 'origin,dest,month'}
                   \DataTypeTok{.SDcols =} \KeywordTok{c}\NormalTok{(}\StringTok{"arr_delay"}\NormalTok{, }\StringTok{"dep_delay"}\NormalTok{)])) }\CommentTok{## for just those specified in .SDcols}
\end{Highlighting}
\end{Shaded}

\begin{longtable}[]{@{}llrrr@{}}
\toprule
origin & dest & month & arr\_delay & dep\_delay\tabularnewline
\midrule
\endhead
JFK & LAX & 1 & 6.590361 & 14.2289157\tabularnewline
LGA & PBI & 1 & -7.758621 & 0.3103448\tabularnewline
EWR & LAX & 1 & 1.366667 & 7.5000000\tabularnewline
JFK & MIA & 1 & 15.720670 & 18.7430168\tabularnewline
JFK & SEA & 1 & 14.357143 & 30.7500000\tabularnewline
EWR & MIA & 1 & 11.011236 & 12.1235955\tabularnewline
\bottomrule
\end{longtable}

\begin{center}\rule{0.5\linewidth}{0.5pt}\end{center}

Clojure

\begin{Shaded}
\begin{Highlighting}[]
\NormalTok{(}\KeywordTok{->}\NormalTok{ flights}
\NormalTok{    (api/select-rows #(}\KeywordTok{=}\NormalTok{ (}\KeywordTok{get} \VariableTok{%} \StringTok{"carrier"}\NormalTok{) }\StringTok{"AA"}\NormalTok{))}
\NormalTok{    (api/group-by [}\StringTok{"origin"} \StringTok{"dest"} \StringTok{"month"}\NormalTok{])}
\NormalTok{    (api/aggregate-columns [}\StringTok{"arr_delay"} \StringTok{"dep_delay"}\NormalTok{] dfn/mean)}
\NormalTok{    (api/head }\DecValTok{6}\NormalTok{))}
\end{Highlighting}
\end{Shaded}

\_unnamed {[}6 5{]}:

\begin{longtable}[]{@{}lllll@{}}
\toprule
month & origin & dest & dep\_delay & arr\_delay\tabularnewline
\midrule
\endhead
9 & LGA & DFW & -0.25575448 & -8.78772379\tabularnewline
10 & LGA & DFW & 4.55276382 & 3.50000000\tabularnewline
1 & JFK & AUS & 27.60000000 & 25.20000000\tabularnewline
4 & JFK & AUS & -0.13333333 & 4.36666667\tabularnewline
5 & JFK & AUS & 14.73333333 & 6.76666667\tabularnewline
2 & JFK & AUS & 21.50000000 & 26.26923077\tabularnewline
\bottomrule
\end{longtable}

\hypertarget{how-can-we-return-the-first-two-rows-for-each-month}{%
\subparagraph{How can we return the first two rows for each
month?}\label{how-can-we-return-the-first-two-rows-for-each-month}}

R

\begin{Shaded}
\begin{Highlighting}[]
\NormalTok{ans <-}\StringTok{ }\NormalTok{flights[, }\KeywordTok{head}\NormalTok{(.SD, }\DecValTok{2}\NormalTok{), by =}\StringTok{ }\NormalTok{month]}
\KeywordTok{kable}\NormalTok{(}\KeywordTok{head}\NormalTok{(ans))}
\end{Highlighting}
\end{Shaded}

\begin{longtable}[]{@{}rrrrrlllrrr@{}}
\toprule
month & year & day & dep\_delay & arr\_delay & carrier & origin & dest &
air\_time & distance & hour\tabularnewline
\midrule
\endhead
1 & 2014 & 1 & 14 & 13 & AA & JFK & LAX & 359 & 2475 & 9\tabularnewline
1 & 2014 & 1 & -3 & 13 & AA & JFK & LAX & 363 & 2475 & 11\tabularnewline
2 & 2014 & 1 & -1 & 1 & AA & JFK & LAX & 358 & 2475 & 8\tabularnewline
2 & 2014 & 1 & -5 & 3 & AA & JFK & LAX & 358 & 2475 & 11\tabularnewline
3 & 2014 & 1 & -11 & 36 & AA & JFK & LAX & 375 & 2475 & 8\tabularnewline
3 & 2014 & 1 & -3 & 14 & AA & JFK & LAX & 368 & 2475 & 11\tabularnewline
\bottomrule
\end{longtable}

\begin{center}\rule{0.5\linewidth}{0.5pt}\end{center}

Clojure

\begin{Shaded}
\begin{Highlighting}[]
\NormalTok{(}\KeywordTok{->}\NormalTok{ flights}
\NormalTok{    (api/group-by [}\StringTok{"month"}\NormalTok{])}
\NormalTok{    (api/head }\DecValTok{2}\NormalTok{) }\CommentTok{;; head applied on each group}
\NormalTok{    (api/ungroup)}
\NormalTok{    (api/head }\DecValTok{6}\NormalTok{))}
\end{Highlighting}
\end{Shaded}

\_unnamed {[}6 11{]}:

\begin{longtable}[]{@{}lllllllllll@{}}
\toprule
dep\_delay & origin & air\_time & hour & arr\_delay & dest & distance &
year & month & day & carrier\tabularnewline
\midrule
\endhead
-8 & LGA & 113 & 18 & -23 & BNA & 764 & 2014 & 4 & 1 & MQ\tabularnewline
-8 & LGA & 71 & 18 & -11 & RDU & 431 & 2014 & 4 & 1 & MQ\tabularnewline
43 & JFK & 288 & 17 & 5 & LAS & 2248 & 2014 & 5 & 1 & AA\tabularnewline
-1 & JFK & 330 & 7 & -38 & SFO & 2586 & 2014 & 5 & 1 & AA\tabularnewline
-9 & JFK & 324 & 8 & -5 & LAX & 2475 & 2014 & 6 & 1 & AA\tabularnewline
-10 & JFK & 329 & 12 & -13 & LAX & 2475 & 2014 & 6 & 1 &
AA\tabularnewline
\bottomrule
\end{longtable}

\hypertarget{how-can-we-concatenate-columns-a-and-b-for-each-group-in-id}{%
\subparagraph{How can we concatenate columns a and b for each group in
ID?}\label{how-can-we-concatenate-columns-a-and-b-for-each-group-in-id}}

R

\begin{Shaded}
\begin{Highlighting}[]
\KeywordTok{kable}\NormalTok{(DT[, .(}\DataTypeTok{val =} \KeywordTok{c}\NormalTok{(a,b)), }\DataTypeTok{by =}\NormalTok{ ID])}
\end{Highlighting}
\end{Shaded}

\begin{longtable}[]{@{}lr@{}}
\toprule
ID & val\tabularnewline
\midrule
\endhead
b & 1\tabularnewline
b & 2\tabularnewline
b & 3\tabularnewline
b & 7\tabularnewline
b & 8\tabularnewline
b & 9\tabularnewline
a & 4\tabularnewline
a & 5\tabularnewline
a & 10\tabularnewline
a & 11\tabularnewline
c & 6\tabularnewline
c & 12\tabularnewline
\bottomrule
\end{longtable}

\begin{center}\rule{0.5\linewidth}{0.5pt}\end{center}

Clojure

\begin{Shaded}
\begin{Highlighting}[]
\NormalTok{(}\KeywordTok{->}\NormalTok{ DT}
\NormalTok{    (api/pivot->longer [}\AttributeTok{:a} \AttributeTok{:b}\NormalTok{] \{}\AttributeTok{:value-column-name} \AttributeTok{:val}\NormalTok{\})}
\NormalTok{    (api/drop-columns [:$column }\AttributeTok{:c}\NormalTok{]))}
\end{Highlighting}
\end{Shaded}

\_unnamed {[}12 2{]}:

\begin{longtable}[]{@{}ll@{}}
\toprule
:ID & :val\tabularnewline
\midrule
\endhead
b & 1\tabularnewline
b & 2\tabularnewline
b & 3\tabularnewline
a & 4\tabularnewline
a & 5\tabularnewline
c & 6\tabularnewline
b & 7\tabularnewline
b & 8\tabularnewline
b & 9\tabularnewline
a & 10\tabularnewline
a & 11\tabularnewline
c & 12\tabularnewline
\bottomrule
\end{longtable}

\hypertarget{what-if-we-would-like-to-have-all-the-values-of-column-a-and-b-concatenated-but-returned-as-a-list-column}{%
\subparagraph{\texorpdfstring{What if we would like to have all the
values of column \texttt{a} and \texttt{b} concatenated, but returned as
a list
column?}{What if we would like to have all the values of column a and b concatenated, but returned as a list column?}}\label{what-if-we-would-like-to-have-all-the-values-of-column-a-and-b-concatenated-but-returned-as-a-list-column}}

R

\begin{Shaded}
\begin{Highlighting}[]
\KeywordTok{kable}\NormalTok{(DT[, .(}\DataTypeTok{val =} \KeywordTok{list}\NormalTok{(}\KeywordTok{c}\NormalTok{(a,b))), }\DataTypeTok{by =}\NormalTok{ ID])}
\end{Highlighting}
\end{Shaded}

\begin{longtable}[]{@{}ll@{}}
\toprule
ID & val\tabularnewline
\midrule
\endhead
b & 1, 2, 3, 7, 8, 9\tabularnewline
a & 4, 5, 10, 11\tabularnewline
c & 6, 12\tabularnewline
\bottomrule
\end{longtable}

\begin{center}\rule{0.5\linewidth}{0.5pt}\end{center}

Clojure

\begin{Shaded}
\begin{Highlighting}[]
\NormalTok{(}\KeywordTok{->}\NormalTok{ DT}
\NormalTok{    (api/pivot->longer [}\AttributeTok{:a} \AttributeTok{:b}\NormalTok{] \{}\AttributeTok{:value-column-name} \AttributeTok{:val}\NormalTok{\})}
\NormalTok{    (api/drop-columns [:$column }\AttributeTok{:c}\NormalTok{])}
\NormalTok{    (api/fold-by }\AttributeTok{:ID}\NormalTok{))}
\end{Highlighting}
\end{Shaded}

\_unnamed {[}3 2{]}:

\begin{longtable}[]{@{}ll@{}}
\toprule
:ID & :val\tabularnewline
\midrule
\endhead
a & {[}4 5 10 11{]}\tabularnewline
b & {[}1 2 3 7 8 9{]}\tabularnewline
c & {[}6 12{]}\tabularnewline
\bottomrule
\end{longtable}

\hypertarget{api-tour}{%
\subsubsection{API tour}\label{api-tour}}

Below snippets are taken from
\href{https://atrebas.github.io/post/2019-03-03-datatable-dplyr/}{A
data.table and dplyr tour} written by Atrebas (permission granted).

I keep structure and subtitles but I skip \texttt{data.table} and
\texttt{dplyr} examples.

Example data

\begin{Shaded}
\begin{Highlighting}[]
\NormalTok{(}\BuiltInTok{def}\FunctionTok{ DS }\NormalTok{(api/dataset \{}\AttributeTok{:V1}\NormalTok{ (}\KeywordTok{take} \DecValTok{9}\NormalTok{ (}\KeywordTok{cycle}\NormalTok{ [}\DecValTok{1} \DecValTok{2}\NormalTok{]))}
                      \AttributeTok{:V2}\NormalTok{ (}\KeywordTok{range} \DecValTok{1} \DecValTok{10}\NormalTok{)}
                      \AttributeTok{:V3}\NormalTok{ (}\KeywordTok{take} \DecValTok{9}\NormalTok{ (}\KeywordTok{cycle}\NormalTok{ [}\FloatTok{0.5} \FloatTok{1.0} \FloatTok{1.5}\NormalTok{]))}
                      \AttributeTok{:V4}\NormalTok{ (}\KeywordTok{take} \DecValTok{9}\NormalTok{ (}\KeywordTok{cycle}\NormalTok{ [}\StringTok{"A"} \StringTok{"B"} \StringTok{"C"}\NormalTok{]))\}))}
\end{Highlighting}
\end{Shaded}

\begin{Shaded}
\begin{Highlighting}[]
\NormalTok{(api/dataset? DS)}
\NormalTok{(}\KeywordTok{class}\NormalTok{ DS)}
\end{Highlighting}
\end{Shaded}

\begin{verbatim}
true
tech.ml.dataset.impl.dataset.Dataset
\end{verbatim}

\begin{Shaded}
\begin{Highlighting}[]
\NormalTok{DS}
\end{Highlighting}
\end{Shaded}

\_unnamed {[}9 4{]}:

\begin{longtable}[]{@{}llll@{}}
\toprule
:V1 & :V2 & :V3 & :V4\tabularnewline
\midrule
\endhead
1 & 1 & 0.5 & A\tabularnewline
2 & 2 & 1.0 & B\tabularnewline
1 & 3 & 1.5 & C\tabularnewline
2 & 4 & 0.5 & A\tabularnewline
1 & 5 & 1.0 & B\tabularnewline
2 & 6 & 1.5 & C\tabularnewline
1 & 7 & 0.5 & A\tabularnewline
2 & 8 & 1.0 & B\tabularnewline
1 & 9 & 1.5 & C\tabularnewline
\bottomrule
\end{longtable}

\hypertarget{basic-operations}{%
\paragraph{Basic Operations}\label{basic-operations}}

\hypertarget{filter-rows}{%
\subparagraph{Filter rows}\label{filter-rows}}

Filter rows using indices

\begin{Shaded}
\begin{Highlighting}[]
\NormalTok{(api/select-rows DS [}\DecValTok{2} \DecValTok{3}\NormalTok{])}
\end{Highlighting}
\end{Shaded}

\_unnamed {[}2 4{]}:

\begin{longtable}[]{@{}llll@{}}
\toprule
:V1 & :V2 & :V3 & :V4\tabularnewline
\midrule
\endhead
1 & 3 & 1.5 & C\tabularnewline
2 & 4 & 0.5 & A\tabularnewline
\bottomrule
\end{longtable}

\begin{center}\rule{0.5\linewidth}{0.5pt}\end{center}

Discard rows using negative indices

In Clojure API we have separate function for that: \texttt{drop-rows}.

\begin{Shaded}
\begin{Highlighting}[]
\NormalTok{(api/drop-rows DS (}\KeywordTok{range} \DecValTok{2} \DecValTok{7}\NormalTok{))}
\end{Highlighting}
\end{Shaded}

\_unnamed {[}4 4{]}:

\begin{longtable}[]{@{}llll@{}}
\toprule
:V1 & :V2 & :V3 & :V4\tabularnewline
\midrule
\endhead
1 & 1 & 0.5 & A\tabularnewline
2 & 2 & 1.0 & B\tabularnewline
2 & 8 & 1.0 & B\tabularnewline
1 & 9 & 1.5 & C\tabularnewline
\bottomrule
\end{longtable}

\begin{center}\rule{0.5\linewidth}{0.5pt}\end{center}

Filter rows using a logical expression

\begin{Shaded}
\begin{Highlighting}[]
\NormalTok{(api/select-rows DS (}\KeywordTok{comp}\NormalTok{ #(}\KeywordTok{>} \VariableTok{%} \DecValTok{5}\NormalTok{) }\AttributeTok{:V2}\NormalTok{))}
\end{Highlighting}
\end{Shaded}

\_unnamed {[}4 4{]}:

\begin{longtable}[]{@{}llll@{}}
\toprule
:V1 & :V2 & :V3 & :V4\tabularnewline
\midrule
\endhead
2 & 6 & 1.5 & C\tabularnewline
1 & 7 & 0.5 & A\tabularnewline
2 & 8 & 1.0 & B\tabularnewline
1 & 9 & 1.5 & C\tabularnewline
\bottomrule
\end{longtable}

\begin{Shaded}
\begin{Highlighting}[]
\NormalTok{(api/select-rows DS (}\KeywordTok{comp}\NormalTok{ #\{}\StringTok{"A"} \StringTok{"C"}\NormalTok{\} }\AttributeTok{:V4}\NormalTok{))}
\end{Highlighting}
\end{Shaded}

\_unnamed {[}6 4{]}:

\begin{longtable}[]{@{}llll@{}}
\toprule
:V1 & :V2 & :V3 & :V4\tabularnewline
\midrule
\endhead
1 & 1 & 0.5 & A\tabularnewline
1 & 3 & 1.5 & C\tabularnewline
2 & 4 & 0.5 & A\tabularnewline
2 & 6 & 1.5 & C\tabularnewline
1 & 7 & 0.5 & A\tabularnewline
1 & 9 & 1.5 & C\tabularnewline
\bottomrule
\end{longtable}

\begin{center}\rule{0.5\linewidth}{0.5pt}\end{center}

Filter rows using multiple conditions

\begin{Shaded}
\begin{Highlighting}[]
\NormalTok{(api/select-rows DS #(}\KeywordTok{and}\NormalTok{ (}\KeywordTok{=}\NormalTok{ (}\AttributeTok{:V1} \VariableTok\NormalTok{) }\StringTok{"A"}\NormalTok{)))}
\end{Highlighting}
\end{Shaded}

\_unnamed {[}2 4{]}:

\begin{longtable}[]{@{}llll@{}}
\toprule
:V1 & :V2 & :V3 & :V4\tabularnewline
\midrule
\endhead
1 & 1 & 0.5 & A\tabularnewline
1 & 7 & 0.5 & A\tabularnewline
\bottomrule
\end{longtable}

\begin{center}\rule{0.5\linewidth}{0.5pt}\end{center}

Filter unique rows

\begin{Shaded}
\begin{Highlighting}[]
\NormalTok{(api/unique-by DS)}
\end{Highlighting}
\end{Shaded}

\_unnamed {[}9 4{]}:

\begin{longtable}[]{@{}llll@{}}
\toprule
:V1 & :V2 & :V3 & :V4\tabularnewline
\midrule
\endhead
1 & 1 & 0.5 & A\tabularnewline
2 & 2 & 1.0 & B\tabularnewline
1 & 3 & 1.5 & C\tabularnewline
2 & 4 & 0.5 & A\tabularnewline
1 & 5 & 1.0 & B\tabularnewline
2 & 6 & 1.5 & C\tabularnewline
1 & 7 & 0.5 & A\tabularnewline
2 & 8 & 1.0 & B\tabularnewline
1 & 9 & 1.5 & C\tabularnewline
\bottomrule
\end{longtable}

\begin{Shaded}
\begin{Highlighting}[]
\NormalTok{(api/unique-by DS [}\AttributeTok{:V1} \AttributeTok{:V4}\NormalTok{])}
\end{Highlighting}
\end{Shaded}

\_unnamed {[}6 4{]}:

\begin{longtable}[]{@{}llll@{}}
\toprule
:V1 & :V2 & :V3 & :V4\tabularnewline
\midrule
\endhead
1 & 1 & 0.5 & A\tabularnewline
2 & 2 & 1.0 & B\tabularnewline
1 & 3 & 1.5 & C\tabularnewline
2 & 4 & 0.5 & A\tabularnewline
1 & 5 & 1.0 & B\tabularnewline
2 & 6 & 1.5 & C\tabularnewline
\bottomrule
\end{longtable}

\begin{center}\rule{0.5\linewidth}{0.5pt}\end{center}

Discard rows with missing values

\begin{Shaded}
\begin{Highlighting}[]
\NormalTok{(api/drop-missing DS)}
\end{Highlighting}
\end{Shaded}

\_unnamed {[}9 4{]}:

\begin{longtable}[]{@{}llll@{}}
\toprule
:V1 & :V2 & :V3 & :V4\tabularnewline
\midrule
\endhead
1 & 1 & 0.5 & A\tabularnewline
2 & 2 & 1.0 & B\tabularnewline
1 & 3 & 1.5 & C\tabularnewline
2 & 4 & 0.5 & A\tabularnewline
1 & 5 & 1.0 & B\tabularnewline
2 & 6 & 1.5 & C\tabularnewline
1 & 7 & 0.5 & A\tabularnewline
2 & 8 & 1.0 & B\tabularnewline
1 & 9 & 1.5 & C\tabularnewline
\bottomrule
\end{longtable}

\begin{center}\rule{0.5\linewidth}{0.5pt}\end{center}

Other filters

\begin{Shaded}
\begin{Highlighting}[]
\NormalTok{(api/random DS }\DecValTok{3}\NormalTok{) }\CommentTok{;; 3 random rows}
\end{Highlighting}
\end{Shaded}

\_unnamed {[}3 4{]}:

\begin{longtable}[]{@{}llll@{}}
\toprule
:V1 & :V2 & :V3 & :V4\tabularnewline
\midrule
\endhead
1 & 3 & 1.5 & C\tabularnewline
1 & 1 & 0.5 & A\tabularnewline
1 & 3 & 1.5 & C\tabularnewline
\bottomrule
\end{longtable}

\begin{Shaded}
\begin{Highlighting}[]
\NormalTok{(api/random DS (}\KeywordTok{/}\NormalTok{ (api/row-count DS) }\DecValTok{2}\NormalTok{)) }\CommentTok{;; fraction of random rows}
\end{Highlighting}
\end{Shaded}

\_unnamed {[}5 4{]}:

\begin{longtable}[]{@{}llll@{}}
\toprule
:V1 & :V2 & :V3 & :V4\tabularnewline
\midrule
\endhead
2 & 6 & 1.5 & C\tabularnewline
2 & 6 & 1.5 & C\tabularnewline
1 & 1 & 0.5 & A\tabularnewline
1 & 9 & 1.5 & C\tabularnewline
1 & 3 & 1.5 & C\tabularnewline
\bottomrule
\end{longtable}

\begin{Shaded}
\begin{Highlighting}[]
\NormalTok{(api/by-rank DS }\AttributeTok{:V1} \KeywordTok{zero?}\NormalTok{) }\CommentTok{;; take top n entries}
\end{Highlighting}
\end{Shaded}

\_unnamed {[}4 4{]}:

\begin{longtable}[]{@{}llll@{}}
\toprule
:V1 & :V2 & :V3 & :V4\tabularnewline
\midrule
\endhead
2 & 2 & 1.0 & B\tabularnewline
2 & 4 & 0.5 & A\tabularnewline
2 & 6 & 1.5 & C\tabularnewline
2 & 8 & 1.0 & B\tabularnewline
\bottomrule
\end{longtable}

\begin{center}\rule{0.5\linewidth}{0.5pt}\end{center}

Convenience functions

\begin{Shaded}
\begin{Highlighting}[]
\NormalTok{(api/select-rows DS (}\KeywordTok{comp}\NormalTok{ (}\KeywordTok{partial} \KeywordTok{re-matches} \SpecialStringTok{#"^B"}\NormalTok{) }\KeywordTok{str} \AttributeTok{:V4}\NormalTok{))}
\end{Highlighting}
\end{Shaded}

\_unnamed {[}3 4{]}:

\begin{longtable}[]{@{}llll@{}}
\toprule
:V1 & :V2 & :V3 & :V4\tabularnewline
\midrule
\endhead
2 & 2 & 1.0 & B\tabularnewline
1 & 5 & 1.0 & B\tabularnewline
2 & 8 & 1.0 & B\tabularnewline
\bottomrule
\end{longtable}

\begin{Shaded}
\begin{Highlighting}[]
\NormalTok{(api/select-rows DS (}\KeywordTok{comp}\NormalTok{ #(}\KeywordTok{<=} \DecValTok{3} \VariableTok \DecValTok{5}\NormalTok{) }\AttributeTok{:V2}\NormalTok{))}
\end{Highlighting}
\end{Shaded}

\_unnamed {[}1 4{]}:

\begin{longtable}[]{@{}llll@{}}
\toprule
:V1 & :V2 & :V3 & :V4\tabularnewline
\midrule
\endhead
2 & 4 & 0.5 & A\tabularnewline
\bottomrule
\end{longtable}

\begin{Shaded}
\begin{Highlighting}[]
\NormalTok{(api/select-rows DS (}\KeywordTok{comp}\NormalTok{ #(}\KeywordTok{<=} \DecValTok{3} \VariableTok{%} \DecValTok{5}\NormalTok{) }\AttributeTok{:V2}\NormalTok{))}
\end{Highlighting}
\end{Shaded}

\_unnamed {[}3 4{]}:

\begin{longtable}[]{@{}llll@{}}
\toprule
:V1 & :V2 & :V3 & :V4\tabularnewline
\midrule
\endhead
1 & 3 & 1.5 & C\tabularnewline
2 & 4 & 0.5 & A\tabularnewline
1 & 5 & 1.0 & B\tabularnewline
\bottomrule
\end{longtable}

Last example skipped.

\hypertarget{sort-rows}{%
\subparagraph{Sort rows}\label{sort-rows}}

Sort rows by column

\begin{Shaded}
\begin{Highlighting}[]
\NormalTok{(api/order-by DS }\AttributeTok{:V3}\NormalTok{)}
\end{Highlighting}
\end{Shaded}

\_unnamed {[}9 4{]}:

\begin{longtable}[]{@{}llll@{}}
\toprule
:V1 & :V2 & :V3 & :V4\tabularnewline
\midrule
\endhead
1 & 1 & 0.5 & A\tabularnewline
2 & 4 & 0.5 & A\tabularnewline
1 & 7 & 0.5 & A\tabularnewline
2 & 2 & 1.0 & B\tabularnewline
1 & 5 & 1.0 & B\tabularnewline
2 & 8 & 1.0 & B\tabularnewline
1 & 3 & 1.5 & C\tabularnewline
2 & 6 & 1.5 & C\tabularnewline
1 & 9 & 1.5 & C\tabularnewline
\bottomrule
\end{longtable}

\begin{center}\rule{0.5\linewidth}{0.5pt}\end{center}

Sort rows in decreasing order

\begin{Shaded}
\begin{Highlighting}[]
\NormalTok{(api/order-by DS }\AttributeTok{:V3} \AttributeTok{:desc}\NormalTok{)}
\end{Highlighting}
\end{Shaded}

\_unnamed {[}9 4{]}:

\begin{longtable}[]{@{}llll@{}}
\toprule
:V1 & :V2 & :V3 & :V4\tabularnewline
\midrule
\endhead
1 & 3 & 1.5 & C\tabularnewline
2 & 6 & 1.5 & C\tabularnewline
1 & 9 & 1.5 & C\tabularnewline
1 & 5 & 1.0 & B\tabularnewline
2 & 2 & 1.0 & B\tabularnewline
2 & 8 & 1.0 & B\tabularnewline
1 & 7 & 0.5 & A\tabularnewline
2 & 4 & 0.5 & A\tabularnewline
1 & 1 & 0.5 & A\tabularnewline
\bottomrule
\end{longtable}

\begin{center}\rule{0.5\linewidth}{0.5pt}\end{center}

Sort rows based on several columns

\begin{Shaded}
\begin{Highlighting}[]
\NormalTok{(api/order-by DS [}\AttributeTok{:V1} \AttributeTok{:V2}\NormalTok{] [}\AttributeTok{:asc} \AttributeTok{:desc}\NormalTok{])}
\end{Highlighting}
\end{Shaded}

\_unnamed {[}9 4{]}:

\begin{longtable}[]{@{}llll@{}}
\toprule
:V1 & :V2 & :V3 & :V4\tabularnewline
\midrule
\endhead
1 & 9 & 1.5 & C\tabularnewline
1 & 7 & 0.5 & A\tabularnewline
1 & 5 & 1.0 & B\tabularnewline
1 & 3 & 1.5 & C\tabularnewline
1 & 1 & 0.5 & A\tabularnewline
2 & 8 & 1.0 & B\tabularnewline
2 & 6 & 1.5 & C\tabularnewline
2 & 4 & 0.5 & A\tabularnewline
2 & 2 & 1.0 & B\tabularnewline
\bottomrule
\end{longtable}

\hypertarget{select-columns}{%
\subparagraph{Select columns}\label{select-columns}}

Select one column using an index (not recommended)

\begin{Shaded}
\begin{Highlighting}[]
\NormalTok{(}\KeywordTok{nth}\NormalTok{ (api/columns DS }\AttributeTok{:as-seq}\NormalTok{) }\DecValTok{2}\NormalTok{) }\CommentTok{;; as column (iterable)}
\end{Highlighting}
\end{Shaded}

\begin{verbatim}
#tech.ml.dataset.column<float64>[9]
:V3
[0.5000, 1.000, 1.500, 0.5000, 1.000, 1.500, 0.5000, 1.000, 1.500, ]
\end{verbatim}

\begin{Shaded}
\begin{Highlighting}[]
\NormalTok{(api/dataset [(}\KeywordTok{nth}\NormalTok{ (api/columns DS }\AttributeTok{:as-seq}\NormalTok{) }\DecValTok{2}\NormalTok{)])}
\end{Highlighting}
\end{Shaded}

\_unnamed {[}9 1{]}:

\begin{longtable}[]{@{}l@{}}
\toprule
:V3\tabularnewline
\midrule
\endhead
0.5\tabularnewline
1.0\tabularnewline
1.5\tabularnewline
0.5\tabularnewline
1.0\tabularnewline
1.5\tabularnewline
0.5\tabularnewline
1.0\tabularnewline
1.5\tabularnewline
\bottomrule
\end{longtable}

\begin{center}\rule{0.5\linewidth}{0.5pt}\end{center}

Select one column using column name

\begin{Shaded}
\begin{Highlighting}[]
\NormalTok{(api/select-columns DS }\AttributeTok{:V2}\NormalTok{) }\CommentTok{;; as dataset}
\end{Highlighting}
\end{Shaded}

\_unnamed {[}9 1{]}:

\begin{longtable}[]{@{}l@{}}
\toprule
:V2\tabularnewline
\midrule
\endhead
1\tabularnewline
2\tabularnewline
3\tabularnewline
4\tabularnewline
5\tabularnewline
6\tabularnewline
7\tabularnewline
8\tabularnewline
9\tabularnewline
\bottomrule
\end{longtable}

\begin{Shaded}
\begin{Highlighting}[]
\NormalTok{(api/select-columns DS [}\AttributeTok{:V2}\NormalTok{]) }\CommentTok{;; as dataset}
\end{Highlighting}
\end{Shaded}

\_unnamed {[}9 1{]}:

\begin{longtable}[]{@{}l@{}}
\toprule
:V2\tabularnewline
\midrule
\endhead
1\tabularnewline
2\tabularnewline
3\tabularnewline
4\tabularnewline
5\tabularnewline
6\tabularnewline
7\tabularnewline
8\tabularnewline
9\tabularnewline
\bottomrule
\end{longtable}

\begin{Shaded}
\begin{Highlighting}[]
\NormalTok{(DS }\AttributeTok{:V2}\NormalTok{) }\CommentTok{;; as column (iterable)}
\end{Highlighting}
\end{Shaded}

\begin{verbatim}
#tech.ml.dataset.column<int64>[9]
:V2
[1, 2, 3, 4, 5, 6, 7, 8, 9, ]
\end{verbatim}

\begin{center}\rule{0.5\linewidth}{0.5pt}\end{center}

Select several columns

\begin{Shaded}
\begin{Highlighting}[]
\NormalTok{(api/select-columns DS [}\AttributeTok{:V2} \AttributeTok{:V3} \AttributeTok{:V4}\NormalTok{])}
\end{Highlighting}
\end{Shaded}

\_unnamed {[}9 3{]}:

\begin{longtable}[]{@{}lll@{}}
\toprule
:V2 & :V3 & :V4\tabularnewline
\midrule
\endhead
1 & 0.5 & A\tabularnewline
2 & 1.0 & B\tabularnewline
3 & 1.5 & C\tabularnewline
4 & 0.5 & A\tabularnewline
5 & 1.0 & B\tabularnewline
6 & 1.5 & C\tabularnewline
7 & 0.5 & A\tabularnewline
8 & 1.0 & B\tabularnewline
9 & 1.5 & C\tabularnewline
\bottomrule
\end{longtable}

\begin{center}\rule{0.5\linewidth}{0.5pt}\end{center}

Exclude columns

\begin{Shaded}
\begin{Highlighting}[]
\NormalTok{(api/select-columns DS (}\KeywordTok{complement}\NormalTok{ #\{}\AttributeTok{:V2} \AttributeTok{:V3} \AttributeTok{:V4}\NormalTok{\}))}
\end{Highlighting}
\end{Shaded}

\_unnamed {[}9 1{]}:

\begin{longtable}[]{@{}l@{}}
\toprule
:V1\tabularnewline
\midrule
\endhead
1\tabularnewline
2\tabularnewline
1\tabularnewline
2\tabularnewline
1\tabularnewline
2\tabularnewline
1\tabularnewline
2\tabularnewline
1\tabularnewline
\bottomrule
\end{longtable}

\begin{Shaded}
\begin{Highlighting}[]
\NormalTok{(api/drop-columns DS [}\AttributeTok{:V2} \AttributeTok{:V3} \AttributeTok{:V4}\NormalTok{])}
\end{Highlighting}
\end{Shaded}

\_unnamed {[}9 1{]}:

\begin{longtable}[]{@{}l@{}}
\toprule
:V1\tabularnewline
\midrule
\endhead
1\tabularnewline
2\tabularnewline
1\tabularnewline
2\tabularnewline
1\tabularnewline
2\tabularnewline
1\tabularnewline
2\tabularnewline
1\tabularnewline
\bottomrule
\end{longtable}

\begin{center}\rule{0.5\linewidth}{0.5pt}\end{center}

Other seletions

\begin{Shaded}
\begin{Highlighting}[]
\NormalTok{(}\KeywordTok{->>}\NormalTok{ (}\KeywordTok{range} \DecValTok{1} \DecValTok{3}\NormalTok{)}
\NormalTok{     (}\KeywordTok{map}\NormalTok{ (}\KeywordTok{comp} \KeywordTok{keyword}\NormalTok{ (}\KeywordTok{partial} \KeywordTok{format} \StringTok{"V%d"}\NormalTok{)))}
\NormalTok{     (api/select-columns DS))}
\end{Highlighting}
\end{Shaded}

\_unnamed {[}9 2{]}:

\begin{longtable}[]{@{}ll@{}}
\toprule
:V1 & :V2\tabularnewline
\midrule
\endhead
1 & 1\tabularnewline
2 & 2\tabularnewline
1 & 3\tabularnewline
2 & 4\tabularnewline
1 & 5\tabularnewline
2 & 6\tabularnewline
1 & 7\tabularnewline
2 & 8\tabularnewline
1 & 9\tabularnewline
\bottomrule
\end{longtable}

\begin{Shaded}
\begin{Highlighting}[]
\NormalTok{(api/reorder-columns DS }\AttributeTok{:V4}\NormalTok{)}
\end{Highlighting}
\end{Shaded}

\_unnamed {[}9 4{]}:

\begin{longtable}[]{@{}llll@{}}
\toprule
:V4 & :V1 & :V2 & :V3\tabularnewline
\midrule
\endhead
A & 1 & 1 & 0.5\tabularnewline
B & 2 & 2 & 1.0\tabularnewline
C & 1 & 3 & 1.5\tabularnewline
A & 2 & 4 & 0.5\tabularnewline
B & 1 & 5 & 1.0\tabularnewline
C & 2 & 6 & 1.5\tabularnewline
A & 1 & 7 & 0.5\tabularnewline
B & 2 & 8 & 1.0\tabularnewline
C & 1 & 9 & 1.5\tabularnewline
\bottomrule
\end{longtable}

\begin{Shaded}
\begin{Highlighting}[]
\NormalTok{(api/select-columns DS #(clojure.string/starts-with? (}\KeywordTok{name} \VariableTok{%}\NormalTok{) }\StringTok{"V"}\NormalTok{))}
\end{Highlighting}
\end{Shaded}

\_unnamed {[}9 4{]}:

\begin{longtable}[]{@{}llll@{}}
\toprule
:V1 & :V2 & :V3 & :V4\tabularnewline
\midrule
\endhead
1 & 1 & 0.5 & A\tabularnewline
2 & 2 & 1.0 & B\tabularnewline
1 & 3 & 1.5 & C\tabularnewline
2 & 4 & 0.5 & A\tabularnewline
1 & 5 & 1.0 & B\tabularnewline
2 & 6 & 1.5 & C\tabularnewline
1 & 7 & 0.5 & A\tabularnewline
2 & 8 & 1.0 & B\tabularnewline
1 & 9 & 1.5 & C\tabularnewline
\bottomrule
\end{longtable}

\begin{Shaded}
\begin{Highlighting}[]
\NormalTok{(api/select-columns DS #(clojure.string/ends-with? (}\KeywordTok{name} \VariableTok{%}\NormalTok{) }\StringTok{"3"}\NormalTok{))}
\end{Highlighting}
\end{Shaded}

\_unnamed {[}9 1{]}:

\begin{longtable}[]{@{}l@{}}
\toprule
:V3\tabularnewline
\midrule
\endhead
0.5\tabularnewline
1.0\tabularnewline
1.5\tabularnewline
0.5\tabularnewline
1.0\tabularnewline
1.5\tabularnewline
0.5\tabularnewline
1.0\tabularnewline
1.5\tabularnewline
\bottomrule
\end{longtable}

\begin{Shaded}
\begin{Highlighting}[]
\NormalTok{(api/select-columns DS }\SpecialStringTok{#"..2"}\NormalTok{) }\CommentTok{;; regex converts to string using `str` function}
\end{Highlighting}
\end{Shaded}

\_unnamed {[}9 1{]}:

\begin{longtable}[]{@{}l@{}}
\toprule
:V2\tabularnewline
\midrule
\endhead
1\tabularnewline
2\tabularnewline
3\tabularnewline
4\tabularnewline
5\tabularnewline
6\tabularnewline
7\tabularnewline
8\tabularnewline
9\tabularnewline
\bottomrule
\end{longtable}

\begin{Shaded}
\begin{Highlighting}[]
\NormalTok{(api/select-columns DS #\{}\AttributeTok{:V1} \StringTok{"X"}\NormalTok{\})}
\end{Highlighting}
\end{Shaded}

\_unnamed {[}9 1{]}:

\begin{longtable}[]{@{}l@{}}
\toprule
:V1\tabularnewline
\midrule
\endhead
1\tabularnewline
2\tabularnewline
1\tabularnewline
2\tabularnewline
1\tabularnewline
2\tabularnewline
1\tabularnewline
2\tabularnewline
1\tabularnewline
\bottomrule
\end{longtable}

\begin{Shaded}
\begin{Highlighting}[]
\NormalTok{(api/select-columns DS #(}\KeywordTok{not}\NormalTok{ (clojure.string/starts-with? (}\KeywordTok{name} \VariableTok{%}\NormalTok{) }\StringTok{"V2"}\NormalTok{)))}
\end{Highlighting}
\end{Shaded}

\_unnamed {[}9 3{]}:

\begin{longtable}[]{@{}lll@{}}
\toprule
:V1 & :V3 & :V4\tabularnewline
\midrule
\endhead
1 & 0.5 & A\tabularnewline
2 & 1.0 & B\tabularnewline
1 & 1.5 & C\tabularnewline
2 & 0.5 & A\tabularnewline
1 & 1.0 & B\tabularnewline
2 & 1.5 & C\tabularnewline
1 & 0.5 & A\tabularnewline
2 & 1.0 & B\tabularnewline
1 & 1.5 & C\tabularnewline
\bottomrule
\end{longtable}

\hypertarget{summarise-data}{%
\subparagraph{Summarise data}\label{summarise-data}}

Summarise one column

\begin{Shaded}
\begin{Highlighting}[]
\NormalTok{(}\KeywordTok{reduce} \KeywordTok{+}\NormalTok{ (DS }\AttributeTok{:V1}\NormalTok{)) }\CommentTok{;; using pure Clojure, as value}
\end{Highlighting}
\end{Shaded}

\begin{verbatim}
13
\end{verbatim}

\begin{Shaded}
\begin{Highlighting}[]
\NormalTok{(api/aggregate-columns DS }\AttributeTok{:V1}\NormalTok{ dfn/sum) }\CommentTok{;; as dataset}
\end{Highlighting}
\end{Shaded}

\_unnamed {[}1 1{]}:

\begin{longtable}[]{@{}l@{}}
\toprule
:V1\tabularnewline
\midrule
\endhead
13.0\tabularnewline
\bottomrule
\end{longtable}

\begin{Shaded}
\begin{Highlighting}[]
\NormalTok{(api/aggregate DS \{}\AttributeTok{:sumV1}\NormalTok{ #(dfn/sum (}\VariableTok \AttributeTok{:V1}\NormalTok{))}
\NormalTok{                   #(dfn/standard-deviation (}\VariableTok{%} \AttributeTok{:V3}\NormalTok{))])}
\end{Highlighting}
\end{Shaded}

\_unnamed {[}1 2{]}:

\begin{longtable}[]{@{}ll@{}}
\toprule
:summary-0 & :summary-1\tabularnewline
\midrule
\endhead
13.0 & 0.4330127\tabularnewline
\bottomrule
\end{longtable}

\begin{Shaded}
\begin{Highlighting}[]
\NormalTok{(api/aggregate-columns DS [}\AttributeTok{:V1} \AttributeTok{:V3}\NormalTok{] [dfn/sum}
\NormalTok{                                     dfn/standard-deviation])}
\end{Highlighting}
\end{Shaded}

\_unnamed {[}1 2{]}:

\begin{longtable}[]{@{}ll@{}}
\toprule
:V1 & :V3\tabularnewline
\midrule
\endhead
13.0 & 0.4330127\tabularnewline
\bottomrule
\end{longtable}

\begin{center}\rule{0.5\linewidth}{0.5pt}\end{center}

Summarise several columns and assign column names

\begin{Shaded}
\begin{Highlighting}[]
\NormalTok{(api/aggregate DS \{}\AttributeTok{:sumv1}\NormalTok{ #(dfn/sum (}\VariableTok \AttributeTok{:V3}\NormalTok{))\})}
\end{Highlighting}
\end{Shaded}

\_unnamed {[}1 2{]}:

\begin{longtable}[]{@{}ll@{}}
\toprule
:sumv1 & :sdv3\tabularnewline
\midrule
\endhead
13.0 & 0.4330127\tabularnewline
\bottomrule
\end{longtable}

\begin{center}\rule{0.5\linewidth}{0.5pt}\end{center}

Summarise a subset of rows

\begin{Shaded}
\begin{Highlighting}[]
\NormalTok{(}\KeywordTok{->}\NormalTok{ DS}
\NormalTok{    (api/select-rows (}\KeywordTok{range} \DecValTok{4}\NormalTok{))}
\NormalTok{    (api/aggregate-columns }\AttributeTok{:V1}\NormalTok{ dfn/sum))}
\end{Highlighting}
\end{Shaded}

\_unnamed {[}1 1{]}:

\begin{longtable}[]{@{}l@{}}
\toprule
:V1\tabularnewline
\midrule
\endhead
6.0\tabularnewline
\bottomrule
\end{longtable}

\hypertarget{additional-helpers}{%
\subparagraph{Additional helpers}\label{additional-helpers}}

\begin{Shaded}
\begin{Highlighting}[]
\NormalTok{(}\KeywordTok{->}\NormalTok{ DS}
\NormalTok{    (api/first)}
\NormalTok{    (api/select-columns }\AttributeTok{:V3}\NormalTok{)) }\CommentTok{;; select first row from `:V3` column}
\end{Highlighting}
\end{Shaded}

\_unnamed {[}1 1{]}:

\begin{longtable}[]{@{}l@{}}
\toprule
:V3\tabularnewline
\midrule
\endhead
0.5\tabularnewline
\bottomrule
\end{longtable}

\begin{Shaded}
\begin{Highlighting}[]
\NormalTok{(}\KeywordTok{->}\NormalTok{ DS}
\NormalTok{    (api/last)}
\NormalTok{    (api/select-columns }\AttributeTok{:V3}\NormalTok{)) }\CommentTok{;; select last row from `:V3` column}
\end{Highlighting}
\end{Shaded}

\_unnamed {[}1 1{]}:

\begin{longtable}[]{@{}l@{}}
\toprule
:V3\tabularnewline
\midrule
\endhead
1.5\tabularnewline
\bottomrule
\end{longtable}

\begin{Shaded}
\begin{Highlighting}[]
\NormalTok{(}\KeywordTok{->}\NormalTok{ DS}
\NormalTok{    (api/select-rows }\DecValTok{4}\NormalTok{)}
\NormalTok{    (api/select-columns }\AttributeTok{:V3}\NormalTok{)) }\CommentTok{;; select forth row from `:V3` column}
\end{Highlighting}
\end{Shaded}

\_unnamed {[}1 1{]}:

\begin{longtable}[]{@{}l@{}}
\toprule
:V3\tabularnewline
\midrule
\endhead
1.0\tabularnewline
\bottomrule
\end{longtable}

\begin{Shaded}
\begin{Highlighting}[]
\NormalTok{(}\KeywordTok{->}\NormalTok{ DS}
\NormalTok{    (api/select }\AttributeTok{:V3} \DecValTok{4}\NormalTok{)) }\CommentTok{;; select forth row from `:V3` column}
\end{Highlighting}
\end{Shaded}

\_unnamed {[}1 1{]}:

\begin{longtable}[]{@{}l@{}}
\toprule
:V3\tabularnewline
\midrule
\endhead
1.0\tabularnewline
\bottomrule
\end{longtable}

\begin{Shaded}
\begin{Highlighting}[]
\NormalTok{(}\KeywordTok{->}\NormalTok{ DS}
\NormalTok{    (api/unique-by }\AttributeTok{:V4}\NormalTok{)}
\NormalTok{    (api/aggregate api/row-count)) }\CommentTok{;; number of unique rows in `:V4` column, as dataset}
\end{Highlighting}
\end{Shaded}

\_unnamed {[}1 1{]}:

\begin{longtable}[]{@{}l@{}}
\toprule
:summary\tabularnewline
\midrule
\endhead
3\tabularnewline
\bottomrule
\end{longtable}

\begin{Shaded}
\begin{Highlighting}[]
\NormalTok{(}\KeywordTok{->}\NormalTok{ DS}
\NormalTok{    (api/unique-by }\AttributeTok{:V4}\NormalTok{)}
\NormalTok{    (api/row-count)) }\CommentTok{;; number of unique rows in `:V4` column, as value}
\end{Highlighting}
\end{Shaded}

\begin{verbatim}
3
\end{verbatim}

\begin{Shaded}
\begin{Highlighting}[]
\NormalTok{(}\KeywordTok{->}\NormalTok{ DS}
\NormalTok{    (api/unique-by)}
\NormalTok{    (api/row-count)) }\CommentTok{;; number of unique rows in dataset, as value}
\end{Highlighting}
\end{Shaded}

\begin{verbatim}
9
\end{verbatim}

\hypertarget{addupdatedelete-columns} \DecValTok{2}\NormalTok{))}
\end{Highlighting}
\end{Shaded}

\_unnamed {[}9 4{]}:

\begin{longtable}[]{@{}llll@{}}
\toprule
:V1 & :V2 & :V3 & :V4\tabularnewline
\midrule
\endhead
1 & 1 & 0.5 & A\tabularnewline
4 & 2 & 1.0 & B\tabularnewline
1 & 3 & 1.5 & C\tabularnewline
4 & 4 & 0.5 & A\tabularnewline
1 & 5 & 1.0 & B\tabularnewline
4 & 6 & 1.5 & C\tabularnewline
1 & 7 & 0.5 & A\tabularnewline
4 & 8 & 1.0 & B\tabularnewline
1 & 9 & 1.5 & C\tabularnewline
\bottomrule
\end{longtable}

\begin{Shaded}
\begin{Highlighting}[]
\NormalTok{(}\BuiltInTok{def}\FunctionTok{ DS }\NormalTok{(api/add-or-replace-column DS }\AttributeTok{:V1}\NormalTok{ (dfn/pow (DS }\AttributeTok{:V1}\NormalTok{) }\DecValTok{2}\NormalTok{)))}
\end{Highlighting}
\end{Shaded}

\begin{Shaded}
\begin{Highlighting}[]
\NormalTok{DS}
\end{Highlighting}
\end{Shaded}

\_unnamed {[}9 4{]}:

\begin{longtable}[]{@{}llll@{}}
\toprule
:V1 & :V2 & :V3 & :V4\tabularnewline
\midrule
\endhead
1.0 & 1 & 0.5 & A\tabularnewline
4.0 & 2 & 1.0 & B\tabularnewline
1.0 & 3 & 1.5 & C\tabularnewline
4.0 & 4 & 0.5 & A\tabularnewline
1.0 & 5 & 1.0 & B\tabularnewline
4.0 & 6 & 1.5 & C\tabularnewline
1.0 & 7 & 0.5 & A\tabularnewline
4.0 & 8 & 1.0 & B\tabularnewline
1.0 & 9 & 1.5 & C\tabularnewline
\bottomrule
\end{longtable}

\begin{center}\rule{0.5\linewidth}{0.5pt}\end{center}

Add one column

\begin{Shaded}
\begin{Highlighting}[]
\NormalTok{(api/map-columns DS }\AttributeTok{:v5}\NormalTok{ [}\AttributeTok{:V1}\NormalTok{] dfn/log)}
\end{Highlighting}
\end{Shaded}

\_unnamed {[}9 5{]}:

\begin{longtable}[]{@{}lllll@{}}
\toprule
:V1 & :V2 & :V3 & :V4 & :v5\tabularnewline
\midrule
\endhead
1.0 & 1 & 0.5 & A & 0.00000000\tabularnewline
4.0 & 2 & 1.0 & B & 1.38629436\tabularnewline
1.0 & 3 & 1.5 & C & 0.00000000\tabularnewline
4.0 & 4 & 0.5 & A & 1.38629436\tabularnewline
1.0 & 5 & 1.0 & B & 0.00000000\tabularnewline
4.0 & 6 & 1.5 & C & 1.38629436\tabularnewline
1.0 & 7 & 0.5 & A & 0.00000000\tabularnewline
4.0 & 8 & 1.0 & B & 1.38629436\tabularnewline
1.0 & 9 & 1.5 & C & 0.00000000\tabularnewline
\bottomrule
\end{longtable}

\begin{Shaded}
\begin{Highlighting}[]
\NormalTok{(}\BuiltInTok{def}\FunctionTok{ DS }\NormalTok{(api/add-or-replace-column DS }\AttributeTok{:v5}\NormalTok{ (dfn/log (DS }\AttributeTok{:V1}\NormalTok{))))}
\end{Highlighting}
\end{Shaded}

\begin{Shaded}
\begin{Highlighting}[]
\NormalTok{DS}
\end{Highlighting}
\end{Shaded}

\_unnamed {[}9 5{]}:

\begin{longtable}[]{@{}lllll@{}}
\toprule
:V1 & :V2 & :V3 & :V4 & :v5\tabularnewline
\midrule
\endhead
1.0 & 1 & 0.5 & A & 0.00000000\tabularnewline
4.0 & 2 & 1.0 & B & 1.38629436\tabularnewline
1.0 & 3 & 1.5 & C & 0.00000000\tabularnewline
4.0 & 4 & 0.5 & A & 1.38629436\tabularnewline
1.0 & 5 & 1.0 & B & 0.00000000\tabularnewline
4.0 & 6 & 1.5 & C & 1.38629436\tabularnewline
1.0 & 7 & 0.5 & A & 0.00000000\tabularnewline
4.0 & 8 & 1.0 & B & 1.38629436\tabularnewline
1.0 & 9 & 1.5 & C & 0.00000000\tabularnewline
\bottomrule
\end{longtable}

\begin{center}\rule{0.5\linewidth}{0.5pt}\end{center}

Add several columns

\begin{Shaded}
\begin{Highlighting}[]
\NormalTok{(}\BuiltInTok{def}\FunctionTok{ DS }\NormalTok{(api/add-or-replace-columns DS \{}\AttributeTok{:v6}\NormalTok{ (dfn/sqrt (DS }\AttributeTok{:V1}\NormalTok{))}
                                       \AttributeTok{:v7} \StringTok{"X"}\NormalTok{\}))}
\end{Highlighting}
\end{Shaded}

\begin{Shaded}
\begin{Highlighting}[]
\NormalTok{DS}
\end{Highlighting}
\end{Shaded}

\_unnamed {[}9 7{]}:

\begin{longtable}[]{@{}lllllll@{}}
\toprule
:V1 & :V2 & :V3 & :V4 & :v5 & :v6 & :v7\tabularnewline
\midrule
\endhead
1.0 & 1 & 0.5 & A & 0.00000000 & 1.0 & X\tabularnewline
4.0 & 2 & 1.0 & B & 1.38629436 & 2.0 & X\tabularnewline
1.0 & 3 & 1.5 & C & 0.00000000 & 1.0 & X\tabularnewline
4.0 & 4 & 0.5 & A & 1.38629436 & 2.0 & X\tabularnewline
1.0 & 5 & 1.0 & B & 0.00000000 & 1.0 & X\tabularnewline
4.0 & 6 & 1.5 & C & 1.38629436 & 2.0 & X\tabularnewline
1.0 & 7 & 0.5 & A & 0.00000000 & 1.0 & X\tabularnewline
4.0 & 8 & 1.0 & B & 1.38629436 & 2.0 & X\tabularnewline
1.0 & 9 & 1.5 & C & 0.00000000 & 1.0 & X\tabularnewline
\bottomrule
\end{longtable}

\begin{center}\rule{0.5\linewidth}{0.5pt}\end{center}

Create one column and remove the others

\begin{Shaded}
\begin{Highlighting}[]
\NormalTok{(api/dataset \{}\AttributeTok{:v8}\NormalTok{ (dfn/+ (DS }\AttributeTok{:V3}\NormalTok{) }\DecValTok{1}\NormalTok{)\})}
\end{Highlighting}
\end{Shaded}

\_unnamed {[}9 1{]}:

\begin{longtable}[]{@{}l@{}}
\toprule
:v8\tabularnewline
\midrule
\endhead
1.5\tabularnewline
2.0\tabularnewline
2.5\tabularnewline
1.5\tabularnewline
2.0\tabularnewline
2.5\tabularnewline
1.5\tabularnewline
2.0\tabularnewline
2.5\tabularnewline
\bottomrule
\end{longtable}

\begin{center}\rule{0.5\linewidth}{0.5pt}\end{center}

Remove one column

\begin{Shaded}
\begin{Highlighting}[]
\NormalTok{(}\BuiltInTok{def}\FunctionTok{ DS }\NormalTok{(api/drop-columns DS }\AttributeTok{:v5}\NormalTok{))}
\end{Highlighting}
\end{Shaded}

\begin{Shaded}
\begin{Highlighting}[]
\NormalTok{DS}
\end{Highlighting}
\end{Shaded}

\_unnamed {[}9 6{]}:

\begin{longtable}[]{@{}llllll@{}}
\toprule
:V1 & :V2 & :V3 & :V4 & :v6 & :v7\tabularnewline
\midrule
\endhead
1.0 & 1 & 0.5 & A & 1.0 & X\tabularnewline
4.0 & 2 & 1.0 & B & 2.0 & X\tabularnewline
1.0 & 3 & 1.5 & C & 1.0 & X\tabularnewline
4.0 & 4 & 0.5 & A & 2.0 & X\tabularnewline
1.0 & 5 & 1.0 & B & 1.0 & X\tabularnewline
4.0 & 6 & 1.5 & C & 2.0 & X\tabularnewline
1.0 & 7 & 0.5 & A & 1.0 & X\tabularnewline
4.0 & 8 & 1.0 & B & 2.0 & X\tabularnewline
1.0 & 9 & 1.5 & C & 1.0 & X\tabularnewline
\bottomrule
\end{longtable}

\begin{center}\rule{0.5\linewidth}{0.5pt}\end{center}

Remove several columns

\begin{Shaded}
\begin{Highlighting}[]
\NormalTok{(}\BuiltInTok{def}\FunctionTok{ DS }\NormalTok{(api/drop-columns DS [}\AttributeTok{:v6} \AttributeTok{:v7}\NormalTok{]))}
\end{Highlighting}
\end{Shaded}

\begin{Shaded}
\begin{Highlighting}[]
\NormalTok{DS}
\end{Highlighting}
\end{Shaded}

\_unnamed {[}9 4{]}:

\begin{longtable}[]{@{}llll@{}}
\toprule
:V1 & :V2 & :V3 & :V4\tabularnewline
\midrule
\endhead
1.0 & 1 & 0.5 & A\tabularnewline
4.0 & 2 & 1.0 & B\tabularnewline
1.0 & 3 & 1.5 & C\tabularnewline
4.0 & 4 & 0.5 & A\tabularnewline
1.0 & 5 & 1.0 & B\tabularnewline
4.0 & 6 & 1.5 & C\tabularnewline
1.0 & 7 & 0.5 & A\tabularnewline
4.0 & 8 & 1.0 & B\tabularnewline
1.0 & 9 & 1.5 & C\tabularnewline
\bottomrule
\end{longtable}

\begin{center}\rule{0.5\linewidth}{0.5pt}\end{center}

Remove columns using a vector of colnames

We use set here.

\begin{Shaded}
\begin{Highlighting}[]
\NormalTok{(}\BuiltInTok{def}\FunctionTok{ DS }\NormalTok{(api/select-columns DS (}\KeywordTok{complement}\NormalTok{ #\{}\AttributeTok{:V3}\NormalTok{\})))}
\end{Highlighting}
\end{Shaded}

\begin{Shaded}
\begin{Highlighting}[]
\NormalTok{DS}
\end{Highlighting}
\end{Shaded}

\_unnamed {[}9 3{]}:

\begin{longtable}[]{@{}lll@{}}
\toprule
:V1 & :V2 & :V4\tabularnewline
\midrule
\endhead
1.0 & 1 & A\tabularnewline
4.0 & 2 & B\tabularnewline
1.0 & 3 & C\tabularnewline
4.0 & 4 & A\tabularnewline
1.0 & 5 & B\tabularnewline
4.0 & 6 & C\tabularnewline
1.0 & 7 & A\tabularnewline
4.0 & 8 & B\tabularnewline
1.0 & 9 & C\tabularnewline
\bottomrule
\end{longtable}

\begin{center}\rule{0.5\linewidth}{0.5pt}\end{center}

Replace values for rows matching a condition

\begin{Shaded}
\begin{Highlighting}[]
\NormalTok{(}\BuiltInTok{def}\FunctionTok{ DS }\NormalTok{(api/map-columns DS }\AttributeTok{:V2}\NormalTok{ [}\AttributeTok{:V2}\NormalTok{] #(}\KeywordTok{if}\NormalTok{ (}\KeywordTok{<} \VariableTok\NormalTok{)))}
\end{Highlighting}
\end{Shaded}

\begin{Shaded}
\begin{Highlighting}[]
\NormalTok{DS}
\end{Highlighting}
\end{Shaded}

\_unnamed {[}9 3{]}:

\begin{longtable}[]{@{}lll@{}}
\toprule
:V1 & :V2 & :V4\tabularnewline
\midrule
\endhead
1.0 & 0 & A\tabularnewline
4.0 & 0 & B\tabularnewline
1.0 & 0 & C\tabularnewline
4.0 & 4 & A\tabularnewline
1.0 & 5 & B\tabularnewline
4.0 & 6 & C\tabularnewline
1.0 & 7 & A\tabularnewline
4.0 & 8 & B\tabularnewline
1.0 & 9 & C\tabularnewline
\bottomrule
\end{longtable}

\hypertarget{by} \AttributeTok{:V2}\NormalTok{))\}))}
\end{Highlighting}
\end{Shaded}

\_unnamed {[}3 2{]}:

\begin{longtable}[]{@{}ll@{}}
\toprule
:V4 & :sumV2\tabularnewline
\midrule
\endhead
B & 13.0\tabularnewline
C & 15.0\tabularnewline
A & 11.0\tabularnewline
\bottomrule
\end{longtable}

\begin{center}\rule{0.5\linewidth}{0.5pt}\end{center}

By several groups

\begin{Shaded}
\begin{Highlighting}[]
\NormalTok{(}\KeywordTok{->}\NormalTok{ DS}
\NormalTok{    (api/group-by [}\AttributeTok{:V4} \AttributeTok{:V1}\NormalTok{])}
\NormalTok{    (api/aggregate \{}\AttributeTok{:sumV2}\NormalTok{ #(dfn/sum (}\VariableTok{%} \AttributeTok{:V2}\NormalTok{))\}))}
\end{Highlighting}
\end{Shaded}

\_unnamed {[}6 3{]}:

\begin{longtable}[]{@{}lll@{}}
\toprule
:V4 & :V1 & :sumV2\tabularnewline
\midrule
\endhead
A & 4.0 & 4.0\tabularnewline
A & 1.0 & 7.0\tabularnewline
B & 1.0 & 5.0\tabularnewline
B & 4.0 & 8.0\tabularnewline
C & 4.0 & 6.0\tabularnewline
C & 1.0 & 9.0\tabularnewline
\bottomrule
\end{longtable}

\begin{center}\rule{0.5\linewidth}{0.5pt}\end{center}

Calling function in by

\begin{Shaded}
\begin{Highlighting}[]
\NormalTok{(}\KeywordTok{->}\NormalTok{ DS}
\NormalTok{    (api/group-by (}\KeywordTok{fn}\NormalTok{ [row]}
\NormalTok{                    (clojure.string/lower-case (}\AttributeTok{:V4}\NormalTok{ row))))}
\NormalTok{    (api/aggregate \{}\AttributeTok{:sumV1}\NormalTok{ #(dfn/sum (}\VariableTok{%} \AttributeTok{:V1}\NormalTok{))\}))}
\end{Highlighting}
\end{Shaded}

\_unnamed {[}3 2{]}:

\begin{longtable}[]{@{}ll@{}}
\toprule
:\$group-name & :sumV1\tabularnewline
\midrule
\endhead
a & 6.0\tabularnewline
b & 9.0\tabularnewline
c & 6.0\tabularnewline
\bottomrule
\end{longtable}

\begin{center}\rule{0.5\linewidth}{0.5pt}\end{center}

Assigning column name in by

\begin{Shaded}
\begin{Highlighting}[]
\NormalTok{(}\KeywordTok{->}\NormalTok{ DS}
\NormalTok{    (api/group-by (}\KeywordTok{fn}\NormalTok{ [row]}
\NormalTok{                    \{}\AttributeTok{:abc}\NormalTok{ (clojure.string/lower-case (}\AttributeTok{:V4}\NormalTok{ row))\}))}
\NormalTok{    (api/aggregate \{}\AttributeTok{:sumV1}\NormalTok{ #(dfn/sum (}\VariableTok{%} \AttributeTok{:V1}\NormalTok{))\}))}
\end{Highlighting}
\end{Shaded}

\_unnamed {[}3 2{]}:

\begin{longtable}[]{@{}ll@{}}
\toprule
:abc & :sumV1\tabularnewline
\midrule
\endhead
a & 6.0\tabularnewline
b & 9.0\tabularnewline
c & 6.0\tabularnewline
\bottomrule
\end{longtable}

\begin{Shaded}
\begin{Highlighting}[]
\NormalTok{(}\KeywordTok{->}\NormalTok{ DS}
\NormalTok{    (api/group-by (}\KeywordTok{fn}\NormalTok{ [row]}
\NormalTok{                    (clojure.string/lower-case (}\AttributeTok{:V4}\NormalTok{ row))))}
\NormalTok{    (api/aggregate \{}\AttributeTok{:sumV1}\NormalTok{ #(dfn/sum (}\VariableTok{%} \AttributeTok{:V1}\NormalTok{))\} \{}\AttributeTok{:add-group-as-column} \AttributeTok{:abc}\NormalTok{\}))}
\end{Highlighting}
\end{Shaded}

\_unnamed {[}3 2{]}:

\begin{longtable}[]{@{}ll@{}}
\toprule
:abc & :sumV1\tabularnewline
\midrule
\endhead
a & 6.0\tabularnewline
b & 9.0\tabularnewline
c & 6.0\tabularnewline
\bottomrule
\end{longtable}

\begin{center}\rule{0.5\linewidth}{0.5pt}\end{center}

Using a condition in by

\begin{Shaded}
\begin{Highlighting}[]
\NormalTok{(}\KeywordTok{->}\NormalTok{ DS}
\NormalTok{    (api/group-by #(}\KeywordTok{=}\NormalTok{ (}\AttributeTok{:V4} \VariableTok \AttributeTok{:V1}\NormalTok{))))}
\end{Highlighting}
\end{Shaded}

\_unnamed {[}2 2{]}:

\begin{longtable}[]{@{}ll@{}}
\toprule
:\$group-name & :summary\tabularnewline
\midrule
\endhead
false & 15.0\tabularnewline
true & 6.0\tabularnewline
\bottomrule
\end{longtable}

\begin{center}\rule{0.5\linewidth}{0.5pt}\end{center}

By on a subset of rows

\begin{Shaded}
\begin{Highlighting}[]
\NormalTok{(}\KeywordTok{->}\NormalTok{ DS}
\NormalTok{    (api/select-rows (}\KeywordTok{range} \DecValTok{5}\NormalTok{))}
\NormalTok{    (api/group-by }\AttributeTok{:V4}\NormalTok{)}
\NormalTok{    (api/aggregate \{}\AttributeTok{:sumV1}\NormalTok{ #(dfn/sum (}\VariableTok{%} \AttributeTok{:V1}\NormalTok{))\}))}
\end{Highlighting}
\end{Shaded}

\_unnamed {[}3 2{]}:

\begin{longtable}[]{@{}ll@{}}
\toprule
:\$group-name & :sumV1\tabularnewline
\midrule
\endhead
A & 5.0\tabularnewline
B & 5.0\tabularnewline
C & 1.0\tabularnewline
\bottomrule
\end{longtable}

\begin{center}\rule{0.5\linewidth}{0.5pt}\end{center}

Count number of observations for each group

\begin{Shaded}
\begin{Highlighting}[]
\NormalTok{(}\KeywordTok{->}\NormalTok{ DS}
\NormalTok{    (api/group-by }\AttributeTok{:V4}\NormalTok{)}
\NormalTok{    (api/aggregate api/row-count))}
\end{Highlighting}
\end{Shaded}

\_unnamed {[}3 2{]}:

\begin{longtable}[]{@{}ll@{}}
\toprule
:\$group-name & :summary\tabularnewline
\midrule
\endhead
A & 3\tabularnewline
B & 3\tabularnewline
C & 3\tabularnewline
\bottomrule
\end{longtable}

\begin{center}\rule{0.5\linewidth}{0.5pt}\end{center}

Add a column with number of observations for each group

\begin{Shaded}
\begin{Highlighting}[]
\NormalTok{(}\KeywordTok{->}\NormalTok{ DS}
\NormalTok{    (api/group-by [}\AttributeTok{:V1}\NormalTok{])}
\NormalTok{    (api/add-or-replace-column }\AttributeTok{:n}\NormalTok{ api/row-count)}
\NormalTok{    (api/ungroup))}
\end{Highlighting}
\end{Shaded}

\_unnamed {[}9 4{]}:

\begin{longtable}[]{@{}llll@{}}
\toprule
:V1 & :V2 & :V4 & :n\tabularnewline
\midrule
\endhead
4.0 & 0 & B & 4\tabularnewline
4.0 & 4 & A & 4\tabularnewline
4.0 & 6 & C & 4\tabularnewline
4.0 & 8 & B & 4\tabularnewline
1.0 & 0 & A & 5\tabularnewline
1.0 & 0 & C & 5\tabularnewline
1.0 & 5 & B & 5\tabularnewline
1.0 & 7 & A & 5\tabularnewline
1.0 & 9 & C & 5\tabularnewline
\bottomrule
\end{longtable}

\begin{center}\rule{0.5\linewidth}{0.5pt}\end{center}

Retrieve the first/last/nth observation for each group

\begin{Shaded}
\begin{Highlighting}[]
\NormalTok{(}\KeywordTok{->}\NormalTok{ DS}
\NormalTok{    (api/group-by [}\AttributeTok{:V4}\NormalTok{])}
\NormalTok{    (api/aggregate-columns }\AttributeTok{:V2} \KeywordTok{first}\NormalTok{))}
\end{Highlighting}
\end{Shaded}

\_unnamed {[}3 2{]}:

\begin{longtable}[]{@{}ll@{}}
\toprule
:V4 & :V2\tabularnewline
\midrule
\endhead
B & 0\tabularnewline
C & 0\tabularnewline
A & 0\tabularnewline
\bottomrule
\end{longtable}

\begin{Shaded}
\begin{Highlighting}[]
\NormalTok{(}\KeywordTok{->}\NormalTok{ DS}
\NormalTok{    (api/group-by [}\AttributeTok{:V4}\NormalTok{])}
\NormalTok{    (api/aggregate-columns }\AttributeTok{:V2} \KeywordTok{last}\NormalTok{))}
\end{Highlighting}
\end{Shaded}

\_unnamed {[}3 2{]}:

\begin{longtable}[]{@{}ll@{}}
\toprule
:V4 & :V2\tabularnewline
\midrule
\endhead
B & 8\tabularnewline
C & 9\tabularnewline
A & 7\tabularnewline
\bottomrule
\end{longtable}

\begin{Shaded}
\begin{Highlighting}[]
\NormalTok{(}\KeywordTok{->}\NormalTok{ DS}
\NormalTok{    (api/group-by [}\AttributeTok{:V4}\NormalTok{])}
\NormalTok{    (api/aggregate-columns }\AttributeTok{:V2}\NormalTok{ #(}\KeywordTok{nth} \VariableTok{%} \DecValTok{1}\NormalTok{)))}
\end{Highlighting}
\end{Shaded}

\_unnamed {[}3 2{]}:

\begin{longtable}[]{@{}ll@{}}
\toprule
:V4 & :V2\tabularnewline
\midrule
\endhead
B & 5\tabularnewline
C & 6\tabularnewline
A & 4\tabularnewline
\bottomrule
\end{longtable}

\hypertarget{going-further}{%
\paragraph{Going further}\label{going-further}}

\hypertarget{advanced-columns-manipulation}{%
\subparagraph{Advanced columns
manipulation}\label{advanced-columns-manipulation}}

Summarise all the columns

\begin{Shaded}
\begin{Highlighting}[]
\CommentTok{;; custom max function which works on every type}
\NormalTok{(api/aggregate-columns DS }\AttributeTok{:all}\NormalTok{ (}\KeywordTok{fn}\NormalTok{ [col] (}\KeywordTok{first}\NormalTok{ (}\KeywordTok{sort}\NormalTok{ #(}\KeywordTok{compare} \VariableTok{%2} \VariableTok{%1}\NormalTok{) col))))}
\end{Highlighting}
\end{Shaded}

\_unnamed {[}1 3{]}:

\begin{longtable}[]{@{}lll@{}}
\toprule
:V1 & :V2 & :V4\tabularnewline
\midrule
\endhead
4.0 & 9 & C\tabularnewline
\bottomrule
\end{longtable}

\begin{center}\rule{0.5\linewidth}{0.5pt}\end{center}

Summarise several columns

\begin{Shaded}
\begin{Highlighting}[]
\NormalTok{(api/aggregate-columns DS [}\AttributeTok{:V1} \AttributeTok{:V2}\NormalTok{] dfn/mean)}
\end{Highlighting}
\end{Shaded}

\_unnamed {[}1 2{]}:

\begin{longtable}[]{@{}ll@{}}
\toprule
:V1 & :V2\tabularnewline
\midrule
\endhead
2.33333333 & 4.33333333\tabularnewline
\bottomrule
\end{longtable}

\begin{center}\rule{0.5\linewidth}{0.5pt}\end{center}

Summarise several columns by group

\begin{Shaded}
\begin{Highlighting}[]
\NormalTok{(}\KeywordTok{->}\NormalTok{ DS}
\NormalTok{    (api/group-by [}\AttributeTok{:V4}\NormalTok{])}
\NormalTok{    (api/aggregate-columns [}\AttributeTok{:V1} \AttributeTok{:V2}\NormalTok{] dfn/mean))}
\end{Highlighting}
\end{Shaded}

\_unnamed {[}3 3{]}:

\begin{longtable}[]{@{}lll@{}}
\toprule
:V4 & :V1 & :V2\tabularnewline
\midrule
\endhead
B & 3.0 & 4.33333333\tabularnewline
C & 2.0 & 5.00000000\tabularnewline
A & 2.0 & 3.66666667\tabularnewline
\bottomrule
\end{longtable}

\begin{center}\rule{0.5\linewidth}{0.5pt}\end{center}

Summarise with more than one function by group

\begin{Shaded}
\begin{Highlighting}[]
\NormalTok{(}\KeywordTok{->}\NormalTok{ DS}
\NormalTok{    (api/group-by [}\AttributeTok{:V4}\NormalTok{])}
\NormalTok{    (api/aggregate-columns [}\AttributeTok{:V1} \AttributeTok{:V2}\NormalTok{] (}\KeywordTok{fn}\NormalTok{ [col]}
\NormalTok{                                       \{}\AttributeTok{:sum}\NormalTok{ (dfn/sum col)}
                                        \AttributeTok{:mean}\NormalTok{ (dfn/mean col)\})))}
\end{Highlighting}
\end{Shaded}

\_unnamed {[}3 5{]}:

\begin{longtable}[]{@{}lllll@{}}
\toprule
:V4 & :V1-sum & :V1-mean & :V2-sum & :V2-mean\tabularnewline
\midrule
\endhead
B & 9.0 & 3.0 & 13.0 & 4.33333333\tabularnewline
C & 6.0 & 2.0 & 15.0 & 5.00000000\tabularnewline
A & 6.0 & 2.0 & 11.0 & 3.66666667\tabularnewline
\bottomrule
\end{longtable}

Summarise using a condition

\begin{Shaded}
\begin{Highlighting}[]
\NormalTok{(}\KeywordTok{->}\NormalTok{ DS}
\NormalTok{    (api/select-columns }\AttributeTok{:type/numerical}\NormalTok{)}
\NormalTok{    (api/aggregate-columns }\AttributeTok{:all}\NormalTok{ dfn/mean))}
\end{Highlighting}
\end{Shaded}

\_unnamed {[}1 2{]}:

\begin{longtable}[]{@{}ll@{}}
\toprule
:V1 & :V2\tabularnewline
\midrule
\endhead
2.33333333 & 4.33333333\tabularnewline
\bottomrule
\end{longtable}

\begin{center}\rule{0.5\linewidth}{0.5pt}\end{center}

Modify all the columns

\begin{Shaded}
\begin{Highlighting}[]
\NormalTok{(api/update-columns DS }\AttributeTok{:all} \KeywordTok{reverse}\NormalTok{)}
\end{Highlighting}
\end{Shaded}

\_unnamed {[}9 3{]}:

\begin{longtable}[]{@{}lll@{}}
\toprule
:V1 & :V2 & :V4\tabularnewline
\midrule
\endhead
1.000 & 9 & C\tabularnewline
4.000 & 8 & B\tabularnewline
1.000 & 7 & A\tabularnewline
4.000 & 6 & C\tabularnewline
1.000 & 5 & B\tabularnewline
4.000 & 4 & A\tabularnewline
1.000 & 0 & C\tabularnewline
4.000 & 0 & B\tabularnewline
1.000 & 0 & A\tabularnewline
\bottomrule
\end{longtable}

\begin{center}\rule{0.5\linewidth}{0.5pt}\end{center}

Modify several columns (dropping the others)

\begin{Shaded}
\begin{Highlighting}[]
\NormalTok{(}\KeywordTok{->}\NormalTok{ DS}
\NormalTok{    (api/select-columns [}\AttributeTok{:V1} \AttributeTok{:V2}\NormalTok{])}
\NormalTok{    (api/update-columns }\AttributeTok{:all}\NormalTok{ dfn/sqrt))}
\end{Highlighting}
\end{Shaded}

\_unnamed {[}9 2{]}:

\begin{longtable}[]{@{}ll@{}}
\toprule
:V1 & :V2\tabularnewline
\midrule
\endhead
1.0 & 0.00000000\tabularnewline
2.0 & 0.00000000\tabularnewline
1.0 & 0.00000000\tabularnewline
2.0 & 2.00000000\tabularnewline
1.0 & 2.23606798\tabularnewline
2.0 & 2.44948974\tabularnewline
1.0 & 2.64575131\tabularnewline
2.0 & 2.82842712\tabularnewline
1.0 & 3.00000000\tabularnewline
\bottomrule
\end{longtable}

\begin{Shaded}
\begin{Highlighting}[]
\NormalTok{(}\KeywordTok{->}\NormalTok{ DS}
\NormalTok{    (api/select-columns (}\KeywordTok{complement}\NormalTok{ #\{}\AttributeTok{:V4}\NormalTok{\}))}
\NormalTok{    (api/update-columns }\AttributeTok{:all}\NormalTok{ dfn/exp))}
\end{Highlighting}
\end{Shaded}

\_unnamed {[}9 2{]}:

\begin{longtable}[]{@{}ll@{}}
\toprule
:V1 & :V2\tabularnewline
\midrule
\endhead
2.71828183 & 1.00000000\tabularnewline
54.59815003 & 1.00000000\tabularnewline
2.71828183 & 1.00000000\tabularnewline
54.59815003 & 54.59815003\tabularnewline
2.71828183 & 148.41315910\tabularnewline
54.59815003 & 403.42879349\tabularnewline
2.71828183 & 1096.63315843\tabularnewline
54.59815003 & 2980.95798704\tabularnewline
2.71828183 & 8103.08392758\tabularnewline
\bottomrule
\end{longtable}

\begin{center}\rule{0.5\linewidth}{0.5pt}\end{center}

Modify several columns (keeping the others)

\begin{Shaded}
\begin{Highlighting}[]
\NormalTok{(}\BuiltInTok{def}\FunctionTok{ DS }\NormalTok{(api/update-columns DS [}\AttributeTok{:V1} \AttributeTok{:V2}\NormalTok{] dfn/sqrt))}
\end{Highlighting}
\end{Shaded}

\begin{Shaded}
\begin{Highlighting}[]
\NormalTok{DS}
\end{Highlighting}
\end{Shaded}

\_unnamed {[}9 3{]}:

\begin{longtable}[]{@{}lll@{}}
\toprule
:V1 & :V2 & :V4\tabularnewline
\midrule
\endhead
1.0 & 0.00000000 & A\tabularnewline
2.0 & 0.00000000 & B\tabularnewline
1.0 & 0.00000000 & C\tabularnewline
2.0 & 2.00000000 & A\tabularnewline
1.0 & 2.23606798 & B\tabularnewline
2.0 & 2.44948974 & C\tabularnewline
1.0 & 2.64575131 & A\tabularnewline
2.0 & 2.82842712 & B\tabularnewline
1.0 & 3.00000000 & C\tabularnewline
\bottomrule
\end{longtable}

\begin{Shaded}
\begin{Highlighting}[]
\NormalTok{(}\BuiltInTok{def}\FunctionTok{ DS }\NormalTok{(api/update-columns DS (}\KeywordTok{complement}\NormalTok{ #\{}\AttributeTok{:V4}\NormalTok{\}) #(dfn/pow }\VariableTok{%} \DecValTok{2}\NormalTok{)))}
\end{Highlighting}
\end{Shaded}

\begin{Shaded}
\begin{Highlighting}[]
\NormalTok{DS}
\end{Highlighting}
\end{Shaded}

\_unnamed {[}9 3{]}:

\begin{longtable}[]{@{}lll@{}}
\toprule
:V1 & :V2 & :V4\tabularnewline
\midrule
\endhead
1.0 & 0.0 & A\tabularnewline
4.0 & 0.0 & B\tabularnewline
1.0 & 0.0 & C\tabularnewline
4.0 & 4.0 & A\tabularnewline
1.0 & 5.0 & B\tabularnewline
4.0 & 6.0 & C\tabularnewline
1.0 & 7.0 & A\tabularnewline
4.0 & 8.0 & B\tabularnewline
1.0 & 9.0 & C\tabularnewline
\bottomrule
\end{longtable}

\begin{center}\rule{0.5\linewidth}{0.5pt}\end{center}

Modify columns using a condition (dropping the others)

\begin{Shaded}
\begin{Highlighting}[]
\NormalTok{(}\KeywordTok{->}\NormalTok{ DS}
\NormalTok{    (api/select-columns }\AttributeTok{:type/numerical}\NormalTok{)}
\NormalTok{    (api/update-columns }\AttributeTok{:all}\NormalTok{ #(dfn/- }\VariableTok{%} \DecValTok{1}\NormalTok{)))}
\end{Highlighting}
\end{Shaded}

\_unnamed {[}9 2{]}:

\begin{longtable}[]{@{}ll@{}}
\toprule
:V1 & :V2\tabularnewline
\midrule
\endhead
0.0 & -1.0\tabularnewline
3.0 & -1.0\tabularnewline
0.0 & -1.0\tabularnewline
3.0 & 3.0\tabularnewline
0.0 & 4.0\tabularnewline
3.0 & 5.0\tabularnewline
0.0 & 6.0\tabularnewline
3.0 & 7.0\tabularnewline
0.0 & 8.0\tabularnewline
\bottomrule
\end{longtable}

\begin{center}\rule{0.5\linewidth}{0.5pt}\end{center}

Modify columns using a condition (keeping the others)

\begin{Shaded}
\begin{Highlighting}[]
\NormalTok{(}\BuiltInTok{def}\FunctionTok{ DS }\NormalTok{(api/convert-types DS }\AttributeTok{:type/numerical} \AttributeTok{:int32}\NormalTok{))}
\end{Highlighting}
\end{Shaded}

\begin{Shaded}
\begin{Highlighting}[]
\NormalTok{DS}
\end{Highlighting}
\end{Shaded}

\_unnamed {[}9 3{]}:

\begin{longtable}[]{@{}lll@{}}
\toprule
:V1 & :V2 & :V4\tabularnewline
\midrule
\endhead
1 & 0 & A\tabularnewline
4 & 0 & B\tabularnewline
1 & 0 & C\tabularnewline
4 & 4 & A\tabularnewline
1 & 5 & B\tabularnewline
4 & 5 & C\tabularnewline
1 & 7 & A\tabularnewline
4 & 8 & B\tabularnewline
1 & 9 & C\tabularnewline
\bottomrule
\end{longtable}

\begin{center}\rule{0.5\linewidth}{0.5pt}\end{center}

Use a complex expression

\begin{Shaded}
\begin{Highlighting}[]
\NormalTok{(}\KeywordTok{->}\NormalTok{ DS}
\NormalTok{    (api/group-by [}\AttributeTok{:V4}\NormalTok{])}
\NormalTok{    (api/head }\DecValTok{2}\NormalTok{)}
\NormalTok{    (api/add-or-replace-column }\AttributeTok{:V2} \StringTok{"X"}\NormalTok{)}
\NormalTok{    (api/ungroup))}
\end{Highlighting}
\end{Shaded}

\_unnamed {[}6 3{]}:

\begin{longtable}[]{@{}lll@{}}
\toprule
:V1 & :V2 & :V4\tabularnewline
\midrule
\endhead
4 & X & B\tabularnewline
1 & X & B\tabularnewline
1 & X & C\tabularnewline
4 & X & C\tabularnewline
1 & X & A\tabularnewline
4 & X & A\tabularnewline
\bottomrule
\end{longtable}

\begin{center}\rule{0.5\linewidth}{0.5pt}\end{center}

Use multiple expressions

\begin{Shaded}
\begin{Highlighting}[]
\NormalTok{(api/dataset (}\KeywordTok{let}\NormalTok{ [x (dfn/+ (DS }\AttributeTok{:V1}\NormalTok{) (dfn/sum (DS }\AttributeTok{:V2}\NormalTok{)))]}
\NormalTok{               (}\KeywordTok{println}\NormalTok{ (}\KeywordTok{seq}\NormalTok{ (DS }\AttributeTok{:V1}\NormalTok{)))}
\NormalTok{               (}\KeywordTok{println}\NormalTok{ (api/info (api/select-columns DS }\AttributeTok{:V1}\NormalTok{)))}
\NormalTok{               \{}\AttributeTok{:A}\NormalTok{ (}\KeywordTok{range} \DecValTok{1}\NormalTok{ (}\KeywordTok{inc}\NormalTok{ (api/row-count DS)))}
                \AttributeTok{:B}\NormalTok{ x\}))}
\end{Highlighting}
\end{Shaded}

(1 4 1 4 1 4 1 4 1) \_unnamed: descriptive-stats {[}1 9{]}:

\begin{longtable}[]{@{}lllllllll@{}}
\toprule
\begin{minipage}[b]{0.08\columnwidth}\raggedright
:col-name\strut
\end{minipage} & \begin{minipage}[b]{0.08\columnwidth}\raggedright
:datatype\strut
\end{minipage} & \begin{minipage}[b]{0.08\columnwidth}\raggedright
:n-valid\strut
\end{minipage} & \begin{minipage}[b]{0.09\columnwidth}\raggedright
:n-missing\strut
\end{minipage} & \begin{minipage}[b]{0.05\columnwidth}\raggedright
:min\strut
\end{minipage} & \begin{minipage}[b]{0.09\columnwidth}\raggedright
:mean\strut
\end{minipage} & \begin{minipage}[b]{0.05\columnwidth}\raggedright
:max\strut
\end{minipage} & \begin{minipage}[b]{0.16\columnwidth}\raggedright
:standard-deviation\strut
\end{minipage} & \begin{minipage}[b]{0.09\columnwidth}\raggedright
:skew\strut
\end{minipage}\tabularnewline
\midrule
\endhead
\begin{minipage}[t]{0.08\columnwidth}\raggedright
:V1\strut
\end{minipage} & \begin{minipage}[t]{0.08\columnwidth}\raggedright
:int32\strut
\end{minipage} & \begin{minipage}[t]{0.08\columnwidth}\raggedright
9\strut
\end{minipage} & \begin{minipage}[t]{0.09\columnwidth}\raggedright
0\strut
\end{minipage} & \begin{minipage}[t]{0.05\columnwidth}\raggedright
1.0\strut
\end{minipage} & \begin{minipage}[t]{0.09\columnwidth}\raggedright
2.33333333\strut
\end{minipage} & \begin{minipage}[t]{0.05\columnwidth}\raggedright
4.0\strut
\end{minipage} & \begin{minipage}[t]{0.16\columnwidth}\raggedright
1.58113883\strut
\end{minipage} & \begin{minipage}[t]{0.09\columnwidth}\raggedright
0.27105237\strut
\end{minipage}\tabularnewline
\bottomrule
\end{longtable}

\_unnamed {[}9 2{]}:

\begin{longtable}[]{@{}ll@{}}
\toprule
:A & :B\tabularnewline
\midrule
\endhead
1 & 39.0\tabularnewline
2 & 42.0\tabularnewline
3 & 39.0\tabularnewline
4 & 42.0\tabularnewline
5 & 39.0\tabularnewline
6 & 42.0\tabularnewline
7 & 39.0\tabularnewline
8 & 42.0\tabularnewline
9 & 39.0\tabularnewline
\bottomrule
\end{longtable}

\hypertarget{chain-expressions} \AttributeTok{:V1}\NormalTok{))\})}
\NormalTok{    (api/select-rows #(}\KeywordTok{>=}\NormalTok{ (}\AttributeTok{:V1sum} \VariableTok{%}\NormalTok{) }\DecValTok{5}\NormalTok{)))}
\end{Highlighting}
\end{Shaded}

\_unnamed {[}3 2{]}:

\begin{longtable}[]{@{}ll@{}}
\toprule
:V4 & :V1sum\tabularnewline
\midrule
\endhead
B & 9.0\tabularnewline
C & 6.0\tabularnewline
A & 6.0\tabularnewline
\bottomrule
\end{longtable}

\begin{Shaded}
\begin{Highlighting}[]
\NormalTok{(}\KeywordTok{->}\NormalTok{ DS}
\NormalTok{    (api/group-by [}\AttributeTok{:V4}\NormalTok{])}
\NormalTok{    (api/aggregate \{}\AttributeTok{:V1sum}\NormalTok{ #(dfn/sum (}\VariableTok{%} \AttributeTok{:V1}\NormalTok{))\})}
\NormalTok{    (api/order-by }\AttributeTok{:V1sum} \AttributeTok{:desc}\NormalTok{))}
\end{Highlighting}
\end{Shaded}

\_unnamed {[}3 2{]}:

\begin{longtable}[]{@{}ll@{}}
\toprule
:V4 & :V1sum\tabularnewline
\midrule
\endhead
B & 9.0\tabularnewline
C & 6.0\tabularnewline
A & 6.0\tabularnewline
\bottomrule
\end{longtable}

\hypertarget{indexing-and-keys}{%
\subparagraph{Indexing and Keys}\label{indexing-and-keys}}

Set the key/index (order)

\begin{Shaded}
\begin{Highlighting}[]
\NormalTok{(}\BuiltInTok{def}\FunctionTok{ DS }\NormalTok{(api/order-by DS }\AttributeTok{:V4}\NormalTok{))}
\end{Highlighting}
\end{Shaded}

\begin{Shaded}
\begin{Highlighting}[]
\NormalTok{DS}
\end{Highlighting}
\end{Shaded}

\_unnamed {[}9 3{]}:

\begin{longtable}[]{@{}lll@{}}
\toprule
:V1 & :V2 & :V4\tabularnewline
\midrule
\endhead
1 & 0 & A\tabularnewline
4 & 4 & A\tabularnewline
1 & 7 & A\tabularnewline
4 & 0 & B\tabularnewline
1 & 5 & B\tabularnewline
4 & 8 & B\tabularnewline
1 & 0 & C\tabularnewline
4 & 5 & C\tabularnewline
1 & 9 & C\tabularnewline
\bottomrule
\end{longtable}

Select the matching rows

\begin{Shaded}
\begin{Highlighting}[]
\NormalTok{(api/select-rows DS #(}\KeywordTok{=}\NormalTok{ (}\AttributeTok{:V4} \VariableTok{%}\NormalTok{) }\StringTok{"A"}\NormalTok{))}
\end{Highlighting}
\end{Shaded}

\_unnamed {[}3 3{]}:

\begin{longtable}[]{@{}lll@{}}
\toprule
:V1 & :V2 & :V4\tabularnewline
\midrule
\endhead
1 & 0 & A\tabularnewline
4 & 4 & A\tabularnewline
1 & 7 & A\tabularnewline
\bottomrule
\end{longtable}

\begin{Shaded}
\begin{Highlighting}[]
\NormalTok{(api/select-rows DS (}\KeywordTok{comp}\NormalTok{ #\{}\StringTok{"A"} \StringTok{"C"}\NormalTok{\} }\AttributeTok{:V4}\NormalTok{))}
\end{Highlighting}
\end{Shaded}

\_unnamed {[}6 3{]}:

\begin{longtable}[]{@{}lll@{}}
\toprule
:V1 & :V2 & :V4\tabularnewline
\midrule
\endhead
1 & 0 & A\tabularnewline
4 & 4 & A\tabularnewline
1 & 7 & A\tabularnewline
1 & 0 & C\tabularnewline
4 & 5 & C\tabularnewline
1 & 9 & C\tabularnewline
\bottomrule
\end{longtable}

\begin{center}\rule{0.5\linewidth}{0.5pt}\end{center}

Select the first matching row

\begin{Shaded}
\begin{Highlighting}[]
\NormalTok{(}\KeywordTok{->}\NormalTok{ DS}
\NormalTok{    (api/select-rows #(}\KeywordTok{=}\NormalTok{ (}\AttributeTok{:V4} \VariableTok{%}\NormalTok{) }\StringTok{"B"}\NormalTok{))}
\NormalTok{    (api/first))}
\end{Highlighting}
\end{Shaded}

\_unnamed {[}1 3{]}:

\begin{longtable}[]{@{}lll@{}}
\toprule
:V1 & :V2 & :V4\tabularnewline
\midrule
\endhead
4 & 0 & B\tabularnewline
\bottomrule
\end{longtable}

\begin{Shaded}
\begin{Highlighting}[]
\NormalTok{(}\KeywordTok{->}\NormalTok{ DS}
\NormalTok{    (api/unique-by }\AttributeTok{:V4}\NormalTok{)}
\NormalTok{    (api/select-rows (}\KeywordTok{comp}\NormalTok{ #\{}\StringTok{"B"} \StringTok{"C"}\NormalTok{\} }\AttributeTok{:V4}\NormalTok{)))}
\end{Highlighting}
\end{Shaded}

\_unnamed {[}2 3{]}:

\begin{longtable}[]{@{}lll@{}}
\toprule
:V1 & :V2 & :V4\tabularnewline
\midrule
\endhead
4 & 0 & B\tabularnewline
1 & 0 & C\tabularnewline
\bottomrule
\end{longtable}

\begin{center}\rule{0.5\linewidth}{0.5pt}\end{center}

Select the last matching row

\begin{Shaded}
\begin{Highlighting}[]
\NormalTok{(}\KeywordTok{->}\NormalTok{ DS}
\NormalTok{    (api/select-rows #(}\KeywordTok{=}\NormalTok{ (}\AttributeTok{:V4} \VariableTok{%}\NormalTok{) }\StringTok{"A"}\NormalTok{))}
\NormalTok{    (api/last))}
\end{Highlighting}
\end{Shaded}

\_unnamed {[}1 3{]}:

\begin{longtable}[]{@{}lll@{}}
\toprule
:V1 & :V2 & :V4\tabularnewline
\midrule
\endhead
1 & 7 & A\tabularnewline
\bottomrule
\end{longtable}

\begin{center}\rule{0.5\linewidth}{0.5pt}\end{center}

Nomatch argument

\begin{Shaded}
\begin{Highlighting}[]
\NormalTok{(api/select-rows DS (}\KeywordTok{comp}\NormalTok{ #\{}\StringTok{"A"} \StringTok{"D"}\NormalTok{\} }\AttributeTok{:V4}\NormalTok{))}
\end{Highlighting}
\end{Shaded}

\_unnamed {[}3 3{]}:

\begin{longtable}[]{@{}lll@{}}
\toprule
:V1 & :V2 & :V4\tabularnewline
\midrule
\endhead
1 & 0 & A\tabularnewline
4 & 4 & A\tabularnewline
1 & 7 & A\tabularnewline
\bottomrule
\end{longtable}

\begin{center}\rule{0.5\linewidth}{0.5pt}\end{center}

Apply a function on the matching rows

\begin{Shaded}
\begin{Highlighting}[]
\NormalTok{(}\KeywordTok{->}\NormalTok{ DS}
\NormalTok{    (api/select-rows (}\KeywordTok{comp}\NormalTok{ #\{}\StringTok{"A"} \StringTok{"C"}\NormalTok{\} }\AttributeTok{:V4}\NormalTok{))}
\NormalTok{    (api/aggregate-columns }\AttributeTok{:V1}\NormalTok{ (}\KeywordTok{fn}\NormalTok{ [col]}
\NormalTok{                                 \{}\AttributeTok{:sum}\NormalTok{ (dfn/sum col)\})))}
\end{Highlighting}
\end{Shaded}

\_unnamed {[}1 1{]}:

\begin{longtable}[]{@{}l@{}}
\toprule
:V1-sum\tabularnewline
\midrule
\endhead
12.0\tabularnewline
\bottomrule
\end{longtable}

\begin{center}\rule{0.5\linewidth}{0.5pt}\end{center}

Modify values for matching rows

\begin{Shaded}
\begin{Highlighting}[]
\NormalTok{(}\BuiltInTok{def}\FunctionTok{ DS }\NormalTok{(}\KeywordTok{->}\NormalTok{ DS}
\NormalTok{            (api/map-columns }\AttributeTok{:V1}\NormalTok{ [}\AttributeTok{:V1} \AttributeTok{:V4}\NormalTok{] #(}\KeywordTok{if}\NormalTok{ (}\KeywordTok{=} \VariableTok{%2} \StringTok{"A"}\NormalTok{) }\DecValTok{0} \VariableTok{%1}\NormalTok{))}
\NormalTok{            (api/order-by }\AttributeTok{:V4}\NormalTok{)))}
\end{Highlighting}
\end{Shaded}

\begin{Shaded}
\begin{Highlighting}[]
\NormalTok{DS}
\end{Highlighting}
\end{Shaded}

\_unnamed {[}9 3{]}:

\begin{longtable}[]{@{}lll@{}}
\toprule
:V1 & :V2 & :V4\tabularnewline
\midrule
\endhead
0 & 0 & A\tabularnewline
0 & 4 & A\tabularnewline
0 & 7 & A\tabularnewline
4 & 0 & B\tabularnewline
1 & 5 & B\tabularnewline
4 & 8 & B\tabularnewline
1 & 0 & C\tabularnewline
4 & 5 & C\tabularnewline
1 & 9 & C\tabularnewline
\bottomrule
\end{longtable}

\begin{center}\rule{0.5\linewidth}{0.5pt}\end{center}

Use keys in by

\begin{Shaded}
\begin{Highlighting}[]
\NormalTok{(}\KeywordTok{->}\NormalTok{ DS}
\NormalTok{    (api/select-rows (}\KeywordTok{comp}\NormalTok{ (}\KeywordTok{complement}\NormalTok{ #\{}\StringTok{"B"}\NormalTok{\}) }\AttributeTok{:V4}\NormalTok{))}
\NormalTok{    (api/group-by [}\AttributeTok{:V4}\NormalTok{])}
\NormalTok{    (api/aggregate-columns }\AttributeTok{:V1}\NormalTok{ dfn/sum))}
\end{Highlighting}
\end{Shaded}

\_unnamed {[}2 2{]}:

\begin{longtable}[]{@{}ll@{}}
\toprule
:V4 & :V1\tabularnewline
\midrule
\endhead
C & 6.0\tabularnewline
A & 0.0\tabularnewline
\bottomrule
\end{longtable}

\begin{center}\rule{0.5\linewidth}{0.5pt}\end{center}

Set keys/indices for multiple columns (ordered)

\begin{Shaded}
\begin{Highlighting}[]
\NormalTok{(api/order-by DS [}\AttributeTok{:V4} \AttributeTok{:V1}\NormalTok{])}
\end{Highlighting}
\end{Shaded}

\_unnamed {[}9 3{]}:

\begin{longtable}[]{@{}lll@{}}
\toprule
:V1 & :V2 & :V4\tabularnewline
\midrule
\endhead
0 & 0 & A\tabularnewline
0 & 4 & A\tabularnewline
0 & 7 & A\tabularnewline
1 & 5 & B\tabularnewline
4 & 0 & B\tabularnewline
4 & 8 & B\tabularnewline
1 & 0 & C\tabularnewline
1 & 9 & C\tabularnewline
4 & 5 & C\tabularnewline
\bottomrule
\end{longtable}

\begin{center}\rule{0.5\linewidth}{0.5pt}\end{center}

Subset using multiple keys/indices

\begin{Shaded}
\begin{Highlighting}[]
\NormalTok{(}\KeywordTok{->}\NormalTok{ DS}
\NormalTok{    (api/select-rows #(}\KeywordTok{and}\NormalTok{ (}\KeywordTok{=}\NormalTok{ (}\AttributeTok{:V1} \VariableTok\NormalTok{) }\StringTok{"C"}\NormalTok{))))}
\end{Highlighting}
\end{Shaded}

\_unnamed {[}2 3{]}:

\begin{longtable}[]{@{}lll@{}}
\toprule
:V1 & :V2 & :V4\tabularnewline
\midrule
\endhead
1 & 0 & C\tabularnewline
1 & 9 & C\tabularnewline
\bottomrule
\end{longtable}

\begin{Shaded}
\begin{Highlighting}[]
\NormalTok{(}\KeywordTok{->}\NormalTok{ DS}
\NormalTok{    (api/select-rows #(}\KeywordTok{and}\NormalTok{ (}\KeywordTok{=}\NormalTok{ (}\AttributeTok{:V1} \VariableTok\NormalTok{)))))}
\end{Highlighting}
\end{Shaded}

\_unnamed {[}3 3{]}:

\begin{longtable}[]{@{}lll@{}}
\toprule
:V1 & :V2 & :V4\tabularnewline
\midrule
\endhead
1 & 5 & B\tabularnewline
1 & 0 & C\tabularnewline
1 & 9 & C\tabularnewline
\bottomrule
\end{longtable}

\begin{Shaded}
\begin{Highlighting}[]
\NormalTok{(}\KeywordTok{->}\NormalTok{ DS}
\NormalTok{    (api/select-rows #(}\KeywordTok{and}\NormalTok{ (}\KeywordTok{=}\NormalTok{ (}\AttributeTok{:V1} \VariableTok\NormalTok{))) \{}\AttributeTok{:result-type} \AttributeTok{:as-indexes}\NormalTok{\}))}
\end{Highlighting}
\end{Shaded}

\begin{verbatim}
(4 6 8)
\end{verbatim}

\hypertarget{set-modifications}{%
\subparagraph{set*() modifications}\label{set-modifications}}

Replace values

There is no mutating operations \texttt{tech.ml.dataset} or easy way to
set value.

\begin{Shaded}
\begin{Highlighting}[]
\NormalTok{(}\BuiltInTok{def}\FunctionTok{ DS }\NormalTok{(api/update-columns DS }\AttributeTok{:V2}\NormalTok{ #(map-indexed (}\KeywordTok{fn}\NormalTok{ [idx v]}
\NormalTok{                                                   (}\KeywordTok{if}\NormalTok{ (}\KeywordTok{zero?}\NormalTok{ idx) }\DecValTok{3}\NormalTok{ v)) }\VariableTok{%}\NormalTok{)))}
\end{Highlighting}
\end{Shaded}

\begin{Shaded}
\begin{Highlighting}[]
\NormalTok{DS}
\end{Highlighting}
\end{Shaded}

\_unnamed {[}9 3{]}:

\begin{longtable}[]{@{}lll@{}}
\toprule
:V1 & :V2 & :V4\tabularnewline
\midrule
\endhead
0 & 3 & A\tabularnewline
0 & 4 & A\tabularnewline
0 & 7 & A\tabularnewline
4 & 0 & B\tabularnewline
1 & 5 & B\tabularnewline
4 & 8 & B\tabularnewline
1 & 0 & C\tabularnewline
4 & 5 & C\tabularnewline
1 & 9 & C\tabularnewline
\bottomrule
\end{longtable}

\begin{center}\rule{0.5\linewidth}{0.5pt}\end{center}

Reorder rows

\begin{Shaded}
\begin{Highlighting}[]
\NormalTok{(}\BuiltInTok{def}\FunctionTok{ DS }\NormalTok{(api/order-by DS [}\AttributeTok{:V4} \AttributeTok{:V1}\NormalTok{] [}\AttributeTok{:asc} \AttributeTok{:desc}\NormalTok{]))}
\end{Highlighting}
\end{Shaded}

\begin{Shaded}
\begin{Highlighting}[]
\NormalTok{DS}
\end{Highlighting}
\end{Shaded}

\_unnamed {[}9 3{]}:

\begin{longtable}[]{@{}lll@{}}
\toprule
:V1 & :V2 & :V4\tabularnewline
\midrule
\endhead
0 & 3 & A\tabularnewline
0 & 4 & A\tabularnewline
0 & 7 & A\tabularnewline
4 & 0 & B\tabularnewline
4 & 8 & B\tabularnewline
1 & 5 & B\tabularnewline
4 & 5 & C\tabularnewline
1 & 0 & C\tabularnewline
1 & 9 & C\tabularnewline
\bottomrule
\end{longtable}

\begin{center}\rule{0.5\linewidth}{0.5pt}\end{center}

Modify colnames

\begin{Shaded}
\begin{Highlighting}[]
\NormalTok{(}\BuiltInTok{def}\FunctionTok{ DS }\NormalTok{(api/rename-columns DS \{}\AttributeTok{:V2} \StringTok{"v2"}\NormalTok{\}))}
\end{Highlighting}
\end{Shaded}

\begin{Shaded}
\begin{Highlighting}[]
\NormalTok{DS}
\end{Highlighting}
\end{Shaded}

\_unnamed {[}9 3{]}:

\begin{longtable}[]{@{}lll@{}}
\toprule
:V1 & v2 & :V4\tabularnewline
\midrule
\endhead
0 & 3 & A\tabularnewline
0 & 4 & A\tabularnewline
0 & 7 & A\tabularnewline
4 & 0 & B\tabularnewline
4 & 8 & B\tabularnewline
1 & 5 & B\tabularnewline
4 & 5 & C\tabularnewline
1 & 0 & C\tabularnewline
1 & 9 & C\tabularnewline
\bottomrule
\end{longtable}

\begin{Shaded}
\begin{Highlighting}[]
\NormalTok{(}\BuiltInTok{def}\FunctionTok{ DS }\NormalTok{(api/rename-columns DS \{}\StringTok{"v2"} \AttributeTok{:V2}\NormalTok{\})) }\CommentTok{;; revert back}
\end{Highlighting}
\end{Shaded}

\begin{center}\rule{0.5\linewidth}{0.5pt}\end{center}

Reorder columns

\begin{Shaded}
\begin{Highlighting}[]
\NormalTok{(}\BuiltInTok{def}\FunctionTok{ DS }\NormalTok{(api/reorder-columns DS }\AttributeTok{:V4} \AttributeTok{:V1} \AttributeTok{:V2}\NormalTok{))}
\end{Highlighting}
\end{Shaded}

\begin{Shaded}
\begin{Highlighting}[]
\NormalTok{DS}
\end{Highlighting}
\end{Shaded}

\_unnamed {[}9 3{]}:

\begin{longtable}[]{@{}lll@{}}
\toprule
:V4 & :V1 & :V2\tabularnewline
\midrule
\endhead
A & 0 & 3\tabularnewline
A & 0 & 4\tabularnewline
A & 0 & 7\tabularnewline
B & 4 & 0\tabularnewline
B & 4 & 8\tabularnewline
B & 1 & 5\tabularnewline
C & 4 & 5\tabularnewline
C & 1 & 0\tabularnewline
C & 1 & 9\tabularnewline
\bottomrule
\end{longtable}

\hypertarget{advanced-use-of-by}{%
\subparagraph{Advanced use of by}\label{advanced-use-of-by}}

Select first/last/\ldots{} row by group

\begin{Shaded}
\begin{Highlighting}[]
\NormalTok{(}\KeywordTok{->}\NormalTok{ DS}
\NormalTok{    (api/group-by }\AttributeTok{:V4}\NormalTok{)}
\NormalTok{    (api/first)}
\NormalTok{    (api/ungroup))}
\end{Highlighting}
\end{Shaded}

\_unnamed {[}3 3{]}:

\begin{longtable}[]{@{}lll@{}}
\toprule
:V4 & :V1 & :V2\tabularnewline
\midrule
\endhead
A & 0 & 3\tabularnewline
B & 4 & 0\tabularnewline
C & 4 & 5\tabularnewline
\bottomrule
\end{longtable}

\begin{Shaded}
\begin{Highlighting}[]
\NormalTok{(}\KeywordTok{->}\NormalTok{ DS}
\NormalTok{    (api/group-by }\AttributeTok{:V4}\NormalTok{)}
\NormalTok{    (api/select-rows [}\DecValTok{0} \DecValTok{2}\NormalTok{])}
\NormalTok{    (api/ungroup))}
\end{Highlighting}
\end{Shaded}

\_unnamed {[}6 3{]}:

\begin{longtable}[]{@{}lll@{}}
\toprule
:V4 & :V1 & :V2\tabularnewline
\midrule
\endhead
A & 0 & 3\tabularnewline
A & 0 & 7\tabularnewline
B & 4 & 0\tabularnewline
B & 1 & 5\tabularnewline
C & 4 & 5\tabularnewline
C & 1 & 9\tabularnewline
\bottomrule
\end{longtable}

\begin{Shaded}
\begin{Highlighting}[]
\NormalTok{(}\KeywordTok{->}\NormalTok{ DS}
\NormalTok{    (api/group-by }\AttributeTok{:V4}\NormalTok{)}
\NormalTok{    (api/tail }\DecValTok{2}\NormalTok{)}
\NormalTok{    (api/ungroup))}
\end{Highlighting}
\end{Shaded}

\_unnamed {[}6 3{]}:

\begin{longtable}[]{@{}lll@{}}
\toprule
:V4 & :V1 & :V2\tabularnewline
\midrule
\endhead
A & 0 & 4\tabularnewline
A & 0 & 7\tabularnewline
B & 4 & 8\tabularnewline
B & 1 & 5\tabularnewline
C & 1 & 0\tabularnewline
C & 1 & 9\tabularnewline
\bottomrule
\end{longtable}

\begin{center}\rule{0.5\linewidth}{0.5pt}\end{center}

Select rows using a nested query

\begin{Shaded}
\begin{Highlighting}[]
\NormalTok{(}\KeywordTok{->}\NormalTok{ DS}
\NormalTok{    (api/group-by }\AttributeTok{:V4}\NormalTok{)}
\NormalTok{    (api/order-by }\AttributeTok{:V2}\NormalTok{)}
\NormalTok{    (api/first)}
\NormalTok{    (api/ungroup))}
\end{Highlighting}
\end{Shaded}

\_unnamed {[}3 3{]}:

\begin{longtable}[]{@{}lll@{}}
\toprule
:V4 & :V1 & :V2\tabularnewline
\midrule
\endhead
A & 0 & 3\tabularnewline
B & 4 & 0\tabularnewline
C & 1 & 0\tabularnewline
\bottomrule
\end{longtable}

Add a group counter column

\begin{Shaded}
\begin{Highlighting}[]
\NormalTok{(}\KeywordTok{->}\NormalTok{ DS}
\NormalTok{    (api/group-by [}\AttributeTok{:V4} \AttributeTok{:V1}\NormalTok{])}
\NormalTok{    (api/ungroup \{}\AttributeTok{:add-group-id-as-column} \AttributeTok{:Grp}\NormalTok{\}))}
\end{Highlighting}
\end{Shaded}

\_unnamed {[}9 4{]}:

\begin{longtable}[]{@{}llll@{}}
\toprule
:Grp & :V4 & :V1 & :V2\tabularnewline
\midrule
\endhead
0 & A & 0 & 3\tabularnewline
0 & A & 0 & 4\tabularnewline
0 & A & 0 & 7\tabularnewline
1 & B & 1 & 5\tabularnewline
2 & C & 1 & 0\tabularnewline
2 & C & 1 & 9\tabularnewline
3 & B & 4 & 0\tabularnewline
3 & B & 4 & 8\tabularnewline
4 & C & 4 & 5\tabularnewline
\bottomrule
\end{longtable}

\begin{center}\rule{0.5\linewidth}{0.5pt}\end{center}

Get row number of first (and last) observation by group

\begin{Shaded}
\begin{Highlighting}[]
\NormalTok{(}\KeywordTok{->}\NormalTok{ DS}
\NormalTok{    (api/add-or-replace-column }\AttributeTok{:row-id}\NormalTok{ (}\KeywordTok{range}\NormalTok{))}
\NormalTok{    (api/select-columns [}\AttributeTok{:V4} \AttributeTok{:row-id}\NormalTok{])}
\NormalTok{    (api/group-by }\AttributeTok{:V4}\NormalTok{)}
\NormalTok{    (api/ungroup))}
\end{Highlighting}
\end{Shaded}

\_unnamed {[}9 2{]}:

\begin{longtable}[]{@{}ll@{}}
\toprule
:V4 & :row-id\tabularnewline
\midrule
\endhead
A & 0\tabularnewline
A & 1\tabularnewline
A & 2\tabularnewline
B & 3\tabularnewline
B & 4\tabularnewline
B & 5\tabularnewline
C & 6\tabularnewline
C & 7\tabularnewline
C & 8\tabularnewline
\bottomrule
\end{longtable}

\begin{Shaded}
\begin{Highlighting}[]
\NormalTok{(}\KeywordTok{->}\NormalTok{ DS}
\NormalTok{    (api/add-or-replace-column }\AttributeTok{:row-id}\NormalTok{ (}\KeywordTok{range}\NormalTok{))}
\NormalTok{    (api/select-columns [}\AttributeTok{:V4} \AttributeTok{:row-id}\NormalTok{])}
\NormalTok{    (api/group-by }\AttributeTok{:V4}\NormalTok{)}
\NormalTok{    (api/first)}
\NormalTok{    (api/ungroup))}
\end{Highlighting}
\end{Shaded}

\_unnamed {[}3 2{]}:

\begin{longtable}[]{@{}ll@{}}
\toprule
:V4 & :row-id\tabularnewline
\midrule
\endhead
A & 0\tabularnewline
B & 3\tabularnewline
C & 6\tabularnewline
\bottomrule
\end{longtable}

\begin{Shaded}
\begin{Highlighting}[]
\NormalTok{(}\KeywordTok{->}\NormalTok{ DS}
\NormalTok{    (api/add-or-replace-column }\AttributeTok{:row-id}\NormalTok{ (}\KeywordTok{range}\NormalTok{))}
\NormalTok{    (api/select-columns [}\AttributeTok{:V4} \AttributeTok{:row-id}\NormalTok{])}
\NormalTok{    (api/group-by }\AttributeTok{:V4}\NormalTok{)}
\NormalTok{    (api/select-rows [}\DecValTok{0} \DecValTok{2}\NormalTok{])}
\NormalTok{    (api/ungroup))}
\end{Highlighting}
\end{Shaded}

\_unnamed {[}6 2{]}:

\begin{longtable}[]{@{}ll@{}}
\toprule
:V4 & :row-id\tabularnewline
\midrule
\endhead
A & 0\tabularnewline
A & 2\tabularnewline
B & 3\tabularnewline
B & 5\tabularnewline
C & 6\tabularnewline
C & 8\tabularnewline
\bottomrule
\end{longtable}

\begin{center}\rule{0.5\linewidth}{0.5pt}\end{center}

Handle list-columns by group

\begin{Shaded}
\begin{Highlighting}[]
\NormalTok{(}\KeywordTok{->}\NormalTok{ DS}
\NormalTok{    (api/select-columns [}\AttributeTok{:V1} \AttributeTok{:V4}\NormalTok{])}
\NormalTok{    (api/fold-by }\AttributeTok{:V4}\NormalTok{))}
\end{Highlighting}
\end{Shaded}

\_unnamed {[}3 2{]}:

\begin{longtable}[]{@{}ll@{}}
\toprule
:V4 & :V1\tabularnewline
\midrule
\endhead
B & {[}4 4 1{]}\tabularnewline
C & {[}4 1 1{]}\tabularnewline
A & {[}0 0 0{]}\tabularnewline
\bottomrule
\end{longtable}

\begin{Shaded}
\begin{Highlighting}[]
\NormalTok{(}\KeywordTok{->}\NormalTok{ DS    }
\NormalTok{    (api/group-by }\AttributeTok{:V4}\NormalTok{)}
\NormalTok{    (api/unmark-group))}
\end{Highlighting}
\end{Shaded}

\_unnamed {[}3 3{]}:

\begin{longtable}[]{@{}lll@{}}
\toprule
:name & :group-id & :data\tabularnewline
\midrule
\endhead
A & 0 & Group: A {[}3 3{]}:\tabularnewline
B & 1 & Group: B {[}3 3{]}:\tabularnewline
C & 2 & Group: C {[}3 3{]}:\tabularnewline
\bottomrule
\end{longtable}

\begin{center}\rule{0.5\linewidth}{0.5pt}\end{center}

Grouping sets (multiple by at once)

Not available.

\hypertarget{miscellaneous}{%
\paragraph{Miscellaneous}\label{miscellaneous}}

\hypertarget{read-write-data}{%
\subparagraph{Read / Write data}\label{read-write-data}}

Write data to a csv file

\begin{Shaded}
\begin{Highlighting}[]
\NormalTok{(api/write-csv! DS }\StringTok{"DF.csv"}\NormalTok{)}
\end{Highlighting}
\end{Shaded}

\begin{verbatim}
nil
\end{verbatim}

\begin{center}\rule{0.5\linewidth}{0.5pt}\end{center}

Write data to a tab-delimited file

\begin{Shaded}
\begin{Highlighting}[]
\NormalTok{(api/write-csv! DS }\StringTok{"DF.txt"}\NormalTok{ \{}\AttributeTok{:separator} \CharTok{\textbackslash{}tab}\NormalTok{\})}
\end{Highlighting}
\end{Shaded}

\begin{verbatim}
nil
\end{verbatim}

or

\begin{Shaded}
\begin{Highlighting}[]
\NormalTok{(api/write-csv! DS }\StringTok{"DF.tsv"}\NormalTok{)}
\end{Highlighting}
\end{Shaded}

\begin{verbatim}
nil
\end{verbatim}

\begin{center}\rule{0.5\linewidth}{0.5pt}\end{center}

Read a csv / tab-delimited file

\begin{Shaded}
\begin{Highlighting}[]
\NormalTok{(api/dataset }\StringTok{"DF.csv"}\NormalTok{ \{}\AttributeTok{:key-fn} \KeywordTok{keyword}\NormalTok{\})}
\end{Highlighting}
\end{Shaded}

DF.csv {[}9 3{]}:

\begin{longtable}[]{@{}lll@{}}
\toprule
:V4 & :V1 & :V2\tabularnewline
\midrule
\endhead
A & 0 & 3\tabularnewline
A & 0 & 4\tabularnewline
A & 0 & 7\tabularnewline
B & 4 & 0\tabularnewline
B & 4 & 8\tabularnewline
B & 1 & 5\tabularnewline
C & 4 & 5\tabularnewline
C & 1 & 0\tabularnewline
C & 1 & 9\tabularnewline
\bottomrule
\end{longtable}

\begin{Shaded}
\begin{Highlighting}[]
\NormalTok{(api/dataset }\StringTok{"DF.txt"}\NormalTok{ \{}\AttributeTok{:key-fn} \KeywordTok{keyword}\NormalTok{\})}
\end{Highlighting}
\end{Shaded}

DF.txt {[}9 3{]}:

\begin{longtable}[]{@{}lll@{}}
\toprule
:V4 & :V1 & :V2\tabularnewline
\midrule
\endhead
A & 0 & 3\tabularnewline
A & 0 & 4\tabularnewline
A & 0 & 7\tabularnewline
B & 4 & 0\tabularnewline
B & 4 & 8\tabularnewline
B & 1 & 5\tabularnewline
C & 4 & 5\tabularnewline
C & 1 & 0\tabularnewline
C & 1 & 9\tabularnewline
\bottomrule
\end{longtable}

\begin{Shaded}
\begin{Highlighting}[]
\NormalTok{(api/dataset }\StringTok{"DF.tsv"}\NormalTok{ \{}\AttributeTok{:key-fn} \KeywordTok{keyword}\NormalTok{\})}
\end{Highlighting}
\end{Shaded}

DF.tsv {[}9 3{]}:

\begin{longtable}[]{@{}lll@{}}
\toprule
:V4 & :V1 & :V2\tabularnewline
\midrule
\endhead
A & 0 & 3\tabularnewline
A & 0 & 4\tabularnewline
A & 0 & 7\tabularnewline
B & 4 & 0\tabularnewline
B & 4 & 8\tabularnewline
B & 1 & 5\tabularnewline
C & 4 & 5\tabularnewline
C & 1 & 0\tabularnewline
C & 1 & 9\tabularnewline
\bottomrule
\end{longtable}

\begin{center}\rule{0.5\linewidth}{0.5pt}\end{center}

Read a csv file selecting / droping columns

\begin{Shaded}
\begin{Highlighting}[]
\NormalTok{(api/dataset }\StringTok{"DF.csv"}\NormalTok{ \{}\AttributeTok{:key-fn} \KeywordTok{keyword}
                       \AttributeTok{:column-whitelist}\NormalTok{ [}\StringTok{"V1"} \StringTok{"V4"}\NormalTok{]\})}
\end{Highlighting}
\end{Shaded}

DF.csv {[}9 2{]}:

\begin{longtable}[]{@{}ll@{}}
\toprule
:V1 & :V4\tabularnewline
\midrule
\endhead
0 & A\tabularnewline
0 & A\tabularnewline
0 & A\tabularnewline
4 & B\tabularnewline
4 & B\tabularnewline
1 & B\tabularnewline
4 & C\tabularnewline
1 & C\tabularnewline
1 & C\tabularnewline
\bottomrule
\end{longtable}

\begin{Shaded}
\begin{Highlighting}[]
\NormalTok{(api/dataset }\StringTok{"DF.csv"}\NormalTok{ \{}\AttributeTok{:key-fn} \KeywordTok{keyword}
                       \AttributeTok{:column-blacklist}\NormalTok{ [}\StringTok{"V4"}\NormalTok{]\})}
\end{Highlighting}
\end{Shaded}

DF.csv {[}9 2{]}:

\begin{longtable}[]{@{}ll@{}}
\toprule
:V1 & :V2\tabularnewline
\midrule
\endhead
0 & 3\tabularnewline
0 & 4\tabularnewline
0 & 7\tabularnewline
4 & 0\tabularnewline
4 & 8\tabularnewline
1 & 5\tabularnewline
4 & 5\tabularnewline
1 & 0\tabularnewline
1 & 9\tabularnewline
\bottomrule
\end{longtable}

\begin{center}\rule{0.5\linewidth}{0.5pt}\end{center}

Read and rbind several files

\begin{Shaded}
\begin{Highlighting}[]
\NormalTok{(}\KeywordTok{apply}\NormalTok{ api/concat (}\KeywordTok{map}\NormalTok{ api/dataset [}\StringTok{"DF.csv"} \StringTok{"DF.csv"}\NormalTok{]))}
\end{Highlighting}
\end{Shaded}

null {[}18 3{]}:

\begin{longtable}[]{@{}lll@{}}
\toprule
V4 & V1 & V2\tabularnewline
\midrule
\endhead
A & 0 & 3\tabularnewline
A & 0 & 4\tabularnewline
A & 0 & 7\tabularnewline
B & 4 & 0\tabularnewline
B & 4 & 8\tabularnewline
B & 1 & 5\tabularnewline
C & 4 & 5\tabularnewline
C & 1 & 0\tabularnewline
C & 1 & 9\tabularnewline
A & 0 & 3\tabularnewline
A & 0 & 4\tabularnewline
A & 0 & 7\tabularnewline
B & 4 & 0\tabularnewline
B & 4 & 8\tabularnewline
B & 1 & 5\tabularnewline
C & 4 & 5\tabularnewline
C & 1 & 0\tabularnewline
C & 1 & 9\tabularnewline
\bottomrule
\end{longtable}

\hypertarget{reshape-data}{%
\subparagraph{Reshape data}\label{reshape-data}}

Melt data (from wide to long)

\begin{Shaded}
\begin{Highlighting}[]
\NormalTok{(}\BuiltInTok{def}\FunctionTok{ mDS }\NormalTok{(api/pivot->longer DS [}\AttributeTok{:V1} \AttributeTok{:V2}\NormalTok{] \{}\AttributeTok{:target-columns} \AttributeTok{:variable}
                                          \AttributeTok{:value-column-name} \AttributeTok{:value}\NormalTok{\}))}
\end{Highlighting}
\end{Shaded}

\begin{Shaded}
\begin{Highlighting}[]
\NormalTok{mDS}
\end{Highlighting}
\end{Shaded}

\_unnamed {[}18 3{]}:

\begin{longtable}[]{@{}lll@{}}
\toprule
:V4 & :variable & :value\tabularnewline
\midrule
\endhead
A & :V1 & 0\tabularnewline
A & :V1 & 0\tabularnewline
A & :V1 & 0\tabularnewline
B & :V1 & 4\tabularnewline
B & :V1 & 4\tabularnewline
B & :V1 & 1\tabularnewline
C & :V1 & 4\tabularnewline
C & :V1 & 1\tabularnewline
C & :V1 & 1\tabularnewline
A & :V2 & 3\tabularnewline
A & :V2 & 4\tabularnewline
A & :V2 & 7\tabularnewline
B & :V2 & 0\tabularnewline
B & :V2 & 8\tabularnewline
B & :V2 & 5\tabularnewline
C & :V2 & 5\tabularnewline
C & :V2 & 0\tabularnewline
C & :V2 & 9\tabularnewline
\bottomrule
\end{longtable}

\begin{center}\rule{0.5\linewidth}{0.5pt}\end{center}

Cast data (from long to wide)

\begin{Shaded}
\begin{Highlighting}[]
\NormalTok{(}\KeywordTok{->}\NormalTok{ mDS}
\NormalTok{    (api/pivot->wider }\AttributeTok{:variable} \AttributeTok{:value}\NormalTok{ \{}\AttributeTok{:fold-fn} \KeywordTok{vec}\NormalTok{\})}
\NormalTok{    (api/update-columns [}\AttributeTok{:V1} \AttributeTok{:V2}\NormalTok{] (}\KeywordTok{partial} \KeywordTok{map} \KeywordTok{count}\NormalTok{)))}
\end{Highlighting}
\end{Shaded}

\_unnamed {[}3 3{]}:

\begin{longtable}[]{@{}lll@{}}
\toprule
:V4 & V1 & V2\tabularnewline
\midrule
\endhead
B & {[}4 4 1{]} & {[}0 8 5{]}\tabularnewline
A & {[}0 0 0{]} & {[}3 4 7{]}\tabularnewline
C & {[}4 1 1{]} & {[}5 0 9{]}\tabularnewline
\bottomrule
\end{longtable}

\begin{Shaded}
\begin{Highlighting}[]
\NormalTok{(}\KeywordTok{->}\NormalTok{ mDS}
\NormalTok{    (api/pivot->wider }\AttributeTok{:variable} \AttributeTok{:value}\NormalTok{ \{}\AttributeTok{:fold-fn} \KeywordTok{vec}\NormalTok{\})}
\NormalTok{    (api/update-columns [}\AttributeTok{:V1} \AttributeTok{:V2}\NormalTok{] (}\KeywordTok{partial} \KeywordTok{map}\NormalTok{ dfn/sum)))}
\end{Highlighting}
\end{Shaded}

\_unnamed {[}3 3{]}:

\begin{longtable}[]{@{}lll@{}}
\toprule
:V4 & V1 & V2\tabularnewline
\midrule
\endhead
B & {[}4 4 1{]} & {[}0 8 5{]}\tabularnewline
A & {[}0 0 0{]} & {[}3 4 7{]}\tabularnewline
C & {[}4 1 1{]} & {[}5 0 9{]}\tabularnewline
\bottomrule
\end{longtable}

\begin{Shaded}
\begin{Highlighting}[]
\NormalTok{(}\KeywordTok{->}\NormalTok{ mDS}
\NormalTok{    (api/map-columns }\AttributeTok{:value}\NormalTok{ #(}\KeywordTok{>} \VariableTok\NormalTok{) (}\KeywordTok{count} \VariableTok{%}\NormalTok{) }\DecValTok{1}\NormalTok{))))}
\end{Highlighting}
\end{Shaded}

\_unnamed {[}3 3{]}:

\begin{longtable}[]{@{}lll@{}}
\toprule
:V4 & true & false\tabularnewline
\midrule
\endhead
C & :V2 & {[}:V1 :V1 :V1 :V2 :V2{]}\tabularnewline
A & :V2 & {[}:V1 :V1 :V1 :V2 :V2{]}\tabularnewline
B & :V2 & {[}:V1 :V1 :V1 :V2 :V2{]}\tabularnewline
\bottomrule
\end{longtable}

\begin{center}\rule{0.5\linewidth}{0.5pt}\end{center}

Split

\begin{Shaded}
\begin{Highlighting}[]
\NormalTok{(api/group-by DS }\AttributeTok{:V4}\NormalTok{ \{}\AttributeTok{:result-type} \AttributeTok{:as-map}\NormalTok{\})}
\end{Highlighting}
\end{Shaded}

\{``A'' Group: A {[}3 3{]}:

\begin{longtable}[]{@{}lll@{}}
\toprule
:V4 & :V1 & :V2\tabularnewline
\midrule
\endhead
A & 0 & 3\tabularnewline
A & 0 & 4\tabularnewline
A & 0 & 7\tabularnewline
\bottomrule
\end{longtable}

, ``B'' Group: B {[}3 3{]}:

\begin{longtable}[]{@{}lll@{}}
\toprule
:V4 & :V1 & :V2\tabularnewline
\midrule
\endhead
B & 4 & 0\tabularnewline
B & 4 & 8\tabularnewline
B & 1 & 5\tabularnewline
\bottomrule
\end{longtable}

, ``C'' Group: C {[}3 3{]}:

\begin{longtable}[]{@{}lll@{}}
\toprule
:V4 & :V1 & :V2\tabularnewline
\midrule
\endhead
C & 4 & 5\tabularnewline
C & 1 & 0\tabularnewline
C & 1 & 9\tabularnewline
\bottomrule
\end{longtable}

\}

\begin{center}\rule{0.5\linewidth}{0.5pt}\end{center}

Split and transpose a vector/column

\begin{Shaded}
\begin{Highlighting}[]
\NormalTok{(}\KeywordTok{->}\NormalTok{ \{}\AttributeTok{:a}\NormalTok{ [}\StringTok{"A:a"} \StringTok{"B:b"} \StringTok{"C:c"}\NormalTok{]\}}
\NormalTok{    (api/dataset)}
\NormalTok{    (api/separate-column }\AttributeTok{:a}\NormalTok{ [}\AttributeTok{:V1} \AttributeTok{:V2}\NormalTok{] }\StringTok{":"}\NormalTok{))}
\end{Highlighting}
\end{Shaded}

\_unnamed {[}3 2{]}:

\begin{longtable}[]{@{}ll@{}}
\toprule
:V1 & :V2\tabularnewline
\midrule
\endhead
A & a\tabularnewline
B & b\tabularnewline
C & c\tabularnewline
\bottomrule
\end{longtable}

\hypertarget{other-1}{%
\subparagraph{Other}\label{other-1}}

Skipped

\hypertarget{joinbind-data-sets}{%
\paragraph{Join/Bind data sets}\label{joinbind-data-sets}}

\begin{Shaded}
\begin{Highlighting}[]
\NormalTok{(}\BuiltInTok{def}\FunctionTok{ x }\NormalTok{(api/dataset \{}\StringTok{"Id"}\NormalTok{ [}\StringTok{"A"} \StringTok{"B"} \StringTok{"C"} \StringTok{"C"}\NormalTok{]}
                     \StringTok{"X1"}\NormalTok{ [}\DecValTok{1} \DecValTok{3} \DecValTok{5} \DecValTok{7}\NormalTok{]}
                     \StringTok{"XY"}\NormalTok{ [}\StringTok{"x2"} \StringTok{"x4"} \StringTok{"x6"} \StringTok{"x8"}\NormalTok{]\}))}
\NormalTok{(}\BuiltInTok{def}\FunctionTok{ y }\NormalTok{(api/dataset \{}\StringTok{"Id"}\NormalTok{ [}\StringTok{"A"} \StringTok{"B"} \StringTok{"B"} \StringTok{"D"}\NormalTok{]}
                     \StringTok{"Y1"}\NormalTok{ [}\DecValTok{1} \DecValTok{3} \DecValTok{5} \DecValTok{7}\NormalTok{]}
                     \StringTok{"XY"}\NormalTok{ [}\StringTok{"y1"} \StringTok{"y3"} \StringTok{"y5"} \StringTok{"y7"}\NormalTok{]\}))}
\end{Highlighting}
\end{Shaded}

\begin{Shaded}
\begin{Highlighting}[]
\NormalTok{x y}
\end{Highlighting}
\end{Shaded}

\_unnamed {[}4 3{]}:

\begin{longtable}[]{@{}lll@{}}
\toprule
Id & X1 & XY\tabularnewline
\midrule
\endhead
A & 1 & x2\tabularnewline
B & 3 & x4\tabularnewline
C & 5 & x6\tabularnewline
C & 7 & x8\tabularnewline
\bottomrule
\end{longtable}

\_unnamed {[}4 3{]}:

\begin{longtable}[]{@{}lll@{}}
\toprule
Id & Y1 & XY\tabularnewline
\midrule
\endhead
A & 1 & y1\tabularnewline
B & 3 & y3\tabularnewline
B & 5 & y5\tabularnewline
D & 7 & y7\tabularnewline
\bottomrule
\end{longtable}

\hypertarget{join-1}{%
\subparagraph{Join}\label{join-1}}

Join matching rows from y to x

\begin{Shaded}
\begin{Highlighting}[]
\NormalTok{(api/left-join x y }\StringTok{"Id"}\NormalTok{)}
\end{Highlighting}
\end{Shaded}

left-outer-join {[}5 6{]}:

\begin{longtable}[]{@{}llllll@{}}
\toprule
Id & X1 & XY & right.Id & Y1 & right.XY\tabularnewline
\midrule
\endhead
A & 1 & x2 & A & 1 & y1\tabularnewline
B & 3 & x4 & B & 3 & y3\tabularnewline
B & 3 & x4 & B & 5 & y5\tabularnewline
C & 5 & x6 & & &\tabularnewline
C & 7 & x8 & & &\tabularnewline
\bottomrule
\end{longtable}

\begin{center}\rule{0.5\linewidth}{0.5pt}\end{center}

Join matching rows from x to y

\begin{Shaded}
\begin{Highlighting}[]
\NormalTok{(api/right-join x y }\StringTok{"Id"}\NormalTok{)}
\end{Highlighting}
\end{Shaded}

right-outer-join {[}4 6{]}:

\begin{longtable}[]{@{}llllll@{}}
\toprule
Id & X1 & XY & right.Id & Y1 & right.XY\tabularnewline
\midrule
\endhead
A & 1 & x2 & A & 1 & y1\tabularnewline
B & 3 & x4 & B & 3 & y3\tabularnewline
B & 3 & x4 & B & 5 & y5\tabularnewline
& & & D & 7 & y7\tabularnewline
\bottomrule
\end{longtable}

\begin{center}\rule{0.5\linewidth}{0.5pt}\end{center}

Join matching rows from both x and y

\begin{Shaded}
\begin{Highlighting}[]
\NormalTok{(api/inner-join x y }\StringTok{"Id"}\NormalTok{)}
\end{Highlighting}
\end{Shaded}

inner-join {[}3 5{]}:

\begin{longtable}[]{@{}lllll@{}}
\toprule
Id & X1 & XY & Y1 & right.XY\tabularnewline
\midrule
\endhead
A & 1 & x2 & 1 & y1\tabularnewline
B & 3 & x4 & 3 & y3\tabularnewline
B & 3 & x4 & 5 & y5\tabularnewline
\bottomrule
\end{longtable}

\begin{center}\rule{0.5\linewidth}{0.5pt}\end{center}

Join keeping all the rows

\begin{Shaded}
\begin{Highlighting}[]
\NormalTok{(api/full-join x y }\StringTok{"Id"}\NormalTok{)}
\end{Highlighting}
\end{Shaded}

full-join {[}6 6{]}:

\begin{longtable}[]{@{}llllll@{}}
\toprule
Id & X1 & XY & right.Id & Y1 & right.XY\tabularnewline
\midrule
\endhead
A & 1 & x2 & A & 1 & y1\tabularnewline
B & 3 & x4 & B & 3 & y3\tabularnewline
B & 3 & x4 & B & 5 & y5\tabularnewline
C & 5 & x6 & & &\tabularnewline
C & 7 & x8 & & &\tabularnewline
& & & D & 7 & y7\tabularnewline
\bottomrule
\end{longtable}

\begin{center}\rule{0.5\linewidth}{0.5pt}\end{center}

Return rows from x matching y

\begin{Shaded}
\begin{Highlighting}[]
\NormalTok{(api/semi-join x y }\StringTok{"Id"}\NormalTok{)}
\end{Highlighting}
\end{Shaded}

semi-join {[}2 3{]}:

\begin{longtable}[]{@{}lll@{}}
\toprule
Id & X1 & XY\tabularnewline
\midrule
\endhead
A & 1 & x2\tabularnewline
B & 3 & x4\tabularnewline
\bottomrule
\end{longtable}

\begin{center}\rule{0.5\linewidth}{0.5pt}\end{center}

Return rows from x not matching y

\begin{Shaded}
\begin{Highlighting}[]
\NormalTok{(api/anti-join x y }\StringTok{"Id"}\NormalTok{)}
\end{Highlighting}
\end{Shaded}

anti-join {[}2 3{]}:

\begin{longtable}[]{@{}lll@{}}
\toprule
Id & X1 & XY\tabularnewline
\midrule
\endhead
C & 5 & x6\tabularnewline
C & 7 & x8\tabularnewline
\bottomrule
\end{longtable}

\hypertarget{more-joins}{%
\subparagraph{More joins}\label{more-joins}}

Select columns while joining

\begin{Shaded}
\begin{Highlighting}[]
\NormalTok{(api/right-join (api/select-columns x [}\StringTok{"Id"} \StringTok{"X1"}\NormalTok{])}
\NormalTok{                (api/select-columns y [}\StringTok{"Id"} \StringTok{"XY"}\NormalTok{])}
                \StringTok{"Id"}\NormalTok{)}
\end{Highlighting}
\end{Shaded}

right-outer-join {[}4 4{]}:

\begin{longtable}[]{@{}llll@{}}
\toprule
Id & X1 & right.Id & XY\tabularnewline
\midrule
\endhead
A & 1 & A & y1\tabularnewline
B & 3 & B & y3\tabularnewline
B & 3 & B & y5\tabularnewline
& & D & y7\tabularnewline
\bottomrule
\end{longtable}

\begin{Shaded}
\begin{Highlighting}[]
\NormalTok{(api/right-join (api/select-columns x [}\StringTok{"Id"} \StringTok{"XY"}\NormalTok{])}
\NormalTok{                (api/select-columns y [}\StringTok{"Id"} \StringTok{"XY"}\NormalTok{])}
                \StringTok{"Id"}\NormalTok{)}
\end{Highlighting}
\end{Shaded}

right-outer-join {[}4 4{]}:

\begin{longtable}[]{@{}llll@{}}
\toprule
Id & XY & right.Id & right.XY\tabularnewline
\midrule
\endhead
A & x2 & A & y1\tabularnewline
B & x4 & B & y3\tabularnewline
B & x4 & B & y5\tabularnewline
& & D & y7\tabularnewline
\bottomrule
\end{longtable}

Aggregate columns while joining

\begin{Shaded}
\begin{Highlighting}[]
\NormalTok{(}\KeywordTok{->}\NormalTok{ y}
\NormalTok{    (api/group-by [}\StringTok{"Id"}\NormalTok{])}
\NormalTok{    (api/aggregate \{}\StringTok{"sumY1"}\NormalTok{ #(dfn/sum (}\VariableTok{%} \StringTok{"Y1"}\NormalTok{))\})}
\NormalTok{    (api/right-join x }\StringTok{"Id"}\NormalTok{)}
\NormalTok{    (api/add-or-replace-column }\StringTok{"X1Y1"}\NormalTok{ (}\KeywordTok{fn}\NormalTok{ [ds] (dfn/* (ds }\StringTok{"sumY1"}\NormalTok{)}
\NormalTok{                                                    (ds }\StringTok{"X1"}\NormalTok{))))}
\NormalTok{    (api/select-columns [}\StringTok{"right.Id"} \StringTok{"X1Y1"}\NormalTok{]))}
\end{Highlighting}
\end{Shaded}

right-outer-join {[}4 2{]}:

\begin{longtable}[]{@{}ll@{}}
\toprule
right.Id & X1Y1\tabularnewline
\midrule
\endhead
A & 1.0\tabularnewline
B & 24.0\tabularnewline
C & NaN\tabularnewline
C & NaN\tabularnewline
\bottomrule
\end{longtable}

Update columns while joining

\begin{Shaded}
\begin{Highlighting}[]
\NormalTok{(}\KeywordTok{->}\NormalTok{ x}
\NormalTok{    (api/select-columns [}\StringTok{"Id"} \StringTok{"X1"}\NormalTok{])}
\NormalTok{    (api/map-columns }\StringTok{"SqX1"} \StringTok{"X1"}\NormalTok{ (}\KeywordTok{fn}\NormalTok{ [x] (}\KeywordTok{*}\NormalTok{ x x)))}
\NormalTok{    (api/right-join y }\StringTok{"Id"}\NormalTok{)}
\NormalTok{    (api/drop-columns [}\StringTok{"X1"} \StringTok{"Id"}\NormalTok{]))}
\end{Highlighting}
\end{Shaded}

right-outer-join {[}4 4{]}:

\begin{longtable}[]{@{}llll@{}}
\toprule
SqX1 & right.Id & Y1 & XY\tabularnewline
\midrule
\endhead
1 & A & 1 & y1\tabularnewline
9 & B & 3 & y3\tabularnewline
9 & B & 5 & y5\tabularnewline
& D & 7 & y7\tabularnewline
\bottomrule
\end{longtable}

\begin{center}\rule{0.5\linewidth}{0.5pt}\end{center}

Adds a list column with rows from y matching x (nest-join)

\begin{Shaded}
\begin{Highlighting}[]
\NormalTok{(}\KeywordTok{->}\NormalTok{ (api/left-join x y }\StringTok{"Id"}\NormalTok{)}
\NormalTok{    (api/drop-columns [}\StringTok{"right.Id"}\NormalTok{])}
\NormalTok{    (api/fold-by (api/column-names x)))}
\end{Highlighting}
\end{Shaded}

\_unnamed {[}4 5{]}:

\begin{longtable}[]{@{}lllll@{}}
\toprule
XY & X1 & Id & Y1 & right.XY\tabularnewline
\midrule
\endhead
x4 & 3 & B & {[}3 5{]} & {[}``y3'' ``y5''{]}\tabularnewline
x6 & 5 & C & {[}{]} & {[}{]}\tabularnewline
x8 & 7 & C & {[}{]} & {[}{]}\tabularnewline
x2 & 1 & A & {[}1{]} & {[}``y1''{]}\tabularnewline
\bottomrule
\end{longtable}

\begin{center}\rule{0.5\linewidth}{0.5pt}\end{center}

Some joins are skipped

\begin{center}\rule{0.5\linewidth}{0.5pt}\end{center}

Cross join

\begin{Shaded}
\begin{Highlighting}[]
\NormalTok{(}\BuiltInTok{def}\FunctionTok{ cjds }\NormalTok{(api/dataset \{}\AttributeTok{:V1}\NormalTok{ [[}\DecValTok{2} \DecValTok{1} \DecValTok{1}\NormalTok{]]}
                        \AttributeTok{:V2}\NormalTok{ [[}\DecValTok{3} \DecValTok{2}\NormalTok{]]\}))}
\end{Highlighting}
\end{Shaded}

\begin{Shaded}
\begin{Highlighting}[]
\NormalTok{cjds}
\end{Highlighting}
\end{Shaded}

\_unnamed {[}1 2{]}:

\begin{longtable}[]{@{}ll@{}}
\toprule
:V1 & :V2\tabularnewline
\midrule
\endhead
{[}2 1 1{]} & {[}3 2{]}\tabularnewline
\bottomrule
\end{longtable}

\begin{Shaded}
\begin{Highlighting}[]
\NormalTok{(}\KeywordTok{reduce}\NormalTok{ #(api/unroll }\VariableTok{%1} \VariableTok{%2}\NormalTok{) cjds (api/column-names cjds))}
\end{Highlighting}
\end{Shaded}

\_unnamed {[}6 2{]}:

\begin{longtable}[]{@{}ll@{}}
\toprule
:V1 & :V2\tabularnewline
\midrule
\endhead
2 & 3\tabularnewline
2 & 2\tabularnewline
1 & 3\tabularnewline
1 & 2\tabularnewline
1 & 3\tabularnewline
1 & 2\tabularnewline
\bottomrule
\end{longtable}

\begin{Shaded}
\begin{Highlighting}[]
\NormalTok{(}\KeywordTok{->}\NormalTok{ (}\KeywordTok{reduce}\NormalTok{ #(api/unroll }\VariableTok{%1} \VariableTok{%2}\NormalTok{) cjds (api/column-names cjds))}
\NormalTok{    (api/unique-by))}
\end{Highlighting}
\end{Shaded}

\_unnamed {[}4 2{]}:

\begin{longtable}[]{@{}ll@{}}
\toprule
:V1 & :V2\tabularnewline
\midrule
\endhead
2 & 3\tabularnewline
2 & 2\tabularnewline
1 & 3\tabularnewline
1 & 2\tabularnewline
\bottomrule
\end{longtable}

\hypertarget{bind-1}{%
\subparagraph{Bind}\label{bind-1}}

\begin{Shaded}
\begin{Highlighting}[]
\NormalTok{(}\BuiltInTok{def}\FunctionTok{ x }\NormalTok{(api/dataset \{}\AttributeTok{:V1}\NormalTok{ [}\DecValTok{1} \DecValTok{2} \DecValTok{3}\NormalTok{]\}))}
\NormalTok{(}\BuiltInTok{def}\FunctionTok{ y }\NormalTok{(api/dataset \{}\AttributeTok{:V1}\NormalTok{ [}\DecValTok{4} \DecValTok{5} \DecValTok{6}\NormalTok{]\}))}
\NormalTok{(}\BuiltInTok{def}\FunctionTok{ z }\NormalTok{(api/dataset \{}\AttributeTok{:V1}\NormalTok{ [}\DecValTok{7} \DecValTok{8} \DecValTok{9}\NormalTok{]}
                     \AttributeTok{:V2}\NormalTok{ [}\DecValTok{0} \DecValTok{0} \DecValTok{0}\NormalTok{]\}))}
\end{Highlighting}
\end{Shaded}

\begin{Shaded}
\begin{Highlighting}[]
\NormalTok{x y z}
\end{Highlighting}
\end{Shaded}

\_unnamed {[}3 1{]}:

\begin{longtable}[]{@{}l@{}}
\toprule
:V1\tabularnewline
\midrule
\endhead
1\tabularnewline
2\tabularnewline
3\tabularnewline
\bottomrule
\end{longtable}

\_unnamed {[}3 1{]}:

\begin{longtable}[]{@{}l@{}}
\toprule
:V1\tabularnewline
\midrule
\endhead
4\tabularnewline
5\tabularnewline
6\tabularnewline
\bottomrule
\end{longtable}

\_unnamed {[}3 2{]}:

\begin{longtable}[]{@{}ll@{}}
\toprule
:V1 & :V2\tabularnewline
\midrule
\endhead
7 & 0\tabularnewline
8 & 0\tabularnewline
9 & 0\tabularnewline
\bottomrule
\end{longtable}

\begin{center}\rule{0.5\linewidth}{0.5pt}\end{center}

Bind rows

\begin{Shaded}
\begin{Highlighting}[]
\NormalTok{(api/bind x y)}
\end{Highlighting}
\end{Shaded}

null {[}6 1{]}:

\begin{longtable}[]{@{}l@{}}
\toprule
:V1\tabularnewline
\midrule
\endhead
1\tabularnewline
2\tabularnewline
3\tabularnewline
4\tabularnewline
5\tabularnewline
6\tabularnewline
\bottomrule
\end{longtable}

\begin{Shaded}
\begin{Highlighting}[]
\NormalTok{(api/bind x z)}
\end{Highlighting}
\end{Shaded}

null {[}6 2{]}:

\begin{longtable}[]{@{}ll@{}}
\toprule
:V1 & :V2\tabularnewline
\midrule
\endhead
1 &\tabularnewline
2 &\tabularnewline
3 &\tabularnewline
7 & 0\tabularnewline
8 & 0\tabularnewline
9 & 0\tabularnewline
\bottomrule
\end{longtable}

\begin{center}\rule{0.5\linewidth}{0.5pt}\end{center}

Bind rows using a list

\begin{Shaded}
\begin{Highlighting}[]
\NormalTok{(}\KeywordTok{->>}\NormalTok{ [x y]}
\NormalTok{     (map-indexed #(api/add-or-replace-column }\VariableTok{%2} \AttributeTok{:id}\NormalTok{ (}\KeywordTok{repeat} \VariableTok{%1}\NormalTok{)))}
\NormalTok{     (}\KeywordTok{apply}\NormalTok{ api/bind))}
\end{Highlighting}
\end{Shaded}

null {[}6 2{]}:

\begin{longtable}[]{@{}ll@{}}
\toprule
:V1 & :id\tabularnewline
\midrule
\endhead
1 & 0\tabularnewline
2 & 0\tabularnewline
3 & 0\tabularnewline
4 & 1\tabularnewline
5 & 1\tabularnewline
6 & 1\tabularnewline
\bottomrule
\end{longtable}

\begin{center}\rule{0.5\linewidth}{0.5pt}\end{center}

Bind columns

\begin{Shaded}
\begin{Highlighting}[]
\NormalTok{(api/append x y)}
\end{Highlighting}
\end{Shaded}

\_unnamed {[}3 2{]}:

\begin{longtable}[]{@{}ll@{}}
\toprule
:V1 & :V1\tabularnewline
\midrule
\endhead
1 & 4\tabularnewline
2 & 5\tabularnewline
3 & 6\tabularnewline
\bottomrule
\end{longtable}

\hypertarget{set-operations}{%
\subparagraph{Set operations}\label{set-operations}}

\begin{Shaded}
\begin{Highlighting}[]
\NormalTok{(}\BuiltInTok{def}\FunctionTok{ x }\NormalTok{(api/dataset \{}\AttributeTok{:V1}\NormalTok{ [}\DecValTok{1} \DecValTok{2} \DecValTok{2} \DecValTok{3} \DecValTok{3}\NormalTok{]\}))}
\NormalTok{(}\BuiltInTok{def}\FunctionTok{ y }\NormalTok{(api/dataset \{}\AttributeTok{:V1}\NormalTok{ [}\DecValTok{2} \DecValTok{2} \DecValTok{3} \DecValTok{4} \DecValTok{4}\NormalTok{]\}))}
\end{Highlighting}
\end{Shaded}

\begin{Shaded}
\begin{Highlighting}[]
\NormalTok{x y}
\end{Highlighting}
\end{Shaded}

\_unnamed {[}5 1{]}:

\begin{longtable}[]{@{}l@{}}
\toprule
:V1\tabularnewline
\midrule
\endhead
1\tabularnewline
2\tabularnewline
2\tabularnewline
3\tabularnewline
3\tabularnewline
\bottomrule
\end{longtable}

\_unnamed {[}5 1{]}:

\begin{longtable}[]{@{}l@{}}
\toprule
:V1\tabularnewline
\midrule
\endhead
2\tabularnewline
2\tabularnewline
3\tabularnewline
4\tabularnewline
4\tabularnewline
\bottomrule
\end{longtable}

\begin{center}\rule{0.5\linewidth}{0.5pt}\end{center}

Intersection

\begin{Shaded}
\begin{Highlighting}[]
\NormalTok{(api/intersect x y)}
\end{Highlighting}
\end{Shaded}

intersection {[}2 1{]}:

\begin{longtable}[]{@{}l@{}}
\toprule
:V1\tabularnewline
\midrule
\endhead
2\tabularnewline
3\tabularnewline
\bottomrule
\end{longtable}

\begin{center}\rule{0.5\linewidth}{0.5pt}\end{center}

Difference

\begin{Shaded}
\begin{Highlighting}[]
\NormalTok{(api/difference x y)}
\end{Highlighting}
\end{Shaded}

difference {[}1 1{]}:

\begin{longtable}[]{@{}l@{}}
\toprule
:V1\tabularnewline
\midrule
\endhead
1\tabularnewline
\bottomrule
\end{longtable}

\begin{center}\rule{0.5\linewidth}{0.5pt}\end{center}

Union

\begin{Shaded}
\begin{Highlighting}[]
\NormalTok{(api/union x y)}
\end{Highlighting}
\end{Shaded}

union {[}4 1{]}:

\begin{longtable}[]{@{}l@{}}
\toprule
:V1\tabularnewline
\midrule
\endhead
1\tabularnewline
2\tabularnewline
3\tabularnewline
4\tabularnewline
\bottomrule
\end{longtable}

\begin{Shaded}
\begin{Highlighting}[]
\NormalTok{(api/concat x y)}
\end{Highlighting}
\end{Shaded}

null {[}10 1{]}:

\begin{longtable}[]{@{}l@{}}
\toprule
:V1\tabularnewline
\midrule
\endhead
1\tabularnewline
2\tabularnewline
2\tabularnewline
3\tabularnewline
3\tabularnewline
2\tabularnewline
2\tabularnewline
3\tabularnewline
4\tabularnewline
4\tabularnewline
\bottomrule
\end{longtable}

\begin{center}\rule{0.5\linewidth}{0.5pt}\end{center}

Equality not implemented

\end{document}
